\documentclass[12pt]{article}
\usepackage{amsmath}
\usepackage{amsfonts}
\usepackage{parskip}
\usepackage{amsthm}
\usepackage{thmtools}
\usepackage[headheight=15pt]{geometry}
\geometry{a4paper, left=20mm, right=20mm, top=30mm, bottom=30mm}
\usepackage{graphicx}
\usepackage{bm} % for bold font in math mode - command is \bm{text}
\usepackage{enumitem}
\usepackage{fancyhdr}
\usepackage{amssymb} % for stacked arrows and other shit
\pagestyle{fancy}
\usepackage{changepage}
\usepackage{mathcomp}
\usepackage{tcolorbox}

\declaretheoremstyle[headfont=\normalfont]{normal}
\declaretheorem[style=normal]{Theorem}
\declaretheorem[style=normal]{Proposition}
\declaretheorem[style=normal]{Lemma}
\newcounter{ProofCounter}
\newcounter{ClaimCounter}[ProofCounter]
\newcounter{SubClaimCounter}[ClaimCounter]
\newenvironment{Proof}{\stepcounter{ProofCounter}\textsc{Proof.}}{\hfill$\square$}
\newenvironment{Solution}{\stepcounter{ProofCounter}\textbf{Solution:}}{\hfill$\square$}
\newenvironment{claim}[1]{\vspace{1mm}\stepcounter{ClaimCounter}\par\noindent\underline{\bf Claim \theClaimCounter:}\space#1}{}
\newenvironment{claimproof}[1]{\par\noindent\underline{Proof of claim \theClaimCounter:}\space#1}{\hfill $\blacksquare$ Claim \theClaimCounter}
\newenvironment{subclaim}[1]{\stepcounter{SubClaimCounter}\par\noindent\emph{Subclaim \theClaimCounter.\theSubClaimCounter:}\space#1}{}
% \newenvironment{subclaimproof}[1]{\begin{adjustwidth}{2em}{0pt}\par\noindent\emph{Proof of subclaim \theClaimCounter.\theSubClaimCounter:}\space#1}{\hfill
% $\blacksquare$ \emph{Subclaim \theClaimCounter.\theSubClaimCounter}\vspace{5mm}\end{adjustwidth}}
\newenvironment{subclaimproof}[1]{\par\noindent\emph{Proof of subclaim \theClaimCounter.\theSubClaimCounter:}\space#1}{\hfill
$\Diamond$ \emph{Subclaim \theClaimCounter.\theSubClaimCounter}}

\allowdisplaybreaks{}

% chktex-file 3

\title{MATH 533: Assignment 1}
\author{Evan ``Pete'' Walsh}
\makeatletter
\makeatother
\lhead{Evan P. Walsh}
\chead{MATH 533: HW 1}
\rhead{\thepage}
\cfoot{}

\begin{document}
\maketitle


\subsection*{1}
\begin{table}[h!]
  \centering
  \begin{tabular}{c|c|c}
    Key length (bits) & Number of keys & Attack Time \\
    \hline 
    56 & $\approx 7.2\times 10^{16}$ & $\approx 7.6\times 10^{-3}$ years ($\approx 67$ hours) \\
    128 & $\approx 3.4\times 10^{38}$ & $\approx 3.6 \times 10^{19}$ years \\
    192 & $\approx 6.3\times 10^{57}$ & $\approx 6.6 \times 10^{38}$ years \\
    256 & $\approx 1.2\times 10^{77}$ & $\approx 1.2 \times 10^{58}$ years \\
  \end{tabular}
\end{table}

The number of keys that this organization could test per second would be 
\[
  K := \exp\left\{\log(1\times 10^5) + \log(3\times 10^{9}) - \log(1\times 10^{3})\right\} = \exp(26.42705) \approx 3\times 10^{11}.
\]
Thus, the attack times (in seconds) were calculated by dividing the total number of possible keys by
$K = \exp(26.42705)$.

Now assume the goverment could use specialized hardware to increase the speed of such an attack by 1000 times. In particular,
let's suppose they could now check $K \times 1000$ keys per second. Then the time it would take to break the 128 bit key would 
still be about $3.6\times 10^{16}$ years, not much of an improvement in the grand scheme of things.
However, if they could cheaply guess half of the bits in the key, then in the 128 bit key there would only be 56 bits remaining to guess,
which would take about 67 hours with the original computing power.



\end{document}

