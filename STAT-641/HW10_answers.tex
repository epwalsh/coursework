\documentclass[12pt]{article}
\usepackage{amsmath}
\usepackage{amsfonts}
\usepackage{parskip}
\usepackage{amsthm}
\usepackage{thmtools}
\usepackage[headheight=15pt]{geometry}
\geometry{a4paper, left=20mm, right=20mm, top=30mm, bottom=30mm}
\usepackage{graphicx}
\usepackage{bm} % for bold font in math mode - command is \bm{text}
\usepackage{enumitem}
\usepackage{fancyhdr}
\usepackage{amssymb} % for stacked arrows and other shit
\pagestyle{fancy}
\usepackage{changepage}
\usepackage{mathcomp}

\declaretheoremstyle[headfont=\normalfont]{normal}
\declaretheorem[style=normal]{Theorem}
\declaretheorem[style=normal]{Proposition}
\declaretheorem[style=normal]{Lemma}
\newcounter{ProofCounter}
\newcounter{ClaimCounter}[ProofCounter]
\newcounter{SubClaimCounter}[ClaimCounter]
\newenvironment{Proof}{\stepcounter{ProofCounter}\textit{Proof.}}{\hfill$\square$}
\newenvironment{claim}[1]{\vspace{3mm}\stepcounter{ClaimCounter}\par\noindent\underline{\bf Claim \theClaimCounter:}\space#1}{}
\newenvironment{claimproof}[1]{\par\noindent\underline{Proof of claim \theClaimCounter:}\space#1}{\hfill $\blacksquare$ Claim \theClaimCounter}
\newenvironment{subclaim}[1]{\stepcounter{SubClaimCounter}\par\noindent\emph{Subclaim \theClaimCounter.\theSubClaimCounter:}\space#1}{}
% \newenvironment{subclaimproof}[1]{\begin{adjustwidth}{2em}{0pt}\par\noindent\emph{Proof of subclaim \theClaimCounter.\theSubClaimCounter:}\space#1}{\hfill
% $\blacksquare$ \emph{Subclaim \theClaimCounter.\theSubClaimCounter}\vspace{5mm}\end{adjustwidth}}
\newenvironment{subclaimproof}[1]{\par\noindent\emph{Proof of subclaim \theClaimCounter.\theSubClaimCounter:}\space#1}{\hfill
$\Diamond$ \emph{Subclaim \theClaimCounter.\theSubClaimCounter}}

\title{STAT 641: HW 10}
\author{Evan P. Walsh}
\makeatletter
\let\runauthor\@author
\let\runtitle\@title
\makeatother
\lhead{\runauthor}
\chead{\runtitle}
\rhead{\thepage}
\cfoot{}

\begin{document}
% \maketitle

\section*{5.12}
Let $\mu$ and $\lambda$ be $\sigma$-finite measures on $(\mathbb{R}, \mathcal{B}(\mathbb{R}))$. Let $\nu(A) = (\mu * \lambda)(A)$.
\begin{enumerate}[label=(\alph*)]
\item Show that for any Borel measurable $f : \mathbb{R} \rightarrow \mathbb{R}_{+}$, $f(x + y)$ is $\langle\mathcal{B}(\mathbb{R})\otimes
\mathcal{B}(\mathbb{R}), \mathcal{B}(\mathbb{R})\rangle$-measurable, and 
\[ \int fd\mu = \int \int f(x + y) d\mu(x) d\lambda(y). \]
\item Show that $\nu(A) = \int \mu(A - t)d\lambda(t)$ for all $A \in \mathcal{B}(\mathbb{R})$.
\item Suppose there exists countable sets $B_{\lambda}, B_{\mu}$ such that $\mu(B_{\mu}^{c}) = 0 = \lambda(B_{\lambda}^{c})$. Show that there exists a
countable $B_{\nu}$ such that $\nu(B_{\nu}^{c}) = 0$.
\item Suppose $\mu\left( \left\{ x \right\} \right) = 0$ for all $x \in \mathbb{R}$. Show that $\nu\left( \left\{ x \right\} \right) = 0$ for all $x
\in \mathbb{R}$.
\item Suppose $\nu << m$ with $\frac{d\mu}{dm} := h$. Show that $\nu << m$ and find $\frac{d\mu}{dm}$ in terms of $h$ and $\lambda$.
\item Suppose that $\nu << m$ and $\lambda << m$. Show that $\frac{d\nu}{dm} = \frac{d\mu}{dm}*\frac{d\lambda}{dm}$. 
\end{enumerate}

\subsection*{Solution}

\begin{enumerate}[label=(\alph*)]
\item 
\begin{Proof}

\begin{claim}
$f(x + y)$ is measurable.
\end{claim}
\begin{claimproof}
Denote $g_{0}(x,y) := x$, $g_{1}(x,y) := y$, and $g(x,y) := g_{0} - g_{1}$. Clearly $g_{0}, g_{1}$ are $\langle\mathcal{B}(\mathbb{R})\otimes
\mathcal{B}(\mathbb{R}), \mathcal{B}(\mathbb{R})\rangle$-measurable, so $g$ is also $\langle\mathcal{B}(\mathbb{R})\otimes
\mathcal{B}(\mathbb{R}), \mathcal{B}(\mathbb{R})\rangle$-measurable. Therefore $f \circ g$ is $\langle\mathcal{B}(\mathbb{R})\otimes
\mathcal{B}(\mathbb{R}), \mathcal{B}(\mathbb{R})\rangle$-measurable as the composition of measurable functions.
\end{claimproof}

\begin{claim}
$\int fd\nu = \int \int f(x + y)d\mu(x) d\lambda(y)$.
\end{claim}
\begin{claimproof}
Let $g := x + y$. Note that for all $A \in \mathcal{B}(\mathbb{R})$, 
\begin{align*}
\nu(A) = \int \int I_{A}(x + y) d\mu(x)d\lambda(y) & = \int \int I_{g^{-1}[A]}(x,y)d\mu(x)d\lambda(y) \\
& = \int_{\mathbb{R}\times\mathbb{R}}I_{g^{-1}[A]}d(\mu\times\lambda) = (\mu\times\lambda)g^{-1}[A].
\end{align*}
Thus, by the change of variables formula,
\begin{align*}
\int f\nu = \int f d(\mu \times \lambda g^{-1}) & = \int f\circ gd(\mu \times \lambda) \\
& = \int \int f\circ g d\mu d\lambda = \int \int f(x + y)d\mu(x) d\lambda(y).
\end{align*}
\end{claimproof}

\end{Proof}

\item 
\begin{Proof}
Let $A \in \mathcal{B}(\mathbb{R})$. Then 
\[ \nu(A) = \int \int I_{A}(x + t) d\mu(x)d\lambda(t) = \int \int I_{A -t}(x)d\mu(x)d\lambda(t) = \int \mu(A - t)d\lambda(t). \]
\end{Proof}

\item 
\begin{Proof}
Let $B\nu := \cup_{b \in B_{\mu}}\cup_{c \in B_{\lambda}}\left\{ b + c \right\}$. Let $g(x,y) := x + y$ for all $(x,y) \in
\mathbb{R}\times\mathbb{R}$. Then 
\begin{align*}
g^{-1}[B_{\mu}^{c}] = \left\{ (x,y) \in \mathbb{R}\times \mathbb{R} : x + y \notin B_{\nu} \right\} & = \left\{ (x,y) : x + y \notin \cup_{b \in B_{\mu}}\cup_{c\in B_{\lambda}}\left\{ b + c \right\} \right\} \\
& \subseteq \left\{ (x,y) : x \notin B_{\mu} \text{ or } y \notin B_{\lambda} \right\} \\
& = \left\{ (x,y) : x \in B_{\mu}^{c} \text{ or } y \in B_{\lambda}^{c} \right\} \\
& = B_{\mu}^{c} \times \mathbb{R} \cup \mathbb{R} \times B_{\lambda}^{c}.
\end{align*}
Therefore,
\[ I_{g^{-1}[B_{\nu}^{c}]}(x,y) \leq I_{B_{\mu}^{c}\times \mathbb{R}\cup \mathbb{R}\times B_{\lambda}^{c}}(x,y) \leq
I_{B_{\mu}^{c}\times\mathbb{R}}(x,y) + I_{\mathbb{R}\times B_{\lambda}^{c}}(x,y) = I_{B_{\mu}^{c}}(x) + I_{B_{\lambda}^{c}}(y). \]
Thus,
\begin{align*}
\nu(B_{\nu}^{c})& = \int \int I_{B_{\nu}^{c}}(x + y)d\mu(x) d\lambda(y)  \\
& = \int \int I_{g^{-1}[B_{\nu}^{c}]}(x,y)d\mu(x)d\lambda(y) \\
& \leq \int\int \left( I_{B_{\mu}^{c}}(x) + I_{B_{\lambda}^{c}}(y) \right)d\mu(x)d\lambda(y) \\
& = \int \int I_{B_{\mu}^{c}}(x)d\mu(x)d\lambda(y) + \int \int I_{B_{\lambda}^{c}}(y)d\lambda(y)d\mu(x) \\
& = \int \mu(B_{\mu}^{c})d\lambda + \int \lambda(B_{\lambda}^{c})d\mu = \int 0d\lambda + \int 0d\mu = 0.
\end{align*}
\end{Proof}

\item
\begin{Proof}
Let $x_{0} \in \mathbb{R}$. By assumption, $\mu\left( \left\{ x_{0} - y \right\} \right) = 0$ for all $y \in \mathbb{R}$. Therefore,
\begin{align*}
\nu\left( \left\{ x_{0} \right\} \right) = \int \int I_{ \left\{ x_{0} \right\} }(x+y)d\mu(x) d\lambda(y) & = \int \int I_{ \left\{ x_{0} - y \right\}
}(x)d\mu(x)d\lambda(y) \\
& = \int \mu\left( \left\{ x_{0} - y \right\} \right)d\lambda(y) = \int 0 d\lambda = 0.
\end{align*}
\end{Proof}

\item 
\begin{Proof}

\begin{claim}
$\nu << m$.
\end{claim}
\begin{claimproof}
Let $A \in \mathcal{B}(\mathbb{R})$ such that $m(A) = 0$. Then $m(A - y) = 0$ for all $y \in \mathbb{R}$. Thus, $\mu(A - y) = 0$ by assumption. Hence,
by part (b),
\[ \nu(A) = \int \mu(A - y)d\lambda(y) = \int 0d\lambda = 0. \]
\end{claimproof}

\begin{claim}
$\frac{d\nu}{dm}(t) = \int h(t- y)d\lambda(y)$ a.e. $(m)$.
\end{claim}
\begin{claimproof}
Let $A \in \mathcal{B}(\mathbb{R})$.
\begin{subclaim}
For each $y \in \mathbb{R}$, 
\[ \int I_{A-y}(t)h(t)dm(t) = \int I_{A}(t)h(t-y)dm. \]
\end{subclaim}
\begin{subclaimproof}
Let $y \in \mathbb{R}$ and let $f,g : \mathbb{R} \rightarrow \mathbb{R}$ be defined by $g(t) := t - y$ and $f(t) := I_{A-y}(t)h(t)$. Then, by the
change of variables formula and since the Legesgue measure is translation invariant,
\[ \int I_{A-y}(t)h(t)dm(t) = \int fdm = \int fdmg^{-1} = \int f\circ gdm = \int I_{A}(t)h(t-y)dm(t). \]
\end{subclaimproof}

Thus,
\begin{align*}
\nu(A) = \int \mu(A - y)d\lambda(y) & = \int \int I_{A-y}(t)h(t)dm(t)d\lambda(y) \\
\text{(claim 1) } & = \int \int I_{A}(t)h(t - y)dm(t)d\lambda(y) \\
\text{(Tonelli's) } & = \int \int I_{A}(t)h(t-y)d\lambda(y)dm(t) \\
& = \int_{A}\int_{\mathbb{R}}h(t-y)d\lambda(y)dm(t).
\end{align*}
So by uniqueness, $\frac{d\nu}{dm}(t) = \int h(t-y)d\lambda(y)$ a.e. $(m)$.
\end{claimproof}

\end{Proof}

\item 
\begin{Proof}
We will use the fact that since $\lambda << m$, $\int fd\lambda = \int f\frac{d\lambda}{dm}dm$ for any non-negative measurable $f$. Let $A \in
\mathcal{B}(\mathbb{R})$. Therefore, by part (d),
\begin{align*}
\nu(A) = \int_{A}\int_{\mathbb{R}}\frac{d\mu}{dm}(t-y)d\lambda(y)dm(t) & =
\int_{A}\int_{\mathbb{R}}\frac{d\mu}{dm}(t-y)\frac{d\lambda}{dm}(y)dm(y)dm(t) \\
& = \int_{A} \frac{d\mu}{dm} * \frac{d\lambda}{dm}(t)dm(t).
\end{align*}
So by uniqueness, $\frac{d\nu}{dm} = \frac{d\mu}{dm}*\frac{d\lambda}{dm}$ a.e. $(m)$.
\end{Proof}
\end{enumerate}


\newpage 
\section*{5.14}
Let $f,g \in L^{1}(\mathbb{R}, \mathcal{B}(\mathbb{R}), m)$.
\begin{enumerate}[label=(\alph*)]
\item Show that if $f$ is continuous and bounded on $\mathbb{R}$ then so is $f*g$.
\item Show that if $f$ is differentiable with a bounded derivative on $\mathbb{R}$, then so is $f * g$.
\end{enumerate}

\subsection*{Solution}
\begin{enumerate}[label=(\alph*)]
\item 
\begin{Proof}
Suppose $f$ is continuous on $\mathbb{R}$ and $|f(x)| \leq c$ for all $x \in \mathbb{R}$.
\begin{claim}
$f*g$ is bounded.
\end{claim}
\begin{claimproof}
Let $x \in \mathbb{R}$. Since $f(x) \leq c$,
\[ |(f*g)(x)| = \left| \int f(x-u)g(u)dm(u)\right| \leq \left| \int cg(u)dm(u) \right| \leq c \int |g|dm < \infty, \]
since $g \in L^{1}(\mathbb{R})$.
\end{claimproof}

\begin{claim}
$f*g$ is continuous.
\end{claim}
\begin{claimproof}
Let $x \in \mathbb{R}$ and suppose $\left\{ x_{n} \right\}_{n=0}^{\infty} \subset \mathbb{R}$ such that $x_{n} \rightarrow x$ as $n \rightarrow
\infty$. We need to show that $(f*g)(x_{n}) \rightarrow (f*g)(x)$. For each $n \in \mathbb{R}$, let $h_{n}(u) := f(x_{n} - u)g(u)$ for all $u \in
\mathbb{R}$. Since $f \leq c$, $|h_{n}| \leq c|g|$, where $c|g| \in L^{1}(\mathbb{R})$ by the work done in claim 1. Thus, by the DCT,
\[ \lim_{n\rightarrow\infty}(f*g)(x_{n}) = \lim_{n\rightarrow\infty}\int h_{n}dm \stackrel{\text{DCT}}{=} \int f(x-u)g(u)dm(u) = (f*g)(x). \]
\end{claimproof}

\end{Proof}

\item 
\begin{Proof}
Suppose $f$ is differentiable and $|f'| \leq c$ for some $c \in \mathbb{R}$. 
\begin{claim}
$(f*g)'(x) = \int f'(x-u)g(u)dm(u)$ for all $x \in \mathbb{R}$.
\end{claim}
\begin{claimproof}
Let $x \in \mathbb{R}$ and suppose $\left\{ x_{n}
\right\}_{n=0}^{\infty} \subset \mathbb{R}$ such that $x_{n} \rightarrow x$. For each $n \in \mathbb{N}$, let $h_{n}(u) := \frac{f(x_{n} -u) -
f(x-u)}{x_{n} - x}g(u)$ for all $u \in \mathbb{R}$. Clearly $h_{n}(u) \rightarrow f'(x-u)g(u)$ and $|h_{n}(u)| \leq c|g(u)|$ for all $u \in
\mathbb{R}$. Also, $\int |cg(u)|dm(u) = c \int |g|dm < \infty$,
since $g \in L^{1}(\mathbb{R})$. Therefore we can apply the DCT to the sequence $\left\{ h_{n} \right\}_{n=0}^{\infty}$. So,
\begin{align*}
\int f'(x-u)g(u)dm(u) & = \lim_{n\rightarrow\infty}\int \frac{f(x_{n} - u) - f(x-u)}{x_{n} - x}g(u)dm(u) \\
& = \lim_{n\rightarrow\infty} \frac{ \int f(x_{n} - u)g(u)dm(u) - \int f(x-u)g(u)dm(u)}{x_{n} - x} \\
& = \lim_{n\rightarrow\infty}\frac{ (f*g)(x_{n}) - (f*g)(x) }{x_{n} - x}  = (f*g)'(x)
\end{align*}
The last line in the above equality is true since the sequence $\left\{ x_{n} \right\}_{n=0}^{\infty}$ was an arbitrary sequence converging to $x$.
\end{claimproof}

\begin{claim}
$(f*g)'(x) \leq c||g||_{1}$ for all $x \in \mathbb{R}$.
\end{claim}
\begin{claimproof}
By claim 1, 
\[ |(f*g)'(x)| = \left| \int f'(x-u)g(u)dm(u)\right| \leq \int |f'(x-u)g(u)|dm(u) \leq c ||g||_{1} < \infty, \]
so $(f*g)'$ is bounded.
\end{claimproof}

\end{Proof}

\end{enumerate}
















\end{document}

