\documentclass[12pt]{article}
\usepackage{amsmath}
\usepackage{amsfonts}
\usepackage{parskip}
\usepackage{amsthm}
\usepackage{thmtools}
\usepackage[headheight=15pt]{geometry}
\geometry{a4paper, left=20mm, right=20mm, top=30mm, bottom=30mm}
\usepackage{graphicx}
\usepackage{bm} % for bold font in math mode - command is \bm{text}
\usepackage{enumitem}
\usepackage{fancyhdr}
\usepackage{amssymb} % for stacked arrows and other shit
\pagestyle{fancy}
\usepackage{changepage}
\usepackage{mathcomp}

\declaretheoremstyle[headfont=\normalfont]{normal}
\declaretheorem[style=normal]{Theorem}
\declaretheorem[style=normal]{Proposition}
\declaretheorem[style=normal]{Lemma}
\newcounter{ProofCounter}
\newcounter{ClaimCounter}[ProofCounter]
\newcounter{SubClaimCounter}[ClaimCounter]
\newenvironment{Proof}{\stepcounter{ProofCounter}\textit{Proof.}}{\hfill$\square$}
\newenvironment{claim}[1]{\vspace{3mm}\stepcounter{ClaimCounter}\par\noindent\underline{\bf Claim \theClaimCounter:}\space#1}{}
\newenvironment{claimproof}[1]{\par\noindent\underline{Proof of claim \theClaimCounter:}\space#1}{\hfill $\blacksquare$ Claim \theClaimCounter}
\newenvironment{subclaim}[1]{\stepcounter{SubClaimCounter}\par\noindent\emph{Subclaim \theClaimCounter.\theSubClaimCounter:}\space#1}{}
% \newenvironment{subclaimproof}[1]{\begin{adjustwidth}{2em}{0pt}\par\noindent\emph{Proof of subclaim \theClaimCounter.\theSubClaimCounter:}\space#1}{\hfill
% $\blacksquare$ \emph{Subclaim \theClaimCounter.\theSubClaimCounter}\vspace{5mm}\end{adjustwidth}}
\newenvironment{subclaimproof}[1]{\par\noindent\emph{Proof of subclaim \theClaimCounter.\theSubClaimCounter:}\space#1}{\hfill
$\Diamond$ \emph{Subclaim \theClaimCounter.\theSubClaimCounter}}

\title{STAT 641: HW 10}
\author{Evan P. Walsh}
\makeatletter
\let\runauthor\@author
\let\runtitle\@title
\makeatother
\lhead{\runauthor}
\chead{\runtitle}
\rhead{\thepage}
\cfoot{}

\begin{document}
% \maketitle

\section*{5.12}
Let $\mu$ and $\lambda$ be $\sigma$-finite measures on $(\mathbb{R}, \mathcal{B}(\mathbb{R}))$. Let $\nu(A) = (\mu * \lambda)(A)$.
\begin{enumerate}[label=(\alph*)]
\item Show that for any Borel measurable $f : \mathbb{R} \rightarrow \mathbb{R}_{+}$, $f(x + y)$ is $\langle\mathcal{B}(\mathbb{R})\otimes
\mathcal{B}(\mathbb{R}), \mathcal{B}(\mathbb{R})\rangle$-measurable, and 
\[ \int f\ d\nu = \int \int f(x + y)\ d\mu(x) d\lambda(y). \]
\item Show that $\nu(A) = \int \mu(A - t)\ d\lambda(t)$ for all $A \in \mathcal{B}(\mathbb{R})$.
\item Suppose there exists countable sets $B_{\lambda}, B_{\mu}$ such that $\mu(B_{\mu}^{c}) = 0 = \lambda(B_{\lambda}^{c})$. Show that there exists a
countable $B_{\nu}$ such that $\nu(B_{\nu}^{c}) = 0$.
\item Suppose $\mu\left( \left\{ x \right\} \right) = 0$ for all $x \in \mathbb{R}$. Show that $\nu\left( \left\{ x \right\} \right) = 0$ for all $x
\in \mathbb{R}$.
\item Suppose $\nu << m$ with $\frac{d\mu}{dm} := h$. Show that $\nu << m$ and find $\frac{d\mu}{dm}$ in terms of $h$ and $\lambda$.
\item Suppose that $\nu << m$ and $\lambda << m$. Show that $\frac{d\nu}{dm} = \frac{d\mu}{dm}*\frac{d\lambda}{dm}$. 
\end{enumerate}

\subsection*{Solution}

\begin{enumerate}[label=(\alph*)]
\item 
\begin{Proof}

\begin{claim}
$f(x + y)$ is measurable.
\end{claim}
\begin{claimproof}
Denote $g_{0}(x,y) := x$, $g_{1}(x,y) := y$, and $g(x,y) := g_{0} - g_{1}$. Clearly $g_{0}, g_{1}$ are $\langle\mathcal{B}(\mathbb{R})\otimes
\mathcal{B}(\mathbb{R}), \mathcal{B}(\mathbb{R})\rangle$-measurable, so $g$ is also $\langle\mathcal{B}(\mathbb{R})\otimes
\mathcal{B}(\mathbb{R}), \mathcal{B}(\mathbb{R})\rangle$-measurable. Therefore $f \circ g$ is $\langle\mathcal{B}(\mathbb{R})\otimes
\mathcal{B}(\mathbb{R}), \mathcal{B}(\mathbb{R})\rangle$-measurable as the composition of measurable functions.
\end{claimproof}

\begin{claim}
$\int fd\nu = \int \int f(x + y)\ d\mu(x) d\lambda(y)$.
\end{claim}
\begin{claimproof}
Let $g := x + y$. Note that for all $A \in \mathcal{B}(\mathbb{R})$, 
\begin{align*}
\nu(A) = \int \int I_{A}(x + y)\  d\mu(x)d\lambda(y) & = \int \int I_{g^{-1}[A]}(x,y)\ d\mu(x)d\lambda(y) \\
& = \int_{\mathbb{R}\times\mathbb{R}}I_{g^{-1}[A]}\ d(\mu\times\lambda) = (\mu\times\lambda)g^{-1}[A].
\end{align*}
Thus, by the change of variables formula,
\begin{align*}
\int f\nu = \int f \ d(\mu \times \lambda)g^{-1} & = \int f\circ g\ d(\mu \times \lambda) \\
& = \int \int f\circ g \ d\mu \ d\lambda = \int \int f(x + y)\ d\mu(x) \ d\lambda(y).
\end{align*}
\end{claimproof}

\end{Proof}

\item % (b)
\begin{Proof}
Let $A \in \mathcal{B}(\mathbb{R})$. Then 
\[ \nu(A) = \int \int I_{A}(x + t) \ d\mu(x)\ d\lambda(t) = \int \int I_{A -t}(x)\ d\mu(x)\ d\lambda(t) = \int \mu(A - t)\ d\lambda(t). \]
\end{Proof}

\item  % (c)
\begin{Proof}
Let $B\nu := \cup_{b \in B_{\mu}}\cup_{c \in B_{\lambda}}\left\{ b + c \right\}$. Let $g(x,y) := x + y$ for all $(x,y) \in
\mathbb{R}\times\mathbb{R}$. Then 
\begin{align*}
g^{-1}[B_{\mu}^{c}] = \left\{ (x,y) \in \mathbb{R}\times \mathbb{R} : x + y \notin B_{\nu} \right\} & = \left\{ (x,y) : x + y \notin \cup_{b \in B_{\mu}}\cup_{c\in B_{\lambda}}\left\{ b + c \right\} \right\} \\
& \subseteq \left\{ (x,y) : x \notin B_{\mu} \text{ or } y \notin B_{\lambda} \right\} \\
& = \left\{ (x,y) : x \in B_{\mu}^{c} \text{ or } y \in B_{\lambda}^{c} \right\} \\
& = B_{\mu}^{c} \times \mathbb{R} \cup \mathbb{R} \times B_{\lambda}^{c}.
\end{align*}
Therefore,
\[ I_{g^{-1}[B_{\nu}^{c}]}(x,y) \leq I_{(B_{\mu}^{c}\times \mathbb{R})\cup (\mathbb{R}\times B_{\lambda}^{c})}(x,y) \leq
I_{B_{\mu}^{c}\times\mathbb{R}}(x,y) + I_{\mathbb{R}\times B_{\lambda}^{c}}(x,y) = I_{B_{\mu}^{c}}(x) + I_{B_{\lambda}^{c}}(y). \]
Thus,
\begin{align*}
\nu(B_{\nu}^{c})& = \int \int I_{B_{\nu}^{c}}(x + y)\ d\mu(x)\ d\lambda(y)  \\
& = \int \int I_{g^{-1}[B_{\nu}^{c}]}(x,y)\ d\mu(x)\ d\lambda(y) \\
& \leq \int\int \left( I_{B_{\mu}^{c}}(x) + I_{B_{\lambda}^{c}}(y) \right)\ d\mu(x)\ d\lambda(y) \\
& = \int \int I_{B_{\mu}^{c}}(x)d\mu(x)d\lambda(y) + \int \int I_{B_{\lambda}^{c}}(y)\ d\lambda(y)\ d\mu(x) \\
& = \int \mu(B_{\mu}^{c})\ d\lambda + \int \lambda(B_{\lambda}^{c})\ d\mu = \int 0\ d\lambda + \int 0\ d\mu = 0.
\end{align*}
\end{Proof}

\item % (d)
\begin{Proof}
Let $x_{0} \in \mathbb{R}$. By assumption, $\mu\left( \left\{ x_{0} - y \right\} \right) = 0$ for all $y \in \mathbb{R}$. Therefore,
\begin{align*}
\nu\left( \left\{ x_{0} \right\} \right) = \int \int I_{ \left\{ x_{0} \right\} }(x+y)\ d\mu(x)\ d\lambda(y) & = \int \int I_{ \left\{ x_{0} - y \right\}
}(x)\ d\mu(x)\ d\lambda(y) \\
& = \int \mu\left( \left\{ x_{0} - y \right\} \right)\ d\lambda(y) = \int 0 \ d\lambda = 0.
\end{align*}
\end{Proof}

\vspace{-5mm}
\item % (e)
\begin{Proof}

\vspace{-5mm}
\begin{claim}
$\nu << m$.
\end{claim}
\begin{claimproof}
Let $A \in \mathcal{B}(\mathbb{R})$ such that $m(A) = 0$. Then $m(A - y) = 0$ for all $y \in \mathbb{R}$. Thus, $\mu(A - y) = 0$ by assumption. Hence,
by part (b),
\[ \nu(A) = \int \mu(A - y)\ d\lambda(y) = \int 0 \ d\lambda = 0. \]
\end{claimproof}

\begin{claim}
$\frac{d\nu}{dm}(t) = \int h(t- y) \ d\lambda(y)$ a.e. $(m)$.
\end{claim}
\begin{claimproof}
Let $A \in \mathcal{B}(\mathbb{R})$.
\begin{subclaim}
For each $y \in \mathbb{R}$, 
\[ \int I_{A-y}(t)h(t)\ dm(t) = \int I_{A}(t)h(t-y)\ dm. \]
\end{subclaim}
\begin{subclaimproof}
Let $y \in \mathbb{R}$ and let $f,g : \mathbb{R} \rightarrow \mathbb{R}$ be defined by $g(t) := t - y$ and $f(t) := I_{A-y}(t)h(t)$. Then, by the
change of variables formula and since the Legesgue measure is translation invariant,
\[ \int I_{A-y}(t)h(t)\ dm(t) = \int f\ dm = \int f\ dmg^{-1} = \int f\circ g\ dm = \int I_{A}(t)h(t-y)\ dm(t). \]
\end{subclaimproof}

Thus,
\begin{align*}
\nu(A) = \int \mu(A - y)\ d\lambda(y) & = \int \int I_{A-y}(t)h(t)\ dm(t)\ d\lambda(y) \\
\text{(claim 1) } & = \int \int I_{A}(t)h(t - y)\ dm(t)\ d\lambda(y) \\
\text{(Tonelli's) } & = \int \int I_{A}(t)h(t-y)\ d\lambda(y)\ dm(t) \\
& = \int_{A}\int_{\mathbb{R}}h(t-y)\ d\lambda(y)\ dm(t).
\end{align*}
So by uniqueness, $\frac{d\nu}{dm}(t) = \int h(t-y)\ d\lambda(y)$ a.e. $(m)$.
\end{claimproof}

\end{Proof}

\item 
\begin{Proof}
We will use the fact that since $\lambda << m$, $\int fd\lambda = \int f\frac{d\lambda}{dm}\ dm$ for any non-negative measurable $f$. Let $A \in
\mathcal{B}(\mathbb{R})$. Therefore, by part (e),
\begin{align*}
\nu(A) = \int_{A}\int_{\mathbb{R}}\frac{d\mu}{dm}(t-y)\ d\lambda(y)\ dm(t) & =
\int_{A}\int_{\mathbb{R}}\frac{d\mu}{dm}(t-y)\frac{d\lambda}{dm}(y)\ dm(y)\ dm(t) \\
& = \int_{A} \left(\frac{d\mu}{dm} * \frac{d\lambda}{dm}\right)(t)\ dm(t).
\end{align*}
So by uniqueness, $\frac{d\nu}{dm} = \frac{d\mu}{dm}*\frac{d\lambda}{dm}$ a.e. $(m)$.
\end{Proof}
\end{enumerate}












\newpage 
\section*{5.14}
Let $f,g \in L^{1}(\mathbb{R}, \mathcal{B}(\mathbb{R}), m)$.
\begin{enumerate}[label=(\alph*)]
\item Show that if $f$ is continuous and bounded on $\mathbb{R}$ then so is $f*g$.
\item Show that if $f$ is differentiable with a bounded derivative on $\mathbb{R}$, then so is $f * g$.
\end{enumerate}

\subsection*{Solution}
\begin{enumerate}[label=(\alph*)]
\item 
\begin{Proof}
Suppose $f$ is continuous on $\mathbb{R}$ and $|f(x)| \leq c$ for all $x \in \mathbb{R}$.
\begin{claim}
$f*g$ is bounded.
\end{claim}
\begin{claimproof}
Let $x \in \mathbb{R}$. Since $f(x) \leq c$,
\[ |(f*g)(x)| = \left| \int f(x-u)g(u)\ dm(u)\right| \leq \left| \int cg(u)\ dm(u) \right| \leq c \int |g|\ dm < \infty, \]
since $g \in L^{1}(\mathbb{R})$.
\end{claimproof}

\begin{claim}
$f*g$ is continuous.
\end{claim}
\begin{claimproof}
Let $x \in \mathbb{R}$ and suppose $\left\{ x_{n} \right\}_{n=0}^{\infty} \subset \mathbb{R}$ such that $x_{n} \rightarrow x$ as $n \rightarrow
\infty$. We need to show that $(f*g)(x_{n}) \rightarrow (f*g)(x)$. For each $n \in \mathbb{R}$, let $h_{n}(u) := f(x_{n} - u)g(u)$ for all $u \in
\mathbb{R}$. Since $f \leq c$, $|h_{n}| \leq c|g|$, where $c|g| \in L^{1}(\mathbb{R})$ by the work done in claim 1. Thus, by the DCT,
\[ \lim_{n\rightarrow\infty}(f*g)(x_{n}) = \lim_{n\rightarrow\infty}\int h_{n}\ dm \stackrel{\text{DCT}}{=} \int f(x-u)g(u)\ dm(u) = (f*g)(x). \]
\end{claimproof}

\end{Proof}

\item 
\begin{Proof}
Suppose $f$ is differentiable and $|f'| \leq c$ for some $c \in \mathbb{R}$. 
\begin{claim}
$(f*g)'(x) = \int f'(x-u)g(u)dm(u)$ for all $x \in \mathbb{R}$.
\end{claim}
\begin{claimproof}
Let $x \in \mathbb{R}$ and suppose $\left\{ x_{n}
\right\}_{n=0}^{\infty} \subset \mathbb{R}$ such that $x_{n} \rightarrow x$. For each $n \in \mathbb{N}$, let $h_{n}(u) := \frac{f(x_{n} -u) -
f(x-u)}{x_{n} - x}g(u)$ for all $u \in \mathbb{R}$. Clearly $h_{n}(u) \rightarrow f'(x-u)g(u)$ and $|h_{n}(u)| \leq c|g(u)|$ for all $u \in
\mathbb{R}$. Also, $\int |cg(u)|\ dm(u) = c \int |g|\ dm < \infty$,
since $g \in L^{1}(\mathbb{R})$. Therefore we can apply the DCT to the sequence $\left\{ h_{n} \right\}_{n=0}^{\infty}$. So,
\begin{align*}
\int f'(x-u)g(u)dm(u) & = \lim_{n\rightarrow\infty}\int \frac{f(x_{n} - u) - f(x-u)}{x_{n} - x}\ g(u)\ dm(u) \\
& = \lim_{n\rightarrow\infty} \frac{ \int f(x_{n} - u)g(u)\ dm(u) - \int f(x-u)g(u)\ dm(u)}{x_{n} - x} \\
& = \lim_{n\rightarrow\infty}\frac{ (f*g)(x_{n}) - (f*g)(x) }{x_{n} - x}  = (f*g)'(x)
\end{align*}
The last line in the above equality is true since the sequence $\left\{ x_{n} \right\}_{n=0}^{\infty}$ was an arbitrary sequence converging to $x$.
\end{claimproof}

\begin{claim}
$(f*g)'(x) \leq c||g||_{1}$ for all $x \in \mathbb{R}$.
\end{claim}
\begin{claimproof}
By claim 1, 
\[ |(f*g)'(x)| = \left| \int f'(x-u)g(u)\ dm(u)\right| \leq \int |f'(x-u)g(u)|\ dm(u) \leq c ||g||_{1} < \infty, \]
so $(f*g)'$ is bounded.
\end{claimproof}

\end{Proof}

\end{enumerate}














\newpage 
\section*{5.15}
Let $f \in L^{1}(\mathbb{R}), g \in L^{p}(\mathbb{R})$ for $1 \leq p < \infty$. Show that for all $x \in \mathbb{R}$,
\[ \left( \int |f(u)g(x-u)|\ dm(u) \right)^{p} \leq \left( \int |f|\ dm \right)^{p-1}\left( \int |g(x-u)|^{p}|f(u)|\ dm(u) \right), \]
and hence that $(f*g)(x)$ is well defined a.e. $(m)$ and $||f*g||_{p} \leq ||f||_{1}\cdot ||g||_{p}$.

\subsection*{Solution}
\begin{Proof}
Without loss of generality assume that $m(\left\{ f > 0 \right\}) > 0$, otherwise the result is trivial. For all $A \in \mathcal{B}(\mathbb{R})$, let
$\mu(A) := \int_{A}\frac{|f|}{||f||_{1}}\ dm$. Clearly $\mu$ is a probability measure on $(\mathbb{R}, \mathcal{B}(\mathbb{R}))$. 
Now, since $g \in L^{p}(m)$, $|g|^{p} \in L^{1}(m)$. Thus, $|g(x-u)|^{p} \in L^{1}(m)$ by the translation invariance of $m$. So, by Proposition 5.4.3,
\begin{equation}
\int |g(x-u)|^{p}|f(u)|\ dm < \infty,
\label{3.1}
\end{equation}
for almost all $x \in \mathbb{R}$. Thus, let $x \in \mathbb{R}$ such that \eqref{3.1} is satisfied
and define $h(u) := |g(x-u)|$ for all $u \in \mathbb{R}$. Let $\phi(\omega) := \omega^{p}$ for all $\omega \in \mathbb{R}$. Since $p \geq
1$, $\phi(\cdot)$ is convex over $\mathbb{R}$.

\begin{claim}
$h$ and $\phi\circ h$ are in $L^{1}(\mathbb{R}, \mathcal{B}(\mathbb{R}), \mu)$.
\end{claim}
\begin{claimproof}
By \eqref{3.1},
\[ \int |\phi\circ h|\ d\mu = \frac{1}{||f||_{1}}\int |g(x-u)|^{p}|f(u)|\ dm(u) < \infty. \]
Further, note that when $|g(x-u)| \geq 1$, $\phi \circ h = |g(x-u)|^{p} \geq |g(x-u)| = h(u)$. Hence,
\[ \int |h|\ d\mu = \int_{\{0\leq h <1\}}|h|\ d\mu + \int_{\{h\geq 1\}}|h|\ d\mu \leq \int_{\{0\leq h<1\}}1\ d\mu + \int_{\{h\geq 1\}}|\phi\circ h|\ d\mu \leq 1
+ \int_{\{h\geq 1\}}|\phi\circ h|\ d\mu < \infty. \]
\end{claimproof}

\begin{claim}
\[ \left( \int |f(u)g(x-u)|\ dm(u) \right)^{p} \leq \left( \int |f|\ dm \right)^{p-1}\left( \int |g(x-u)|^{p}|f(u)|\ dm(u) \right). \]
\end{claim}
\begin{claimproof}
By claim 1 we can apply Jensen's Inequality to $h$ and $\phi$ with the probability measure $\mu$. Thus,
\[ \left( \frac{\int |g(x-u)||f(u)|\ dm(u)}{||f||_{1}} \right)^{p} = \phi\left( \int h\ d\mu \right) \leq \int \phi\circ h\ d\mu = \frac{\int
|g(x-u)|^{p}|f(u)|\ dm(u)}{||f||_{1}}. \]
By rearranging terms, we are done.
\end{claimproof}

By claim 2, Proposition 5.4.4, and \eqref{3.1},
\[ |(f*g)(x)| = |(g*f)(x)| = \left| \int f(u)g(x-u)\ dm(u)\right| \leq \left( ||f||_{1}^{p-1}\int |g(x-u)|^{p}|f(u)|\ dm(u) \right)^{1/p} < \infty. \]
Therefore $f*g$ is well defined a.e. $(m)$.

\begin{claim}
$||f*g||_{p} \leq ||f||_{1}\cdot ||g||_{p}$.
\end{claim}
\begin{claimproof}
By claim 2,
\begin{align*}
||f*g||_{p} = \left( \int |(f*g)|^{p}\ dm \right)^{1/p} & = \left( \int \left| \int f(u)g(x-u)\ dm(u)\right|^{p}\ dm(x) \right)^{1/p} \\
& \leq \left( \int ||f||_{1}^{p-1}\int |g(x-u)|^{p}|f(u)|\ dm(u)\ dm(x) \right)^{1/p} \\
\text{(Tonelli's) } & = \left( ||f||_{1}^{p-1}\int\int |g(x-u)|^{p}|f(u)|\ dm(x)\ dm(u) \right)^{1/p} \\
& = \left( ||f||_{1}^{p-1}\int |f(u)|\int |g(x)|^{p}\ dm(x)\ dm(u) \right)^{1/p} \\
& = \left( ||f||_{1}^{p-1}\cdot ||f||_{1}\cdot ||g||_{p}^{p} \right)^{1/p} = ||f||_{1}\cdot ||g||_{p}.
\end{align*}
\end{claimproof}

\end{Proof}











\newpage 
\section*{A}
Let $\mu, \lambda$ be two probabilities on $(\mathbb{R}, \mathcal{B}(\mathbb{R}))$. Consider the functions $f(x,y) = x$, $g(x,y) = y$ defined from
$(\mathbb{R}^{2}, \mathcal{B}(\mathbb{R}^{2}))$, $\nu := \mu \times \lambda)$. Assume $f,g \in L^{1}(\mathbb{R}^{2}, \mathcal{B}(\mathbb{R}^{2}),
\nu)$. Then show that 
\[ \nu\left( \left\{ (f,g) \in A\times B \right\} \right) = \nu\left( \left\{ f\in A \right\} \right)\nu\left( \left\{ g \in B \right\} \right), \]
for all $A,B \in \mathcal{B}(\mathbb{R})$, and 
\[ \int fg\  d\nu = \left( \int f\ d\nu \right)\left( \int g\ d\nu \right). \]

\subsection*{Solution}
\begin{Proof}

\begin{claim}
$\nu\left( \left\{ (f,g) \in A\times B \right\} \right) = \nu\left( \left\{ f\in A \right\} \right)\nu\left( \left\{ g \in B \right\} \right)$
for all $A, B \in \mathcal{B}(\mathbb{R})$.
\end{claim}
\begin{claimproof}
Let $A, B \in \mathcal{B}(\mathbb{R})$. Then 
\begin{align*}
\nu\left( \left\{ (f,g) \in A\times B \right\} \right) = \nu\left(\left\{ (x,y) : x \in A \wedge y \in B \right\}\right) 
& = \nu\left( A\times B \right) \\
& = \mu(A)\cdot\lambda(B) \\
\text{(since $\mu, \lambda$ probability measures) } & = \mu(A)\cdot\lambda(\mathbb{R})\cdot\mu(\mathbb{R})\cdot \lambda(B) \\
& = \nu\left( A\times \mathbb{R} \right)\cdot \nu\left( \mathbb{R} \times B \right) \\
& = \nu\left( \left\{ (x,y) : x \in A \right\} \right)\cdot \nu\left( \left\{ (x,y) : y\in B \right\} \right) \\
& = \nu\left( \left\{ f \in A \right\} \right)\cdot \nu\left( \left\{ g \in B \right\} \right).
\end{align*}
\end{claimproof}

\begin{claim}
$f\cdot g, f, g \in L^{1}\left( \mathbb{R}^{2}, \mathcal{B}(\mathbb{R}^{2}), \nu \right)$.
\end{claim}
\begin{claimproof}
By the Cauchy-Schwarz Inequality, $||f\cdot g||_{1} \leq ||f||_{2}\cdot ||g||_{2} < \infty$, since by assumption $f,g \in L^{2}(\nu)$. Further, by
Exercise 3.18 (a), $f, g \in L^{1}(\nu)$.
\end{claimproof}

\begin{claim}
$\int fg\ d\nu = \left( \int f\ d\nu \right)\left( \int g\ d\nu \right)$.
\end{claim}
\begin{claimproof}
Due to claim 2, we can apply Fubini's Theorem. Therefore,
\begin{align*}
\int fg\ d\nu \stackrel{\text{Fubini's}}{=} \int \int fg\ d\mu\  d\lambda = \int \int x\cdot y\ d\mu(x)d\lambda(y) & = \left( \int x\ d\mu(x) \right)\left( \int y\ d\lambda(y) \right) \\
\text{(since $\mu, \lambda$ probability measures) } & = \left(\int \int x\ d\mu(x)\lambda(y)\right)\left( \int \int y\ d\lambda(y)d\mu(x) \right) \\
\text{(Fubini's again) } & = \left( \int f\ d\nu \right)\left( \int g\ d\nu \right).
\end{align*}
\end{claimproof}

\end{Proof}

In terms of random variables, this means that if $X$ and $Y$ are independent random variables, then $E(X\cdot Y) = (EX)\cdot(EY)$.







\newpage 
\section*{6.1}
Let $\mu_{1} = \mu_{2}$ be the probability distribution on $\Omega = \left\{ 1,2 \right\}$ were $\mu_{1}(\left\{ 1 \right\}) = 1/2$. Find two distinct
probability distributions on $\Omega \times \Omega$ with $\mu_{1}$ and $\mu_{2}$ as the set of marginals.

\subsection*{Solution}
Let $\mu$ and $\lambda$ be probability measures on $(\Omega \times \Omega, \mathcal{P}(\Omega\times \Omega))$ defined as follows. 
Let $\mu(1,1) = \mu(2,2) = 1/2$, while $\mu(1,2) = \mu(2,1) = 0$. On the other hand, let $\lambda(1,1) = \lambda(2,2) = 0$, while $\lambda(1,2) =
\lambda(2,1) = 1/2$.






\newpage 
\section*{6.3}
\begin{enumerate}[label=(\alph*)]
\item Let $Z \sim N(0,1), X := Z^{2}$, and $Y := e^{-X}$. Find the distributions of $P_{X}$ and $P_{Y}$ on $(\mathbb{R}, \mathcal{B}(\mathbb{R}))$ and
compute the integrals
\[ \int_{\mathbb{R}}e^{-z^{2}}\varphi(z)\ dm(z), \qquad \int_{\mathbb{R}}e^{-x}\ dP_{X}, \qquad \text{and} \qquad \int_{\mathbb{R}}y\ dP_{Y}. \]
\item Let $X_{1}, X_{2}, \hdots, X_{k}$ be iid $N(0,1)$ random variables. Let $Y := (X_{1}, \dots, X_{n})$ and $Z := Y^{2}$. Find the distributions
on $Y$ and $Z$ and evaluate the the integrals
\[ \int_{\mathbb{R}^{k}}(x_{1} + x_{2} + \dots + x_{k})^{2}\ dP_{X_{1},\hdots,X_{k}}(x_{1},\hdots,x_{k}), \qquad \int_{\mathbb{R}_{+}}z\ dP_{Z}(z),
\qquad\text{and}\qquad \int_{\mathbb{R}}y^{2}\ dP_{Y}(y).  \]
\item Let $X_{1}, \hdots, X_{k}$ be independent Binomial$(n_{i}, p)$, $i = 1, \hdots, k$ random variables. Let $Y := (X_{1} + \dots + X_{k})$. Find
the distribution $P_{Y}$ of $Y$ and evaluate the integrals 
\[ \int_{\mathbb{R}^{k}}(x_{1} + \dots + x_{k})\ dP_{X_{1},\hdots,X_{k}}(x_{1},\hdots,x_{k}) \qquad \text{and} \qquad \int_{\mathbb{R}}y\ dP_{Y}(y). \]
\end{enumerate}

\subsection*{Solution}
\begin{enumerate}[label=(\alph*)]
\item 
Let $a \geq 0$. Then 
\begin{align*}
F_{X}(a) = P_{X}\left( (-\infty, a] \right) = P(X^{-1}[(-\infty, a]]) & = P\left( \left\{ z : z^{2} \leq z \right\} \right) \\
& = P\left( \left\{ z : |z| \leq \sqrt{a} \right\} \right) \\
& = \Phi(\sqrt{a}) - \Phi(-\sqrt{a}). 
\end{align*}
For $a < 0$, clearly $F_{X}(a) = 0$. Thus,
\[ F_{X}(a) = \left\{ \begin{array}{cl}
0 & \text{ for } a < 0 \\
\Phi(\sqrt{a}) - \Phi(-\sqrt{a}) & \text{ for } a \geq 0 \\
\end{array} \right.. \]
Then $P_{X}$ is the Lebesgue-Stieltjes measure generated by $F_{X}$. Now let $0 < a \leq 1$. Then 
\begin{align*}
F_{Y}(a) = P_{Y}\left( (-\infty, a] \right) = P(Y^{-1}[(-\infty, a]]) & = P\left( \left\{ z : e^{-z^{2}} \leq a \right\} \right) \\
& = P\left( \left\{ z : |z| \geq \sqrt{-\log(a)} \right\} \right) \\
& = 2\Phi(-\sqrt{-\log(a)}).
\end{align*}
Clearly if $a > 1$, $F_{Y}(a) = 1$ and if $a \leq 0$, $F_{Y}(a) = 0$. Thus,
\[ F_{Y}(a) = \left\{ \begin{array}{cl}
0 & \text{ for } a \leq 0 \\
2\Phi(-\sqrt{-\log(a)}) & \text{ for } 0 < a \leq 1 \\
1 & \text{ for } a > 1 \\
\end{array} \right. . \]
Then $P_{Y}$ is the Legesgue-Stieltjes measure generated by $F_{Y}$.

We will now prove that 
\[ \int_{\mathbb{R}}e^{-z^{2}}\varphi(z)\ dm(z) = \int_{\mathbb{R}}e^{-x}\ dP_{X}(x) = \int_{\mathbb{R}}y\ dP_{Y}(y) = \frac{1}{\sqrt{3}}. \]
\begin{Proof} 

\vspace{-5mm}
\begin{claim}
$\int_{\mathbb{R}}e^{-z^{2}}\varphi(z)\ dm(z) = \frac{1}{\sqrt{3}}$.
\end{claim}
\begin{claimproof}
Note that by Exercise 5.10, 
\begin{equation}
\int_{\mathbb{R}}e^{-u^{2}/2}\ dm(u) = 2\int_{\mathbb{R}_{+}}e^{-u^{2}/2}\ dm(u) = \sqrt{2\pi}. 
\label{5.1}
\end{equation}
This also implies that $e^{-u^{2}/2} \in L^{1}(m)$. Thus, by applying Theorem 4.4.6 with $u \equiv T(z) := z\sqrt{3}$, we have 
\[ \int_{\mathbb{R}}e^{-z^{2}}\varphi(z)\ dm(z) = \frac{1}{\sqrt{2\pi}}\int_{\mathbb{R}}e^{-3z^{2}/2}\ dm(z) =
\frac{1}{\sqrt{6\pi}}\int_{\mathbb{R}}e^{-u^{2}/2}\ dm(u) \stackrel{\eqref{5.1}}{=} \frac{\sqrt{2\pi}}{\sqrt{6\pi}} = \frac{1}{\sqrt{3}}. \]
\end{claimproof}

Now, let $h,f : \mathbb{R} \rightarrow \mathbb{R}$ be defined by $h(x) := e^{-x}$ and $f(z) := z^{2}$ for all $x,z \in \mathbb{R}$, and $\mu_{1} :=
P_{Z}$, $\mu_{2} := P_{Z}f^{-1} = P_{Z}X^{-1} = P_{X}$, and $\mu_{3} := \mu_{2} h^{-1} = P_{X}Y^{-1} = P_{Y}$.

\begin{claim}
$h \in L^{1}(\mathbb{R}, \mathcal{B}(\mathbb{R}), \mu_{2})$.
\end{claim}
\begin{claimproof}
Note that $h(x) \leq 1$ for all $x \in \mathbb{R}_{+}$ and $\mu_{2}\left( \mathbb{R}_{-} \right) = P_{X}(\mathbb{R}_{-}) = 0$. Thus, since $\mu_{2}$
is a probability measure, 
\[ \int_{\mathbb{R}}|h|\ d\mu_{2} = \int_{\mathbb{R}_{+}}h\ d\mu_{2} \leq 1. \]
\end{claimproof}

\begin{claim}
$\int_{\mathbb{R}}e^{-x}\ dP_{X} = \int_{\mathbb{R}}y\ dP_{Y} = \frac{1}{\sqrt{3}}$.
\end{claim}
\begin{claimproof}
By claim 2 we can apply the change of variables formula (Exercise 2.42/Proposition 6.2.1). Therefore
\[ \int_{\mathbb{R}}y\ d\mu_{3} = \int_{\mathbb{R}}h\ d\mu_{2} = \int_{\mathbb{R}}h\circ f\ d\mu_{1}, \]
which implies
\[ \int_{\mathbb{R}}y\ dP_{Y}(y) = \int_{\mathbb{R}}e^{-x}\ dP_{X}(x) = \int_{\mathbb{R}}e^{-z^{2}}\ dP_{Z}(z) = \int_{\mathbb{R}}e^{-z^{2}}\varphi(z)\ dm(z) 
\stackrel{\text{claim 1}}{=} \frac{1}{\sqrt{3}}. \]
\end{claimproof}

\end{Proof}

\item 
We will use the following proposition to find the distribution of $Y$.

{\bf Proposition.} If $Y_{1} \sim N(0, \sigma^{2})$ and $Y_{2} \sim N(0,1)$, then $Y_{1} + Y_{2} \sim N(0, \sigma^{2} + 1)$.

\begin{Proof}
By Exercise 5.12 (f), the density function of $Y_{1} + Y_{2}$ is given by the convolution of $f_{1}$ and $f_{2}$, where $f_{i}$ is the pdf of $Y_{i}$,
$i = 1,2$. Thus,
\begin{align*}
f_{Y_{1} + Y_{2}}(y) = (f_{1} * f_{2})(y) & = \frac{1}{2\pi\sigma}\int_{\mathbb{R}}e^{-\frac{(x-u)^{2}}{2\sigma^{2}}}e^{-\frac{u^{2}}{2}}dm(u) \\
& = \frac{1}{2\pi\sigma} \int_{\mathbb{R}}\exp\left[ -\frac{\sigma^{2} + 1}{2\sigma^{2}}\left( \frac{\sigma^{2}x^{2}}{(\sigma^{2} + 1)^{2}} + \left(
u - \frac{x}{\sigma^{2} + 1} \right)^{2} \right) \right]dm(u) \\
& = \frac{1}{2\pi\sigma}e^{-\frac{x^{2}}{2(\sigma^{2} +1)}} \int_{\mathbb{R}} \exp \left[ -\frac{\sigma^{2}+1}{2\sigma^{2}}\left( u -
\frac{x}{\sigma^{2} + 1} \right)^{2} \right]dm(u) \\
& = \frac{1}{2\pi\sigma}e^{-\frac{x^{2}}{2(\sigma^{2} + 1)}} \times \frac{\sigma\sqrt{2\pi}}{\sqrt{\sigma^{2} + 1}} \\
& = \frac{1}{\sqrt{2\pi(\sigma^{2} + 1)}} e^{-\frac{x^{2}}{2(\sigma^{2} + 1)}} \ a.e.\  (m).
\end{align*}
Since this is the pdf of a $N(0, \sigma^{2} + 1)$ RV, we are done.
\end{Proof}

By iterative use of the above proposition, the distribution of $Y$ is $N(0,k)$. Now, if $a < 0$, then clearly $F_{Z}(a) = 0$. Now suppose $a \geq 0$. Then 
\[ F_{Z}(a) = P_{Y}\left( Z^{-1}[(-\infty, a]] \right) = P_{Y}\left( \left\{ z : |z| \leq \sqrt{a} \right\} \right) = F_{Y}(\sqrt{a}) -
F_{Y}(-\sqrt{a}). \]
So 
\[ F_{Z}(a) = \left\{ \begin{array}{cl}
0 & \text{ for } a < 0 \\
F_{Y}(\sqrt{a}) - F_{Y}(-\sqrt{a}) & \text{ for } a \geq 0 
\end{array} \right. .\]
The distribution of $Z$, $P_{Z}$, is the Legesgue-Stieltjes measure generated by $F_{Z}$. Next we will show that 
\[ \int_{\mathbb{R}^{k}}(x_{1} + \dots + x_{k})^{2}\ dP_{X_{1}, \hdots, X_{k}}(x_{1}, \hdots, x_{k}) = \int_{\mathbb{R}}y^{2}\ dP_{Y}(y) =
\int_{\mathbb{R}_{+}}z \ dP_{Z}(z) = k.\]
\begin{Proof}
Since the distribution of $Y$ is $N(0,k)$, the mgf of $Y$ is $e^{\frac{t^{2}k}{2}}$ by Table 6.2.1. Thus, 
$\int_{\mathbb{R}}y^{2}\ dP_{Y}(y)$ is the $2^{nd}$ moment of $Y$, which is given by 
\[ \frac{d^{2}}{dt} \left( e^{\frac{t^{2}k}{2}}\right)\bigg|_{t=0} = \left( ke^{\frac{t^{2}k}{2}} + t^{2}k^{2}e^{\frac{t^{2}k}{2}} \right)\bigg|_{t=0} = k. \]
Thus, by applying the change of variables formula, 
\[ k = \int_{\mathbb{R}}y^{2}\ dP_{Y}(y) = \int_{\mathbb{R}^{k}}(x_{1} + \dots + x_{k})^{2}\ dP_{X_{1}, \hdots, X_{k}}(x_{1}, \hdots, x_{k}) = 
\int_{\mathbb{R}_{+}}z \ dP_{Z}(z).\]
\end{Proof}

\item 
First we will derive the moment generating function of $Y$. Since $e^{tY} > 0$ for all $Y \in \mathbb{R}$, we can apply the change of variables
formula. Further, since $X_{1}, \hdots, X_{k}$ are independent, $P_{X_{1}, \hdots, X_{k}} = P_{X_{1}}\times \dots \times P_{X_{k}}$. Thus,
\begin{align*}
Ee^{tY} = \int_{\mathbb{R}}e^{ty}\ dP_{Y}(y) & = \int_{\mathbb{R}^{k}}e^{t(x_{1} + \dots + x_{k})}\ dP_{X_{1}, \hdots, X_{k}}(x_{1}, \hdots, x_{k}) \\
\text{(Tonelli's) } & = \int_{\mathbb{R}}\dots \int_{\mathbb{R}}e^{tx_{1}}\dots e^{tx_{k}}\ dP_{X_{1}}(x_{1})\dots dP_{X_{k}}(x_{k}) \\
& = \Pi_{i=1}^{k} Ee^{tX_{i}} \\
\text{(Table 6.2.1) } & = \Pi_{i=1}^{k} [(1-p) + pe^{t}]^{n_{i}} = [(1-p) + pe^{t}]^{\sum_{i=1}^{k}n_{i}}.
\end{align*}
This is the mgf of a Bin$(\sum_{i=1}^{k}n_{i}, p)$ RV, so that is the distribution of $Y$. Therefore $\int_{\mathbb{R}}y\ dP_{Y}(y) = EY =
p\sum_{i=1}^{k}n_{i}$. At this point we could apply the change of variables formula again, but the other integral is easy enough to evaluate 
using Tonelli's. Thus,
\begin{align*}
\int_{\mathbb{R}^{k}}(x_{1} + \dots + x_{k})\ dP_{X_{1},\hdots, X_{k}}(x_{1}, \hdots, x_{k}) & = \int_{\mathbb{R}}\dots \int_{\mathbb{R}} 
(x_{1} + \dots + x_{k})\ dP_{X_{1}}(x_{1})\dots dP_{X_{k}}(x_{k}) \\
& = \Pi_{i=1}^{k}EX_{i} \\
& = p\sum_{i=1}^{k}n_{i}.
\end{align*}

\end{enumerate}






\newpage 
\section*{6.6}
Let $X$ and $Y$ be random variables on $(\Omega, \mathcal{F}, P)$. Show that if $|Cov(X,Y)| = \sqrt{Var(X)}\sqrt{Var(Y)}$, then there exists constants
$a$ and $b$ such that $P(Y = aX + b) = 1$.

\subsection*{Solution}
\begin{Proof}
Suppose $|Cov(X,Y)| = \sqrt{Var(X)}\sqrt{Var(Y)}$. WLOG assume $X$ and $Y$ are not constant.

\begin{claim}
$Var(Y - aX) = 0$, where $a = \frac{E(XY) - (EX)(EY)}{E(X^{2}) - (EX)^{2}}$.
\end{claim}
\begin{claimproof}
By assumption,
\begin{align*}
0 & = Var(X)Var(Y) - Cov(X,Y)^{2} \\
& = [E(X^{2}) - (EX)^{2}][E(Y^{2}) - (EY)^{2}] - (EXY - (EX)(EY))^{2} \\
& = E(X^{2})E(Y^{2}) - (EX)^{2}E(Y^{2}) - E(X^{2})(EY)^{2} + (EX)^{2}(EY)^{2} \\ 
& \qquad - (EXY)^{2} - 2(EXY)(EX)(EY) - (EX)^{2}(EY)^{2} \\
& = E(Y^{2})[E(X^{2}) - (EX)^{2}] - (EY)^{2}E(X^{2}) - (EXY)^{2}+ 2(EXY)(EX)(EXY).
\end{align*}
Which implies
\begin{align*}
0 & = E(Y^{2}) - (EY)^{2} - \frac{[(EXY)^{2} - 2(EX)(EY)(EXY) + (EX)^{2}(EY)^{2}]}{E(X^{2}) - (EX)^{2}} \\
& = E(Y^{2}) - (EY)^{2} - \frac{[EXY - (EX)(EY)]^{2}}{E(X^{2}) - (EX)^{2}} \\
& = E(Y^{2}) - (EY)^{2} - 2\frac{[EXY - (EX)(EY)]^{2}}{E(X^{2}) - (EX)^{2}} + \frac{[EXY - (EX)(EXY)]^{2}}{E(X^{2}) - (EX)^{2}} \\
& = E(Y^{2}) - (EY)^{2} - 2\left[ \frac{EXY - (EX)(EY)}{E(X^{2}) - (EX)^{2}} \right](EXY - (EX)(EY)) \\
& \qquad + \left[ \frac{EXY - (EX)(EY)}{E(X^{2}) -
(EX)^{2}} \right]^{2}(E(X^{2}) - (EX)^{2}) \\
& = E(Y^{2}) - 2aEXY + a^{2}E(X^{2}) - (EY)^{2} + 2a(EX)(EY) - a^{2}(EX)^{2} \\
& = E[Y^{2} - 2aXY + a^{2}X^{2}] - [EY - aEX]^{2} \\
& = E[Y - aX]^{2} - [E(Y - aX)]^{2} = Var(Y - aX).
\end{align*}
\end{claimproof}

\begin{claim}
$Y = aX + EY - aEX$ a.e. $(P)$.
\end{claim}
\begin{claimproof}
Let $f := Y - aX - (EY - aEX)$. By claim 1, $0 = Var(Y - aX) = \int_{\Omega}f^{2}\ dP$,
which implies $f = 0$ a.e. $(P)$ since $f^{2} \geq 0$.  Thus, $Y = aX + b$ a.e. $(P)$ where $b = EY - aEX$. 

\end{claimproof}

\end{Proof}




\end{document}

