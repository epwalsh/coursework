\documentclass[12pt]{article}
\usepackage{amsmath}
\usepackage{amsfonts}
\usepackage{parskip}
\usepackage{amsthm}
\usepackage{thmtools}
\usepackage[headheight=15pt]{geometry}
\geometry{a4paper, left=20mm, right=20mm, top=30mm, bottom=30mm}
\usepackage{graphicx}
\usepackage{bm} % for bold font in math mode - command is \bm{text}
\usepackage{enumitem}
\usepackage{fancyhdr}
\usepackage{amssymb} % for stacked arrows and other shit
\pagestyle{fancy}

\declaretheoremstyle[headfont=\normalfont]{normal}
\declaretheorem[style=normal]{Theorem}
\declaretheorem[style=normal]{Proposition}
\declaretheorem[style=normal]{Lemma}
\newenvironment{claimproof}[1]{\par\noindent\underline{Proof of claim:}\space#1}{\hfill $\blacksquare$}

\title{STAT 641: HW 4}
\author{Evan ``Pete'' Walsh}
\makeatletter
\let\runauthor\@author
\let\runtitle\@title
\makeatother
\lhead{\runauthor}
\chead{\runtitle}
\rhead{\thepage}
\cfoot{}

\begin{document}
\maketitle

\section*{2.1}
Let $\Omega_{i}$, $i = 1,2$ be two non-empty sets and $T : \Omega_{1} \rightarrow \Omega_{2}$ be a map. Let $\left\{ A_{\alpha} : \alpha \in I
\right\}$ be any collection of subsets of $\Omega_{2}$. Show that

(i) $T^{-1}\left( \bigcup_{\alpha \in I}A_{\alpha} \right) = \bigcup_{\alpha\in I}T^{-1}(A_{\alpha})$,

(ii) $T^{-1}\left( \bigcap_{\alpha \in I} A_\alpha \right) = \bigcap_{\alpha \in I}T^{-1}(A_\alpha)$, and 

(iii) $\left( T^{-1}(A) \right)^{c} = T^{-1}(A^{c})$ for every $A \in \Omega_2$.

\section*{Solution}

{\bf (i)}
\begin{proof}
Let $\omega_0 \in T^{-1}(\cup_{\alpha \in I}A_{\alpha}) = \left\{ \omega \in \Omega_{1} : T(\omega) \in \cup_{\alpha \in I}A_{\alpha} \right\}$. Then
$T(\omega_0) \in \cup_{\alpha \in I}A_{\alpha}$, so $T(\omega_{0}) \in A_{\alpha_0}$ for some $\alpha_{0} \in I$. Thus, $w_0 \in T^{-1}(A_{\alpha_{0}})$, so
$w_{0} \in \cup_{\alpha \in I}T^{-1}(A_\alpha)$. Hence 
\begin{equation}
T^{-1}\left( \bigcup_{\alpha\in I}A_{\alpha} \right) \subseteq \bigcup_{\alpha\in I}T^{-1}(A_\alpha).
\label{1}
\end{equation}
Now, let 
\begin{align*}
\omega_{1} \in \cup_{\alpha \in I}T^{-1}(A_\alpha) & = \cup_{\alpha\in I}\left\{ \omega \in \Omega_{1} : T(\omega) \in A_\alpha \right\} \\
& = \left\{ \omega \in \Omega_{1} : T(\omega) \in A_{\alpha}, \text{ for some }\alpha \in I \right\} \\
& = \left\{ \omega \in \Omega_1 : T(\omega) \in \cup_{\alpha \in I}A_{\alpha} \right\} \\
& = T^{-1}\left( \bigcup_{\alpha \in I}A_{\alpha} \right).
\end{align*}
Therefore 
\begin{equation}
\bigcup_{\alpha \in I}T^{-1}(A_{\alpha}) \subseteq T^{-1}\left( \bigcup_{\alpha \in I}A_{\alpha} \right).
\label{2}
\end{equation}
By \ref{1} and \ref{2} we have equality.
\end{proof}

{\bf (ii)}
\begin{proof}
Let $\omega_{0} \in T^{-1}\left( \cap_{\alpha \in I}A_{\alpha} \right) = \left\{ \omega \in \Omega_{1} : T(\omega) \in \cap_{\alpha \in I}A_{\alpha}
\right\}$. Then $T(\omega_0) \in \cap_{\alpha \in I}A_\alpha$, so $T(\omega_0) \in A_{\alpha}$ for every $\alpha \in I$. Thus $\omega_0 \in
T^{-1}(A_\alpha)$ for every $\alpha \in I$. Hence $\omega_0 \in \cap_{\alpha\in I}T^{-1}(A_\alpha)$. So 
\begin{equation}
T^{-1}\left( \bigcap_{\alpha\in I}A_{\alpha} \right) \subseteq \bigcap_{\alpha\in I}T^{-1}(A_\alpha).
\label{3}
\end{equation}
Now, let $\omega_{1} \in \cap_{\alpha\in I}T^{-1}(A_\alpha)$. Then $\omega_1 \in T^{-1}(A_\alpha) = \left\{ \omega \in \Omega_1 : T(\omega) \in
A_\alpha \right\}$ for each $\alpha \in I$. So $\omega_1 \in \left\{ \omega \in \Omega_1 : T(\omega) \in \cap_{\alpha \in I}A_\alpha \right\} =
T^{-1}\left( \cap_{\alpha \in I}A_\alpha \right)$. Thus 
\begin{equation}
\bigcap_{\alpha \in I}T^{-1}(A_\alpha) \subseteq T^{-1}\left( \bigcap_{\alpha\in I}A_\alpha \right).
\label{4}
\end{equation}
So by \ref{3} and \ref{4} we have equality.
\end{proof}

{\bf (iii)}
\begin{proof}
Let $\omega_0 \in \left( T^{-1}(A) \right)^{c}$. Then $\omega_0 \notin T^{-1}(A)$. So 
\[ \omega_0 \in \left\{ \omega \in \Omega_1 : T(\omega) \notin A \right\} =  \left\{ \omega \in \Omega_1 : T(\omega) \in A^{c} \right\} =
T^{-1}(A^{c}). \]
Hence 
\begin{equation}
\left( T^{-1}(A) \right)^{c} \subseteq T^{-1}(A^{c}).
\label{5}
\end{equation}
Now, let $\omega_1 \in T^{-1}(A^{c}) = \left\{ \omega\in \Omega_1 : T(\omega) \in A^{c} \right\} = \left\{ \omega \in \Omega_1 : T(\omega) \notin A
\right\}$. Then $\omega_1 \notin T^{-1}(A)$, so $\omega_1 \in \left( T^{-1}(A) \right)^{c}$. Hence 
\begin{equation}
T^{-1}(A^{c}) \subseteq \left( T^{-1}(A) \right)^{c}.
\label{6}
\end{equation}
So by \ref{5} and \ref{6} we have equality.
\end{proof}



\newpage
\section*{2.3}
Let $f, g : \Omega \rightarrow \mathbb{R}$ be $\langle\mathcal{F}, \mathcal{B}(\mathbb{R})\rangle$-measurable. Set 
\[ h(\omega) = \frac{f(\omega)}{g(\omega)}I(g(\omega)\neq 0), \omega \in \Omega. \]
Verify that $h$ is $\langle\mathcal{F}, \mathcal{B}(\mathbb{R})\rangle$-measurable.

\section*{Solution}

\begin{proof}
Define $g_1 : \mathbb{R} \rightarrow \mathbb{R}$ by $g_1(x) = x^{-1}I(x \neq 0)$.

\underline{Claim 1:} $g_1$ is $\langle\mathcal{B}(\mathbb{R}), \mathcal{B}(\mathbb{R})\rangle$-measurable.

\begin{claimproof}
Let $\mathcal{A} = \left\{ A\subseteq \mathbb{R} : A \text{ open} \right\}$, and let $A \in \mathcal{A}$. There are two cases to consider: either $0
\notin A$ or $0 \in A$.

{\bf Case 1:} First let us consider the case where $0 \notin A$. Note that $g_1^{-1}(\left\{ 0 \right\}) = \left\{ 0 \right\}$, so $g_1(0) \notin A$.
Thus, since $g_1$ is continuous everywhere except at $x = 0$, $g_1$ is continouos over $g_{1}^{-1}(A)$. Hence $g_1^{-1}(A)$ is open, so $g_1^{-1}(A) \in
\mathcal{B}(\mathbb{R})$.

{\bf Case 2:} Now assume that $0 \in A$. Then $A = (A \setminus \left\{ 0 \right\}) \cup \left\{ 0 \right\}$. Since $A$ is open, $A \setminus \left\{
0 \right\} = A \cap \left\{ 0 \right\}^{c}$ is still open. Also, $g_{1}^{-1}(\left\{ 0 \right\}) = \left\{ 0 \right\}$. Therefore $g_{1}^{-1}(A\setminus \left\{ 0 \right\})$ 
is open since $g_{1}$ is continuous every except at $x = 0$. So $g_{1}^{-1}(A\setminus \left\{ 0 \right\}), g_{1}^{-1}(\left\{ 0 \right\}) \in
\mathcal{B}(\mathbb{R})$. Hence, by Exercise 2.1 and the fact that $\mathcal{B}(\mathbb{R})$ is closed under finite unions,
\[ g_{1}^{-1}(A) = g_{1}^{-1}\bigg( (A\setminus \left\{ 0 \right\}) \cup \left\{ 0 \right\}\bigg) = g_{1}^{-1}(A\setminus \left\{ 0 \right\}) \cup
g_{1}^{-1}(\left\{ 0 \right\}) \in \mathcal{B}(\mathbb{R}). \]

In both cases $g_{1}^{-1}(A) \in \mathcal{B}(\mathbb{R})$. Hence, by Proposition 2.1.1(i), $g_{1}$ is $\langle\mathcal{B}(\mathbb{R}),
\mathcal{B}(\mathbb{R})\rangle$-measurable since $\sigma\langle\mathcal{A}\rangle = \mathcal{B}(\mathbb{R})$.
\end{claimproof}

\underline{Claim 2:} $g_{2} \equiv g_{1} \circ g$ is $\langle\mathcal{F}, \mathcal{B}(\mathbb{R})\rangle$-measurable.

\begin{claimproof}
Follows immediately from Proposition 2.1.1 (ii).
\end{claimproof}

\underline{Claim 3:} $h$ is $\langle\mathcal{F}, \mathcal{B}(\mathbb{R})\rangle$-measurable.

\begin{claimproof}
Note that 
\[ h(\omega) = \frac{f(\omega)}{g(\omega)}I(g(\omega)\neq 0) = f(\omega)\times g_{2}(\omega), \omega \in \Omega. \]
So, since $f$ and $g_{2}$ are $\langle\mathcal{F}, \mathcal{B}(\mathbb{R})\rangle$-measurable, $f\times g_{2}$ is $\langle\mathcal{F},
\mathcal{B}(\mathbb{R})\rangle$-measurable by Proposition 2.1.3 (iii).
\end{claimproof}

\end{proof}

\newpage
\section*{2.6}
Let $X_{i}, i = 1,2,3$ be random variables on a probability space $(\Omega, \mathcal{F}, P)$. Consider the random equation (in $t \in \mathbb{R}$):
\begin{equation}
X_{1}(\omega)t^{2} + X_{2}(\omega)t + X_{3}(\omega) = 0. 
\label{2.6.1}
\end{equation}
(a) Show that $A \equiv \left\{ \omega \in \Omega : \text{ Equation \ref{2.6.1} has 2 distinct real roots} \right\} \in \mathcal{F}$.

(b) Let $T_{1}(\omega)$ and $T_{2}(\omega)$ denote the two roots of \ref{2.6.1} on $A$. Let 
\[ f_{i}(\omega) = \left\{ \begin{array}{cl}
T_{i}(\omega) & \text{ on } A \\
0 & \text{ on }A^{c} \\
\end{array} \right., \]
$i = 1,2$. Show that $(f_{1}, f_{2})$ is $\langle\mathcal{F}, \mathcal{B}(\mathbb{R}^{2})\rangle$-measurable.

\section*{Solution}
{\bf (a)}
\begin{proof}
Note that 
\begin{align*}
A & = \left\{ \omega \in \Omega : X_{2}(\omega)^{2} - 4X_{1}(\omega)X_{3}(\omega) > 0, X_{1}(\omega) \neq 0 \right\} \\
& = \left\{ \omega \in \Omega : X_{2}(\omega)^{2} - 4X_{1}(\omega)X_{3}(\omega) > 0 \right\} \cap \left\{ \omega \in \Omega : X_{1}(\omega) \neq 0
\right\} \\
& = g^{-1}\bigg(\left( 0, \infty \right)\bigg) \cap X_{1}^{-1}(\mathbb{R}\setminus \left\{ 0 \right\}),
\end{align*}
where $g : \Omega \rightarrow\mathbb{R}$ is defined by $g(\omega) = X_{2}(\omega)^{2} - 4X_{1}(\omega)X_{3}(\omega)$. However, since $X_{1}$ is $\langle\mathcal{F}, \mathcal{B}(\mathbb{R})\rangle$-measurable,
$X_{1}^{-1}(\mathbb{R}\setminus \left\{ 0 \right\}) \in \mathcal{F}$ because $\mathbb{R}\setminus \left\{ 0 \right\} \in \mathcal{B}(\mathbb{R})$.
Thus, we just need to show that $g^{-1}\bigg( (0,\infty) \bigg) \in \mathcal{F}$.

\underline{Claim:} $f(\omega) \equiv -4$ is $\langle\mathcal{F}, \mathcal{B}(\mathbb{R})\rangle$-measurable.

\begin{claimproof}
Let $X \in \mathcal{B}(\mathbb{R})$. If $-4 \in X$, then $f^{-1}(X) = \Omega \in \mathcal{F}$. If $-4 \notin X$, then $f^{-1}(X) = \emptyset \in
\mathcal{F}$. Thus $f$ is $\langle\mathcal{F}, \mathcal{B}(\mathbb{R})\rangle$-measurable.
\end{claimproof}

Thus, since $X_{1}, X_{2}, X_{3}$, and $f$ are all $\langle\mathcal{F}, \mathcal{B}(\mathbb{R})\rangle$-measurable, 
\[ g(\omega) = X_{2}(\omega)X_{2}(\omega) + f(\omega)X_{1}(\omega)X_{3}(\omega) \]
is $\langle\mathcal{F}, \mathcal{B}(\mathbb{R})\rangle$-measurable by Proposition 2.1.3. So $g^{-1}\bigg( (0, \infty) \bigg) \in \mathcal{F}$ since
$(0, \infty) \in \mathcal{B}(\mathbb{R})$. Hence $A = g^{-1}\bigg( (0,\infty) \bigg) \cap X_{1}^{-1}(\mathbb{R}\setminus \left\{ 0 \right\}) \in
\mathcal{F}$.
\end{proof}

{\bf (b)} 
\begin{proof}
Note that 
\begin{align*}
f_{1}(\omega) & = T_{1}(\omega)I(\omega \in A) \\
& = \frac{-X_{2}(\omega) + \sqrt{X_{2}(\omega)^{2} - 4X_{1}(\omega)X_{3}(\omega)}}{2X_{1}(\omega)}\times I(X_{1}(\omega)\neq 0, X_{2}(\omega)^{2} -
4X_{1}(\omega)X_{3}(\omega) > 0) \\
& = \frac{I(X_{1}(\omega) \neq 0)}{2X_{1}(\omega)} \times \bigg[ \sqrt{X_{2}(\omega)^{2} - 4X_{1}(\omega)X_{3}(\omega)}I\big(X_{2}(\omega)^{2} -
4X_{1}(\omega)X_{3}(\omega) > 0\big) - \\
& \qquad X_{2}(\omega)I\big(X_{2}(\omega)^{2} - 4X_{1}(\omega)X_{3}(\omega) > 0\big) \bigg].
\end{align*}
We will show that each part of the above function is measurable and then tie that all together with Proposition 2.1.1 (ii) and Corollary 2.1.4. First,
let $g_{1} : \mathbb{R} \rightarrow \mathbb{R}$ be defined by $g_{1}(x) = \frac{I(x \neq 0)}{2x}$.

\underline{Claim 1:} $g_{1}$ is $\langle\mathcal{B}(\mathbb{R}), \mathcal{B}(\mathbb{R})\rangle$-measurable.

\begin{claimproof}
Let $(a,b) \in \mathcal{B}(\mathbb{R})$ be an open interval. There are two cases to consider: either $0 \in (a,b)$ or $0 \notin (a,b)$.

{\bf Case 1:} First assume that $0 \in (a,b)$. Then 
\begin{align*}
g_{1}^{-1}\big( (a,b) \big) & = \left\{ x \in \mathbb{R} : g_{1}(x) \in (a,b) \right\} \\
& = \left\{ x \in \mathbb{R} : g_{1}(x) \in (a,0) \cup \left\{ 0 \right\} \cup (0, b) \right\} \\
& = \left\{ x \in \mathbb{R} : g_{1}(x) \in (a,0) \right\} \cup \left\{ x \in \mathbb{R} : g_{1}(x) = 0 \right\} \cup \left\{ x \in \mathbb{R} :
g_{1}(x) \in (0,b) \right\} \\
& = g_{1}^{-1}\big( (a,0) \big) \cup \left\{ 0 \right\} \cup g_{1}^{-1}\big( (0,b) \big).
\end{align*}
But since $g_{1}$ is only discontinuous at $g_{1}^{-1}(\left\{ 0 \right\}) = \left\{ 0 \right\}$, $g_{1}^{-1}\big( (a,0)\big)$ and $g_{1}^{-1}\big(
(0,b)\big)$ are both open, and hence are in $\mathcal{B}(\mathbb{R})$. So $g_{1}^{-1}\big( (a,b)\big) \in \mathcal{B}(\mathbb{R})$ by closure under
finite unions.

{\bf Case 2:} Now assume that $0 \notin (a,b)$. Then $g_{1}^{-1}\big( (a,b) \big) \in \mathcal{B}(\mathbb{R})$ since $g_{1}$ is continuous over
$g_{1}^{-1}\big( (a,b)\big)$.

In both cases $g_{1}^{-1}\big( (a,b)\big) \in \mathcal{B}(\mathbb{R})$. Thus, by Proposition 2.1.1(i), $g_{1}$ is $\langle\mathcal{B}(\mathbb{R}),
\mathcal{B}(\mathbb{R})\rangle$-measurable since $\sigma\langle (a,b)\rangle = \mathcal{B}(\mathbb{R})$.
\end{claimproof}

Now, since $X_{1} : \Omega \rightarrow \mathbb{R}$ is $\langle\mathcal{F}, \mathcal{B}(\mathbb{R})\rangle$-measurable, $g_{1}\circ X_{1}$
is $\langle\mathcal{B}(\mathbb{R}), \mathcal{B}(\mathbb{R})\rangle$-measurable by Proposition 2.1.1(ii). Now, let 
$g_{2}, g_{3} : \mathbb{R}^{3} \rightarrow \mathbb{R}$ be defined by $g_{2}(x_{1}, x_{2}, x_3) = \sqrt{x_2^2 - 4x_1x_3}\times
I(x_2^2 - 4x_1x_3 > 0)$ and $g_3(x_1, x_2, x_3) = -x_2\times I(x_2^2 - 4x_1x_3 > 0)$. 

\underline{Claim 2:} $g_2$ and $g_3$ are 
$\langle\mathcal{B}(\mathbb{R}^3), \mathcal{B}(\mathbb{R})\rangle$-measurable.

\begin{claimproof}
Since $g_{2}$ is continouous everywhere on $\mathbb{R}^{3}$, $g_{2}$ is $\langle\mathcal{B}(\mathbb{R}^{3}),
\mathcal{B}(\mathbb{R})\rangle$-measurable by Prop 2.1.2. Let $\psi_{1} :
\mathbb{R} \rightarrow \mathbb{R}$ be defined by $\psi_{1}(x) = I(x > 0)$, and $\psi_{2} : \mathbb{R}^{3} \rightarrow \mathbb{R}$ be defined by
$\psi_{2}(x_{1}, x_{2}, x_{3}) = x_{2}^{2} - 4x_1x_3$.  Since $\psi_2$ is continuous, it is $\langle\mathcal{B}(\mathbb{R}^{3}),
\mathcal{B}(\mathbb{R})\rangle$-measurable. Now, we want to show that $\psi_1$ is $\langle\mathcal{B}(\mathbb{R}),
\mathcal{B}(\mathbb{R})\rangle$-measurable. So, let $(a,b) \in \mathcal{B}(\mathbb{R})$ be an open interval. There are 4 cases to consider: either (i)
$0,1 \notin (a,b)$, (ii) $0 \in (a,b)$ and $1 \notin (a,b)$, (iii) $0 \notin (a,b)$ and $1 \in (a,b)$, or (iv) $0,1 \in (a,b)$.

{\bf Case (i)} $\psi_{1}^{-1}\big( (a,b)\big) = \emptyset \in \mathcal{B}(\mathbb{R})$.

{\bf Case (ii)} $\psi_{1}^{-1}\big( (a,b) \big) = \left\{ x \in \mathbb{R} : x \leq 0 \right\} = (-\infty, 0] \in \mathbb{R}$.

{\bf Case (iii)} $\psi_{1}^{-1}\big( (a,b) \big) = \left\{ x \in \mathbb{R} : x > 0 \right\} = (0,\infty) \in \mathcal{B}(\mathbb{R})$.

{\bf Case (iv)} $\psi_{1}^{-1}\big( (a,b) \big) = \mathbb{R} \in \mathcal{B}(\mathbb{R})$.

Thus, by Proposition 2.1.1(i), $\psi_{1}$ is $\langle\mathcal{B}(\mathbb{R}), \mathcal{B}(\mathbb{R})\rangle$-measurable. Now, let $\phi_{1}(x_1, x_2, x_3) = -x_2$. 
Since $\phi_1$ is continuous, $\phi_1$ is $\langle\mathcal{B}(\mathbb{R}^{3}), \mathcal{B}(\mathbb{R})\rangle$-measurable. Hence, by Propositions
2.1.1(ii) and 2.1.3(iii),
\begin{align*}
g_{3}(x_1, x_2, x_3) & = -x_2 I(x_2^2 - 4x_1 x_3 > 0) \\
& = \phi_1(x_1,x_2,x_3)\psi_{1}(\psi_{2}(x_1,x_2,x_3))
\end{align*}
is $\langle\mathcal{B}(\mathbb{R}^{3}), \mathcal{B}(\mathbb{R})\rangle$-measurable.
\end{claimproof}

Thus, by Proposition 2.1.1(ii) and Corollary 2.1.4, 
\[ f_{1}(\omega) = g_{1}(X_{1}(\omega))\times \left[ g_{2}\left( X_{1}(\omega), X_{2}(\omega), X_{2}(\omega) \right) + g_{3}\left( X_{1}(\omega),
X_{2}(\omega), X_{2}(\omega) \right) \right] \]
is $\langle\mathcal{F}, \mathcal{B}(\mathbb{R})\rangle$-measurable. By a similar argument, $f_{2}$ is $\langle\mathcal{F}, \mathcal{B}(\mathbb{R})\rangle$-measurable.
So by Proposition 2.1.3(i), $(f_{1}, f_{2})$ is $\langle\mathcal{F}, \mathcal{B}(\mathbb{R}^{2})\rangle$-measurable.
\end{proof}


\newpage
\section*{(A)}
Let $X : \Omega_{1} \rightarrow \Omega_{2}$ be a $\langle\mathcal{F}_{1}, \mathcal{F}_{2}\rangle$-measurable transformation. Consider the collection
(of subsets of $\Omega_{1}$) defined by $\sigma\langle X\rangle = \left\{ X^{-1}(B) : B \in \mathcal{F}_{2} \right\}$. Then show that 

(a) $\sigma\langle X\rangle$ is a $\sigma$-algebra, and 

(b) $\sigma\langle X\rangle$ is the smallest $\sigma$-algebra that makes $X$ measurable.

\section*{Solution}
{\bf (a)} 
\begin{proof}
First note that $\Omega_{2} \in \mathcal{F}_{2}$ and $X^{-1}(\Omega_{2}) = \Omega_{1}$. So $\Omega_{1} \in \sigma \langle X \rangle$. Now let $A \in
\sigma \langle X \rangle$. Then $A = X^{-1}(B)$ for some $B \in \mathcal{F}_{2}$. Thus,
\begin{align*}
A^{c} = \left( X^{-1}(B) \right)^{c} & = \left( \left\{ a \in \Omega_1 : X(a) \in B \right\} \right)^{c} \\
& = \left\{ a \in \Omega_1 : X(a) \notin B \right\} \\
& = \left\{ a \in \Omega_1 : X(a) \in B^{c} \right\} \\
& = X^{-1}(B^{c}).
\end{align*}
Thus $A^{c} \in \sigma\langle X\rangle$ since $B^{c} \in \mathcal{F}_{2}$. Now, let $\left\{ A_{n} \right\}_{n\geq 1} \subseteq \sigma\langle
X\rangle$. Then for each $n \in \mathbb{R}$, $A_{n} = X^{-1}(B_{n})$ for some $B_{n} \in \mathcal{F}_{2}$. So,
\begin{align*}
\bigcup_{n=1}^{\infty}A_{n} = \bigcup_{n=1}^{\infty}X^{-1}(B_{n}) & = \bigcup_{n=1}^{\infty}\left\{ a \in \Omega_{1} : X(a) \in B_{n} \right\} \\
& = \left\{ a \in \Omega_{1} : X(a) \in \cup_{n=1}^{\infty}B_{n} \right\} \\
& = X^{-1}\left( \bigcup_{n=1}^{\infty}B_{n} \right).
\end{align*}
Thus, since $\bigcup_{n=1}^{\infty}B_n \in \mathcal{F}_{2}$, $\bigcup_{n=1}^{\infty}A_{n} \in \sigma\langle X\rangle$.
\end{proof}

{\bf (b)}
\begin{proof}
Let $\mathcal{G}$ be a $\sigma$-algebra such that $X$ is $\langle \mathcal{G}, \mathcal{F}_{2}\rangle$-measurable. Let $A \in \sigma\langle X\rangle$.
Then $A = X^{-1}(B)$ for some $B\in \mathcal{F}_{2}$. Then by the $\langle \mathcal{G}, \mathcal{F}_{2}\rangle$-measurability of $X$, $X^{-1}(B) = A
\in \mathcal{G}$. So $\sigma\langle X\rangle \subseteq \mathcal{G}$.
\end{proof}


\newpage
\section*{(B)}
Give an example of a function $f$ that is not $\langle\mathcal{B}(\mathbb{R}), \mathcal{B}(\mathbb{R})\rangle$-measurable.

\section*{Solution}
Let $x,y \in \mathbb{R}$. Then say $x \sim_{\mathbb{Q}} y$ if $x - y \in \mathbb{Q}$. It is not hard to see that $\sim_{\mathbb{Q}}$ forms an
equivalence relation on $\mathbb{R}$. Then let $R$ be any system of representatives (choice set) on the set of equivalence classes of
$\sim_{\mathbb{Q}}$. That is,
every equivalence class of $\sim_{\mathbb{Q}}$ contains exactly one element of $R$ (which is possible provided we accept the Axiom of Choice). It turns out that $R$ is not Lebesgue measurable, and so is not
Borel (see \emph{Real Analysis} by Royden and Fitzpatrick for a proof). Thus, we can define a function $f : \mathbb{R} \rightarrow \mathbb{R}$ by 
\[ f(x) = \left\{ \begin{array}{cl}
1 & \text{ if } x \in R \\
0 & \text{ if } x \notin R \\
\end{array} \right. \]
Then $f^{-1}(\left\{ 1 \right\}) = R \notin \mathcal{B}(\mathbb{R})$. So $f$ is not measurable.



\end{document}

