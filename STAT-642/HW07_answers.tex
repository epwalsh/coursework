\documentclass[12pt]{article}
\usepackage{amsmath}
\usepackage{amsfonts}
\usepackage{parskip}
\usepackage{amsthm}
\usepackage{thmtools}
\usepackage[headheight=15pt]{geometry}
\geometry{a4paper, left=20mm, right=20mm, top=30mm, bottom=30mm}
\usepackage{graphicx}
\usepackage{bm} % for bold font in math mode - command is \bm{text}
\usepackage{enumitem}
\usepackage{fancyhdr}
\usepackage{amssymb} % for stacked arrows and other shit
\pagestyle{fancy}
\usepackage{changepage}
\usepackage{mathcomp}
\usepackage{tcolorbox}

\declaretheoremstyle[headfont=\normalfont]{normal}
\declaretheorem[style=normal]{Theorem}
\declaretheorem[style=normal]{Proposition}
\declaretheorem[style=normal]{Lemma}
\newcounter{ProofCounter}
\newcounter{ClaimCounter}[ProofCounter]
\newcounter{SubClaimCounter}[ClaimCounter]
\newenvironment{Proof}{\stepcounter{ProofCounter}\textsc{Proof.}}{\hfill$\square$}
\newenvironment{claim}[1]{\vspace{1mm}\stepcounter{ClaimCounter}\par\noindent\underline{\bf Claim \theClaimCounter:}\space#1}{}
\newenvironment{claimproof}[1]{\par\noindent\underline{Proof of claim \theClaimCounter:}\space#1}{\hfill $\blacksquare$ Claim \theClaimCounter}
\newenvironment{subclaim}[1]{\stepcounter{SubClaimCounter}\par\noindent\emph{Subclaim \theClaimCounter.\theSubClaimCounter:}\space#1}{}
% \newenvironment{subclaimproof}[1]{\begin{adjustwidth}{2em}{0pt}\par\noindent\emph{Proof of subclaim \theClaimCounter.\theSubClaimCounter:}\space#1}{\hfill
% $\blacksquare$ \emph{Subclaim \theClaimCounter.\theSubClaimCounter}\vspace{5mm}\end{adjustwidth}}
\newenvironment{subclaimproof}[1]{\par\noindent\emph{Proof of subclaim \theClaimCounter.\theSubClaimCounter:}\space#1}{\hfill
$\Diamond$ \emph{Subclaim \theClaimCounter.\theSubClaimCounter}}

\allowdisplaybreaks

\title{STAT 642: HW 7}
\author{Evan P. Walsh}
\makeatletter
\let\runauthor\@author
\let\runtitle\@title
\makeatother
\lhead{\runauthor}
\chead{\runtitle}
\rhead{\thepage}
\cfoot{}

\begin{document}
\maketitle


\subsection*{1 [AL 9.33]}
\begin{tcolorbox}
Let $X_{n} \rightarrow^{d} X$ and $x_{n} \rightarrow x$ as $n\rightarrow\infty$. If $P(X = x) = 0$, then show that $P(X_{n}\leq x_{n}) \rightarrow P(X
\leq x)$.
\end{tcolorbox}
\begin{Proof}

\end{Proof}

\newpage
\subsection*{3 [Proposition 5.9 (ii)]}
\begin{tcolorbox}
If $\left\{ X_{n} \right\}_{n=0}^{\infty}$ is tight and $Y_{n} \stackrel{p}{\rightarrow} 0$ ($X_{n}, Y_{n}$ defined on $(\Omega_{n}, \mathcal{F}_{n},
P_{n})$), then $X_{n}Y_{n} \stackrel{p}{\rightarrow} 0$.
\end{tcolorbox}
\begin{Proof}
Let $\epsilon, \delta > 0$. Since $\left\{ X_{n} \right\}_{n=0}^{\infty}$ is tight, there exists some $M > 0$ such that $sup_{\mathbb{N}}P_{n}(|X_{n}|
> M) < \delta / 2$. Further, since $Y_{n} \stackrel{p}{\rightarrow} 0$, there exists some $N \in \mathbb{N}$ such that $P_{n}(|Y_{n}| > \epsilon / M)
< \delta / 2$ for all $n \geq N$. Thus, for $n \geq N$,
\begin{align*}
P_{n}\left( |X_{n}Y_{n}| > \epsilon \right) & = P_{n}\left( |X_{n}Y_{n}| > \epsilon, |X_{n}| > M \right) + P_{n}\left( |X_{n}Y_{n}| > \epsilon, |X_{n}|
\leq M \right) \\
& \leq P_{n}\left( |X_{n}| > M \right) + P_{n}\left( M|Y_{n}| > \epsilon, |X_{n}| \leq M \right) \\
& \leq P_{n}\left( |X_{n}| > M \right) + P_{n}\left( |Y_n| > \epsilon / M \right) \\
& < \delta / 2 + \delta / 2 = \delta.
\end{align*}
So $X_{n}Y_{n} \stackrel{p}{\rightarrow} 0$.
\end{Proof}


\end{document}

