\documentclass[12pt]{article}
\usepackage{amsmath}
\usepackage{amsfonts}
\usepackage{parskip}
\usepackage{amsthm}
\usepackage{thmtools}
\usepackage[headheight=15pt]{geometry}
\geometry{a4paper, left=20mm, right=20mm, top=30mm, bottom=30mm}
\usepackage{graphicx}
\usepackage{bm} % for bold font in math mode - command is \bm{text}
\usepackage{enumitem}
\usepackage{fancyhdr}
\usepackage{amssymb} % for stacked arrows and other shit
\pagestyle{fancy}
\usepackage{changepage}
\usepackage{mathcomp}
\usepackage{tcolorbox}

\declaretheoremstyle[headfont=\normalfont]{normal}
\declaretheorem[style=normal]{Theorem}
\declaretheorem[style=normal]{Proposition}
\declaretheorem[style=normal]{Lemma}
\newcounter{ProofCounter}
\newcounter{ClaimCounter}[ProofCounter]
\newcounter{SubClaimCounter}[ClaimCounter]
\newenvironment{Proof}{\stepcounter{ProofCounter}\textsc{Proof.}}{\hfill$\square$}
\newenvironment{claim}[1]{\vspace{1mm}\stepcounter{ClaimCounter}\par\noindent\underline{\bf Claim \theClaimCounter:}\space#1}{}
\newenvironment{claimproof}[1]{\par\noindent\underline{Proof of claim \theClaimCounter:}\space#1}{\hfill $\blacksquare$ Claim \theClaimCounter}
\newenvironment{subclaim}[1]{\stepcounter{SubClaimCounter}\par\noindent\emph{Subclaim \theClaimCounter.\theSubClaimCounter:}\space#1}{}
% \newenvironment{subclaimproof}[1]{\begin{adjustwidth}{2em}{0pt}\par\noindent\emph{Proof of subclaim \theClaimCounter.\theSubClaimCounter:}\space#1}{\hfill
% $\blacksquare$ \emph{Subclaim \theClaimCounter.\theSubClaimCounter}\vspace{5mm}\end{adjustwidth}}
\newenvironment{subclaimproof}[1]{\par\noindent\emph{Proof of subclaim \theClaimCounter.\theSubClaimCounter:}\space#1}{\hfill
$\Diamond$ \emph{Subclaim \theClaimCounter.\theSubClaimCounter}}

\title{STAT 642: HW 2}
\author{Evan P. Walsh}
\makeatletter
\let\runauthor\@author
\let\runtitle\@title
\makeatother
\lhead{\runauthor}
\chead{\runtitle}
\rhead{\thepage}
\cfoot{}

\begin{document}
\maketitle


\subsection*{1 [AL 2.28]}
\begin{tcolorbox}
Let $\mu$ be the Lebesgue measure on $\left( [-1,1], \mathcal{B}([-1,1]) \right)$. For $n \geq 1$, define $f_{n}(x) = nI_{(0,n^{-1})}(x) -
nI_{(-n^{-1},0)}(x)$ and $f(x) = 0$ for $x \in [-1,1]$. Show that $f_{n} \rightarrow f$ a.e. ($\mu$) and $\int f_{n}d\mu \rightarrow \int fd\mu$, but
$\left\{ f_{n} \right\}_{n\geq 1}$ is not uniformly integrable.
\end{tcolorbox}

\begin{Proof}
$\ $

\begin{claim}
$f_{n} \rightarrow f$ a.e. ($\mu$).
\end{claim}
\begin{claimproof}
Let $x \in [-1,1]$. If $x = 0$, then $f_{n}(x) = 0$ for all $n \in \mathbb{N}$. If $x \neq 0$, then there exists $N \in \mathbb{N}$ such that $|x| >
N^{-1}$. Therefore $x \notin (0,n^{-1}) \cup (-n^{-1},0)$, so $f_{n}(x) = 0$ for all $n \geq N$. Thus $f_{n}(x) \rightarrow 0$ as $n \rightarrow
\infty$ for all $x \in [-1,1]$.
\end{claimproof}

\begin{claim}
$\int f_{n}d\mu \rightarrow \int fd\mu$.
\end{claim}
\begin{claimproof}
Let $n \geq 1$. Then 
\begin{align*}
\int f_{n}d\mu & = 0\cdot \mu\left( [-1,-n^{-1}] \right) -n\cdot \mu\left( (-n^{-1},0) \right) + 0\cdot \mu\left( \{0\} \right) + 
n\cdot \mu\left( (0,n^{-1}) \right) + 0\cdot \mu\left( [n^{-1},1] \right) \\
& = -n(1/n) + n(1/n) = 0.
\end{align*}
Therefore $\int f_{n}d\mu \rightarrow 0 = \int fd\mu$.
\end{claimproof}

\begin{claim}
$\left\{ f_{n} \right\}_{n\geq 1}$ is not uniformly integrable.
\end{claim}
\begin{claimproof}
Let $t > 0$. Then there exists $N \in \mathbb{N}$ such that $N > t$. Therefore 
\[ A_{n,t} = \left\{ x \in [-1,1] : |f_{n}(x)| > t \right\} = (-1/n, 0) \cup (0, 1/n), \ \forall \ n \geq N. \]
Thus,
\[ a_{n}(t) = \int_{A_{n,t}} |f_{n}|d\mu = n\cdot \mu\left( (-n^{-1},0) \right) + n\cdot \mu\left( (0,n^{-1} \right) = 1 + 1 = 2, \]
for all $n \geq N$. Hence, $\sup_{n\geq 1}a_{n}(t) = 2$ for all $t > 0$, so $\sup_{n\geq 1}a_{n}(t) \rightarrow 2$ as $t\rightarrow \infty$.
\end{claimproof}

\end{Proof}


\newpage
\subsection*{2}
\begin{tcolorbox}
Let $f_{n}(x) := n^{-1}\mathbb{I}_{(0,n^2)}(x)$ and $f(x) \equiv 0$ for all $x \in \mathbb{R}$. Let $m$ denote the Lebesgue measure on $(\mathbb{R},
\mathcal{B}(\mathbb{R}))$. Show that $f_{n} \rightarrow f$ a.e. ($m$) and $\left\{ f_{n} \right\}_{n\geq 1}$ is UI but $\int f_{n}dm \nrightarrow \int
f\ dm$.
\end{tcolorbox}
\begin{Proof}
Let $\epsilon > 0$. Then there exists $N \geq 1$ such that $n^{-1} < \epsilon$. Therefore $f_{n}(x) < \epsilon$ for all $n \geq N$.
Thus $f_n \rightarrow 0$. Now note that for all $t > 1$, $m(\left\{ |f_{n}| > t \right\}) = 0$ for all $n \geq 1$. Hence 
\[ \sup_{n \geq 1} \int_{\{|f_{n}| > t\}}|f_{n}|\ dm = 0, \]
when $t > 1$. Thus $\left\{ f_{n} \right\}_{n\geq 1}$ is UI. However,
\[ \int f_{n}\ dm = n^{-1}\cdot n^{2} = n \rightarrow \infty. \]
\end{Proof}






\newpage
\subsection*{3}
\begin{tcolorbox}
Let $\left\{ f_{n} \right\}_{n=0}^{\infty} \subset L^{1}(\Omega, \mathcal{F}, \mu)$ and $f \in L^{1}(\Omega, \mathcal{F}, \mu)$ such that $f_{n}
\rightarrow f$ a.e. ($\mu$) and $\int |f_{n}|d\mu \rightarrow \int |f|d\mu$. Show that $\left\{ f_{n} \right\}_{n=0}^{\infty}$ is UI.
\end{tcolorbox}

\begin{Proof}
Since $f_{n} \rightarrow f$ a.e.($\mu$), $g_{n} := |f_{n}| \rightarrow g := |f|$ a.e.($\mu$). Thus, since $f_{n} \leq g_n$, $g_{n}, g \in
L^{1}(\Omega, \mathcal{F}, \mu)$, and $\|g_{n}\|_{1} \rightarrow \|g\|_{1}$,
\begin{equation}
\lim_{n\rightarrow\infty}\int |f_{n} - f|\ d\mu = 0,
\label{3.1}
\end{equation}
by the Extended Dominated Convergence Theorem. Let $\epsilon > 0$. By \eqref{3.1}, there exists $N_{1} \in \mathbb{N}$ such that 
\begin{equation}
\|f_{n} - f\|_{1} < \epsilon/2 \ \ \forall \ n \geq N_{1}. 
\label{3.2}
\end{equation}
Now, since $f \in L^{1}(\mu)$, there exists $\delta > 0$ such that $\int_{E}|f|d\mu < \epsilon / 2$ whenever $\mu(E) < \delta$.
Further, by \eqref{3.1} we can apply Theorem 2.5.3, which says $f_{n} \rightarrow^{m} f$. Therefore there exists $N_{2} \in \mathbb{N}$ such that 
$\mu\left( \{|f_{n} - f| > 1\} \right) < \delta$. Let $N := \max\{N_{1}, N_{2}\}$.
Fix $t > 0$. For $n \geq N$, since
$\{|f_{n}| > t\} \subseteq \{|f| + 1 > t\} \cup \{ |f_{n} - t| > 1\}$,
\begin{equation}
\int_{|f_{n}| > t}|f|\ d\mu \leq \int_{|f| > t - 1}|f| \ d\mu + \int_{|f_{n}-f|>1}|f|\ d\mu < \int_{|f| > t -1}|f|\ d\mu + \epsilon/2.
\label{3.3}
\end{equation}
Thus
\begin{align*}
\sup_{\mathbb{N}} \int_{|f_{n}| > t}|f_{n}|\ d\mu & \leq \sup_{0\leq n < N}\int_{|f_{n}| > t}|f_{n}|\ d\mu + \sup_{n\geq N}\int_{|f_{n}| > t}|f_{n}|\
d\mu \\
& \leq \sup_{0\leq n < N}\int_{|f_{n}| > t}|f_{n}|\ d\mu + \sup_{n\geq N}\left(\int_{|f_{n}| > t}|f_{n} - f|\ d\mu + \int_{|f_{n}|>t}|f|\ d\mu\right) \\
& \leq \sup_{0\leq n < N}\int_{|f_{n}| > t}|f_{n}|\ d\mu + \int_{\Omega}|f_{n} - f|\ d\mu + \sup_{n\geq N}\int_{|f_{n}|>t}|f|\ d\mu \\
& \leq \sup_{0\leq n < N}\int_{|f_{n}| > t}|f_{n}|\ d\mu + \epsilon/2 + \sup_{n\geq N}\left(\int_{|f|>t-1}|f|\ d\mu + \int_{|f_{n}-f|>1}|f|\
d\mu\right) \\
& \leq \sup_{0\leq n < N}\int_{|f_{n}| > t}|f_{n}|\ d\mu + \epsilon/2 + \int_{|f|>t-1}|f|\ d\mu + \epsilon / 2.
\end{align*}
Further, since $\left\{ f_{n} \right\}_{n=0}^{N}$ and $\left\{ f \right\}$ are finite collections, they are UI. Hence 
\[ \lim_{t\rightarrow\infty}\sup_{\mathbb{N}}\int_{|f_{n}| > t}|f_{n}|\ d\mu \leq \lim_{t\rightarrow\infty}\left( \sup_{0\leq n < N}\int_{|f_{n}| > t}
|f_{n}|\ d\mu + \int_{|f| > t -1}|f|\ d\mu + \epsilon \right) = \epsilon. \]
Since $\epsilon > 0$ was arbitrary, $\lim_{t\rightarrow\infty}\sup_{\mathbb{N}}\int_{|f_{n}| > t}|f_{n}|\ d\mu = 0$.
\end{Proof}






\newpage
\subsection*{4 [AL 2.44]}
\begin{tcolorbox}
Let $f_{n}(x) := c_{n}\left( 1 - \frac{x}{n} \right)^{n}\mathbb{I}_{[0,n]}(x)$, for $x \in \mathbb{R}$ and $n\geq 1$.
\begin{enumerate}[label=(\alph*)]
\item Find $c_{n}$ such that $\int f_{n}\ d\mu = 1$.
\item Show that $\lim_{n\rightarrow\infty}f_{n}(x) := f(x)$ exists for all $x \in \mathbb{R}$ and that $f$ is a pdf.
\item For $A \in \mathcal{B}(\mathbb{R})$, let 
\[ \nu_{n}(A) := \int_{A}f_{n}\ dm \qquad \text{and} \qquad \nu(A) := \int_{A}f\ dm. \]
Show that $\nu_{n} \rightarrow \nu$ uniformly on $\mathcal{B}(\mathbb{R})$.
\end{enumerate}
\end{tcolorbox}

\begin{enumerate}[label=(\alph*)]
\item Let $c_{n} := (n+1)/n$. Then 
\begin{align*}
\int f_{n}\ d\mu = c_{n} \int_{[0,n]}\left( 1-\frac{x}{n} \right)^{n}\ d\mu(x) & = c_{n} \int_{0}^{n}\left( 1-\frac{x}{n} \right)^{n}\ dx \\
& = c_{n}\cdot \frac{(-n)}{n+1}\left( 1-\frac{x}{n} \right)^{n+1}\bigg|_{0}^{n} \\
& = c_{n}\cdot \frac{n}{n+1} = 1.
\end{align*}

\item 
\begin{Proof} Set 
\[ f(x) := \left\{ \begin{array}{cl}
e^{-x} & \text{ if } x \geq 0 \\
0 & \text{ if } x < 0
\end{array} \right., \ x \in \mathbb{R}. \]
If $x < 0$, then $f_{n}(x) \equiv 0$ for all $n \in \mathbb{N}$. If $x \geq 0$, then 
\[ \lim_{n\rightarrow\infty}f_{n}(x) = \lim_{n\rightarrow\infty} \frac{n+1}{n}\left( 1-\frac{x}{n} \right)^{n}\mathbb{I}_{[0,n]}(x) = e^{-x}. \]
Thus $f_{n} \rightarrow f$.
\end{Proof}
\item 
\begin{Proof}
We need to show that $\sup_{A\in \mathcal{B}(\mathbb{R})}|\nu_{n}(A) - \nu(A)| \rightarrow 0$, but by a result on the previous homework,
\[ \sup_{A\in \mathcal{B}(\mathbb{R})}|\nu_{n}(A) - \nu(A)| = \frac{1}{2}\int |f_{n} - f|\ d\mu. \]
Therefore we need to show $\lim_{n\rightarrow\infty}\|f_{n} - f\|_{1} = 0$. Now, since $1 \equiv \|f_{n}\|_{1} \equiv \|f\|_{1}$ and $f_{n} \rightarrow f$
from part (b), $\|f_{n} - f\|_{1} \rightarrow 0$ by Scheffe's Theorem.
\end{Proof}
\end{enumerate}



\newpage 
\subsection*{5 [AL 2.47]}
\begin{tcolorbox}
Let $\left\{ f_{n} \right\}_{n=0}^{\infty}$ be a sequence of continuous functions from $[0,1]$ to $[0,1]$ such that $f_{n}(x) \rightarrow 0$ as $n
\rightarrow \infty$ for all $0 \leq x \leq 1$. Show that $\int_{0}^{1}f_{n}(x)dx \rightarrow 0$ as $n\rightarrow \infty$ by two methods: one using BCT
and one without using BCT. Show also that if $\mu$ is a finite measure on $([0,1], \mathcal{B}([0,1]))$, then $\int_{[0,1]}f_{n}\ d\mu \rightarrow 0$.
\end{tcolorbox}

First we will show the result using the BCT.

\begin{Proof}
By Theorem 2.4.1, the Lebesgue integral of $f_{n}$ coincides with the Riemann integral of $f_{n}$ over $[0,1]$ for each $n \in \mathbb{N}$. 
Then, since each $f_{n}$ is bounded above by 1 and $f_{n} \rightarrow 0$, 
\[ \lim_{n\rightarrow\infty}\int_{0}^{1}f(x)\ dx = \lim_{n\rightarrow\infty}\int_{[0,1]}f_{n}\ dm = 0 \]
by the bounded convergence theorem.
\end{Proof}

Now we will show the result without using the BCT.

\begin{Proof}
Let $\epsilon > 0$. By Egorov's Theorem, there exists some $A \in \mathcal{B}([0,1])$ such that $m(A) < \epsilon/2$ and $f_{n} \rightarrow f$ uniformly
on $A^{c}$. Thus there exists $N \in \mathbb{N}$ such that $|f_{n}(x) - 0| = f_{n}(x) < \epsilon / 2$ whenever $x \in A^{c}$ and $n \geq N$. Thus,
for $n\geq N$,
\[ \int_{0}^{1}f_{n}\ dm = \int_{[0,1]}f\ dm = \int_{A}1\ dm + \int_{A^{c}}\frac{\epsilon}{2}\ dm \leq 
\frac{\epsilon}{2} + \int_{[0,1]}\frac{\epsilon}{2}\ dm = \epsilon. \]
Thus $\int f_{n} dx \rightarrow 0$ as $n \rightarrow \infty$.
\end{Proof}

Now suppose $\mu$ is a finite measure on $([0,1], \mathcal{B}([0,1]))$. We will show $\int_{[0,1]}f_{n}\ d\mu \rightarrow 0$.

\begin{Proof}
Apply the BCT once again.
\end{Proof}



\newpage 
\subsection*{6 [AL 7.1]}
\begin{tcolorbox}
Give an example of three events $A_{1}, A_{2}, A_{3}$ on some probability space such that they are pairwise independent but not independent.
\end{tcolorbox}

Suppose $X_{i}$, for $i = 1,2,3$, are iid random variables such that $P(X_{i} = 0) = P(X_{i} = 1) = 1/2$. Then define 
\[ A_{1} := \{X_{1} = X_{2}\}, \qquad A_{2} := \left\{ X_{1} = X_{3} \right\}, \qquad A_{3} := \left\{ X_{2} = X_{3} \right\}. \]
Since $X_{1}$ and $X_{2}$ are independent, 
\[ P(A_{1}) = P(X_{1} = 0 \wedge X_{2} = 0) + P(X_{1} = 1 \wedge X_{2} = 1) = \left( \frac{1}{2} \right)\left( \frac{1}{2} \right) + 
\left( \frac{1}{2} \right)\left( \frac{1}{2} \right) = \frac{1}{2}. \]
Similarly $P(A_{2}) = P(A_{3}) = 1/2$. Further,
\begin{align*}
P(A_{1}\cap A_{2}) = P(X_{1} = X_{2} \wedge X_{1} = X_{3}) & = P(X_{1} = X_{2} = X_{3} = 0) + P(X_{1} = X_{2} = X_{3} = 1) \\
& = \left( \frac{1}{2} \right)\left( \frac{1}{2} \right)\left( \frac{1}{2} \right) + \left( \frac{1}{2} \right)\left( \frac{1}{2} \right)\left(
\frac{1}{2} \right) \\
& = \frac{1}{4} = P(A_{1})\cdot P(A_{2}).
\end{align*}
By similar calculation, $P(A_{i}\cap A_{j}) = P(A_{i})\cdot P(A_{j})$ for all $i \neq j$. Hence $\left\{ A_{i} \right\}_{i=1}^{3}$ are pairwise
independent. However,
\begin{align*}
P(A_{1} \cap A_{2} \cap A_{3}) & = P(X_{1} = X_{2} \wedge X_{1} = X_{3} \wedge X_{3} = X_{2}) \\
& = P(X_{1} = X_{2} = X_{3}) \\
& = P(X_{1} = X_{2} = X_{3} = 0) + P(X_{1} = X_{2} = X_{3} = 1) \\
& = \left( \frac{1}{8} \right) + \left( \frac{1}{8} \right) \\
& = \frac{1}{4} \neq P(A_{1})\cdot P(A_{2}) \cdot P(A_{3}) = \frac{1}{8}.
\end{align*}


\end{document}

