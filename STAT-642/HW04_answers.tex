\documentclass[12pt]{article}
\usepackage{amsmath}
\usepackage{amsfonts}
\usepackage{parskip}
\usepackage{amsthm}
\usepackage{thmtools}
\usepackage[headheight=15pt]{geometry}
\geometry{a4paper, left=20mm, right=20mm, top=30mm, bottom=30mm}
\usepackage{graphicx}
\usepackage{bm} % for bold font in math mode - command is \bm{text}
\usepackage{enumitem}
\usepackage{fancyhdr}
\usepackage{amssymb} % for stacked arrows and other shit
\pagestyle{fancy}
\usepackage{changepage}
\usepackage{mathcomp}
\usepackage{tcolorbox}

\declaretheoremstyle[headfont=\normalfont]{normal}
\declaretheorem[style=normal]{Theorem}
\declaretheorem[style=normal]{Proposition}
\declaretheorem[style=normal]{Lemma}
\newcounter{ProofCounter}
\newcounter{ClaimCounter}[ProofCounter]
\newcounter{SubClaimCounter}[ClaimCounter]
\newenvironment{Proof}{\stepcounter{ProofCounter}\textsc{Proof.}}{\hfill$\square$}
\newenvironment{claim}[1]{\vspace{1mm}\stepcounter{ClaimCounter}\par\noindent\underline{\bf Claim \theClaimCounter:}\space#1}{}
\newenvironment{claimproof}[1]{\par\noindent\underline{Proof of claim \theClaimCounter:}\space#1}{\hfill $\blacksquare$ Claim \theClaimCounter}
\newenvironment{subclaim}[1]{\stepcounter{SubClaimCounter}\par\noindent\emph{Subclaim \theClaimCounter.\theSubClaimCounter:}\space#1}{}
% \newenvironment{subclaimproof}[1]{\begin{adjustwidth}{2em}{0pt}\par\noindent\emph{Proof of subclaim \theClaimCounter.\theSubClaimCounter:}\space#1}{\hfill
% $\blacksquare$ \emph{Subclaim \theClaimCounter.\theSubClaimCounter}\vspace{5mm}\end{adjustwidth}}
\newenvironment{subclaimproof}[1]{\par\noindent\emph{Proof of subclaim \theClaimCounter.\theSubClaimCounter:}\space#1}{\hfill
$\Diamond$ \emph{Subclaim \theClaimCounter.\theSubClaimCounter}}

\title{STAT 642: HW 4}
\author{Evan P. Walsh}
\makeatletter
\let\runauthor\@author
\let\runtitle\@title
\makeatother
\lhead{\runauthor}
\chead{\runtitle}
\rhead{\thepage}
\cfoot{}



\begin{document}
% \maketitle

\def\cp{\stackrel{p}{\rightarrow}}
\def\as{\stackrel{\text{a.s.}}{\longrightarrow}}

\subsection*{1}
\begin{tcolorbox}
Suppose $X, X_{n}, Y, Y_{n}$ are random variables on a common probability space. Show that
\begin{enumerate}[label=(\alph*)]
\item if $X_{n} \stackrel{p}{\rightarrow} X$ and $X_{n}\stackrel{p}{\rightarrow}Y$, then $P(X = Y) = 1$,
\item if $X_{n}\cp X$ and $Y_{n} \cp Y$, then $X_{n} + Y_{n} \cp X + Y$,
\item if $X_{n}\cp X$ and $Y_{n}\cp Y$, then $X_{n}Y_{n}\cp XY$, 
\item and if $X_{n}\cp X$, $Y_{n}\cp Y$ and $g$ is a continuous function on $\mathbb{R}^{2}$, then $g(X_{n},Y_{n})\cp g(X,Y)$.
\end{enumerate}
\end{tcolorbox}
\begin{enumerate}[label=(\alph*)]
\item \begin{Proof}
We need to show that $P(X - y = 0) = 1$, or $P(|X - Y| > 0) = 0$, or $P(|X - Y| > \epsilon) = 0$ for all $\epsilon > 0$. But, since $|X - Y| \leq |X -
X_{n}| + |X_{n} - Y|$,
\[ P(|X - Y| > \epsilon) \leq P(|X - X_{n}| + |X_{n} - Y| > \epsilon) \leq P(|X - X_{n}| > \epsilon/2) + P(|X_{n} - Y| > \epsilon/2).\]
But the right-hand side of the above inequality goes to 0 as $n\rightarrow \infty$, independent of $\epsilon$. Hence $P(|X-Y| > 0) = 0$.
\end{Proof}

\item \begin{Proof} Let $\epsilon > 0$. Since $|(X_{n} + Y_{n}) - (X+Y)| \leq |X_{n} - X| + |Y_{n} - Y|$,
\[ P(|(X_{n} + Y_{n}) - (X + Y) > \epsilon) \leq P(|X_{n} - X| + |Y_{n} - Y| > \epsilon) \leq P(|X_{n} - X|>\epsilon/2) + P(|Y_{n}-Y| > \epsilon/2).
\]
But the right-hand side of the above inequality goes to 0 as $n\rightarrow\infty$. Hence $X_{n} + Y_{n}\cp X + Y$.
\end{Proof}

\item \begin{Proof}
Let $\epsilon > 0$. Since 
\[ |X_{n}Y_{n} - XY| = |X_{n}Y_{n} - XY + X_{n}Y - X_{n}Y| \leq |X_{n}|\cdot |Y_{n} - Y| + |Y|\cdot |X_{n} - X|, \]
\begin{align*}
P(|X_{n}Y_{n} - XY| > \epsilon) & \leq P(|X_{n}|\cdot |Y_{n} - Y| + |Y|\cdot |X_{n} - X| > \epsilon \\
& \leq P(|X_{n}|\cdot |Y_{n}-Y| > \epsilon/2) + P(|Y|\cdot |X_{n} - X| > \epsilon/2) \\
& = P(\left\{ |X_{n}|\cdot |Y_{n}-Y| > \epsilon/2 \right\}\cap \left\{ |X_{n}-X|\leq 1 \right\}) \\
& \qquad + P(\left\{ |X_{n}|\cdot |X_{n}-Y| > \epsilon/2 \right\} 
\cap\left\{ |X_{n}-X|>1 \right\}) \\
& \qquad + P(|Y|\cdot |X_{n} -X| > \epsilon/2) \\
& = P([|X| + 1]|Y_{n} - Y| > \epsilon/2)  \rightarrow 0\\
& \qquad + P(|X_{n} - X| > 1) \rightarrow 0\\
& \qquad + P(|Y|\cdot |X_{n} - X| > \epsilon/2) \rightarrow 0
\end{align*}
Since everything on the right-hand side goes to 0 as $n\rightarrow\infty$, $X_{n}Y_{n}\cp XY$.
\end{Proof}

\item \begin{Proof}
Let $\epsilon > 0$. Since $g$ is continuous, there exists $\delta > 0$ such that $|X_{n} - X| + |Y_{n} - Y| \leq \delta$ implies $|g(X_{n},Y_{n}) -
g(X,Y)| \leq \epsilon$. Thus,
\begin{align*}
P(|g(X_{n},Y_{n}) - g(X,Y)| > \epsilon) & \leq P(|X_{n} - X| + |Y_{n} - Y| > \delta) \\
& \leq P(|X_{n} - X| > \delta/2) + P(|Y_{n} - Y|>\delta/2).
\end{align*}
But everything on the right-hand side goes to 0 as $n\rightarrow\infty$. Therefore $g(X_{n},Y_{n}) \cp g(X,Y)$. 
\end{Proof}
\end{enumerate}

\subsection*{2}
\begin{tcolorbox}
Show that if $\left\{ X_{n} \right\}_{n=0}^{\infty}$ is a sequence of independent random variables and $X_{n} \cp X$, then $X$ is degenerate.
\end{tcolorbox}
\begin{Proof}
Since $X_{n}\cp X$, there exists a subsequence 
$\left\{ X_{n_{k}} \right\}_{k=0}^{\infty}$ such that $X_{n_{k}}\rightarrow X$ a.s. Therefore $T := \limsup_{k\rightarrow\infty}X_{n_{k}}
= X$ a.e. $(P)$. By an example from class, $T$ is a tail random variable, and since $\left\{ X_{n_{k}} \right\}_{k=0}^{\infty}$ is an independent
sequence of random variables, $T$ is degenerate by Corollary 3.5. Therefore there exists some $c \in \overline{\mathbb{R}}$ such that $P(T = c) = 1$.
But since $P(T = X) = 1$, $P(X = c) = 1$. Hence $X$ is degenerate.
\end{Proof}

\subsection*{3}
\begin{tcolorbox}
For any sequence $\left\{ X_{n} \right\}_{n=0}^{\infty}$ of random variables with $X_{n}\cp X$, show that 
\[ P\left(\liminf_{n\rightarrow\infty}X_{n}\leq X\leq \limsup_{n\rightarrow\infty}X_{n}\right) = 1. \]
\end{tcolorbox}
\begin{Proof}
Since $X_{n} \cp X$, there exists a subsequence $\left\{ X_{n_{k}} \right\}_{k=0}^{\infty}$ such that $X_{n_{k}}\rightarrow X$ a.s. 
Without loss of generality assume $k\leq n_{k}$ for all $k \in \mathbb{N}$.
Let $E := \left\{ \omega : X_{n_{k}}(\omega) \rightarrow X(\omega) \right\}$. Note that
\[ X_{n_{k}} \leq \sup_{j\geq k}X_{n_{j}} \leq \sup_{j\geq k}X_{k}. \]
Therefore $X(\omega) = \lim_{k\rightarrow\infty}X_{n_{k}}(\omega) \leq \lim_{k\rightarrow\infty}\sup_{j\geq k}X_{k} =
\limsup_{k\rightarrow\infty}X_{k}$ for all $\omega \in E$. Since $P(E) = 1$, $P(X \leq \limsup X_{n}) = 1$, and therefore $P(X > \limsup X_{n}) = 0$.
By a similar calculation, $P(X < \liminf X_{n}) = 0$. Thus 
\begin{align*}
P(\liminf X_{n} \leq X \leq \limsup X_{n}) & = 1 - P(\liminf X_{n} > X \text{ or } \limsup X_{n} < X) \\
& \geq 1 - P(\liminf X_{n} > X) - P(\limsup X_{n}
< X) = 1. 
\end{align*}
\end{Proof}

\newpage
\subsection*{4}
\begin{tcolorbox}
Let $X, X_{0}, X_{1}, \hdots$ be random variables and suppose that for some $r > 0$,
\[ E|X_{n} - X|^{r} \rightarrow 0 \ \text{ as } \ n \rightarrow \infty. \]
Prove that $X_{n}\cp X$.
\end{tcolorbox}
\begin{Proof}
By way of contradiction, suppose there exists $\epsilon, \delta > 0$ such that for all $N \in \mathbb{N}$, there exists an $n \geq N$ such that 
\begin{equation}
P(|X_{n} - X| > \epsilon) > \delta, \qquad \text{and thus}\qquad P(|X_{n} - X|^{r} > \epsilon^{r}) > \delta. 
\label{4.1}
\end{equation}
Therefore we can create a subsequence $n_{0} < n_{1} < n_{2} < \hdots$ such that \eqref{4.1} holds for $n_{k}$, for all $k \in \mathbb{N}$.
But then for each $k \in \mathbb{N}$,
\[ E|X_{n_{k}} - X|^{r} = \int_{\Omega}|X_n - X|^{r}\ dP \geq \int_{E_{k}}|X_{n} - X|^{r}\ dP \geq \int_{E_{k}}\epsilon^{r}\ dP \geq \delta\cdot
\epsilon^{r}, \]
where $E_{k} := \left\{ \omega \in \Omega : |X_{n_{k}}(\omega)- X(\omega)|^{r} > \epsilon \right\}$.
So $\limsup_{k\rightarrow\infty}E|X_{n_{k}}-X|^{r} \geq \delta\cdot \epsilon^{r}$. This is a contradiction.
\end{Proof}

\subsection*{5}
\begin{tcolorbox}
Let $X, X_{0}, X_{1}, \hdots$ be random variables and suppose that for some $r > 0$,
\[ \sum_{n=0}^{\infty}E|X_{n} - X|^{r} < \infty. \]
Prove that $X_{n} \as X$.
\end{tcolorbox}
\begin{Proof}
Let $\epsilon > 0$ and define $A_{n} := \left\{ \omega \in \Omega : |X_{n}(\omega) - X(\omega)| > \epsilon \right\} = \left\{ 
\omega \in \Omega : |X_{n}(\omega) - X(\omega)|^{r} > \epsilon^r \right\}$. Therefore 
\[ \epsilon^{r}P(A_{n}) \leq \int_{A_{n}}|X_{n}-X|^{r}\ dP \leq \int_{\Omega}|X_{n} - X|^{r}\ dP = E|X_{n} - X|^{r}. \]
Hence $\sum_{n=0}^{\infty}P(A_{n}) \leq \epsilon^{-r}\sum_{n=0}^{\infty}E|X_{n} - X|^{r} < \infty$. Therefore, by the Borel-Cantelli Lemma,
$P(A_{n}\text{ i.o.}) = 0$. So $P(|X_{n} - X|\leq \epsilon\text{ eventually}) = 1$. Thus, by Theorem 4.1, $X_{n} - X \as 0$, and so $X_{n} \as X$.
\end{Proof}

\newpage
\subsection*{6}
\begin{tcolorbox}
Suppose $X_{1}, X_{2}, \hdots$ are random variables. For each $n \geq 1$, let $S_{n} := \sum_{j=1}^{n}X_{j}$. Let $b_{1}, b_{2}, \hdots \in
\mathbb{R}$ such that $0 < b_{n} < \infty$ and $b_{n} = \mathcal{O}(b_{n+1})$ for all $n \geq 1$. Prove that if 
$\frac{S_{n}}{b_{n}} \as 0$, then $\frac{X_{n}}{b_{n}} \as 0$.
\end{tcolorbox}
\begin{Proof}
Since $b_{n} = \mathcal{O}(b_{n+1})$, there exists some $M > 0$ such that $b_{n} \leq M b_{n+1}$ for all $n \in \mathbb{N}$. This implies 
\begin{equation}
b_{n-1} \leq M\cdot b_{n} \ \text{ for all } \ n\geq 1.
\label{6.1}
\end{equation}
Now, since $S_{n}/b_{n} \as 0$, clearly $M\cdot S_{n-1}/b_{n-1} \as 0$. Therefore, since
\[ \left|\frac{S_{n-1}}{b_{n}}\right| = \left|\frac{\sum_{j=0}^{n-1}X_{j}}{b_{n}}\right| \leq \left| \frac{\sum_{j=0}^{n-1}X_{j}}{b_{n-1}}\right|\times M  = \left|
\frac{S_{n-1}}{b_{n-1}}\right|\times M,\]
$S_{n-1}/b_{n} \as 0$. Hence by the results in question 1,
\[ \frac{S_{n}}{b_{n}} - \frac{S_{n-1}}{b_{n}} = \frac{X_{n}}{b_{n}} \as 0. \]
\end{Proof}

\end{document}

