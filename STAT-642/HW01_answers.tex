\documentclass[12pt]{article}
\usepackage{amsmath}
\usepackage{amsfonts}
\usepackage{parskip}
\usepackage{amsthm}
\usepackage{thmtools}
\usepackage[headheight=15pt]{geometry}
\geometry{a4paper, left=20mm, right=20mm, top=30mm, bottom=30mm}
\usepackage{graphicx}
\usepackage{bm} % for bold font in math mode - command is \bm{text}
\usepackage{enumitem}
\usepackage{fancyhdr}
\usepackage{amssymb} % for stacked arrows and other shit
\pagestyle{fancy}
\usepackage{changepage}
\usepackage{mathcomp}

\declaretheoremstyle[headfont=\normalfont]{normal}
\declaretheorem[style=normal]{Theorem}
\declaretheorem[style=normal]{Proposition}
\declaretheorem[style=normal]{Lemma}
\newcounter{ProofCounter}
\newcounter{ClaimCounter}[ProofCounter]
\newcounter{SubClaimCounter}[ClaimCounter]
\newenvironment{Proof}{\stepcounter{ProofCounter}\textit{Proof.}}{\hfill$\square$}
\newenvironment{claim}[1]{\vspace{1mm}\stepcounter{ClaimCounter}\par\noindent\underline{\bf Claim \theClaimCounter:}\space#1}{}
\newenvironment{claimproof}[1]{\par\noindent\underline{Proof of claim \theClaimCounter:}\space#1}{\hfill $\blacksquare$ Claim \theClaimCounter}
\newenvironment{subclaim}[1]{\stepcounter{SubClaimCounter}\par\noindent\emph{Subclaim \theClaimCounter.\theSubClaimCounter:}\space#1}{}
% \newenvironment{subclaimproof}[1]{\begin{adjustwidth}{2em}{0pt}\par\noindent\emph{Proof of subclaim \theClaimCounter.\theSubClaimCounter:}\space#1}{\hfill
% $\blacksquare$ \emph{Subclaim \theClaimCounter.\theSubClaimCounter}\vspace{5mm}\end{adjustwidth}}
\newenvironment{subclaimproof}[1]{\par\noindent\emph{Proof of subclaim \theClaimCounter.\theSubClaimCounter:}\space#1}{\hfill
$\Diamond$ \emph{Subclaim \theClaimCounter.\theSubClaimCounter}}

\title{STAT 642: HW 1}
\author{Evan P. Walsh}
\makeatletter
\let\runauthor\@author
\let\runtitle\@title
\makeatother
\lhead{\runauthor}
\chead{\runtitle}
\rhead{\thepage}
\cfoot{}

\begin{document}
\maketitle

\subsection*{1}

\begin{Proof} Let $\Omega := (0,1]$ and $\mathcal{F}$ consist of all finite disjoint intervals of the form $(a,a'] \subseteq (0,1]$.

\begin{claim}
$\mathcal{F}$ is an algebra.
\end{claim}
\begin{claimproof}
Clearly $\Omega \in \mathcal{F}$. Now suppose $0 \leq a \leq a' \leq 1$ and $0 \leq b \leq b' \leq 1$. Then 
\[ (0,1] - (a,a'] = (0,a] \cup (a', 1] \in \mathcal{F}, \]
so $\mathcal{F}$ is closed under complementation. Further, if $a' < b$ or $b' < a$, then clearly 
\begin{equation}
(a,a'] \cap (b,b'] = \emptyset = \Omega^{c} \in \mathcal{F}.
\label{1.1}
\end{equation}
On the other hand, if $b \leq a'$ and $a \leq b'$, then 
\begin{equation}
(a,a'] \cap (b,b'] = (\max\left\{ a,b \right\}, \min\left\{ a',b' \right\}] \in \mathcal{F}.
\label{1.2}
\end{equation}
So by \eqref{1.1} and \eqref{1.2}, the intersection of any two such intervals is in $\mathcal{F}$. Now suppose $A, B \in \mathcal{F}$. Then $A =
\cup_{i=1}^{n}(a_{i}, a_{i}']$ and $B = \cup_{j=1}^{m}(b_{j},b_{j}']$. Therefore 
\[ A\cap B = \cup_{i=1}^{n}\cup_{j=1}^{m}(a_{i},a_{i}']\cap (b_{j},b_{j}'] \in \mathcal{F} \]
by the above considerations. This proves that $\mathcal{F}$ is an algebra.
\end{claimproof}

\begin{claim}
$\mathcal{F}$ is NOT a $\sigma$-algebra.
\end{claim}
\begin{claimproof}
Note that $\cup_{n=0}^{\infty}(0,1-2^{-n}] = (0,1) \notin \mathcal{F}$. Thus $\mathcal{F}$ is not closed under countable unions, so cannot be a $\sigma$-algebra.
\end{claimproof}

\end{Proof}





\subsection*{2}

\begin{Proof}
It suffices to show that $\mathcal{L}$ is closed under countable unions. So, let $\left\{ A_{n} \right\}_{n=0}^{\infty} \subset \mathcal{L}$. Set
$B_{0} := A_{0}$, and for each $n \geq 1$, recursively define $B_{n} := B_{n-1} \cup A_{n}$. By the properties of $\lambda$ and $\pi$-classes, 
$B_{n} = (B_{n-1}^{c} \cap A_{n}^{c})^{c} \in \mathcal{F}$ for each $n \in \mathbb{N}$, since $\mathcal{L}$ is closed under complementation and finite
intersections. Further, by construction we have $B_{n} \subset B_{n+1}$ for every $n \in \mathbb{N}$, so 
\[ \cup_{n=0}^{\infty}A_{n} = \cup_{n=0}^{\infty}B_{n} \in \mathcal{L}, \]
since $\mathcal{L}$ is a $\lambda$-class.
\end{Proof}




\subsection*{3 [A/L 2.1]}
\begin{enumerate}[label=(\alph*)]
\item 

\begin{Proof}
Let $\omega_0 \in T^{-1}(\cup_{\alpha \in I}A_{\alpha}) = \left\{ \omega \in \Omega_{1} : T(\omega) \in \cup_{\alpha \in I}A_{\alpha} \right\}$. Then
$T(\omega_0) \in \cup_{\alpha \in I}A_{\alpha}$, so $T(\omega_{0}) \in A_{\alpha_0}$ for some $\alpha_{0} \in I$. Thus, $w_0 \in T^{-1}(A_{\alpha_{0}})$, so
$w_{0} \in \cup_{\alpha \in I}T^{-1}(A_\alpha)$. Hence 
\begin{equation}
T^{-1}\left( \bigcup_{\alpha\in I}A_{\alpha} \right) \subseteq \bigcup_{\alpha\in I}T^{-1}(A_\alpha).
\label{1}
\end{equation}
Now, let 
\begin{align*}
\omega_{1} \in \cup_{\alpha \in I}T^{-1}(A_\alpha) & = \cup_{\alpha\in I}\left\{ \omega \in \Omega_{1} : T(\omega) \in A_\alpha \right\} \\
& = \left\{ \omega \in \Omega_{1} : T(\omega) \in A_{\alpha}, \text{ for some }\alpha \in I \right\} \\
& = \left\{ \omega \in \Omega_1 : T(\omega) \in \cup_{\alpha \in I}A_{\alpha} \right\} \\
& = T^{-1}\left( \bigcup_{\alpha \in I}A_{\alpha} \right).
\end{align*}
Therefore 
\begin{equation}
\bigcup_{\alpha \in I}T^{-1}(A_{\alpha}) \subseteq T^{-1}\left( \bigcup_{\alpha \in I}A_{\alpha} \right).
\label{2}
\end{equation}
By \ref{1} and \ref{2} we have equality.
\end{Proof}

\item 

\begin{Proof}
Let $\omega_{0} \in T^{-1}\left( \cap_{\alpha \in I}A_{\alpha} \right) = \left\{ \omega \in \Omega_{1} : T(\omega) \in \cap_{\alpha \in I}A_{\alpha}
\right\}$. Then $T(\omega_0) \in \cap_{\alpha \in I}A_\alpha$, so $T(\omega_0) \in A_{\alpha}$ for every $\alpha \in I$. Thus $\omega_0 \in
T^{-1}(A_\alpha)$ for every $\alpha \in I$. Hence $\omega_0 \in \cap_{\alpha\in I}T^{-1}(A_\alpha)$. So 
\begin{equation}
T^{-1}\left( \bigcap_{\alpha\in I}A_{\alpha} \right) \subseteq \bigcap_{\alpha\in I}T^{-1}(A_\alpha).
\label{3}
\end{equation}
Now, let $\omega_{1} \in \cap_{\alpha\in I}T^{-1}(A_\alpha)$. Then $\omega_1 \in T^{-1}(A_\alpha) = \left\{ \omega \in \Omega_1 : T(\omega) \in
A_\alpha \right\}$ for each $\alpha \in I$. So $\omega_1 \in \left\{ \omega \in \Omega_1 : T(\omega) \in \cap_{\alpha \in I}A_\alpha \right\} =
T^{-1}\left( \cap_{\alpha \in I}A_\alpha \right)$. Thus 
\begin{equation}
\bigcap_{\alpha \in I}T^{-1}(A_\alpha) \subseteq T^{-1}\left( \bigcap_{\alpha\in I}A_\alpha \right).
\label{4}
\end{equation}
So by \ref{3} and \ref{4} we have equality.
\end{Proof}

\item

\begin{Proof}
Let $\omega_0 \in \left( T^{-1}(A) \right)^{c}$. Then $\omega_0 \notin T^{-1}(A)$. So 
\[ \omega_0 \in \left\{ \omega \in \Omega_1 : T(\omega) \notin A \right\} =  \left\{ \omega \in \Omega_1 : T(\omega) \in A^{c} \right\} =
T^{-1}(A^{c}). \]
Hence 
\begin{equation}
\left( T^{-1}(A) \right)^{c} \subseteq T^{-1}(A^{c}).
\label{5}
\end{equation}
Now, let $\omega_1 \in T^{-1}(A^{c}) = \left\{ \omega\in \Omega_1 : T(\omega) \in A^{c} \right\} = \left\{ \omega \in \Omega_1 : T(\omega) \notin A
\right\}$. Then $\omega_1 \notin T^{-1}(A)$, so $\omega_1 \in \left( T^{-1}(A) \right)^{c}$. Hence 
\begin{equation}
T^{-1}(A^{c}) \subseteq \left( T^{-1}(A) \right)^{c}.
\label{6}
\end{equation}
So by \ref{5} and \ref{6} we have equality.
\end{Proof}
\end{enumerate}



\subsection*{4}
\begin{enumerate}[label=(\alph*)]
\item 
\begin{Proof}
$(\Rightarrow)$ Suppose $A \subset \mathbb{R}$ is measurable. Then for any $B \in \mathcal{B}(\mathbb{R})$,
\[ f^{-1}[B] = \left\{ \begin{array}{cl}
A & \text{ if } 1 \in B \wedge 0 \notin B \\
A^{c} & \text{ if } 1 \notin B \wedge 0 \in B \\
\emptyset & \text{ if } 1,0 \notin B \\
\mathbb{R} & \text{ if } 1,0 \in B.
\end{array} \right. \]
In each case above, $f^{-1}[B] \in \mathcal{B}(\mathbb{R})$, so $f$ is measurable.

$(\Leftarrow)$ Now suppose $f$ is measurable. Then $A = f^{-1}[\{1\}] \in \mathcal{B}(\mathbb{R})$.
\end{Proof}

\item 
\begin{Proof}
Let $a \in \mathbb{R}$. It suffices to show that $f^{-1}[(-\infty, a)] \in \mathcal{B}(\mathbb{R})$. Let $c := \sup\left\{ x : f(x) < a \right\}$.
Then 
\[ f^{-1}[(-\infty, a)] = \left\{ x : f(x) < a \right\} = (-\infty, c) \in \mathcal{B}(\mathbb{R}). \]
\end{Proof}

\end{enumerate}



\subsection*{5}

Let $x \in \mathbb{R}$. If $x < 0$, then $f_{n}(x) = 0$ for all $n \in \mathbb{N}$. If $x \geq 0$, then there exists $N \in \mathbb{N}$ such that $N >
x$. Therefore $f_{n}(x) = 0$ for all $n \geq N$. This implies that $f(x) \equiv 0$. However, for each $n \in \mathbb{N}$,
\[ \int f_{n}(x) \ dm(x) = \frac{1}{n}\cdot m([n,2n]) = 1, \]
so $\lim_{n\rightarrow\infty} \int f_{n}\ dm = 1 \neq \int f\ dm$.


\subsection*{6}

\begin{Proof}
Let $\epsilon > 0$. Since $f^{+}$ and $f^{-}$ are nonnegative and integrable, there exists simple functions $\varphi \leq f^{+}$ and $\xi(x) \leq
f^{-}$ such that 
\begin{equation}
\int f^{+}\ d\mu - \int \varphi\ d\mu < \epsilon / 2, \qquad \text{and} \qquad \int f^{-}\ d\mu - \int \xi\ d\mu < \epsilon / 2.
\label{6.1}
\end{equation}
(This is just a consequence of the definition of Legesgue integral) 

Now set $\psi(x) = \varphi(x) + \xi(x)$. Clearly $\psi$ is a simple function and by \eqref{6.1},
\[ \int |f - \psi|\ d\mu = \int |f^{+} - f^{-} - \varphi + \xi|\ d\mu \leq \int |f^{+} - \varphi|\ d\mu + \int |f^{-} - \xi|\ d\mu < \epsilon. \]
\end{Proof}



\subsection*{7}

\begin{Proof}
Let $f := p -q$. Then 
\[ 0 = P(\Omega) - Q(\Omega) = \int f \ d\mu = \int f^{+}\ d\mu - \int f^{-}\ d\mu, \]
so 
\begin{equation}
\int f^{+}\ d\mu = \int f^{-}\ d\mu.
\label{7.1}
\end{equation}
Let $A \in \mathcal{F}$. Then 
\begin{align*}
|P(A) - Q(A)| = \left| \int_{A} f\ d\mu \right| & \leq \max\left\{ \int_{A}f^{+}\ d\mu, \int_{A}f^{-}\ d\mu \right\} \\
& \leq \max\left\{ \int f^{+}\ d\mu, \int f^{-} \ d\mu \right\} \\
\text{(by \eqref{7.1}) } & = \frac{1}{2}\left[ \int f^{+}\ d\mu + \int f^{-} \ d\mu \right] \\
& = \frac{1}{2}\int |f| \ d\mu.
\end{align*}
Now let $A' := \left\{ \omega \in \Omega : f^{+}(\omega) > 0 \right\}$. Since $f$ is measurable, $A' \in \mathcal{F}$. Following the above computation
with $A'$, we get 
\[ |P(A') - Q(A')| = \left| \int_{A'} f\ d\mu\right| = \int f^{+} \ d\mu = \max\left\{ \int f^{+}\ d\mu, \int f^{-}\ d\mu \right\} = \frac{1}{2}\int
|f|\ d\mu. \]
Therefore 
\[ \sup_{A}|P(A) - Q(A)| = \frac{1}{2}\int |f|\ d\mu. \]
\end{Proof}



\subsection*{8}










\end{document}

