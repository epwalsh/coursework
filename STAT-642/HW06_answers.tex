\documentclass[12pt]{article}
\usepackage{amsmath}
\usepackage{amsfonts}
\usepackage{parskip}
\usepackage{amsthm}
\usepackage{thmtools}
\usepackage[headheight=15pt]{geometry}
\geometry{a4paper, left=20mm, right=20mm, top=30mm, bottom=30mm}
\usepackage{graphicx}
\usepackage{bm} % for bold font in math mode - command is \bm{text}
\usepackage{enumitem}
\usepackage{fancyhdr}
\usepackage{amssymb} % for stacked arrows and other shit
\pagestyle{fancy}
\usepackage{changepage}
\usepackage{mathcomp}
\usepackage{tcolorbox}

\declaretheoremstyle[headfont=\normalfont]{normal}
\declaretheorem[style=normal]{Theorem}
\declaretheorem[style=normal]{Proposition}
\declaretheorem[style=normal]{Lemma}
\newcounter{ProofCounter}
\newcounter{ClaimCounter}[ProofCounter]
\newcounter{SubClaimCounter}[ClaimCounter]
\newenvironment{Proof}{\stepcounter{ProofCounter}\textit{Proof.}}{\hfill$\square$}
\newenvironment{claim}[1]{\vspace{1mm}\stepcounter{ClaimCounter}\par\noindent\underline{\bf Claim \theClaimCounter:}\space#1}{}
\newenvironment{claimproof}[1]{\par\noindent\underline{Proof of claim \theClaimCounter:}\space#1}{\hfill $\blacksquare$ Claim \theClaimCounter}
\newenvironment{subclaim}[1]{\stepcounter{SubClaimCounter}\par\noindent\emph{Subclaim \theClaimCounter.\theSubClaimCounter:}\space#1}{}
% \newenvironment{subclaimproof}[1]{\begin{adjustwidth}{2em}{0pt}\par\noindent\emph{Proof of subclaim \theClaimCounter.\theSubClaimCounter:}\space#1}{\hfill
% $\blacksquare$ \emph{Subclaim \theClaimCounter.\theSubClaimCounter}\vspace{5mm}\end{adjustwidth}}
\newenvironment{subclaimproof}[1]{\par\noindent\emph{Proof of subclaim \theClaimCounter.\theSubClaimCounter:}\space#1}{\hfill
$\Diamond$ \emph{Subclaim \theClaimCounter.\theSubClaimCounter}}

\allowdisplaybreaks

\title{STAT 642: HW 6}
\author{Evan P. Walsh}
\makeatletter
\let\runauthor\@author
\let\runtitle\@title
\makeatother
\lhead{\runauthor}
\chead{\runtitle}
\rhead{\thepage}
\cfoot{}

\begin{document}
% \maketitle


\subsection*{1 [AL 9.1]}
\begin{tcolorbox}
If $X_{n}\rightarrow^{d}X_{0}$ and $P(X_0 = c) = 1$ for some $c \in \mathbb{R}$, then $X_n \rightarrow_p c$.
\end{tcolorbox}
\begin{Proof}
Since $X_0$ is degenerate, the cdf of $X_0$, $F_{X_{0}}$, is only discontinuous at $c$. Therefore $F_{X_{n}}(x) \rightarrow F_{X_{0}}(x)$ for all $x \neq c$, where
$F_{X_{n}}$ is the cdf of $X_n$ for all $n \geq 1$. Let $\epsilon > 0$. Then 
\[ \lim_{n\rightarrow\infty}F_{X_{n}}(c + \epsilon) = F_{X_{0}}(c+\epsilon) = 1, \qquad \lim_{n\rightarrow\infty}F_{X_{n}}(x - \epsilon) = F_{X_{0}}(x
- \epsilon) = 0. \]
Thus,
\begin{align*}
& \lim_{n\rightarrow\infty}P(X_{n} > c + \epsilon) = 1 - \lim_{n\rightarrow\infty}P(X_{n} \leq c + \epsilon) = 0, \text{ and} \\
& \lim_{n\rightarrow\infty}P(X_{n} < c - \epsilon) \leq \lim_{n\rightarrow\infty}P(X_{n} \leq c - \epsilon) = 0.
\end{align*}
Hence 
\[ \lim_{n\rightarrow\infty}P(|X_{n} - X_{0}| > \epsilon) = \lim_{n\rightarrow\infty}P(|X_{n} - c| > \epsilon) = 0. \]
\end{Proof}



\subsection*{2 [AL 9.11]}
\begin{tcolorbox}
For any cdf $F$, let $F^{-1}(p) := \inf\left\{ x \in \mathbb{R} : F(x) \geq p \right\}$, for $p \in (0,1)$. Show that for any $0 < p_{0} < 1$ and
$t_{0} \in \mathbb{R}$,
\[ F^{-1}(p_{0}) \leq t_{0} \Leftrightarrow F(t_{0}) \geq p_{0}. \]
\end{tcolorbox}
\begin{Proof}
Suppose $p_{0} \in (0,1)$ and $t_{0} \in \mathbb{R}$. 

$(\Leftarrow)$ First suppose $F(t_{0}) \geq p_{0}$. Since $F$ is non-decreasing, 
\[ t_{0} \geq \inf\left\{ x \in \mathbb{R} : F(x) \geq p_{0} \right\}  = F^{-1}(p_{0}). \] 
$(\Rightarrow)$ On the other hand, if $t_{0} \geq F^{-1}(p_{0})$ then either $t_{0} = F^{-1}(p_{0})$ or $t_{0} > F^{-1}(p_{0})$. The latter case is
trivial since $F$ is non-decreasing. For the former, suppose $t_{1}, t_{2}, \hdots \in \mathbb{R}$ such that $t_{n} > t_{0}$ for all $n \geq 1$ and $t_{n} \rightarrow t_{0}$.
Clearly $F(t_{n}) \geq p_{0}$ since $t_{n} > F^{-1}(p_{0})$ for each $n \geq 1$. Thus,
by the right-continuity of $F$, $F(t_{0}) = \lim_{n\rightarrow\infty}F(t_{n}) \geq p_{0}$.
\end{Proof}



\newpage
\subsection*{3}
\begin{tcolorbox}
Let $F$ be a cdf. Show that the set of points where is discontinuous is countable.
\end{tcolorbox}
\begin{Proof}
Since $F$ is non-decreasing, the only type of discontinuities that $F$ can have is jump discontinuities. That is, the discontinuous points are 
\[ D := \left\{ x \in \mathbb{R} : \lim_{y\uparrow x}F(y) < F(x) \right\} = \bigcup_{n=0}^{\infty}\underbrace{\left\{ x \in \mathbb{R} : \lim_{y\uparrow x}F(y) <
F(x) - 2^{-n} \right\}}_{D_{n}}. \]
\begin{claim}
Each set $D_{n}$ is finite.
\end{claim}
\begin{claimproof}
By way of contradiction, suppose there exists some $n_{0} \in \mathbb{N}$ such that $D_{n_{0}}$ is infinite. Without loss of generality we can assume
$D_{n_0}$ is countably infinite, otherwise we could just consider a countably infinite subset of $D_{n_0}$. Suppose $D_{n_0} = \left\{ x_k
\right\}_{k=0}^{\infty}$, where $x_k < x_{k+1}$ for each $k \in \mathbb{N}$. By definition of $D_{n_0}$ and since $F$ is non-decreasing, 
\[ F(x_k) + 2^{-n_0} < F(x_{k+1}). \]
But then $\lim_{k\rightarrow\infty}F(x_{k}) = \infty$, a contradiction.
\end{claimproof}

By claim 1 and since the countable union of finite sets is countable, $D$ is countable.
\end{Proof}



\subsection*{4}
\begin{tcolorbox}
Show that a distribution function is uniquely determined by its values on any dense set.
\end{tcolorbox}
\begin{Proof}
Let $F$ and $G$ be two distribution functions such that $F \equiv G$ and some dense set $D \subseteq \mathbb{R}$. We need to show that $F \equiv G$ on
$D^{c}$. Well, let $x \in D^{c}$. Since $D$ is dense in $\mathbb{R}$, we can choose a sequence $\left\{ x_{n} \right\}_{n=0}^{\infty}$ such that
$x < x_{n} \in D$ for all $n \in \mathbb{N}$ and so that $x_{n} \rightarrow x$. By the right-continuity of $F$ and $G$ and since $F(x_{n}) = G(x_{n})$
for all $n \in \mathbb{N}$,
\[ F(x) = \lim_{n\rightarrow\infty}F(x_{n}) = \lim_{n\rightarrow\infty}G(x_{n}) = G(x). \]
Thus $F \equiv G$.
\end{Proof}


\subsection*{5}
\begin{tcolorbox}
Show that if $F_{n} \Rightarrow F$ and $F_{n}\Rightarrow G$, then $F \equiv G$.
\end{tcolorbox}
\begin{Proof}
By the result in question 3, $C(F)^{c}$ and $C(G)^{c}$ are countable. So $C(F)^{c} \cup C(G)^{c}$ is countable. Thus, $A := [C(F)^{c}\cap
C(F)^{c}]^{c} = C(F) \cap C(G)$ is dense in
$\mathbb{R}$ and $F_{n}(x) \rightarrow F(x)$, $F_{n}(x) \rightarrow G(x)$ for every $x \in A$. So $F \equiv G$ on $A$. By the result in question 4, $F
\equiv G$.
\end{Proof}



\end{document}

