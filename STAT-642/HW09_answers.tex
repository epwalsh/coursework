\documentclass[12pt]{article}
\usepackage{amsmath}
\usepackage{amsfonts}
\usepackage{parskip}
\usepackage{amsthm}
\usepackage{thmtools}
\usepackage[headheight=15pt]{geometry}
\geometry{a4paper, left=20mm, right=20mm, top=30mm, bottom=30mm}
\usepackage{graphicx}
\usepackage{bm} % for bold font in math mode - command is \bm{text}
\usepackage{enumitem}
\usepackage{fancyhdr}
\usepackage{amssymb} % for stacked arrows and other shit
\pagestyle{fancy}
\usepackage{changepage}
\usepackage{mathcomp}
\usepackage{tcolorbox}

\declaretheoremstyle[headfont=\normalfont]{normal}
\declaretheorem[style=normal]{Theorem}
\declaretheorem[style=normal]{Proposition}
\declaretheorem[style=normal]{Lemma}
\newcounter{ProofCounter}
\newcounter{ClaimCounter}[ProofCounter]
\newcounter{SubClaimCounter}[ClaimCounter]
\newenvironment{Proof}{\stepcounter{ProofCounter}\textsc{Proof.}}{\hfill$\square$}
\newenvironment{claim}[1]{\vspace{1mm}\stepcounter{ClaimCounter}\par\noindent\underline{\bf Claim \theClaimCounter:}\space#1}{}
\newenvironment{claimproof}[1]{\par\noindent\underline{Proof of claim \theClaimCounter:}\space#1}{\hfill $\blacksquare$ Claim \theClaimCounter}
\newenvironment{subclaim}[1]{\stepcounter{SubClaimCounter}\par\noindent\emph{Subclaim \theClaimCounter.\theSubClaimCounter:}\space#1}{}
% \newenvironment{subclaimproof}[1]{\begin{adjustwidth}{2em}{0pt}\par\noindent\emph{Proof of subclaim \theClaimCounter.\theSubClaimCounter:}\space#1}{\hfill
% $\blacksquare$ \emph{Subclaim \theClaimCounter.\theSubClaimCounter}\vspace{5mm}\end{adjustwidth}}
\newenvironment{subclaimproof}[1]{\par\noindent\emph{Proof of subclaim \theClaimCounter.\theSubClaimCounter:}\space#1}{\hfill
$\Diamond$ \emph{Subclaim \theClaimCounter.\theSubClaimCounter}}

\allowdisplaybreaks{}

% chktex-file 3

\title{STAT 642: HW 9}
\author{Evan P. Walsh}
\makeatletter
\makeatother
\lhead{Evan P. Walsh}
\chead{STAT 642: HW 9}
\rhead{\thepage}
\cfoot{}

\begin{document}
\maketitle


\subsection*{1 [AL 10.10]}
\begin{tcolorbox}
  Let $\mu$ be a probability measure on $\mathbb{R}$ with characteristic function $\phi$. Prove that
  \[ \frac{1}{\pi} \int_{-\infty}^{\infty}[1 - \text{Re}(\phi(t))]t^{-2}dt = \int_{-\infty}^{\infty}|x|d\mu(x). \]
\end{tcolorbox}
\begin{Proof}
  By Tonelli's Theorem,
  \[
    \frac{1}{\pi}\int_{-\infty}^{\infty}[1 - \text{Re}(\phi(t))]t^{-2}dt = \frac{1}{\pi}\int_{-\infty}^{\infty}\int_{-\infty}^{\infty}\frac{1 -
    \cos(|xt|)}{t^{2}}d\mu(x) dt = \frac{1}{\pi}\int_{-\infty}^{\infty}\int_{-\infty}^{\infty}\frac{1 - \cos(|xt|)}{t^{2}}dt d\mu(x)
  \]
  Thus, it suffices to show that
  \begin{equation}
    \int_{-\infty}^{\infty}\frac{1 - \cos(|xt|)}{t^{2}}dt = \pi|x|, \qquad \text{i.e.} \qquad \int_{0}^{\infty}\frac{1 - \cos(|x|t)}{t^{2}}dt =
    \frac{\pi}{2}|x|.
    \label{1.2}
  \end{equation}
  We do integration by parts with $u := 1 - \cos(|x|t)$ and $dv := t^{-2}$. Thus,
  \[ \int_{0}^{\infty}\frac{1 - \cos(|x|t)}{t^{2}}dt = \lim_{T\rightarrow\infty}\int_{0}^{T}\frac{1 - \cos(|x|t)}{t^{2}}dt =
  \lim_{T\rightarrow\infty}\frac{\cos(|x|t) - 1}{t}\bigg|_{0}^{T} + |x|\int_{0}^{T}\frac{\sin(|x|t)}{t}dt. \]
  Now, $\lim_{t \rightarrow 0}\frac{\cos(|x|t) - 1}{t} = \lim_{t\rightarrow 0}-|x|\sin(|x|t) = 0$ by L'Hopitals, so
  \[
    \lim_{T\rightarrow\infty}\left|\frac{\cos(|x|t) - 1}{t}\bigg|_{0}^{T}\right| = \lim_{T\rightarrow\infty}\left|\frac{\cos(|x|T) - 1}{T}\right| 
    \leq \lim_{T\rightarrow\infty}\frac{2}{T} = 0,
  \]
  and by Equation 4.7 in Chapter 10.4 of AL,
  \[ \lim_{T\rightarrow\infty}|x|\int_{0}^{T}\frac{\sin(|x|t)}{t}dt = \frac{\pi}{2}|x|. \]
  Hence~\eqref{1.2} is established.
\end{Proof}


\newpage
\subsection*{2 [AL 10.11]}
\begin{tcolorbox}
  Let $\phi$ be the characteristic function of a random variable $X$. If $|\phi(t)| = 1 = |\phi(\alpha t)|$ for some $t \neq 0$ and $\alpha \in
  \mathbb{R}$ irrational, then there exists $x_0 \in \mathbb{R}$ such that $P(X = x_0) = 1$.
\end{tcolorbox}
\begin{Proof}
  By the proof of Proposition 10.1.1 [AL], there exists $a_0, a_1$ such that
  \[
    P\left(X \in \left\{ a_0 + \frac{2\pi k}{t} : k \in \mathbb{Z} \right\}\right) = 1 \qquad \text{ and } \qquad P\left(X \in \left\{ a_1 + \frac{2\pi k}{t\alpha} :
    k \in \mathbb{Z}\right\}\right) = 1.
  \]
  Now, by way of contradiction suppose that there exists at least two integers $i_0 \neq j_0$ such that $P(X = a_0 + \frac{2\pi i_0}{t}), P(X = a_0 +
  \frac{2\pi j_0}{t}) > 0$. Then there exists $i_1, j_1 \in \mathbb{Z}$ such that
  \[
    a_0 + \frac{2\pi i_0}{t} = a_1 + \frac{2\pi i_1}{t\alpha} \qquad \text{ and } \qquad a_0 + \frac{2\pi j_0}{t} = a_1 + \frac{2\pi j_1}{t\alpha}.
  \]
  But then
  \[
    \frac{2\pi}{t}(i_0 - j_0) = \frac{2\pi}{t\alpha}(i_1 - j_1),
  \]
  which implies $\alpha = (i_1 - j_1) / (i_0 - j_0) \in \mathbb{Q}$, a contradiction.
\end{Proof}



\newpage
\subsection*{3 [AL 10.18]}
\begin{tcolorbox}
  Let $\phi(\cdot)$ be a characteristic function on $\mathbb{R}$ such that $\phi(t) \rightarrow 0$ as $|t| \rightarrow \infty$. Let $X$ be a random
  variable with $\phi$ as its characteristic function. For each $n \geq 1$, let $X_n = \frac{k}{n}$ if $\frac{k}{n} \leq X \leq \frac{k+1}{n}$, for $k
  \in \mathbb{Z}$. Show that if $\phi_n(t) := Ee^{itX_n}$, then $\phi_n(t) \rightarrow \phi(t)$ for all $t \in \mathbb{R}$ but for each $n \geq 1$,
  \[ \sup\left\{ |\phi_n(t) - \phi(t)| : t \in \mathbb{R} \right\} \geq 1. \]
\end{tcolorbox}
\begin{Proof}
  \begin{claim}
    $X_n \rightarrow X$ almost surely.
  \end{claim}
  \begin{claimproof}
    We can write $X_n = \lfloor n X\rfloor / n$. Thus, $\frac{nX - 1}{n} \leq X_n \leq \frac{nX}{n} = X$, so $\limsup X_n \leq X$ and $\liminf X_n
    \geq X$. Hence $X_n \rightarrow X$.
  \end{claimproof}

  By Claim 1, $X_n \rightarrow^{d} X$ since almost sure convergence implies convergence in distribution, and so $\phi_n(t) \rightarrow \phi(t)$ for
  all $t \in \mathbb{R}$ by the Levy continuity theorem.

  \begin{claim}
    For each $n \geq 1$, $\limsup_{|t|\rightarrow\infty}|\phi_n(t)| \geq 1$.
  \end{claim}
  \begin{claimproof}
    Let $n \geq 1$. Note that $P\left(X \in \left\{ \frac{k}{n} : k \in \mathbb{Z} \right\}\right) = 1$. Thus, $e^{i2\pi nmX_n} = 1$ with probability 1,
    for any $m \in \mathbb{Z}$. Hence,
    \[ \limsup_{|t|\rightarrow\infty} |\phi_n(t)| \geq \limsup_{|m|\rightarrow\infty, m \in \mathbb{Z}}|\phi_n(2\pi n m)| = 1. \]
  \end{claimproof}

  By Claim 2,
  \[
    \sup_{t\in\mathbb{R}}|\phi_n(t) - \phi(t)| \geq \limsup_{|t|\rightarrow\infty}|\phi_n(t) - \phi(t)| \geq \limsup_{|t|\rightarrow\infty}\big|
    |\phi_n(t)| - |\phi(t)|\big| \geq 1.
  \]

\end{Proof}


\newpage
\subsection*{4 [AL 10.19 (b)]}
\begin{tcolorbox}
  Let $\left\{ \delta_i \right\}_{i\geq 1}$ be iid random variables with distribution
  \[ P(\delta_1 = 1) = P(\delta_1 = -1) = 1/2. \]
  Let $X_n = \sum_{i=1}^{n}\delta_i / 2^{i}$ and $X = \lim_{n\rightarrow\infty}X_n$. Show that the characteristic function of $X$ is $\phi_X(t) :=
  \frac{\sin t}{t}$ and that $X$ must be Uniform (-1, 1).
\end{tcolorbox}
\begin{Proof}
  \begin{claim}
    $X_n \rightarrow X$ almost surely.
  \end{claim}
  \begin{claimproof}
    Note that $E\delta_i = 0$ for all $i \geq 1$, so $E\left(\frac{\delta_i}{2^{i}}\right) = 0$ as well. Further,
    \[ \sum_{i=1}^{\infty}E\left( \frac{\delta_i}{2^{i}} \right)^{2} = \sum_{i=1}^{\infty}\frac{1}{2^{2i}} < \infty, \]
    so by independence and the 1-series theorem, $X_n$ converges almost surely (to $X$).

  \end{claimproof}

  By the hint given,
  \[
    \phi_n(t) \rightarrow \phi(t) := \left\{ \begin{array}{cl}
        \frac{\sin t}{t} & \text{ if } t \neq 0 \\
        \\
        1 & \text{ if } t = 0.
    \end{array} \right.
  \]
  By Claim 1 and the Levy continuity theorem, $X$ has characteristic function $\phi$. But for a Uniform (-1, 1) random variable $Y$,
  \[
    Ee^{itY} = \int_{-1}^{1}e^{ity}\cdot \frac{1}{2} dy = \left\{ \begin{array}{cl}
        \frac{e^{ity}}{2it}\bigg|_{-1}^{1} = \frac{\cos(t) + i\sin(t) - \cos(-t) - i\sin(-t)}{2it} = \frac{\sin t}{t} & \ t \neq 0 \\
        \\
        1 & \ t = 0.
    \end{array} \right.
  \]
  By the uniqueness of characteristic functions, $X$ has a Uniform (-1,1) distribution.
\end{Proof}

\end{document}
