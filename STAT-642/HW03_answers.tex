\documentclass[12pt]{article}
\usepackage{amsmath}
\usepackage{amsfonts}
\usepackage{parskip}
\usepackage{amsthm}
\usepackage{thmtools}
\usepackage[headheight=15pt]{geometry}
\geometry{a4paper, left=20mm, right=20mm, top=30mm, bottom=30mm}
\usepackage{graphicx}
\usepackage{bm} % for bold font in math mode - command is \bm{text}
\usepackage{enumitem}
\usepackage{fancyhdr}
\usepackage{amssymb} % for stacked arrows and other shit
\pagestyle{fancy}
\usepackage{changepage}
\usepackage{mathcomp}
\usepackage{tcolorbox}

\declaretheoremstyle[headfont=\normalfont]{normal}
\declaretheorem[style=normal]{Theorem}
\declaretheorem[style=normal]{Proposition}
\declaretheorem[style=normal]{Lemma}
\newcounter{ProofCounter}
\newcounter{ClaimCounter}[ProofCounter]
\newcounter{SubClaimCounter}[ClaimCounter]
\newenvironment{Proof}{\stepcounter{ProofCounter}\textsc{Proof.}}{\hfill$\square$}
\newenvironment{claim}[1]{\vspace{1mm}\stepcounter{ClaimCounter}\par\noindent\underline{\bf Claim \theClaimCounter:}\space#1}{}
\newenvironment{claimproof}[1]{\par\noindent\underline{Proof of claim \theClaimCounter:}\space#1}{\hfill $\blacksquare$ Claim \theClaimCounter}
\newenvironment{subclaim}[1]{\stepcounter{SubClaimCounter}\par\noindent\emph{Subclaim \theClaimCounter.\theSubClaimCounter:}\space#1}{}
% \newenvironment{subclaimproof}[1]{\begin{adjustwidth}{2em}{0pt}\par\noindent\emph{Proof of subclaim \theClaimCounter.\theSubClaimCounter:}\space#1}{\hfill
% $\blacksquare$ \emph{Subclaim \theClaimCounter.\theSubClaimCounter}\vspace{5mm}\end{adjustwidth}}
\newenvironment{subclaimproof}[1]{\par\noindent\emph{Proof of subclaim \theClaimCounter.\theSubClaimCounter:}\space#1}{\hfill
$\Diamond$ \emph{Subclaim \theClaimCounter.\theSubClaimCounter}}

\title{STAT 642: HW 3}
\author{Evan P. Walsh}
\makeatletter
\let\runauthor\@author
\let\runtitle\@title
\makeatother
\lhead{\runauthor}
\chead{\runtitle}
\rhead{\thepage}
\cfoot{}

\begin{document}
\maketitle

\subsection*{1 [AL 7.6 (a)]}
\begin{tcolorbox}
Let $X_{1}$ and $X_{2}$ be independent random variables. Show that for any $p > 0$,
\[ E|X_{1} + X_{2}|^{p} < \infty \qquad \text{if and only if} \qquad E|X_{1}| < \infty,\  E|X_{2}| < \infty. \]
\end{tcolorbox}
\begin{Proof}
Since $X_{1}$ and $X_{2}$ are independent, $P_{X_{1},X_{2}} = P_{X}\times P_{Y}$. The backwards direction is easy. That is, if we assume that
$E|X_{1}|^{p} < \infty$ and $E|X_{2}|^p < \infty$, then $E|X_{1} + EX_{2}|^{p} < \infty$ by Proposition 3.2.1 [AL]. Now suppose $E|X_{1} + X_{2}|^{p}
< \infty$. By Tonelli's Theorem,
\[ E|X_{1} + X_{2}|^{p} = \int \int |x_{1} + x_{2}|^{p}\ dP_{X_{1}}(x_{1})\ dP_{X_{2}}(x_{2}) = \int E|X_{1} + x_{2}|^{p} \ dP_{X_{2}}(x_{2}) < \infty. \]
Therefore there exists some $x_{2} \in \mathbb{R}$ such that $E|X_{1} + x_{2}|^{p} < \infty$. But then $E|X_{1}|^{p} < \infty$ since $X_{1} = 
X_{1} + x_{2} - Y$, where $Y \equiv x_{2} \in L^{p}$. By the symmetry of the
above argument, $E|X_{2}|^{p} < \infty$ as well.
\end{Proof}

For an example of how the above result fails when $X_{1}$ and $X_{2}$ are not independent, consider $X_{1} \sim $Cauchy(0,1) and $X_{2} = -X_{1}$. Then
$E|X_{1} - X_{2}| = 0$ but $E|X_{1}| = E|X_{2}| = \infty$.


\subsection*{2 [AL 7.11]}
\begin{tcolorbox}
For any nonnegative random variable $X$, show that $E|X| < \infty$ if and only if $\sum_{n=1}^{\infty}P(|X| > \epsilon n) < \infty$ for every
$\epsilon > 0$.
\end{tcolorbox}
\begin{Proof}
By the hint given
\begin{equation}
E|X| = \sum_{k=0}^{\infty}\int_{k\epsilon}^{(k+1)\epsilon}P(|X| > t)\ dt.
\label{2.1}
\end{equation}
$(\Rightarrow)$ First assume that $E|X| < \infty$. Then using \eqref{2.1},
\[ E|X| = \sum_{k=0}^{\infty}\int_{k\epsilon}^{(k+1)\epsilon}P(|X| > t)\ dt \geq \sum_{k=0}^{\infty}\int_{k\epsilon}^{(k+1)\epsilon}P(|X| >
(k+1)\epsilon)\ dt = \epsilon\sum_{n=1}^{\infty}P(|X| > n\epsilon). \]
$(\Leftarrow)$ Similarly, if we assume $\sum_{n=1}^{\infty}P(|X| > n\epsilon) < \infty$, then 
\[ E|X| = \sum_{n=0}^{\infty}\int_{n\epsilon}^{(n+1)\epsilon}P(|X| > t)\ dt \leq \sum_{n=0}^{\infty}\int_{n\epsilon}^{(n+1)\epsilon}P(|X| >
n\epsilon)\ dt \leq \epsilon + \epsilon\sum_{n=1}^{\infty}P(|X| > n\epsilon) < \infty. \]
\end{Proof}


\subsection*{3 [AL 7.12 (a)]}
\begin{tcolorbox}
Let $\left\{ X_{i} \right\}_{i\geq 1}$ be a sequence of independent and identically distributed random variables. Show that
$\lim_{n\rightarrow\infty}\frac{X_{n}}{n} = 0$ with probability 1 if and only if $E|X_{1}| < \infty$.
\end{tcolorbox}
\begin{Proof}
By question 2, $E|X_{1}| < \infty$ if and only if, for each $\epsilon > 0$,
\begin{equation}
\sum_{n=1}^{\infty}P(|X_{1}| > n\epsilon) = \sum_{n=1}^{\infty}P(|X_{n}| > n\epsilon) < \infty. 
\label{3.1}
\end{equation}
But by the Borel 0-1 Law, \eqref{3.1} holds if and only if $P(\{\omega : |X_{n}| > n\epsilon \text{ i.o.}\}) = 0$ for every $\epsilon > 0$, 
if and only if $P(\left\{ \omega : |X_{n}| \leq n\epsilon \text{ eventually} \right\}) = 1$ for every $\epsilon > 0$,
if and only if 
\[ P\left(\lim_{n\rightarrow\infty}\frac{|X_{n}|}{n} \leq \epsilon\right) = 1 \ \forall \ \epsilon > 0
\qquad \text{if and only if} \qquad P\left(\lim_{n\rightarrow\infty}\frac{X_{n}}{n} = 0\right) = 1. \]
\end{Proof}


\subsection*{4 [AL 7.13]}
\begin{tcolorbox}
Let $\left\{ X_{i} \right\}_{i\geq 1}$ be sequence of identically distributed random variables and let $M_{n} := \max\left\{ |X_{j}| : 1 \leq j \leq n
\right\}$.
\begin{enumerate}[label=(\alph*)]
\item If $E|X_{1}|^{\alpha} < \infty$ for some $\alpha \in (0,\infty)$, then show that 
\begin{equation}
\frac{M_{n}}{n^{1/\alpha}} \rightarrow 0 \text{ with probability 1}. 
\label{4.1}
\end{equation}
\item Show that if $\left\{ X_{i} \right\}_{i\geq 1}$ are iid satisfying \eqref{4.1} for some $\alpha > 0$, then $E|X_{1}|^{\alpha} < \infty$.
\end{enumerate}
\end{tcolorbox}

\begin{enumerate}[label=(\alph*)]
\item \begin{Proof}
Suppose $E|X_{1}|^{\alpha} < \infty$, where $\alpha > 0$. Let $\epsilon > 0$. By question 2,
\begin{equation}
\infty > \sum_{n=1}^{\infty}P(|X_{1}|^{\alpha} > \epsilon^{\alpha}n) = \sum_{n=1}^{\infty}P(|X_{1}| > \epsilon n^{1/\alpha}). 
\label{4.2}
\end{equation}
Therefore $P(|X_{n}| \leq \epsilon n^{1/\alpha} \text{ eventually}) = 1$. Hence there exists some $k \geq 1$ such that $|X_{n}| \leq \epsilon
n^{1/\alpha}$ with probability 1 whenever $n \geq k$. Thus, for any $n \geq k$,
\begin{equation}
P(M_{n} > \epsilon n^{1/\alpha}) = P\left( \cup_{i=1}^{k}|X_{i}| > \epsilon n^{1/\alpha} \right) \leq \sum_{i=1}^{k}P(|X_{1}| > \epsilon
n^{1/\alpha}) = kP(|X_{1}| > \epsilon n^{1/\alpha}). 
\label{4.3}
\end{equation}
Also, for $n < k$, 
\begin{equation}
P(M_{n} > \epsilon n^{1/\alpha}) \leq \sum_{i=1}^{n}P(|X_{1}| > \epsilon n^{1/\alpha}) \leq \sum_{i=1}^{n}P(|X_{1}| > \epsilon i^{1/\alpha}).
\label{4.4}
\end{equation}
Therefore 
\begin{align*}
\sum_{n=1}^{\infty}P(M_{n} > \epsilon n^{1/\alpha}) & = \sum_{n=1}^{k-1}P(M_{n} > \epsilon n^{1/\alpha}) + \sum_{n=k}^{\infty}P(M_{n} > \epsilon
n^{1/\alpha})  \\
\text{by \eqref{4.3} and \eqref{4.4} }\ & \leq \sum_{n=1}^{k-1}\sum_{i=1}^{n}P(|X_{1}| > \epsilon i^{1/\alpha}) + \sum_{n=k}^{\infty}\sum_{i=1}^{k}P(|X_{1}| > \epsilon n^{1/\alpha}) \\
& = \sum_{n=1}^{k-1}\sum_{i=1}^{n}P(|X_{1}| > \epsilon i^{1/\alpha}) + k\sum_{n=k}^{\infty}P(|X_{1}| > \epsilon n^{1/\alpha}) \\
\text{by \eqref{4.2} }\ & < \infty.
\end{align*}
\end{Proof}

\item \begin{Proof}
Let $\epsilon > 0$. Note that $P\left( \dfrac{M_{n}}{n^{\alpha}} < \epsilon \right) \leq P\left( \dfrac{|X_{n}|}{n^{\alpha}} < \epsilon \right)$ for
all $n \geq 1$. 
Since \eqref{4.1} holds,
\[ 1 = P\left( \lim_{n\rightarrow\infty}\frac{M_{n}}{n^{\alpha}} < \epsilon \right) \leq P\left( \lim_{n\rightarrow\infty}|X_{n}| < n^{1/\alpha}\epsilon
\right). \]
Now, by the Borel 0-1 Law, $\sum_{n=1}^{\infty}P(|X_{n}| \geq n^{1/\alpha}\epsilon) < \infty$. Thus,
\begin{align*}
\infty > \epsilon + \epsilon\sum_{n=1}^{\infty}P\left( |X_{n}| \geq n^{1/\alpha}\epsilon \right) & = \epsilon + \epsilon\sum_{n=1}^{\infty}P(|X_{n}|^{\alpha} 
\geq n\epsilon^{1/\alpha}) \\
& \geq \sum_{n=0}^{\infty}\int_{n\epsilon^{1/\alpha}}^{(n+1)\epsilon^{1/\alpha}}P(|X_{n}|^{\alpha} \geq n\epsilon^{1/\alpha})\ dt \\
& \geq \sum_{n=0}^{\infty}\int_{n\epsilon^{1/\alpha}}^{(n+1)\epsilon^{1/\alpha}}P(|X_{n}|^{\alpha} \geq t)\ dt  \\
& = \sum_{n=0}^{\infty}\int_{n\epsilon^{1/\alpha}}^{(n+1)\epsilon^{1/\alpha}}P(|X_{1}|^{\alpha} \geq t)\ dt  \\
& = E|X_{1}|^{\alpha}.
\end{align*}
\end{Proof}
\end{enumerate}

\newpage
\subsection*{5}
\begin{tcolorbox}
Let $p_{n} := 1 - 2^{-n}$ for $n \geq 1$. Let $\left\{ X_{n} : n\geq 1 \right\}$ be random variables such that for each $n \geq 1$,
\[ P(X_{n} = j) = (1 - p_{n})^{j-1}p_{n}, \ \ j = 1, 2, \hdots \]
\begin{enumerate}[label=(\alph*)]
\item Prove that $P(X_{n} = 1 \text{ eventually}) = 1$.
\item Prove that $P(\lim_{n\rightarrow\infty} X_{n} = 1) = 1$.
\end{enumerate}
\end{tcolorbox}
\begin{enumerate}[label=(\alph*)]
\item Let $A_{n} := \left\{ \omega : X_{n} > 1 \right\}$. Then 
$P\left( \left\{ \omega : X_{n} = 1 \text{ eventually}\right\}^{c} \right) = P\left( A_{n} \text{ i.o.} \right)$. Now,
\[ P(A_{n}) = \sum_{j=2}^{\infty}(2^{-n})^{j-1}(1-2^{-n}) = (1-2^{-n})\sum_{i=1}^{\infty}\left( \frac{1}{2^{n}} \right)^{i} = 2^{-n}. \]
Thus $\sum_{n=1}^{\infty}P(A_{n}) = \frac{1}{2} < \infty$. So by the Borel 0-1 Law, $P(A_{n}\text{ i.o.}) = 0$, and thus 
$P(A_{n}^{c}\text{ eventually}) = P(X_{n} = 1 \text{ eventually}) = 1$.

\item Note that $P(\lim_{n\rightarrow\infty} X_{n} = 1) = 1$ if and only if for all $\epsilon > 0$, $P(\lim_{n\rightarrow\infty}|X_{n} - 1| > \epsilon) =
0$. So let $\epsilon > 0$. Without loss of generality assume $\epsilon < 1$. Then $\left\{ \omega : |X_{n} - 1| > \epsilon \right\} = \left\{ \omega : 
X_{n} > 1\right\} = A_{n}$, where $A_{n}$ is defined as in part (a). But by part (a), $P(A_{n} \text{ i.o.}) = 0$. Therefore 
\[1 =  P(A_{n}^{c}\text{ eventually}) = P(|X_{n} - 1 | \leq \epsilon \text{ eventually}) = P(X_{n} = 1 \text{ eventually}). \]
So $P(\lim_{n\rightarrow\infty}X_{n} = 1) = 1$.
\end{enumerate}

\end{document}

