\documentclass[12pt]{article}
\usepackage{amsmath}
\usepackage{amsfonts}
\usepackage{parskip}
\usepackage{amsthm}
\usepackage{thmtools}
\usepackage[headheight=15pt]{geometry}
\geometry{a4paper, left=20mm, right=20mm, top=30mm, bottom=30mm}
\usepackage{graphicx}
\usepackage{bm} % for bold font in math mode - command is \bm{text}
\usepackage{enumitem}
\usepackage{fancyhdr}
\usepackage{amssymb} % for stacked arrows and other shit
\pagestyle{fancy}
\usepackage{changepage}
\usepackage{mathcomp}
\usepackage{tcolorbox}

\declaretheoremstyle[headfont=\normalfont]{normal}
\declaretheorem[style=normal]{Theorem}
\declaretheorem[style=normal]{Proposition}
\declaretheorem[style=normal]{Lemma}
\newcounter{ProofCounter}
\newcounter{ClaimCounter}[ProofCounter]
\newcounter{SubClaimCounter}[ClaimCounter]
\newenvironment{Proof}{\stepcounter{ProofCounter}\textsc{Proof.}}{\hfill$\square$}
\newenvironment{claim}[1]{\vspace{1mm}\stepcounter{ClaimCounter}\par\noindent\underline{\bf Claim \theClaimCounter:}\space#1}{}
\newenvironment{claimproof}[1]{\par\noindent\underline{Proof of claim \theClaimCounter:}\space#1}{\hfill $\blacksquare$ Claim \theClaimCounter}
\newenvironment{subclaim}[1]{\stepcounter{SubClaimCounter}\par\noindent\emph{Subclaim \theClaimCounter.\theSubClaimCounter:}\space#1}{}
% \newenvironment{subclaimproof}[1]{\begin{adjustwidth}{2em}{0pt}\par\noindent\emph{Proof of subclaim \theClaimCounter.\theSubClaimCounter:}\space#1}{\hfill
% $\blacksquare$ \emph{Subclaim \theClaimCounter.\theSubClaimCounter}\vspace{5mm}\end{adjustwidth}}
\newenvironment{subclaimproof}[1]{\par\noindent\emph{Proof of subclaim \theClaimCounter.\theSubClaimCounter:}\space#1}{\hfill
$\Diamond$ \emph{Subclaim \theClaimCounter.\theSubClaimCounter}}

\allowdisplaybreaks

\title{STAT 642: HW 8}
\author{Evan P. Walsh}
\makeatletter
\let\runauthor\@author
\let\runtitle\@title
\makeatother
\lhead{\runauthor}
\chead{\runtitle}
\rhead{\thepage}
\cfoot{}

\begin{document}
% \maketitle

\subsection*{1 [AL 9.22]}
\begin{tcolorbox}
Let $\mu_n$, $\mu$ be probability measures on a countable set $D := \left\{ a_j \right\}_{j\geq 1}$, where $a_j \in \mathbb{R}$ for all $j \geq 1$.
Let $p_{n,j} := \mu_{n}\left( \left\{ a_j \right\} \right)$ for all $n,j\geq 1$ and $p_j := \mu\left( \left\{ a_j \right\} \right)$. Show that, as $n
\rightarrow \infty$, 
\begin{enumerate}[label=(\roman*)]
\item $\sup_{x\in\mathbb{R}}|F_n(x) - F(x)| \rightarrow 0$ if and only if 
\item $p_{n,j} \rightarrow p_{j}$ for each $j \geq 1$ if and only if 
\item $\sum_{j=1}^{\infty}|p_{n,j} - p_j| \rightarrow 0$.
\end{enumerate}
\end{tcolorbox}

\begin{Proof}
{\bf (i) $\Rightarrow$ (ii)} Suppose $\sup_{x\in\mathbb{R}}|F_{n}(x) - F(x)| \rightarrow 0$ as $n \rightarrow \infty$.
Without loss of generality suppose $a_j < a_{j+1}$ for all $j \geq 1$. Let $j_0 \geq 1$. Then there exists $x_{j_0} \in
\mathbb{R}$ such that $x_{j_0} < a_{j_0}$ and $j_0 = \min\left\{ j \geq 1 : a_J \geq x_{j_0} \right\}$. Then $p_{n,j_0} = F_n(a_{j_0}) - F_n(x_{j_0})$
and $p_j = F(a_{j_0}) - F(x_{j_0})$. Therefore 
\[ |p_{n,j_0} - p_{j_0}| = |(F_n(a_{j_0}) - F_n(x_{j_0})) - (F(a_{j_0}) - F(x_{j_0}))| \leq |F_n(a_{j_0}) - F(a_{j_0})| + |F_n(x_{j_0}) - F(x_{j_0})|
\rightarrow 0 \]
as $n \rightarrow \infty$.

{\bf (ii) $\Rightarrow$ (iii)} Suppose $p_{n,j} \rightarrow p_{j}$ for all $j\geq 1$. Consider the measure space $(\mathbb{N},
\mathcal{P}(\mathbb{N}), \lambda)$\footnote{For simplicity, let $\mathbb{N}$ denote the positive integers}, where $\lambda$ is the counting measure.
For each $n,j \geq 1$, let $f,f_{n} : \mathbb{N} \rightarrow \mathbb{R}$ be defined by $f_{n}(j) := p_{n,j}$ and $f(j) := p_{j}$. By construction,
$f_{n} \rightarrow f$ a.e. ($\lambda$) and $\int f_nd\lambda = \sum_{j=1}^{\infty}p_{n,j} = 1 = \sum_{j=1}^{\infty}p_{j} = \int fd\lambda$.
Therefore, by Scheffe's Theorem,
\[ \lim_{n\rightarrow \infty}\int|f_n - f|d\lambda = \lim_{n\rightarrow\infty}\sum_{j=1}^{\infty}|p_{n,j} - p_{j}| = 0. \]

{\bf (iii) $\Rightarrow$ (i)} Suppose $\sum_{j=1}^{\infty}|p_{n,j} - p_{j}| \rightarrow 0$ as $n\rightarrow \infty$. Let $\epsilon > 0$. Note that 
$F_n(x) = \sum_{\{j : a_{j} \leq x\}}p_{n,j}$ and $F(x) = \sum_{\{j : a_{j} \leq x\}}p_{j}$. By assumption, we can choose some $N \geq 1$ such that
\[\sum_{j=1}^{\infty}|p_{n,j} - p_{j}| < \epsilon\] whenever $n \geq N$. So for $n \geq N$ and any $x \in \mathbb{R}$,
\[ |F_{n}(x) - F(x)| \leq \sum_{\{j : a_{j} \leq x\}}|p_{n,j} - p_{j}| \leq \sum_{j=1}^{\infty}|p_{n,j} - p_{j}| < \epsilon. \]
Thus $\sup_{x\in\mathbb{R}}|F_{n}(x) - F(x)| \leq \epsilon$ for all $n \geq N$.
\end{Proof}


\newpage
\subsection*{2 [AL 9.28]}
\begin{tcolorbox}
Let $\mu$ be a probability distribution on $\mathbb{R}$ such that $\varphi(t_0) := \int e^{t_0|x|}d\mu(x) < \infty$ for some $t_0 > 0$. Show that
Carleman's condition is satisfied, i.e.
\[ \sum_{k=1}^{\infty}m_{2k}^{-1/2k} = \infty, \]
where $m_{j} := \int x^{j}d\mu(x)$.
\end{tcolorbox}




\newpage 
\subsection*{3 [AL 9.24 (a)]}
\begin{tcolorbox}
Let $X_n \sim \text{Geo}(p_n)$ for $n\in\mathbb{N}$ and suppose $p_n \rightarrow 0$. Show that $p_n X_n \rightarrow^{d} X$, where $X \sim
\text{Exp}(1)$.
\end{tcolorbox}
\begin{Proof}
By the result from Problem 9.29, it suffices to show that for some $\delta > 0$, 
\[ Ee^{tp_nX_n} \rightarrow Ee^{tX} = \frac{1}{1 - t}\] 
for all $t$ such that $|t| < \delta$. Well, by Table 6.2.1 and L'Hopitals rule, 
\[ \lim_{n\rightarrow\infty}Ee^{tp_nX_n} = \lim_{n\rightarrow\infty}\frac{p_n e^{tp_n}}{1 - (1 - p_n)e^{tp_n}} =
\lim_{n\rightarrow\infty}\frac{e^{tp_n} + tp_n e^{tp_n}}{e^{tp_n} - t e^{tp_n} + tp_n e^{tp_n}} = \frac{1}{1 - t}, \]
for all $t$ such that $|t| < 1$.
\end{Proof}



\newpage 
\subsection*{4 [AL 9.26]}
\begin{tcolorbox}
Let $Y_n$ have the discrete uniform distribution on the integers $\left\{ 1,2,\hdots, n \right\}$. Let $X_n := Y_n / n$ and let $X \sim$ Uniform(0,1).
Show that $X_n \rightarrow^{d} X$ using three different methods as follows:
\begin{enumerate}[label=(\alph*)]
\item Helly-Bray Theorem,
\item the method of moments,
\item using the cdfs.
\end{enumerate}
\end{tcolorbox}
\begin{enumerate}[label=(\alph*)]
\item Let $\mu_n$ denote the probability distribution of $X_n$ for all $n \in \mathbb{N}$ and $\mu$ the probability distribution of $X$. 
Suppose $f : \mathbb{R} \rightarrow \mathbb{R}$ is a bounded continuous function. For each $n \in \mathbb{N}$, define the simple function $s_n :
\mathbb{R} \rightarrow \mathbb{R}$ by $s_n(x) := \sum_{k=1}^{n}f\left( \frac{k}{n} \right)\mathbb{I}_{[k-1,k)}(x)$ for all $x \in (0,1)$. Since $f$ is continuous,
$s_n(x) \rightarrow f(x)$ for all $x \in (0,1)$. Further, since $f$ is bounded, there exists some $M > 0$ such that $s_n \leq M$ for all $n \in
\mathbb{N}$. Thus, by the BCT,
\[ \lim_{n\rightarrow\infty}\int f d\mu_n = \lim_{n\rightarrow\infty}\frac{1}{n}\sum_{k=1}^{n}f\left( \frac{k}{n} \right) = 
\lim_{n\rightarrow\infty}\int s_n d\mu \stackrel{\text{BCT}}{=} \int fd\mu. \]
Hence, by the Helly-Bray Theorem, $\mu_n \Rightarrow \mu$.
\item Let $k \in \mathbb{N}$. By the work done above and by taking $f(x) = x^{k}$,
\[ \lim_{n\rightarrow\infty}EX_n^{k} = EX^{k}. \]
Now, since the moment generating function of $X$ exists, the probability distribution of $X$ is uniquely determined by its moments. From this
consideration and the convergence of moments as shown above, $X_n \rightarrow^{d} X$ by the Frech\'{e}t-Shohat theorem.
\item Let $\epsilon > 0$. Choose $N \in \mathbb{N}$ such that $N^{-1} < \epsilon$. Let $x \in (0,1)$ and $n \geq N$. Let $k_x := \max\left\{ k :
k/n \leq x\right\}$. Then
\[ |F_n(x) - F(x)| = \left| \sum_{i=1}^{k_x}\frac{i}{n} - x\right| = \left| \frac{k_x}{n} - x\right| \leq \frac{1}{n} \leq \frac{1}{N} < \epsilon. \]
Therefore $F_n \rightarrow F$.
\end{enumerate}





\end{document}

