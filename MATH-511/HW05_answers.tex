\documentclass[12pt]{article}
\usepackage{amsmath}
\usepackage{amsfonts}
\usepackage{parskip}
\usepackage{amsthm}
\usepackage{thmtools}
\usepackage[headheight=15pt]{geometry}
\geometry{a4paper, left=20mm, right=20mm, top=30mm, bottom=30mm}
\usepackage{graphicx}
\usepackage{bm} % for bold font in math mode - command is \bm{text}
\usepackage{enumitem}
\usepackage{fancyhdr}
\usepackage{amssymb} % for stacked arrows and other shit
\pagestyle{fancy}
\usepackage{changepage}
\usepackage{mathcomp}
\usepackage{tcolorbox}

\declaretheoremstyle[headfont=\normalfont]{normal}
\declaretheorem[style=normal]{Theorem}
\declaretheorem[style=normal]{Proposition}
\declaretheorem[style=normal]{Lemma}
\newcounter{ProofCounter}
\newcounter{ClaimCounter}[ProofCounter]
\newcounter{SubClaimCounter}[ClaimCounter]
\newenvironment{Proof}{\stepcounter{ProofCounter}\textsc{Proof.}}{\hfill$\square$}
\newenvironment{claim}[1]{\vspace{1mm}\stepcounter{ClaimCounter}\par\noindent\underline{\bf Claim \theClaimCounter:}\space#1}{}
\newenvironment{claimproof}[1]{\par\noindent\underline{Proof of claim \theClaimCounter:}\space#1}{\hfill $\blacksquare$ Claim \theClaimCounter}
\newenvironment{subclaim}[1]{\stepcounter{SubClaimCounter}\par\noindent\emph{Subclaim \theClaimCounter.\theSubClaimCounter:}\space#1}{}
% \newenvironment{subclaimproof}[1]{\begin{adjustwidth}{2em}{0pt}\par\noindent\emph{Proof of subclaim \theClaimCounter.\theSubClaimCounter:}\space#1}{\hfill
% $\blacksquare$ \emph{Subclaim \theClaimCounter.\theSubClaimCounter}\vspace{5mm}\end{adjustwidth}}
\newenvironment{subclaimproof}[1]{\par\noindent\emph{Proof of subclaim \theClaimCounter.\theSubClaimCounter:}\space#1}{\hfill
$\Diamond$ \emph{Subclaim \theClaimCounter.\theSubClaimCounter}}

\allowdisplaybreaks{}

% chktex-file 3

\title{MATH 511: HW 5}
\author{Evan P. Walsh}
\makeatletter
\makeatother
\lhead{Evan P. Walsh}
\chead{MATH 511: HW 5}
\rhead{\thepage}
\cfoot{}

\begin{document}
% \maketitle

\subsubsection*{1}
\begin{tcolorbox}
  Let $g$ be a holomorphic function on an open set $U \subseteq \mathbb{C}$ and $f$ a meromorphic function on $U$. Suppose $\overline{D(P,r)}
  \subset U$ such that $f$ has zeros $z_{1}, \hdots, z_{p}$ and poles $w_{1}, \hdots, w_{q}$ in $D(P,r)$ and $f$ has neither zeros nor poles on
  $\partial D(P,r)$. Prove that
  \[
    \frac{1}{2\pi i}\oint_{\partial D(P,r)}g(\zeta)\frac{f'(\zeta)}{f(\zeta)}d\zeta = \sum_{j=1}^{p}n_{j}g(z_{j}) - \sum_{k=1}^{q}m_{k}g(w_{k}),
  \]
  where $n_{j}$ is the multiplicity of $z_j$, $1 \leq j \leq p$, and $m_k$ is the order of $w_k$, $1 \leq k \leq q$.
\end{tcolorbox}
\begin{Proof}
  Let $H(z)$ be the holomorphic extension of
  \[ f(z)\cdot \frac{(z - w_1)^{m_1}\cdots (z - w_q)^{m_{q}}}{(z-z_1)^{n_{1}}\cdots (z - z_p)^{n_{p}}} \]
  on $U$. So $H(z)$ is holomorphic on $U$ and $H(z) \neq 0$ for all $z \in \overline{D(P,r)}$. Therefore $H'(z) / H(z)$ is well-defined and
  holomorphic on $\overline{D(p,r)}$. Further, we see that
  \begin{align*}
    f'(z) = H'(z)\cdot \frac{(z - z_1)^{n_1}\cdots (z - z_p)^{n_p}}{(z - w_1)^{m_1}\cdots (z - w_q)^{m_q}} & +
    H(z)\sum_{i=1}^{p}\frac{n_i(z - z_i)^{n_i-1}}{(z - w_1)^{m_1}\cdots(z-w_q)^{m_q}}\prod_{j\neq i}(z - z_j)^{n_j} \\
    & - H(z)\sum_{k=1}^{q}\frac{m_k(z-z_1)^{n_1}\cdots(z-z_p)^{n_p}}{(z - w_k)^{m_k + 1}}\prod_{l\neq k}\frac{1}{(z-w_l)^{m_l + 1}}.
  \end{align*}
  Therefore
  \[ g(z)\cdot\frac{f'(z)}{f(z)} = g(z)\cdot\frac{H'(z)}{H(z)} + g(z)\sum_{j=1}^{p}\frac{n_j}{z - z_j} - g(z)\sum_{k=1}^{q}\frac{m_k}{z - w_k}. \]
  Since $g(z) H'(z) / H(z)$ is holomorphic on $U$,
  \[
    \oint_{\partial D(P,r)}\frac{g(z)H'(z)}{H(z)}\ dz = 0.
  \]
  Thus, by the Cauchy integral formula,
  \begin{align*}
    \frac{1}{2\pi i}\oint_{\partial D(P,r)}g(z)\frac{f'(z)}{f(z)} dz & = \frac{1}{2\pi i}\sum_{j=1}^{p}n_j \oint_{\partial D(P,r)}\frac{g(z)}{z - z_j}dz -
    \frac{1}{2\pi i}\sum_{k=1}^{q}m_k \oint_{\partial D(P,r)}\frac{g(z)}{z - w_k}dz \\
    & = \sum_{j=1}^{p}n_{j}g(z_{j}) - \sum_{k=1}^{q}m_{k}g(w_{k}).
  \end{align*}
\end{Proof}


\newpage
\subsubsection*{2}
\begin{tcolorbox}
  Find the number of zeros of $f(z) = z^{10} + 10ze^{z + 1} - 9$ in $\left\{ z : |z| < 1 \right\}$.
\end{tcolorbox}
Define $g(z) := 10ze^{z+1}$. Then, for $z = -1$, we have
\[ |f(z) - g(z)| = |z^{10} - 9| = 8 < 10 + 18 = |g(z)| + |f(z)|. \]
In general, when $|z| = 1$ but $z \neq -1$, $|e^{z + 1}| > 1$, so
\[ |f(z) - g(z)| = |z^{10} - 9| \leq |z|^{10} + 9 = 10 < 10|e^{z+1}| = |g(z)| \leq |g(z)| + |f(z)|. \]
Thus, by Rouch\'{e}'s theorem, $f$ has the same number of zeros in $D(0,1)$ as $g$, which is 1.

\subsubsection*{3}
\begin{tcolorbox}
  Find the number of zeros of $f(z) = 2z^5 - 6z^2 + z + 1$ in $\left\{ z : 1 < |z| < 2 \right\}$.
\end{tcolorbox}
Let $g_1(z) := -6z^2$ and $g_2(z) := 2z^5$. For $z \in \partial D(0,1)$,
\begin{equation}
  |f(z) - g_1(z)| = |2z^5 + z + 1| \leq 2|z|^5 + |z| + 1 = 4 < 6 = |g_1(z)|.
  \label{3.1}
\end{equation}
Similarly, for $z \in \partial D(0,2)$, we have
\begin{equation}
  |f(z) - g_2(z)| = |-6z^2 + z + 1| \leq 6|z|^2 + |z| + 1 = 27 < 64 = |g_2(z)|.
  \label{3.2}
\end{equation}
Thus, by~\eqref{3.1} and~\eqref{3.2} $f$ has 2 zeros in $D(0,1)$ and 5 zeros in $D(0,2)$ (counting multiplicity) by Rouch\'{e}'s theorem. Hence, since
$f$ has no zeros on $\partial D(0,1)$, $f$ has 3 zeros in $\left\{ z : 1 < |z| < 2 \right\}$.


\subsubsection*{4}
\begin{tcolorbox}
  Find the number of zeros of $f(z) = z^8 + 3z^3 + 7z + 5$ in $\left\{ x + iy : x > 0, y > 0 \right\}$.
\end{tcolorbox}
Let $R > 2$. Let $\Gamma_{1}^{R}$ be the vertical path from $iR$ to $0$, $\Gamma_{2}^{R}$ be the horizontal path from $0$ to $R$, and $\Gamma_{3}^{R}$
be the quarter circle of radius $R$ centered at $0$ from $R$ to $iR$. Let $\Gamma := \cup_{i=1}^{3}\Gamma_{i}^{R}$.
Let $g(z) := z^{8} + 5$. Note that $g$ has two zeros within $\Gamma$, namely $z = 5^{1/8}e^{i\pi/8}$ and $z = 5^{1/8}e^{i3\pi/8}$. We will show that
$f$ also has two zeros by showing that $|f(z) - g(z)| < |f(z)| + |g(z)|$ for all $z \in \Gamma$. First note that on $\Gamma_{1}^{R}$, $z = iy$. Thus,
\[ |f(z) - g(z)| = |i(-3y^{3} + 7y)| < |y^{8} + 5 + i(-3y^{3} + 7y)| + |y^{8} + 5| = |f(z)| + |g(z)|. \]
Similary, on $\Gamma_{2}^{R}$, $z = x > 0$. Thus,
\[ |f(z) - g(z)| = |3x^{3} + 7x| < |x^{8} + 3x^{3} + 7x + 5| + |x^{8} + 5| = |f(z)| + |g(z)|. \]
Finally, on $\Gamma_{3}^{R}$,
\[ |f(z) - g(z)| \leq 3R^{3} + 7R < R^{8} + 5 = |g(z)| < |f(z)| + |g(z)|. \]



\newpage
\subsection*{5}
\begin{tcolorbox}
  Let $P(z) = z^n + a_{n-1}z^{n-1} + \cdots + a_0$. Show that all the zeros of $P$ lie inside $D(0,R)$, where $R := 1 + \max\left\{
  |a_k| : 0 \leq k \leq n-1\right\}$.
\end{tcolorbox}
\begin{Proof}
  Let $g(z) := z^{n}$, $A := \max\left\{ |a_k| : 0\leq k \leq n-1 \right\}$ so that $R = A + 1$. We will show that $|f(z) - g(z)| < |f(z)| + |g(z)|$ on
  $\partial D(0,R)$, which implies that $f$ and $g$ have the same number of zeros (counting multiplicity) within $D(0,R)$ by Rouch\'{e}'s theorem.
  Well, for $z \in \partial D(0,R)$,
  \begin{align*}
    |P(z) - g(z)| = |a_{n-1}z^{n-1} + \cdots + a_0| & \leq |a_{n-1}||z|^{n-1} + |a_{n-1}||z|^{n-2} + \cdots + |a_{0}| \\
    & \leq AR^{n-1} + AR^{n-2} + \cdots + A \\
    & = A[(A+1)^{n-1} + (A + 1)^{n-2} + \cdots + 1].
  \end{align*}
  Since $|g(z)| = R^{n} = (A + 1)^{n}$ for $z\in \partial D(0,R)$, it suffices to show that
  \begin{equation}
    \frac{A[(A+1)^{n-1} + (A + 1)^{n-2} + \cdots + 1]}{(A + 1)^{n}} < 1.
    \label{5.1}
  \end{equation}
  But
  \begin{align*}
    \frac{A[(A+1)^{n-1} + (A + 1)^{n-2} + \cdots + 1]}{(A + 1)^{n}} & = A\left[ \frac{1}{A+1} + \frac{1}{(A+1)^{2}} + \cdots + \frac{1}{(A+1)^{n}}
    \right] \\
    & < A\sum_{k=1}^{\infty}\left( \frac{1}{A + 1} \right)^{k} \\
    & = A\cdot \frac{(A + 1)^{-1}}{1 - (A+1)^{-1}} = 1.
  \end{align*}
  Therefore~\eqref{5.1} holds, and so $|f(z) - g(z)| < (A + 1)^{n} = R^{n} = |g(z)| \leq |f(z)| + |g(z)|$. So by Rouch\'{e}'s theorem, $f$ has $n$
  zeros within $D(0,R)$. Since $f$ only has $n$ zeros, all the zeros of $f$ are within $D(0,R)$.
\end{Proof}


\newpage
\subsection*{6}
\begin{tcolorbox}
  Suppose $\left\{ f_j \right\}_{j=0}^{\infty}$ is a sequence of holomorphic funtions on $D(0,1)$ such that each $f_j$ has at most $k$ zeros (counting
  multiplicity) in $D(0,1)$ and $\left\{ f_j \right\}_{j=0}^{\infty}$ converges uniformly on compact subsets of $D(0,1)$ to $f$ (necessarily
  holomorphic). Prove that if $f
  \not\equiv 0$, then $f$ has at most $k$ zeros (counting multiplicity) in $D(0,1)$.
\end{tcolorbox}
\begin{Proof}
  By way of contradiction, suppose $f \not\equiv 0$ and $f$ has $m > k$ zeros (counting multiplicity). Then there exists some $0 < R < 1$ such that
  all of the zeros of $f$ are contained within $D(0,R)$. Since $\left\{ f_j \right\}_{n=0}^{\infty}$ converges uniformly on compact subsets, there
  exists some $N \in \mathbb{N}$ such that $|f(z) - f_j(z)| < \min\left\{ |f(z)| : z \in \partial D(0,R) \right\}$ for all $j \geq N$, i.e. $f_j(z) \neq
  0$ for all $z \in \partial D(0,R)$ whenever $j \geq N$. So, by Proposition 5.1.2,
  \begin{equation}
    \frac{1}{2\pi i}\oint_{\partial D(0,R)}\frac{f'_j(\zeta)}{f_j(\zeta)}d\zeta \leq k,
    \label{6.1}
  \end{equation}
  for all $j \geq N$, and
  \begin{equation}
    \frac{1}{2\pi i}\oint_{\partial D(0,R)}\frac{f'(\zeta)}{f(\zeta)}d\zeta = m.
    \label{6.2}
  \end{equation}
  However, since $f_j \rightarrow f$ and $f_j' \rightarrow f$ uniformly on $\partial D(0,R)$, the integrals in~\eqref{6.1} converge to the integral
  in~\eqref{6.2}. This is a contradiction.
\end{Proof}

\subsection*{7}
\begin{tcolorbox}
  Let $k > 0$. For each $0 \leq \ell \leq k$, construct a sequence $\left\{ f_j \right\}_{j=0}^{\infty}$ on $D(0,1)$ such that each $f_j$ has at least
  $k$ zeros in $D(0,1)$ and $f_j$ converges uniformly on compact subsets of $D(0,1)$ to $f$ but $f$ has only $\ell$ zeros in $D(0,1)$.
\end{tcolorbox}
For $j \geq 0$, $0 \leq \ell \leq k$, let $f_{j}^{\ell}(z) := z^{\ell}(z - (1 - 2^{-j}))^{k}$ and $f^{\ell}(z) := z^{\ell}(z - 1)^{k}$. Then
$f_{j}^{\ell} \rightarrow f^{\ell}$ uniformly on compact subsets of $D(0,1)$, while each $f_{j}^{\ell}$ has $\ell + k$ zeros in $D(0,1)$ but $f^{\ell}$
has only $\ell$ zeros in $D(0,1)$.

\subsection*{8}
\begin{tcolorbox}
  Let $R > 0$ and $U = \mathbb{C} \setminus \overline{D(0,R)}$. Suppose $f$ is holomorphic on $U$, continuous and bounded on $\overline{U}$. Prove
  that there exists $z_0 \in \partial D(0,R)$ such that $|f(z_0)| \geq |f(z)|$ for all $z \in \overline{U}$.
\end{tcolorbox}
\begin{Proof}
  Define $\tilde{g}(z) = f(1/z)$ for all $z \in D(0,R^{-1})\setminus \{0\}$. Then $\tilde{g}$ is holomorphic on \\
  $D(0,R^{-1})\setminus\{0\}$,
  and since $f$ is bounded on $U$, $\tilde{g}$ has a holomorphic extension $g$ on $D(0,R^{-1})$. Clearly $g$ must also be continuous on $\partial
  D(0,R^{-1})$, so by the maximum modulus theorem, there exists $p_0 \in \partial D(0,R^{-1})$ such that $|g(p_0)| \geq |g(z)|$ for all $z \in
  \overline{D(0,R^{-1})}$, i.e. $|f(1/p_0)| \geq |f(1/z)|$ for all $z \in \overline{D(0,R^{-1})}\setminus\{0\}$. Thus $|f(z_0)| \geq |f(z)|$ for all
  $z \in \overline{U}$, where $z_0 := 1/p_0$.
\end{Proof}


\newpage
\subsection*{9}
\begin{tcolorbox}
  Suppose $f$ is 1-1 and holomorphic on $D(0,1)$, $D(0,1) \subseteq f(D(0,1))$ and $f(0) = 0$. Show that $|f'(0)| \geq 1$ and the equality % chktex 8
  holds if and only if $f(z) = \alpha z$ for some $\alpha$ with $|\alpha| = 1$.
\end{tcolorbox}
\begin{Proof}
  Let $U = f(D(0,1))$. Since $f$ is 1-1 and holomorphic, $f^{-1} : U \rightarrow D(0,1)$ is holomorphic. % chktex 8
  Further, $f^{-1}(D(0,1)) \subseteq f^{-1}(U) = D(0,1)$, so $|f^{-1}(z)| \leq 1$ for all $z \in D(0,1)$ and $f^{-1}(0) = 0$. Hence, by Schwarz's
  Lemma, $|(f^{-1})'(0)| \leq 1$, which implies $|f'(0)| \geq 1$, and $|(f^{-1})'(0)| = 1$ if and only if $f^{-1}(z) \equiv \alpha z$ for some
  $\alpha \in \mathbb{C}$ with $|\alpha| = 1$, if and only if $f(z) \equiv \tilde{\alpha}z$, where $\tilde{\alpha} = \frac{1}{\alpha}$.
\end{Proof}



\subsection*{10}
\begin{tcolorbox}
  Let $f$ be a holomorphic map on $D(0,1)$ such that $f(-1/2) = 0$ and $|f(z)| \leq |1 + z^2|$ for all $z \in D(0,1)$. Show that $|f(1/2)| \leq 1$.
\end{tcolorbox}
\begin{Proof}
  Let $g(z) = \frac{f(z)}{1 + z^2}$. Then $g$ is holomorphic on $D(0,1)$ and $|g(z)| \leq 1$ for all $z \in D(0,1)$. Let $a_1 = -1/2$, $b_1 =
  g(-1/2) = 0$, $a_2 = 1/2$, and $b_2 = g(1/2) = f(1/2) / (1 + 1/4)$. By the Schwars-Pick theorem,
  \[ \left| \frac{b_2 - b_1}{1 - \bar{b}_1 b_2} \right| \leq \left| \frac{a_2 - a_1}{1 - \bar{a}_1 a_2}\right|, \]
  which implies
  \[ |b_2| \leq \frac{1}{1 + \frac{1}{4}}\qquad \Rightarrow\qquad \frac{\left|f\left(\frac{1}{2}\right)\right|}{1 + \frac{1}{4}} \leq \frac{1}{1 + \frac{1}{4}}
  \qquad\Rightarrow\qquad \left|f\left(\frac{1}{2}\right)\right| \leq 1. \]
\end{Proof}

\end{document}
