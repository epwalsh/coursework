\documentclass[12pt]{article}
\usepackage{amsmath}
\usepackage{amsfonts}
\usepackage{parskip}
\usepackage{amsthm}
\usepackage{thmtools}
\usepackage[headheight=15pt]{geometry}
\geometry{a4paper, left=20mm, right=20mm, top=30mm, bottom=30mm}
\usepackage{graphicx}
\usepackage{bm} % for bold font in math mode - command is \bm{text}
\usepackage{enumitem}
\usepackage{fancyhdr}
\usepackage{amssymb} % for stacked arrows and other shit
\pagestyle{fancy}
\usepackage{changepage}
\usepackage{mathcomp}
\usepackage{tcolorbox}

\declaretheoremstyle[headfont=\normalfont]{normal}
\declaretheorem[style=normal]{Theorem}
\declaretheorem[style=normal]{Proposition}
\declaretheorem[style=normal]{Lemma}
\newcounter{ProofCounter}
\newcounter{ClaimCounter}[ProofCounter]
\newcounter{SubClaimCounter}[ClaimCounter]
\newenvironment{Proof}{\stepcounter{ProofCounter}\textsc{Proof.}}{\hfill$\square$}
\newenvironment{claim}[1]{\vspace{1mm}\stepcounter{ClaimCounter}\par\noindent\underline{\bf Claim \theClaimCounter:}\space#1}{}
\newenvironment{claimproof}[1]{\par\noindent\underline{Proof of claim \theClaimCounter:}\space#1}{\hfill $\blacksquare$ Claim \theClaimCounter}
\newenvironment{subclaim}[1]{\stepcounter{SubClaimCounter}\par\noindent\emph{Subclaim \theClaimCounter.\theSubClaimCounter:}\space#1}{}
% \newenvironment{subclaimproof}[1]{\begin{adjustwidth}{2em}{0pt}\par\noindent\emph{Proof of subclaim \theClaimCounter.\theSubClaimCounter:}\space#1}{\hfill
% $\blacksquare$ \emph{Subclaim \theClaimCounter.\theSubClaimCounter}\vspace{5mm}\end{adjustwidth}}
\newenvironment{subclaimproof}[1]{\par\noindent\emph{Proof of subclaim \theClaimCounter.\theSubClaimCounter:}\space#1}{\hfill
$\Diamond$ \emph{Subclaim \theClaimCounter.\theSubClaimCounter}}

\allowdisplaybreaks{}

% chktex-file 3

\title{MATH 511: HW 5}
\author{Evan P. Walsh}
\makeatletter
\makeatother
\lhead{Evan P. Walsh}
\chead{MATH 511: HW 5}
\rhead{\thepage}
\cfoot{}

\begin{document}
\maketitle

\subsubsection*{1}
\begin{tcolorbox}
  Let $g$ be a holomorphic function on an open set $U \subseteq \mathbb{C}$ and $f$ a meromorphic function on $U$. Suppose $\overline{D(P,r)}
  \subset U$ such that $f$ has zeros $z_{1}, \hdots, z_{p}$ and poles $w_{1}, \hdots, w_{q}$ in $D(P,r)$ and $f$ has neither zeros nor poles on
  $\partial D(P,r)$. Prove that
  \[
    \frac{1}{2\pi i}\oint_{\partial D(P,r)}g(\zeta)\frac{f'(\zeta)}{f(\zeta)}d\zeta = \sum_{j=1}^{p}n_{j}g(z_{j}) - \sum_{k=1}^{q}m_{k}g(w_{k}),
  \]
  where $n_{j}$ is the multiplicity of $z_j$, $1 \leq j \leq p$, and $m_k$ is the order of $w_k$, $1 \leq k \leq q$.
\end{tcolorbox}
\begin{Proof}
  Let $H(z)$ be the holomorphic extension of 
  \[ f(z)\cdot \frac{(z - w_1)^{m_1}\cdots (z - w_q)^{m_{q}}}{(z-z_1)^{n_{1}}\cdots (z - z_p)^{n_{p}}} \]
  on $U$. So $H(z)$ is holomorphic on $U$ and $H(z) \neq 0$ for all $z \in \overline{D(P,r)}$. Therefore $H'(z) / H(z)$ is well-defined and
  holomorphic on $\overline{D(p,r)}$. Further, we see that 
  \begin{align*}
    f'(z) = H'(z)\cdot \frac{(z - z_1)^{n_1}\cdots (z - z_p)^{n_p}}{(z - w_1)^{m_1}\cdots (z - w_q)^{m_q}} & + 
    H(z)\sum_{i=1}^{p}\frac{n_i(z - z_i)^{n_i-1}}{(z - w_1)^{m_1}\cdots(z-w_q)^{m_q}}\prod_{j\neq i}(z - z_j)^{n_j} \\
    & - H(z)\sum_{k=1}^{q}\frac{(z-z_1)^{n_1}\cdots(z-z_p)^{n_p}}{(z - w_k)^{m_k + 1}}\prod_{l\neq k}\frac{1}{(z-w_l)^{m_l + 1}}.
  \end{align*}
  Therefore 
  \[ g(z)\cdot\frac{f'(z)}{f(z)} = g(z)\cdot\frac{H'(z)}{H(z)} + g(z)\sum_{j=1}^{p}\frac{n_j}{z - z_j} - g(z)\sum_{k=1}^{q}\frac{m_k}{z - w_k}. \]
  Since $g(z) H'(z) / H(z)$ is holomorphic on $U$, $\oint_{\partial D(P,r)}g(z)H'(z) / H(z)\ dz = 0$. Thus,
  \[ 
    \oint_{\partial D(P,r)}g(z)\frac{f'(z)}{f(z)} dz = \sum_{j=1}^{p}\oint_{\partial D(P,r)}\frac{n_j g(z)}{z - z_j}dz -
    \sum_{k=1}^{q}\oint_{\partial D(P,r)}\frac{m_k g(z)}{z - w_k}dz = \sum_{j=1}^{p}n_{j}g(z_{j}) - \sum_{k=1}^{q}m_{k}g(w_{k}).
  \]
\end{Proof}


\newpage
\subsubsection*{2}
\begin{tcolorbox}
  Find the number of zeros of $f(z) = z^{10} + 10ze^{z + 1} - 9$ in $\left\{ z : |z| < 1 \right\}$.
\end{tcolorbox}
Define $g(z) := 10ze^{z+1}$. Then, for $z = -1$, we have
\[ |f(z) - g(z)| = |z^{10} - 9| = 8 < 10 + 18 = |g(z)| + |f(z)|. \]
In general, when $|z| = 1$ but $z \neq -1$, $|e^{z + 1}| > 1$, so
\[ |f(z) - g(z)| = |z^{10} - 9| \leq |z|^{10} + 9 = 10 < 10|e^{z+1}| = |g(z)| \leq |g(z)| + |f(z)|. \]
Thus, by Rouch\'{e}'s theorem, $f$ has the same number of zeros in $D(0,1)$ as $g$, which is 1.

\subsubsection*{3}
\begin{tcolorbox}
  Find the number of zeros of $f(z) = 2z^5 - 6z^2 + z + 1$ in $\left\{ z : 1 < |z| < 2 \right\}$.
\end{tcolorbox}
Let $g_1(z) := -6z^2$ and $g_2(z) := 2z^5$. For $z \in \partial D(0,1)$,
\begin{equation}
  |f(z) - g_1(z)| = |2z^5 + z + 1| \leq 2|z|^5 + |z| + 1 = 4 < 6 = |g_1(z)|.
  \label{3.1}
\end{equation}
Similarly, for $z \in \partial D(0,2)$, we have 
\begin{equation}
  |f(z) - g_2(z)| = |-6z^2 + z + 1| \leq 6|z|^2 + |z| + 1 = 27 < 64 = |g_2(z)|.
  \label{3.2}
\end{equation}
Thus, by~\eqref{3.1} and~\eqref{3.2} $f$ has 2 zeros in $D(0,1)$ and 5 zeros in $D(0,2)$ by Rouch\'{e}'s theorem (counting multiplicity). Hence, since
$f$ has no zeros on $\partial D(0,1)$, $f$ has 3 zeros (counting multiplicity) in $\left\{ z : 1 < |z| < 2 \right\}$.


\subsubsection*{4}
\begin{tcolorbox}
  Find the number of zeros of $f(z) = z^8 + 3z^3 + 7z + 5$ in $\left\{ x + iy : x > 0, y > 0 \right\}$.
\end{tcolorbox}


\end{document}

