\documentclass[12pt]{article}
\usepackage{amsmath}
\usepackage{amsfonts}
\usepackage{parskip}
\usepackage{amsthm}
\usepackage{thmtools}
\usepackage[headheight=15pt]{geometry}
\geometry{a4paper, left=20mm, right=20mm, top=30mm, bottom=30mm}
\usepackage{graphicx}
\usepackage{bm} % for bold font in math mode - command is \bm{text}
\usepackage{enumitem}
\usepackage{fancyhdr}
\usepackage{amssymb} % for stacked arrows and other shit
\pagestyle{fancy}
\usepackage{changepage}
\usepackage{mathcomp}
\usepackage{tcolorbox}

\declaretheoremstyle[headfont=\normalfont]{normal}
\declaretheorem[style=normal]{Theorem}
\declaretheorem[style=normal]{Proposition}
\declaretheorem[style=normal]{Lemma}
\newcounter{ProofCounter}
\newcounter{ClaimCounter}[ProofCounter]
\newcounter{SubClaimCounter}[ClaimCounter]
\newenvironment{Proof}{\stepcounter{ProofCounter}\textsc{Proof.}}{\hfill$\square$}
\newenvironment{claim}[1]{\vspace{1mm}\stepcounter{ClaimCounter}\par\noindent\underline{\bf Claim \theClaimCounter:}\space#1}{}
\newenvironment{claimproof}[1]{\par\noindent\underline{Proof of claim \theClaimCounter:}\space#1}{\hfill $\blacksquare$ Claim \theClaimCounter}
\newenvironment{subclaim}[1]{\stepcounter{SubClaimCounter}\par\noindent\emph{Subclaim \theClaimCounter.\theSubClaimCounter:}\space#1}{}
% \newenvironment{subclaimproof}[1]{\begin{adjustwidth}{2em}{0pt}\par\noindent\emph{Proof of subclaim \theClaimCounter.\theSubClaimCounter:}\space#1}{\hfill
% $\blacksquare$ \emph{Subclaim \theClaimCounter.\theSubClaimCounter}\vspace{5mm}\end{adjustwidth}}
\newenvironment{subclaimproof}[1]{\par\noindent\emph{Proof of subclaim \theClaimCounter.\theSubClaimCounter:}\space#1}{\hfill
$\Diamond$ \emph{Subclaim \theClaimCounter.\theSubClaimCounter}}

\allowdisplaybreaks

\title{MATH 511: HW 4}
\author{Evan P. Walsh}
\makeatletter
\let\runauthor\@author
\let\runtitle\@title
\makeatother
\lhead{\runauthor}
\chead{\runtitle}
\rhead{\thepage}
\cfoot{}

\begin{document}
\maketitle

\subsection*{1}
\begin{tcolorbox}
Let $R(z) := \frac{P(z)}{Q(z)}$, where $P$ and $Q$ are two polynomials with no zeros in common. Let $P_1, \hdots, P_k$ be the zeros of $Q$. Suppose
$f$ is holomorphic on $\mathbb{C} \setminus \left\{ P_1, \hdots, P_k \right\}$ such that $|f(z)| \leq |R(z)|$ for all $z \in \mathbb{C} \setminus
\left\{ P_1,\hdots, P_k \right\}$. Prove that $f(z) = CR(z)$ for some constant $C$.
\end{tcolorbox}
\begin{Proof}
Let $z_0 \in \mathbb{C}$. If $z_0$ is not a zero of $P$ or $Q$, then $|f(z_0)| / |R(z_0)| \leq 1$. If $z_0$ is a zero of $P$ or $Q$, 
then there exists some $r_0 > 0$ such that $D(z_0,r_0)$ does not contain any other zeros of $P$ or $Q$ besides $z_0$. Therefore $f(z) / R(z)$ is
holomorphic and $|f(z)| / |R(z)| \leq 1$ for all $z$ in $D(z_0, r) \setminus \{z_0\}$. By the Riemann removable singularities theorem, we can extend
$f(z) / R(z)$ to function $g_0$ that is holomorphic on $D(z_0, r_0)$ and defined by 
\[ g_0(z) := \left\{ \begin{array}{cl}
f(z) / R(z) & \text{ if } z \neq z_0 \\
\lim_{\zeta\rightarrow z_0} f(z) / R(z) \leq 1 & \text{ if } z = z_0. 
\end{array} \right. \]
Since this can be done at every zero of $P$ and $Q$, we can construct an entire function $g$ that is equal to $f / R$ except at the zeros of $P$ and
$Q$. By since $g$ is entire and bounded above by 1, $g(z) \equiv C$ for some constant $C$. Therefore $f(z) \equiv CR(z)$ except at the zeros of $P$
and $Q$. But since $f(z)$ and $CR(z)$ are holomorphic and equal except at a finite number of isolated points, $f(z) \equiv CR(z)$ for all $z \in \mathbb{C}
\setminus \left\{ P_1,\hdots, P_k \right\}$ by Corollary 3.6.3.
\end{Proof}


\newpage
\subsection*{4}
\begin{tcolorbox}
Find the Laurent series for 
\[ f(x) = \frac{1}{z(z-1)(z-2)} \]
centered at $z = 0$ and converging in the annulus $\left\{ z : 1 \leq |z| < 2 \right\}$.
\end{tcolorbox}
We have 
\begin{align*}
f(z) = \frac{1}{z(1-z)(z-2)} = \frac{1}{z}\left[ \frac{1}{z - 2} - \frac{1}{z - 1} \right] & = \frac{1}{z}\left[ 
\left( -\frac{1}{2}\right)\left( \frac{1}{1 - \frac{z}{2}} \right)  - \left( \frac{1}{z} \right)\left( \frac{1}{1 - \frac{1}{z}} \right) \right] \\
& = -\frac{1}{2z}\sum_{n=0}^{\infty}\left( \frac{z}{2} \right)^{n} - \frac{1}{z^{2}}\sum_{n=0}^{\infty}\left( \frac{1}{z} \right)^{n} \\
& = \sum_{n=0}^{\infty}\left( -\frac{1}{2} \right)\frac{z^{n-1}}{2^{n}} + \sum_{n=0}^{\infty}(-1)\frac{1}{z^{n+2}} \\
& = \sum_{n=-\infty}^{-2}(-1)z^{n} + \sum_{n=-1}^{\infty}\left( -\frac{1}{4} \right)\left( \frac{z}{2} \right)^{n}.
\end{align*}

\newpage
\subsection*{5}
\begin{tcolorbox}
Find the function $f(z)$ satisfying the following conditions:
\begin{enumerate}[label=(\alph*)]
\item $f$ has a pole of order 2 as $z = 0$ with residue 2, a simple pole at $z = 1$ with residue 2.
\item $f$ is holomorphic on $\mathbb{C} \setminus \left\{ 0,1 \right\}$.
\item There exists $R$ and $M > 0$ such that $|f(z)| < M$ for all $z$ with $|z| \geq R$.
\item $f(2) = 5$, $f(-1) = 2$.
\end{enumerate}
\end{tcolorbox}
Since $f$ has a pole of order 2 at 0 and a simple pole at 1, we can express $f$ as 
\[ f(z) = \frac{1}{z^{2}(z-1)}g(z), \]
where $g$ is holomorphic on $\mathbb{C}$. Then,
\begin{align}
2 = \text{Res}_{f}(0) = \frac{\partial }{\partial z}\left( z^{2}f(z) \right)\bigg|_{z=0} & = \frac{\partial }{\partial z}\left( \frac{g(z)}{z-1}
\right)\bigg|_{z = 0} \nonumber \\
& = \frac{(z-1)g'(z) - g(z)}{(z-1)^{2}}\bigg|_{z=0}\nonumber \\
& = (-1)g'(0) - g(0) \nonumber \\
\Rightarrow  \qquad g(0) & = -2 - g'(0), \text{ and }\label{5.1} 
\end{align}
\begin{equation}
2 = \text{Res}_{f}(1) = \frac{1}{z^{2}}f(z)\bigg|_{z = 1} = g(1), \ \text{ i.e. } g(1) = 2.
\label{5.2}
\end{equation}
Further, we must have $f(2) = \frac{1}{4}g(2) = 5$ and $f(-1) = -\frac{1}{2}g(-1) = 2$, implying 
\begin{equation}
g(2) = 20, \qquad g(-1) = -4.
\label{5.3}
\end{equation}
Now, if there exists $M, R > 0$ such that $|f(z)| < M$ for all $z$ with $|z| \geq R$, then $|g(z)| < M|z^{2}(z-1)| \leq C_{1}|z|^{3}$ for some $C_{1} > 0$.
We can assume that $R > 1$ since $f$ is unbounded at $z = 1$. Since $g$ is holomorphic on $\mathbb{C}$, there exists some $C_{2} > 0$ such that $|g| <
C_{2}$ on $\left\{ z : |z| \leq R \right\}$. Let $C := \max\left\{ C_1, C_2 \right\}$. Then $|g(z)| < C < C|z|^{3}$ for all $z$ with $|z| > 1$. Hence,
by Theorem 3.4.4, $g$ is a polynomial in $z$ with degree at most $3$. Therefore we can write 
\[ g(z) = az^{3} - bz^{2} + cz + d, \]
where $a,b,c,d \in \mathbb{C}$. By \eqref{5.1}, \eqref{5.2}, and \eqref{5.3}, we must have 
\[
\begin{bmatrix}
0 & 0 & 1 & 1 \\
1 & 1 & 1 & 1 \\
-1 & 1 & -1 & 1 \\
8 & 4 & 2 & 1 \\
\end{bmatrix}\left( \begin{array}{c}
a \\
b \\ 
c \\
d \\
\end{array} \right) = \left( \begin{array}{c}
-2 \\
2 \\
-4 \\
20 \\
\end{array} \right)
\]
which implies $a = 1, b = 3, c = 2, d = -4$. Therefore 
\[ f(z) = \frac{z^{3} + 3z^{2} + 2z - 4}{z^{2}(z-1)}. \]



\end{document}

