\documentclass[12pt]{article}
\usepackage{amsmath}
\usepackage{amsfonts}
\usepackage{parskip}
\usepackage{amsthm}
\usepackage{thmtools}
\usepackage[headheight=15pt]{geometry}
\geometry{a4paper, left=20mm, right=20mm, top=30mm, bottom=30mm}
\usepackage{graphicx}
\usepackage{bm} % for bold font in math mode - command is \bm{text}
\usepackage{enumitem}
\usepackage{fancyhdr}
\usepackage{amssymb} % for stacked arrows and other shit
\pagestyle{fancy}
\usepackage{changepage}
\usepackage{mathcomp}
\usepackage{tcolorbox}

\declaretheoremstyle[headfont=\normalfont]{normal}
\declaretheorem[style=normal]{Theorem}
\declaretheorem[style=normal]{Proposition}
\declaretheorem[style=normal]{Lemma}
\newcounter{ProofCounter}
\newcounter{ClaimCounter}[ProofCounter]
\newcounter{SubClaimCounter}[ClaimCounter]
\newenvironment{Proof}{\stepcounter{ProofCounter}\textsc{Proof.}}{\hfill$\square$}
\newenvironment{claim}[1]{\vspace{1mm}\stepcounter{ClaimCounter}\par\noindent\underline{\bf Claim \theClaimCounter:}\space#1}{}
\newenvironment{claimproof}[1]{\par\noindent\underline{Proof of claim \theClaimCounter:}\space#1}{\hfill $\blacksquare$ Claim \theClaimCounter}
\newenvironment{subclaim}[1]{\stepcounter{SubClaimCounter}\par\noindent\emph{Subclaim \theClaimCounter.\theSubClaimCounter:}\space#1}{}
% \newenvironment{subclaimproof}[1]{\begin{adjustwidth}{2em}{0pt}\par\noindent\emph{Proof of subclaim \theClaimCounter.\theSubClaimCounter:}\space#1}{\hfill
% $\blacksquare$ \emph{Subclaim \theClaimCounter.\theSubClaimCounter}\vspace{5mm}\end{adjustwidth}}
\newenvironment{subclaimproof}[1]{\par\noindent\emph{Proof of subclaim \theClaimCounter.\theSubClaimCounter:}\space#1}{\hfill
$\Diamond$ \emph{Subclaim \theClaimCounter.\theSubClaimCounter}}

\allowdisplaybreaks

\title{MATH 511: HW 4}
\author{Evan P. Walsh}
\makeatletter
\let\runauthor\@author
\let\runtitle\@title
\makeatother
\lhead{\runauthor}
\chead{\runtitle}
\rhead{\thepage}
\cfoot{}

\begin{document}
\maketitle

\subsection*{1}
\begin{tcolorbox}
Let $R(z) := \frac{P(z)}{Q(z)}$, where $P$ and $Q$ are two polynomials with no zeros in common. Let $P_1, \hdots, P_k$ be the zeros of $Q$. Suppose
$f$ is holomorphic on $\mathbb{C} \setminus \left\{ P_1, \hdots, P_k \right\}$ such that $|f(z)| \leq |R(z)|$ for all $z \in \mathbb{C} \setminus
\left\{ P_1,\hdots, P_k \right\}$. Prove that $f(z) = CR(z)$ for some constant $C$.
\end{tcolorbox}
\begin{Proof}
Let $z_0 \in \mathbb{C}$. If $z_0$ is not a zero of $P$ or $Q$, then $|f(z_0)| / |R(z_0)| \leq 1$. If $z_0$ is a zero of $P$ or $Q$,
then there exists some $r_0 > 0$ such that $D(z_0,r_0)$ does not contain any other zeros of $P$ or $Q$ besides $z_0$. Therefore $f(z) / R(z)$ is
holomorphic and $|f(z)| / |R(z)| \leq 1$ for all $z$ in $D(z_0, r) \setminus \{z_0\}$. By the Riemann removable singularities theorem, we can extend
$f(z) / R(z)$ to function $g_0$ that is holomorphic on $D(z_0, r_0)$ and defined by
\[ g_0(z) := \left\{ \begin{array}{cl}
f(z) / R(z) & \text{ if } z \neq z_0 \\
\lim_{\zeta\rightarrow z_0} f(z) / R(z) \leq 1 & \text{ if } z = z_0.
\end{array} \right. \]
Since this can be done at every zero of $P$ and $Q$, we can construct an entire function $g$ that is equal to $f / R$ except at the zeros of $P$ and
$Q$. By since $g$ is entire and bounded above by 1, $g(z) \equiv C$ for some constant $C$. Therefore $f(z) \equiv CR(z)$ except at the zeros of $P$
and $Q$. But since $f(z)$ and $CR(z)$ are holomorphic and equal except at a finite number of isolated points, $f(z) \equiv CR(z)$ for all $z \in \mathbb{C}
\setminus \left\{ P_1,\hdots, P_k \right\}$ by Corollary 3.6.3.
\end{Proof}

\newpage
\subsection*{2}
\begin{tcolorbox}
Prove that $f(z) := z\cdot e^{\frac{1}{z}}\cdot e^{-\frac{1}{z^{2}}}$ has an essential singularity at $z = 0$.
\end{tcolorbox}
\begin{Proof}
We will show that $f$ cannot have either a removable singularity or a pole at 0.

\begin{claim}
$f$ does not have a removable singularity at 0.
\end{claim}
\begin{claimproof}
Let $z_n := \frac{i}{n\pi}$ for $n \geq 1$. Then $z_n \rightarrow 0$ and
\[ |f(z_n)| = \frac{1}{n\pi}\left| e^{-in\pi}\right|e^{\pi^2n^2} = \frac{1}{n\pi}e^{n^{2}}\left| \cos(-n\pi) + i\sin(-n\pi)\right| =
\frac{1}{n\pi}e^{n^{2}} \rightarrow +\infty \]
as $n \rightarrow \infty$. So $f$ is unbounded near 0, and so 0 is not a removable singularity.
\end{claimproof}

\begin{claim}
$f$ does not have a pole at 0.
\end{claim}
\begin{claimproof}
Let $z_n := \frac{1}{n}$, $n\geq 1$. Then $z_n \rightarrow 0$ and $|f(z_n)| = \frac{1}{n}e^{n(1-n)} \rightarrow 0$ as $n \rightarrow \infty$. Therefore $f$ can
not have a pole at 0.
\end{claimproof}

By claims 1 and 2, $f$ must have an essential singularity at 0.
\end{Proof}

\newpage
\subsection*{3}
\begin{tcolorbox}
Let $P$ and $Q$ be two polynomials with no zeros in common and let $a$ be a zero of $Q$. Express the residue of $P(z) / Q(z)$ at $a$ in terms of
$P^{(k)}(a)$ and $Q^{(k)}(a)$, $k = 0,1,2,\hdots$ if $Q$ has a zero of order 2 at $a$.
\end{tcolorbox}
Let $f(z) := P(z) / Q(z)$. By Proposition 4.5.6,
\begin{align*}
\text{Res}_{f}(a) = \frac{\partial}{\partial z}\left( (z-a)^{2}f(z) \right)\bigg|_{z = a} & = \lim_{z\rightarrow a}\frac{\partial }{\partial z}\left(
\frac{P(z)}{\frac{Q(z)}{(z-a)^{2}}} \right) \\
& = \lim_{z\rightarrow a}\frac{ \frac{Q(z)}{(z-a)^2}P'(z) - P(z)\left( \frac{ (z-a)^2Q'(z) - 2(z-a)Q(z)}{(z-a)^4} \right)}{\left(
\frac{Q(z)}{(z-a)^2} \right)^4} \\
& = \lim_{z\rightarrow a}\left[ \frac{(z-a)^2}{Q(z)}P'(z) - P(z)\left( \frac{(z-a)^2Q'(z) - 2(z-a)Q(z)}{[Q(z)]^2} \right) \right] \\
& = \lim_{z\rightarrow a}\frac{(z-a)^2Q(z)P'(z) - (z-a)^2Q'(z)P(z) + 2(z-a)Q(z)P(z)}{[Q(z)]^2}
\end{align*}



\newpage
\subsection*{4}
\begin{tcolorbox}
Find the Laurent series for
\[ f(x) = \frac{1}{z(z-1)(z-2)} \]
centered at $z = 0$ and converging in the annulus $\left\{ z : 1 < |z| < 2 \right\}$.
\end{tcolorbox}
We have
\begin{align*}
f(z) = \frac{1}{z(1-z)(z-2)} = \frac{1}{z}\left[ \frac{1}{z - 2} - \frac{1}{z - 1} \right] & = \frac{1}{z}\left[
\left( -\frac{1}{2}\right)\left( \frac{1}{1 - \frac{z}{2}} \right)  - \left( \frac{1}{z} \right)\left( \frac{1}{1 - \frac{1}{z}} \right) \right] \\
& = -\frac{1}{2z}\sum_{n=0}^{\infty}\left( \frac{z}{2} \right)^{n} - \frac{1}{z^{2}}\sum_{n=0}^{\infty}\left( \frac{1}{z} \right)^{n} \\
& = \sum_{n=0}^{\infty}\left( -\frac{1}{2} \right)\frac{z^{n-1}}{2^{n}} + \sum_{n=0}^{\infty}(-1)\frac{1}{z^{n+2}} \\
& = \sum_{n=-\infty}^{-2}(-1)z^{n} + \sum_{n=-1}^{\infty}\left( -\frac{1}{4} \right)\left( \frac{z}{2} \right)^{n}.
\end{align*}

\newpage
\subsection*{5}
\begin{tcolorbox}
Find the function $f(z)$ satisfying the following conditions:
\begin{enumerate}[label=(\alph*)]
\item $f$ has a pole of order 2 as $z = 0$ with residue 2, a simple pole at $z = 1$ with residue 2.
\item $f$ is holomorphic on $\mathbb{C} \setminus \left\{ 0,1 \right\}$.
\item There exists $R$ and $M > 0$ such that $|f(z)| < M$ for all $z$ with $|z| \geq R$.
\item $f(2) = 5$, $f(-1) = 2$.
\end{enumerate}
\end{tcolorbox}
Since $f$ has a pole of order 2 at 0 and a simple pole at 1, we can express $f$ as
\[ f(z) = \frac{1}{z^{2}(z-1)}g(z), \]
where $g$ is holomorphic on $\mathbb{C}$. Then,
\begin{align}
2 = \text{Res}_{f}(0) = \frac{\partial }{\partial z}\left( z^{2}f(z) \right)\bigg|_{z=0} & = \frac{\partial }{\partial z}\left( \frac{g(z)}{z-1}
\right)\bigg|_{z = 0} \nonumber \\
& = \frac{(z-1)g'(z) - g(z)}{(z-1)^{2}}\bigg|_{z=0}\nonumber \\
& = (-1)g'(0) - g(0) \nonumber \\
\Rightarrow  \qquad g(0) & = -2 - g'(0), \text{ and }\label{5.1}
\end{align}
\begin{equation}
2 = \text{Res}_{f}(1) = \frac{1}{z^{2}}f(z)\bigg|_{z = 1} = g(1), \ \text{ i.e. } g(1) = 2.
\label{5.2}
\end{equation}
Further, we must have $f(2) = \frac{1}{4}g(2) = 5$ and $f(-1) = -\frac{1}{2}g(-1) = 2$, implying
\begin{equation}
g(2) = 20, \qquad g(-1) = -4.
\label{5.3}
\end{equation}
Now, if there exists $M, R > 0$ such that $|f(z)| < M$ for all $z$ with $|z| \geq R$, then $|g(z)| < M|z^{2}(z-1)| \leq C_{1}|z|^{3}$ for some $C_{1} > 0$.
We can assume that $R > 1$ since $f$ is unbounded at $z = 1$. Since $g$ is holomorphic on $\mathbb{C}$, there exists some $C_{2} > 0$ such that $|g| <
C_{2}$ on $\left\{ z : |z| \leq R \right\}$. Let $C := \max\left\{ C_1, C_2 \right\}$. Then $|g(z)| < C < C|z|^{3}$ for all $z$ with $|z| > 1$. Hence,
by Theorem 3.4.4, $g$ is a polynomial in $z$ with degree at most $3$. Therefore we can write
\[ g(z) = az^{3} - bz^{2} + cz + d, \]
where $a,b,c,d \in \mathbb{C}$. By \eqref{5.1}, \eqref{5.2}, and \eqref{5.3}, we must have
\[
\begin{bmatrix}
0 & 0 & 1 & 1 \\
1 & 1 & 1 & 1 \\
-1 & 1 & -1 & 1 \\
8 & 4 & 2 & 1 \\
\end{bmatrix}\left( \begin{array}{c}
a \\
b \\
c \\
d \\
\end{array} \right) = \left( \begin{array}{r}
-2 \\
2 \\
-4 \\
20 \\
\end{array} \right)
\]
which implies $a = 1, b = 3, c = 2, d = -4$. Therefore
\[ f(z) = \frac{z^{3} + 3z^{2} + 2z - 4}{z^{2}(z-1)}. \]


\newpage
\subsection*{6}
\begin{tcolorbox}
Let $f(z)$ be meromorphic on $\mathbb{C}$. Assume that $f(z) = f(z + 2i) = f(z + (1-3i))$ for all $z$ where $f(z)$, $f(z + 2i)$, and $f(z + (1-3i)$
are defined. Let $\mathcal{P}$ be the parallelogram with vertices $0, 2i, 1- i$, and $1 - 3i$, and assume that $f$ has no poles on $\mathcal{P}$. Let
$\zeta_{1}, \zeta_{2}, \hdots, \zeta_{n}$ be the poles of $f$ inside $\mathcal{P}$. Prove that
\[ \sum_{k=1}^{n}\text{Res}_{f}(\zeta_{k}) = 0. \]
\end{tcolorbox}
\begin{Proof}
We parameterize $\mathcal{P}$ in the following way. Let
\[
\begin{array}{ll}
\gamma_{1}(t) := 2i - ti, \ \  0 \leq t \leq 2, & \gamma_{2}(t) := (1-3i), \ \ 0 \leq t \leq 1, \\
\gamma_{3}(t) := 1 - 3i + ti, \ \ 0 \leq t \leq 2, & \gamma_{4}(t) := 1 - i + t(-1 + 3i), \ \ 0 \leq t \leq 1.
\end{array}
\]
Let $\gamma = \cup_{i=1}^{4}\gamma_{i}$. By assumption,
\begin{align}
\oint_{\gamma_{1}}f(z)dz = \int_{0}^{2}f(2i - ti)(-i)dt = \int_{0}^{2}f(2 - ti + 1 - 3i)(-i)dt & = \int_{0}^{2}f(1 - i -ti)(-i)dt \nonumber \\
& = -\oint_{\gamma_{3}}f(z)dz, \label{6.1} \\
\oint_{\gamma_{2}}f(z)dz = \int_{0}^{1}f(t(1-3i))(1-3i)dt = \int_{0}^{1}f(t - 3it + 2i)(1 - 3i)dt & = -\oint_{\gamma_{4}}f(z)dz. \label{6.2}
\end{align}
Now note that $\text{Ind}_{\gamma}(\zeta_{k}) = 1$ for all $1 \leq k \leq n$. Hence,
by \eqref{6.1}, \eqref{6.2}, and the residue theorem,
\[ \sum_{k=1}^{n}\text{Res}_{f}(\zeta_{k})  = \frac{1}{2\pi i}\oint_{\gamma}f(z)dz = \frac{1}{2\pi i}\sum_{i=1}^{4}\oint_{\gamma_{i}}f(z)dz = 0. \]
\end{Proof}

\newpage
\subsection*{7}
\begin{tcolorbox}
Evaluate the integral
\[ \int_{0}^{\pi/2}\frac{dx}{9 + 7\sin^{2}(x)}. \]
\end{tcolorbox}
Note that $\sin^2(t) = \frac{1 - \cos(2t)}{2}$, and
\[ \cos(t) = \frac{e^{it} - e^{-it}}{2} = \frac{z - z^{-1}}{2} \]
for $z = e^{it}$, $0 \leq t \leq 2\pi$. Thus,
\begin{align*}
\int_{0}^{\pi/2}\frac{dt}{9 + 7\sin^2(t)} = \int_{0}^{\pi/2}\frac{2dt}{25 - 7\cos(2t)} = \int_{0}^{\pi}\frac{dt}{25 - 7\cos(t)} & =
\frac{1}{2}\int_{0}^{2\pi}\frac{dt}{25 - 7\cos(t)} \\
& = \frac{1}{2}\int_{0}^{2\pi}\frac{dt}{25 - 7\left( \frac{e^{it} - e^{-it}}{2} \right)} \\
& = \frac{1}{2}\oint_{|z| = 1}\frac{dz}{iz\left[ 25 - 7\left( \frac{z + z^{-1}}{2} \right) \right]} \\
& = \frac{1}{2}\oint_{|z| = 1}\frac{dz}{-\frac{i}{2}(7z^2 - 50z + 7)} \\
& = \oint_{|z| = 1}\frac{idz}{7(z - 7)(z - \frac{1}{7})} \\
\text{by the residue theorem } \qquad & = 2\pi i \left( \frac{i}{7(z - 7)} \right)\bigg|_{z = 1/7} = \frac{\pi}{24}.
\end{align*}




\newpage
\subsection*{8}
\begin{tcolorbox}
Let $a > 0$. Compute the value of the integral
\[ \int_{0}^{\infty}\frac{x\sin(x)}{x^2 + a^2}dx. \]
\end{tcolorbox}
Let $f(z) := \frac{ze^{iz}}{z^{2} + a^{2}}$. Then $f$ has a pole at $\pm ia$. For $r > a$, define $\gamma_{r}^{1}(t) := t$, $-r \leq t \leq r$, and
$\gamma_{r}^{2}(t) := re^{it}$, $0 \leq t \leq \pi$. Then $\gamma := \gamma_{r}^{1} \cup \gamma_{r}^{2}$ is the semicircle centered at 0 with radius
$r$ on the top half of the plane. Therefore $f$ has only one (simple) pole at $ia$ within $\gamma$, and so
\begin{equation}
\oint_{\gamma}f(z)dz = 2\pi i \cdot \text{Res}_{f}(ia)\cdot 1 = 2\pi i \frac{ze^{iz}}{(z + ia)}\bigg|_{z=ia} = 2\pi i\frac{ia e^{-a}}{2ia} = \pi i
e^{-a}.
\label{8.0}
\end{equation}
Further,
\begin{equation}
\oint_{\gamma_{r}^{1}}f(z)dz = \int_{-r}^{r}\frac{x\cos(x)}{x^{2} + a^{2}}dx + i \int_{-r}^{r}\frac{x\sin(x)}{x^{2} + a^{2}}dx =
i\int_{-r}^{r}\frac{x\sin(x)}{x^{2} + a^{2}}dx.
\label{8.1}
\end{equation}
On the other hand, for $0 \leq t \leq \pi$,
\[ |f(z)| = \left|\frac{re^{it}e^{ir(\cos(t) + i\sin(t))}}{(re^{it})^{2} + a^{2}}\right| \leq \frac{re^{-r\sin(t)}}{r^{2} + a^{2}} \leq \frac{r}{r^{2}
+ a^{2}}. \]
Therefore
\begin{equation}
\left| \oint_{\gamma_{r}^{2}}f(z)dz \right| \leq \pi r\left( \frac{r}{r^{2} + a^{2}} \right) \rightarrow 0 \ \ \text{ as } r\rightarrow \infty.
\label{8.2}
\end{equation}
By \eqref{8.0}, \eqref{8.1}, and \eqref{8.2},
\[ \int_{0}^{\infty}\frac{x\sin(x)}{x^{2} + a^{2}}dx = \lim_{r\rightarrow\infty}\frac{1}{2}\int_{-r}^{r}\frac{x\sin(x)}{x^{2} + a^{2}}dx = \frac{\pi
e^{-a}}{2}. \]

\newpage
\subsection*{9}
\begin{tcolorbox}
Evaluate the integral
\[ \int_{0}^{\infty}\frac{x^{1/2}}{1 + x^{4}}dx. \]
\end{tcolorbox}
Define $f(z) := \frac{z^{1/2}}{1 + z^{4}}$, where $z^{1/2}$ is defined by $z^{1/2} = (Re^{i\theta})^{1/2} := R^{1/2}e^{i\theta/2}$. Then $z^{1/2}$ is
well-defined and holomorphic except on $\left\{ x : x \leq 0 \right\}$. For $r > 1$, define the contours 
\begin{align*}
& \gamma_{r}^{(1)}(t) := t,\  \frac{1}{r} \leq t \leq r, & \gamma_{r}^{(2)}(t) := re^{it}, \ 0 \leq t \leq \frac{\pi}{2}, \\
& \gamma_{r}^{(3)}(t) := i(r - t), \ 0\leq t \leq r - \frac{1}{r}, & \gamma_{r}^{(4)}(t) := \frac{1}{r}e^{i\left( \frac{\pi}{2} - r \right)},\  0
\leq t \leq \frac{\pi}{2}.
\end{align*}
Let $\gamma_{r} := \cup_{i=1}^{r}\gamma_{r}^{(i)}$. Note that $f$ has one pole of order 1 at $P := \frac{1}{\sqrt{2}}(1 + i) = e^{i\pi / 4}$ in
$\gamma_{r}$, and 
\[\text{Res}_{f}(P) = \lim_{z\rightarrow P}\frac{(z - P)z^{1/2}}{1 + z^{4}} 
= \frac{e^{i\pi / 8}}{\left( \frac{1 + i + 1 - i}{\sqrt{2}} \right)\left( \frac{1 + i + 1 + i}{\sqrt{2}} \right)\left( \frac{1 + i - 1 + i}{\sqrt{2}} \right)} 
= \frac{e^{i\pi / 8}}{\frac{4}{\sqrt{2}}(i - 1)} = \frac{\sqrt{2}}{4}e^{-i\frac{5\pi}{8}}.\]
Thus, by the residue thereom,
\begin{equation}
\oint_{\gamma_{r}}f(z)dz = 2\pi i \cdot \text{Res}_{f}(P) \cdot 1 = \frac{\pi i}{\sqrt{2}}e^{-i\frac{5\pi}{8}} = -\frac{\pi}{\sqrt{2}}
\sin\left( -\frac{5\pi}{8} \right) + i\frac{\pi}{\sqrt{2}}\cos\left( -\frac{5\pi}{8} \right).
\label{9.1}
\end{equation}
Now,
\begin{equation}
\oint_{\gamma_{r}^{(1)}}f(z)dz = \int_{\frac{1}{r}}^{r}\frac{t^{1/2}}{1 + t^{4}}dt,
\label{9.2}
\end{equation}
and 
\begin{equation}
\oint_{\gamma_{r}^{(3)}}f(z)dz = -\int_{\frac{1}{r}}^{r}\frac{(it)^{1/2}}{1 + (it)^{4}}\cdot i \ dt =
-ie^{i\pi/4}\int_{\frac{1}{r}}^{r}\frac{t^{1/2}}{1 + t^{4}}dt = \left(\frac{1}{\sqrt{2}} -i\frac{1}{\sqrt{2}}\right)\int_{\frac{1}{r}}^{r}\frac{t^{1/2}}{1 + t^{4}}dt. 
\label{9.3}
\end{equation}
Further,
\begin{equation}
\left|\oint_{\gamma_{r}^{(2)}}f(z)dz\right| \leq \frac{\pi r}{2}\cdot \frac{r^{1/2}}{1 + r^{4}} \rightarrow 0 \ \text{ as } r\rightarrow \infty,
\label{9.4}
\end{equation}
and similarly,
\begin{equation}
\left|\oint_{\gamma_{r}^{(4)}}f(z)dz\right| = \left|\int_{0}^{\frac{\pi}{2}}\frac{(e^{it}/r)^{1/2}}{1 + e^{i4t}/r^{4}}\cdot \frac{ie^{it}}{r}dt\right|
\leq -r^{-3/2} \cdot \frac{\pi}{2} \rightarrow 0 \ \text{ as } r\rightarrow \infty.
\label{9.5}
\end{equation}
By \eqref{9.1} - \eqref{9.5},
\[ -\frac{\pi}{\sqrt{2}}\sin\left( -\frac{5\pi}{8} \right) + i\frac{\pi}{\sqrt{2}}\cos\left( -\frac{5\pi}{8} \right) = 
\lim_{n\rightarrow \infty}\oint_{\gamma_{r}}f(z)dz = 
\left(\frac{1 + \sqrt{2}}{\sqrt{2}} -i\frac{1}{\sqrt{2}}\right)\int_{0}^{\infty}\frac{t^{1/2}}{1 + t^{4}}dt \]
Therefore $\int_{0}^{\infty}\frac{x^{1/2}}{1 + x^{4}}dx = - \pi \cos \left( -\frac{5\pi}{8} \right) = \pi \sin\left( \frac{\pi}{8} \right)$.



\newpage
\subsection*{10}
\begin{tcolorbox}
Evaluate
\[ \sum_{j=-\infty}^{\infty}\frac{1}{j^{3} + 2}. \]
\end{tcolorbox}
Let
\[ f(z) := \frac{\cot(z)}{\left( \frac{z}{\pi} \right)^{3} + 2}. \]
Note that $f$ has poles of order 1 at $\left\{ j\pi : j \in \mathbb{Z} \right\}\cup \left\{ P_{i} : i \in \left\{ 1,2,3 \right\} \right\}$,
where $P_{1} := 2^{1/3}\pi e^{i\pi/3}, P_{2} := -2^{1/3}\pi$, and $P_{3} := 2^{1/3}\pi e^{i5\pi /3}$.
At $j\pi$, $j \in \mathbb{Z}$,
\begin{align*}
\text{Res}_{f}(j\pi) = \lim_{z \rightarrow j\pi}(z - j\pi)\frac{1}{\left( \frac{z}{\pi} \right)^{3} + 2}\cdot \frac{\cos(z)}{\sin(z)} & =
\lim_{z\rightarrow j\pi} \frac{\cos(z)}{\left( \frac{z}{\pi} \right)^{3} + 2}\cdot \frac{1}{\left( \frac{\sin(z) + \sin(j\pi)}{z - j\pi} \right)} \\
& = \frac{\cos(j\pi)}{j^{3} + 2}\cdot \frac{1}{\cos(j\pi)} \\
& = \frac{1}{j^{3} + 2}.
\end{align*}
Let
\[ M := \sum_{i=1}^{3}\text{Res}_{f}(P_{i}) = \sum_{i=1}^{3}\left( \lim_{z \rightarrow P_{i}} (z - P_{i})\frac{\cot(z)}{\left( \frac{z}{\pi}
\right)^{3} + 2}\right). \]
Let $\gamma_n$ be the contour consisting of a counterclockwise square with corners at $\left\{ (\pm z \pm i)(n+1/2)\pi \right\}$ for $n \geq 2$. For
each $n$, let
$\gamma_{n}^{(1)}$ be the vertical segment on the left, $\gamma_{n}^{(2)}$ be the horizontal segment on the bottom, $\gamma_{n}^{(3)}$ be the vertical
segment on the right, and $\gamma_{n}^{(4)}$ be the horizontal segment on the top.
We claim that 
\begin{equation}
\sum_{j=-\infty}^{\infty}\frac{1}{j^{3} + 2} = - M.
\label{10.1}
\end{equation}
Indeed, since 
\[ \frac{1}{2\pi i}\oint_{\gamma_n}f(z)dz = \sum_{j=-n}^{n}\text{Res}_{f}(j\pi) + \sum_{i=1}^{3}\text{Res}_{f}(P_{i}) = \sum_{j=-n}^{n}\frac{1}{j^{3}
+ 2} + M, \]
it remains to show that $\lim_{n\rightarrow \infty}\oint_{\gamma_{n}}f(z)dz = 0$ in order to verify \eqref{10.1}. Well, in clas we proved that along
$\gamma_{n}^{(1)}$ and $\gamma_{n}^{(3)}$, $|\cot(z)| < 1$, and along $\gamma_{n}^{(2)}$ and $\gamma_{n}^{(4)}$, $|\cot(z)| \leq (1 + e^{-\pi}) / (1 -
e^{-\pi})$. Thus,
\begin{align*}
\oint_{\gamma_{n}}f(z)dz & = \oint_{\gamma_{n}^{(1)}\cup\gamma_{n}^{(3)}}f(z)dz + \oint_{\gamma_{n}^{(2)}\cup\gamma_{n}^{(4)}}f(z)dz \\
& \leq \left(\frac{1}{n^{3} + 2}\right) 4(n + 1/2)\pi + \left( \frac{1 + e^{-\pi}}{1 - e^{-\pi}} \right)\left( \frac{1}{n^{3} + 2} \right)4(n + 1/2)\pi
\rightarrow 0 \ \ \text{ as } n\rightarrow \infty, 
\end{align*}
so we are done.




\end{document}

