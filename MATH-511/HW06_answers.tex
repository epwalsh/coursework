\documentclass[12pt]{article}
\usepackage{amsmath}
\usepackage{amsfonts}
\usepackage{parskip}
\usepackage{amsthm}
\usepackage{thmtools}
\usepackage[headheight=15pt]{geometry}
\geometry{a4paper, left=20mm, right=20mm, top=30mm, bottom=30mm}
\usepackage{graphicx}
\usepackage{bm} % for bold font in math mode - command is \bm{text}
\usepackage{enumitem}
\usepackage{fancyhdr}
\usepackage{amssymb} % for stacked arrows and other shit
\pagestyle{fancy}
\usepackage{changepage}
\usepackage{mathcomp}
\usepackage{tcolorbox}

\declaretheoremstyle[headfont=\normalfont]{normal}
\declaretheorem[style=normal]{Theorem}
\declaretheorem[style=normal]{Proposition}
\declaretheorem[style=normal]{Lemma}
\newcounter{ProofCounter}
\newcounter{ClaimCounter}[ProofCounter]
\newcounter{SubClaimCounter}[ClaimCounter]
\newenvironment{Proof}{\stepcounter{ProofCounter}\textsc{Proof.}}{\hfill$\square$}
\newenvironment{claim}[1]{\vspace{1mm}\stepcounter{ClaimCounter}\par\noindent\underline{\bf Claim \theClaimCounter:}\space#1}{}
\newenvironment{claimproof}[1]{\par\noindent\underline{Proof of claim \theClaimCounter:}\space#1}{\hfill $\blacksquare$ Claim \theClaimCounter}
\newenvironment{subclaim}[1]{\stepcounter{SubClaimCounter}\par\noindent\emph{Subclaim \theClaimCounter.\theSubClaimCounter:}\space#1}{}
% \newenvironment{subclaimproof}[1]{\begin{adjustwidth}{2em}{0pt}\par\noindent\emph{Proof of subclaim \theClaimCounter.\theSubClaimCounter:}\space#1}{\hfill
% $\blacksquare$ \emph{Subclaim \theClaimCounter.\theSubClaimCounter}\vspace{5mm}\end{adjustwidth}}
\newenvironment{subclaimproof}[1]{\par\noindent\emph{Proof of subclaim \theClaimCounter.\theSubClaimCounter:}\space#1}{\hfill
$\Diamond$ \emph{Subclaim \theClaimCounter.\theSubClaimCounter}}

\allowdisplaybreaks{}

% chktex-file 3

\title{MATH 511: HW 6}
\author{Evan P. Walsh}
\makeatletter
\makeatother
\lhead{Evan P. Walsh}
\chead{MATH 511: HW 6}
\rhead{\thepage}
\cfoot{}

\begin{document}
\maketitle


\subsection*{1}
\begin{tcolorbox}
  Suppose $f : \mathbb{C} \rightarrow \mathbb{C}$ is $1-1$ and holomorphic. Prove that $f(z) = az + b$ for some $a \neq 0$, $b \in \mathbb{C}$.  
\end{tcolorbox}
\begin{Proof}
  \begin{claim}
    $f$ is not bounded.
  \end{claim}
  \begin{claimproof}
    If $f$ is bounded then $f \equiv C \in \mathbb{C}$ since $f$ is entire. But then $f$ is not $1-1$, a contradiction.
  \end{claimproof}

  \begin{claim}
    $f$ does not have an essential singularity at $\infty$.
  \end{claim}
  \begin{claimproof}
    By way of contradiction suppose $f$ has an essential singularity at $\infty$. But then $g(z) := f(1/z)$ has an essential singularity at $0$ (note 
    that $g$ is $1-1$ on $\mathbb{C} \setminus \{0\}$ since $f$ is $1-1$). 
    Thus, by the Casorati-Weierstrass Theorem, $g(D(0,R)\setminus \{0\})$ is dense in $\mathbb{C}$ for all $R > 0$. 
    But then $g$ cannot be $1-1$, a contradiction.
  \end{claimproof}

  By claims 1 and 2, $\lim_{|z|\rightarrow\infty}|f(z)| = \infty$, and so by Theorem 4.7.5, $f$ is a polynomial. So it suffices to show that $f$ has
  degree 1. By way of contradiction, suppose the degree of $f$ is greater than 1. Then $f'$ is also a non-constant polyomial with degree at least 1, and
  so has a zero $P \in \mathbb{C}$. But this means that $f$ has a zero at $P$ of degree at least 2. But then by Theorem 5.2.2, there exists $\epsilon,
  \delta > 0$ such that each $q \in D(f(P),\epsilon) \setminus \{f(P)\}$ has at least 2 distinct preimages in $D(P,\delta)$. This is a contradiction.
  Hence $f$ is linear.
\end{Proof}


\newpage
\subsection*{2}
\begin{tcolorbox}
  Let $S$ be the sphere of radius $1$, centered at the origin in $\mathbb{R}^{3}$ and $N = (0,0,1)$. For each point $(x,y,z)$, let $p(x,y,z)$ be the
  point in $\mathbb{C}\cup \{\infty\}$ under the stereographic projection. Let $\gamma$ be a circle on $S$. Prove that $p(\gamma)$ is a line if
  $N \in \gamma$ and $p(\gamma)$ is a circle $N\notin \gamma$, and that every line or circle in $\mathbb{C}\cup\{\infty\}$ is equal to $p(\gamma)$ for
  some circle $\gamma$ on $S$.
\end{tcolorbox}
\begin{Proof}
  Note that by Assignment 1, Question 3,
  \begin{equation}
    p(x,y,z) = \left( \frac{x}{1 - z}, \frac{y}{1 - z}\right) \in \mathbb{C},
    \label{2.1}
  \end{equation}
  for any $(x,y,z) \in S$, and for any $(u,v) \in \mathbb{C}$,
  \begin{equation}
    q(u,v) := \left( \frac{2u}{u^{2} + v^{2} + 1}, \frac{2v}{u^{2} + v^{2} + 1}, \frac{u^{2} + v^{2} -1}{u^{2} + v^{2} + 1}\right) \in S. 
    \label{2.2}
  \end{equation}
  \begin{claim}
    If $\gamma$ is a circle on $S$ then $p(\gamma)$ is a line if $N \in \gamma$ or a circle if $N \notin \gamma$.
  \end{claim}
  \begin{claimproof}
    Since $\gamma$ is a circle on $S$, $\gamma$ is the intersection of a plane with $S$. Thus, there exists scalars $A,B,C,D \in \mathbb{R}$ such that 
    $Ax + By + Cz = D$ for all $(x,y,z) \in \gamma$, where $A^2 + B^2 + C^2 > D^{2}$. Thus by~\eqref{2.1}, if $(u,v) \in p(\gamma)$, we must have
    \[ A\left( \frac{2u}{u^{2} + v^{2} + 1} \right) + B\left( \frac{2v}{u^{2} + v^{2} + 1} \right) + C\left( \frac{u^{2} + v^{2} - 1}{u^{2} + v^{2} +
    1}\right) = D, \]
    which implies 
    \begin{equation}
      (C - D)u^{2} + 2Au + (C-D)v^{2} + 2Bv = D + C. 
      \label{2.3}
    \end{equation}
    Now, if $N \in \gamma$, then $C = D$, so~\eqref{2.3} becomes 
    \begin{equation}
      Au + Bv = C, 
      \label{2.4}
    \end{equation}
    which is the equation of a line in $\mathbb{C}$. If $N \notin C$, then $C \neq D$,
    so we can write~\eqref{2.3} as 
    \[ u^{2} + 2\left( \frac{A}{C - D}\right)u + v^{2} + 2\left( \frac{B}{C-D} \right)v = \frac{C + D}{C - D}, \]
    which, by completing the square, becomes 
    \begin{equation}
      \left( u - \frac{A}{D - C} \right)^{2} + \left( v - \frac{B}{D - C} \right)^{2} = \frac{A^2 + B^2 + C^2 - D^2}{(C - D)^{2}}, 
      \label{2.5}
    \end{equation}
    the equation of a circle in $\mathbb{C}$.
  \end{claimproof}

  \begin{claim}
    Any circle or line $\Gamma \subset \mathbb{C}$ is equal to $p(\gamma)$ for some circle $\gamma$ on $S$.
  \end{claim}
  \begin{claimproof}
    Suppose $\Gamma = \{(u,v) \in \mathbb{C} : (u - u_0)^{2} + (v - v_0)^{2} = R\}$ is a circle. Then take the circle $\gamma \in S$ given by the
    equation $Ax + By + Cz = D$, where $A := u_0$, $B := v_0$, and $C := D + 1$, and $D := R - 1 - u_{0}^{2} - v_{0}^{2}$. Then by~\eqref{2.5} 
    we see that $p(\gamma) = \Gamma$. Similarly, if $\Gamma = \{(u,v) \in \mathbb{C} : Fu + Gv = H\}$ is a line, then take $A := F$, $B := G$, and $C
    := D := H$. Then by~\eqref{2.4} we see that $p(\gamma) = \Gamma$.
  \end{claimproof}

\end{Proof}








\newpage
\subsection*{3}
\begin{tcolorbox}
  Suppose $f : \mathbb{C}\setminus \{0\}\rightarrow \mathbb{C}\setminus\{0\}$ is conformal. Prove that there exists $a \in \mathbb{C} \setminus \{0\}$
  such that $f(z) = az$ or $f(z) = \frac{a}{z}$ for all $z \in \mathbb{C}\setminus \{0\}$.
\end{tcolorbox}
\begin{Proof}
  By the same reasoning as in Question 1 (Claim 2), $f$ cannot have an essential singularity at $0$. So $f$ either has a pole or a removable
  singularity at $0$.  
  \begin{description}
    \item[Case 1:] Suppose $f$ has a removable singularity at $0$

      Then $f$ has a holomorphic extension $g$ on $\mathbb{C}$ defined by
      \[ 
        g(z) := \left\{ \begin{array}{cl}
            f(z) & \text{ if } z \neq 0 \\
            \lim_{\zeta\rightarrow 0}f(\zeta) & \text{ if } z = 0.
        \end{array} \right.
      \]
      \begin{claim}
        $g(0) = 0$
      \end{claim}
      \begin{claimproof}
        By way of contradiction, suppose $g(0) = b \in \mathbb{C}\setminus\{0\}$. Let $z_0 := f^{-1}(b)$ and $r > 0$ such that $z_0 \notin
        \overline{D(0,r)}$ ($z_0$ is well-defined since $f$ is onto $\mathbb{C}\setminus \{0\}$). 
        By the open mapping theorem and since $f$ is non-constant, there exists some $\epsilon > 0$ such that $D(z_0, \epsilon)
        \cap D(0,r) = \emptyset$ and $f[D(z_0,\epsilon)] \subseteq g[D(0,r)]$, which implies
        \[ f[D(z_0,\epsilon)]\setminus\{b\} \subseteq g[D(0,r)]\setminus\{b\} \ \text{ and so } \ f[D(z_0, \epsilon)\setminus\{z_0\}] \subseteq
        g[D(0,r)\setminus\{0\}] = f[D(0,r)\setminus\{0\}]. \]
        This contradicts the fact that $f$ is 1-1. % chktex 8
      \end{claimproof}

      Therefore, by the result in Question 1, $f(z) = az$ for some $a \in \mathbb{C}\setminus\{0\}$.

    \item[Case 2:] Now suppose $\lim_{z\rightarrow 0}|f(z)| = \infty$.

      Then the function $g$ defined by
      \[
        g(z) := \left\{ \begin{array}{cl}
            \frac{1}{f(z)} & \text{ if } z \neq 0 \\
            \\
            0 & \text{ if } z = 0
        \end{array} \right.
      \]
      is well-defined, 1-1, and holomorphic on $\mathbb{C}$. % chktex 8
      Hence, by the result in Question 1, $g(z) = \tilde{a}z$ for some $\tilde{a} \in \mathbb{C}\setminus\{0\}$. Hence $f(z) = \frac{a}{z}$, 
      where $a := \tilde{a}$.
  \end{description}
\end{Proof}


\newpage
\subsection*{4}
\begin{tcolorbox}
  Describe $f(U)$ for each of the following domains $U$ under the indicated maps.
  \begin{enumerate}[label = (\alph*)]
    \item $U = \left\{ z \in \mathbb{C} : \text{Im}(z) > 0 \right\}$, $f(z) = \frac{2z - i}{2 + iz}$.
    \item $U = \left\{ z \in \mathbb{C} : 0 < \text{Re}(z) < 1 \right\}$, $f(z) = \frac{z - 1}{z}$.
  \end{enumerate}
\end{tcolorbox}
\begin{enumerate}[label = (\alph*)]
  \item Note that 
    \[ f(1) = \frac{2 - i}{2 + i} = \frac{3 - 4i}{5} \qquad f(-1) = \frac{-2 - i}{2 - i} = \frac{-3 - 4i}{5}, \]
    and 
    \[ f(0) = \frac{-i}{2} \qquad f(\infty) = -2i. \]
    Therefore $f$ maps the real line to the circle centered at $-\frac{5}{4}i$ with radius $\frac{3}{4}$. But since $f(i) = \frac{i}{1} = i$,
    \[ f[U] = \left\{ z \in \mathbb{C} : \left|z + \frac{5}{4}i\right| > \frac{3}{4} \right\}. \]
  \item Note that 
    \[ f(1 + i) = \frac{i}{1 + i} = \frac{1 + i}{2} \qquad f(1 - i) = \frac{-i}{1 - i} = \frac{1 - i}{2}, \]
    and $f(1) = 0$ while $f(\infty) = 1$. Thus $f$ maps the line $\left\{ z : \text{Re}(z) = 0 \right\}$ to the line $\left\{ z : \text{Re}(z) = 1
    \right\}$ and the line $\left\{ z : \text{Re}(z) = 1 \right\}$ to the circle $\left\{ z : \left|z - \frac{1}{2}\right| = \frac{1}{2} \right\}$.
    But since $f(\frac{1}{2}) = \frac{-\frac{1}{2}}{\frac{1}{2}} = -1$, 
    \[ f[U] = \left\{ z \in \mathbb{C} : \left|z - \frac{1}{2}\right| > \frac{1}{2} \text{ and } \text{Re}({z}) < 1 \right\}. \]
\end{enumerate}


\newpage
\subsection*{5}
\begin{tcolorbox}
  Let $f : \mathbb{C} \cup \{\infty\} \rightarrow \mathbb{C}\cup \{\infty\}$ be a linear fractional transformation. Prove that $f$ has a unique fixed
  point $z_0$ with $z_0 \in \mathbb{C}$ if and only if $\frac{1}{f(z) - z_0} = \frac{1}{z - z_0} + h$, with $h \neq 0$.
\end{tcolorbox}
\begin{Proof}
  Suppose we have $f(z) = \frac{az + b}{cz + d}$, where $a,b,c,d \in \mathbb{C}$. For $f$ to have a unique fixed point in $\mathbb{C}$, we need $c
  \neq 0$, otherwise $f$ would be a linear which implies $f(\infty) = \infty$. Further, by solving $z = \frac{az + b}{cz + d}$, we obtain the equation 
  $cz^{2} + (d - a)z - b = 0$, which offers the solutions given by
  \[ z = \frac{(a-d) \pm \sqrt{(d - a)^{2} + 4bc}}{2c}. \]
  Hence a linear fractional transformation $f$ has a unique fixed point in $\mathbb{C}$ if and only if 
  \begin{equation}
    c \neq 0\qquad \text{ and } \qquad (d - a)^{2} + 4bc = 0,
    \label{5.1}
  \end{equation}
  in which case the solution is given by $z_0 = \frac{a - d}{2c}$.

  \vspace{5mm}
  $(\Leftarrow)$ Now suppose $f$ is a LFT such that 
  \[ \frac{1}{f(z) - z_0} = \frac{1}{z - z_0} + h, \]
  where $h \neq 0$. Then by rearranging we obtain
  \begin{equation}
    f(z) = \frac{(1 + hz_0)z - hz_{0}^{2}}{hz + 1 - hz_{0}}.
    \label{5.2}
  \end{equation}
  Thus, with $a := 1 + hz_{0}, b := -hz_{0}^{2}, c := h$, and $d := 1 - hz_{0}$, we see that $c \neq 0$ and 
  \[ (d - a)^{2} + 4bc = (1 - hz_{0} - 1 - hz_{0})^{2} + 4(-hz_{0}^{2})h = 4h^{2}z_{0}^{2} - 4h^{2}z_{0}{2} = 0. \]
  Hence the conditions in~\eqref{5.1} are satisfied, and $f$ has a unique fixed point in $\mathbb{C}$ given by 
  \[
    \frac{a - d}{2c} = \frac{1 + hz_0 - 1 + hz_0}{2h} = z_0.
  \]

  $(\Rightarrow)$ On the other hand, suppose $f(z) = \frac{az + b}{cz + d}$ has a unique fixed point $z_0 = \frac{a - d}{2c}$. Then let $h := c$.
  By~\eqref{5.1}, $h\neq 0$ and we can write $b = -c\left(\frac{d-a}{2c}\right)^{2} = -hz_{0}^{2}$. Thus $f(z)$ is of the form in~\eqref{5.2}.

\end{Proof}


\newpage
\subsection*{6}
\begin{tcolorbox}
  Find a linear fractional transformation $f$ such that $i$ is the only fixed point and $f(1) = \infty$.
\end{tcolorbox}
For $f(1) = \infty$, we need $f$ to be of the form $f(z) = \frac{az + b}{z - 1}$. Further, for $i$ to be the unique fixed point of $f$, we
need~\eqref{5.1} satisfied, which implies 
\[ (a + 1)^{2} + 4b = 0. \]
But by the work done in Question 5, $i = \frac{1 + 1}{2}$, and so $a = 2i - 1$. Solving for $b$, we get $b = 1$. Hence
\[ f(z) = \frac{(2i - 1)z + 1}{z - 1}. \]


\subsection*{7}
\begin{tcolorbox}
  Find a linear fractional transformation $f$ that takes the upper half plane to itself and satisfies $f(0) = 1$, $f(i) = 2i$.
\end{tcolorbox}
Since this mapping will preserve the symmetry of points about the real axis, we have 
\begin{align*}
  0 & \mapsto 1 \\
  i & \mapsto 2i \\
  -i & \mapsto -2i
\end{align*}
Thus, $(z, 0, i, -i) = (w, 1, 2i, -2i)$ where $w := f(z)$, i.e.
\[ \left( \frac{w - 2i}{w + 2i} \right)\left( \frac{1 + 2i}{1 - 2i} \right) = \left( \frac{z - i}{z + i} \right)\left( \frac{i}{-i} \right). \]
After simplifying, we have 
\[ f(z) = w = \frac{-4z - 2}{z - 2}. \]

\newpage
\subsection*{8}
\begin{tcolorbox}
  Find a linear fractional transformation that takes the half plane $P = \left\{ x + iy : x + y > 2 \right\}$ to $D(0,1)$.
\end{tcolorbox}
Consider the mapping $f$ that takes
\begin{align*}
  2 + 2i & \mapsto 0 \\
  1 + i & \mapsto \frac{1}{\sqrt{2}}(1 + i) \\
  \infty & \mapsto -\frac{1}{\sqrt{2}}(1 + i)
\end{align*}
Then $(z, 2 + 2i, 1 + i, \infty) = \left(w, 0, \frac{1}{\sqrt{2}}(1 + i), -\frac{1}{\sqrt{2}}(1 + i)\right)$, where $w := f(z)$, i.e.
\[ \left( \frac{w - \frac{1}{\sqrt{2}}(1 + i)}{w + \frac{1}{\sqrt{2}}(1 + i)} \right)(-1) = \frac{z - (1 + i)}{1 + i}. \]
After simplifying, we have 
\[ f(z) = w = \frac{-z + 2(1+i)}{\frac{1-i}{\sqrt{2}}z}. \]


\subsection*{9}
\begin{tcolorbox}
  Find a conformal map $f : \left\{ z \in \mathbb{C} : |z - 1| > 1 \text{ and } |z - 2| < 2 \right\} \rightarrow D(0,1)$.
\end{tcolorbox}


\subsection*{10}
\begin{tcolorbox}
  Find a conformal map $f : \left\{ x + iy \in \mathbb{C} : x > 0 \text{ and } 0 < y < 1 \right\} \rightarrow D(0,1)$.
\end{tcolorbox}




\end{document}

