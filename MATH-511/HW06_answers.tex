\documentclass[12pt]{article}
\usepackage{amsmath}
\usepackage{amsfonts}
\usepackage{parskip}
\usepackage{amsthm}
\usepackage{thmtools}
\usepackage[headheight=15pt]{geometry}
\geometry{a4paper, left=20mm, right=20mm, top=30mm, bottom=30mm}
\usepackage{graphicx}
\usepackage{bm} % for bold font in math mode - command is \bm{text}
\usepackage{enumitem}
\usepackage{fancyhdr}
\usepackage{amssymb} % for stacked arrows and other shit
\pagestyle{fancy}
\usepackage{changepage}
\usepackage{mathcomp}
\usepackage{tcolorbox}

\declaretheoremstyle[headfont=\normalfont]{normal}
\declaretheorem[style=normal]{Theorem}
\declaretheorem[style=normal]{Proposition}
\declaretheorem[style=normal]{Lemma}
\newcounter{ProofCounter}
\newcounter{ClaimCounter}[ProofCounter]
\newcounter{SubClaimCounter}[ClaimCounter]
\newenvironment{Proof}{\stepcounter{ProofCounter}\textsc{Proof.}}{\hfill$\square$}
\newenvironment{claim}[1]{\vspace{1mm}\stepcounter{ClaimCounter}\par\noindent\underline{\bf Claim \theClaimCounter:}\space#1}{}
\newenvironment{claimproof}[1]{\par\noindent\underline{Proof of claim \theClaimCounter:}\space#1}{\hfill $\blacksquare$ Claim \theClaimCounter}
\newenvironment{subclaim}[1]{\stepcounter{SubClaimCounter}\par\noindent\emph{Subclaim \theClaimCounter.\theSubClaimCounter:}\space#1}{}
% \newenvironment{subclaimproof}[1]{\begin{adjustwidth}{2em}{0pt}\par\noindent\emph{Proof of subclaim \theClaimCounter.\theSubClaimCounter:}\space#1}{\hfill
% $\blacksquare$ \emph{Subclaim \theClaimCounter.\theSubClaimCounter}\vspace{5mm}\end{adjustwidth}}
\newenvironment{subclaimproof}[1]{\par\noindent\emph{Proof of subclaim \theClaimCounter.\theSubClaimCounter:}\space#1}{\hfill
$\Diamond$ \emph{Subclaim \theClaimCounter.\theSubClaimCounter}}

\allowdisplaybreaks{}

% chktex-file 3

\title{MATH 511: HW 6}
\author{Evan P. Walsh}
\makeatletter
\makeatother
\lhead{Evan P. Walsh}
\chead{MATH 511: HW 6}
\rhead{\thepage}
\cfoot{}

\begin{document}
\maketitle


\subsection*{1}
\begin{tcolorbox}
  Suppose $f : \mathbb{C} \rightarrow \mathbb{C}$ is $1-1$ and holomorphic. Prove that $f(z) = az + b$ for some $a \neq 0$, $b \in \mathbb{C}$.  
\end{tcolorbox}
\begin{Proof}
  \begin{claim}
    $f$ is not bounded.
  \end{claim}
  \begin{claimproof}
    If $f$ is bounded then $f \equiv C \in \mathbb{C}$ since $f$ is entire. But then $f$ is not $1-1$, a contradiction.
  \end{claimproof}

  \begin{claim}
    $f$ does not have an essential singularity at $\infty$.
  \end{claim}
  \begin{claimproof}
    By way of contradiction suppose $f$ has an essential singularity at $\infty$. But then $g(z) := f(1/z)$ has an essential singularity at $0$ (note 
    that $g$ is $1-1$ on $\mathbb{C} \setminus \{0\}$ since $f$ is $1-1$). 
    Thus, by the Casorati-Weierstrass Theorem, $g(D(0,R)\setminus \{0\})$ is dense in $\mathbb{C}$ for all $R > 0$. 
    But then $g$ cannot be $1-1$, a contradiction.
  \end{claimproof}

  By claims 1 and 2, $\lim_{|z|\rightarrow\infty}|f(z)| = \infty$, and so by Theorem 4.7.5, $f$ is a polynomial. So it suffices to show that $f$ has
  degree 1. By way of contradiction, suppose the degree of $f$ is greater than 1. Then $f'$ is also a non-constant polyomial with degree at least 1, and
  so has a zero $P \in \mathbb{C}$. But this means that $f$ has a zero at $P$ of degree at least 2. But then by Theorem 5.2.2, there exists $\epsilon,
  \delta > 0$ such that each $q \in D(f(P),\epsilon) \setminus \{f(P)\}$ has at least 2 distinct preimages in $D(P,\delta)$. This is a contradiction.
  Hence $f$ is linear.
\end{Proof}







\end{document}

