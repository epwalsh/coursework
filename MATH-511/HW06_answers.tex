\documentclass[12pt]{article}
\usepackage{amsmath}
\usepackage{amsfonts}
\usepackage{parskip}
\usepackage{amsthm}
\usepackage{thmtools}
\usepackage[headheight=15pt]{geometry}
\geometry{a4paper, left=20mm, right=20mm, top=30mm, bottom=30mm}
\usepackage{graphicx}
\usepackage{bm} % for bold font in math mode - command is \bm{text}
\usepackage{enumitem}
\usepackage{fancyhdr}
\usepackage{amssymb} % for stacked arrows and other shit
\pagestyle{fancy}
\usepackage{changepage}
\usepackage{mathcomp}
\usepackage{tcolorbox}

\declaretheoremstyle[headfont=\normalfont]{normal}
\declaretheorem[style=normal]{Theorem}
\declaretheorem[style=normal]{Proposition}
\declaretheorem[style=normal]{Lemma}
\newcounter{ProofCounter}
\newcounter{ClaimCounter}[ProofCounter]
\newcounter{SubClaimCounter}[ClaimCounter]
\newenvironment{Proof}{\stepcounter{ProofCounter}\textsc{Proof.}}{\hfill$\square$}
\newenvironment{claim}[1]{\vspace{1mm}\stepcounter{ClaimCounter}\par\noindent\underline{\bf Claim \theClaimCounter:}\space#1}{}
\newenvironment{claimproof}[1]{\par\noindent\underline{Proof of claim \theClaimCounter:}\space#1}{\hfill $\blacksquare$ Claim \theClaimCounter}
\newenvironment{subclaim}[1]{\stepcounter{SubClaimCounter}\par\noindent\emph{Subclaim \theClaimCounter.\theSubClaimCounter:}\space#1}{}
% \newenvironment{subclaimproof}[1]{\begin{adjustwidth}{2em}{0pt}\par\noindent\emph{Proof of subclaim \theClaimCounter.\theSubClaimCounter:}\space#1}{\hfill
% $\blacksquare$ \emph{Subclaim \theClaimCounter.\theSubClaimCounter}\vspace{5mm}\end{adjustwidth}}
\newenvironment{subclaimproof}[1]{\par\noindent\emph{Proof of subclaim \theClaimCounter.\theSubClaimCounter:}\space#1}{\hfill
$\Diamond$ \emph{Subclaim \theClaimCounter.\theSubClaimCounter}}

\allowdisplaybreaks{}

% chktex-file 3

\title{MATH 511: HW 6}
\author{Evan P. Walsh}
\makeatletter
\makeatother
\lhead{Evan P. Walsh}
\chead{MATH 511: HW 6}
\rhead{\thepage}
\cfoot{}

\begin{document}
\maketitle


\subsection*{1}
\begin{tcolorbox}
  Suppose $f : \mathbb{C} \rightarrow \mathbb{C}$ is $1-1$ and holomorphic. Prove that $f(z) = az + b$ for some $a \neq 0$, $b \in \mathbb{C}$.  
\end{tcolorbox}
\begin{Proof}
  \begin{claim}
    $f$ is not bounded.
  \end{claim}
  \begin{claimproof}
    If $f$ is bounded then $f \equiv C \in \mathbb{C}$ since $f$ is entire. But then $f$ is not $1-1$, a contradiction.
  \end{claimproof}

  \begin{claim}
    $f$ does not have an essential singularity at $\infty$.
  \end{claim}
  \begin{claimproof}
    By way of contradiction suppose $f$ has an essential singularity at $\infty$. But then $g(z) := f(1/z)$ has an essential singularity at $0$ (note 
    that $g$ is $1-1$ on $\mathbb{C} \setminus \{0\}$ since $f$ is $1-1$). 
    Thus, by the Casorati-Weierstrass Theorem, $g(D(0,R)\setminus \{0\})$ is dense in $\mathbb{C}$ for all $R > 0$. 
    But then $g$ cannot be $1-1$, a contradiction.
  \end{claimproof}

  By claims 1 and 2, $\lim_{|z|\rightarrow\infty}|f(z)| = \infty$, and so by Theorem 4.7.5, $f$ is a polynomial. So it suffices to show that $f$ has
  degree 1. By way of contradiction, suppose the degree of $f$ is greater than 1. Then $f'$ is also a non-constant polyomial with degree at least 1, and
  so has a zero $P \in \mathbb{C}$. But this means that $f$ has a zero at $P$ of degree at least 2. But then by Theorem 5.2.2, there exists $\epsilon,
  \delta > 0$ such that each $q \in D(f(P),\epsilon) \setminus \{f(P)\}$ has at least 2 distinct preimages in $D(P,\delta)$. This is a contradiction.
  Hence $f$ is linear.
\end{Proof}


\newpage
\subsection*{2}
\begin{tcolorbox}
  Let $S$ be the sphere of radius $1$, centered at the origin in $\mathbb{R}^{3}$ and $N = (0,0,1)$. For each point $(x,y,z)$, let $p(x,y,z)$ be the
  point in $\mathbb{C}\cup \{\infty\}$ under the stereographic projection. Let $\gamma$ be a circle on $S$. Prove that $p(\gamma)$ is a line if
  $N \in \gamma$ and $p(\gamma)$ is a circle $N\notin \gamma$, and that every line or circle in $\mathbb{C}\cup\{\infty\}$ is equal to $p(\gamma)$ for
  some circle $\gamma$ on $S$.
\end{tcolorbox}
\begin{Proof}
  Note that by Assignment 1, Question 3,
  \begin{equation}
    p(x,y,z) = \left( \frac{x}{1 - z}, \frac{y}{1 - z}\right) \in \mathbb{C},
    \label{2.1}
  \end{equation}
  for any $(x,y,z) \in S$, and for any $(u,v) \in \mathbb{C}$,
  \begin{equation}
    q(u,v) := \left( \frac{2u}{u^{2} + v^{2} + 1}, \frac{2v}{u^{2} + v^{2} + 1}, \frac{u^{2} + v^{2} -1}{u^{2} + v^{2} + 1}\right) \in S. 
    \label{2.2}
  \end{equation}
  \begin{claim}
    If $\gamma$ is a circle on $S$ then $p(\gamma)$ is a line if $N \in \gamma$ or a circle if $N \notin \gamma$.
  \end{claim}
  \begin{claimproof}
    Since $\gamma$ is a circle on $S$, $\gamma$ is the intersection of a plane with $S$. Thus, there exists scalars $A,B,C,D \in \mathbb{R}$ such that 
    $Ax + By + Cz = D$ for all $(x,y,z) \in \gamma$, where $A^2 + B^2 + C^2 > D^{2}$. Thus by~\eqref{2.1}, if $(u,v) \in p(\gamma)$, we must have
    \[ A\left( \frac{2u}{u^{2} + v^{2} + 1} \right) + B\left( \frac{2v}{u^{2} + v^{2} + 1} \right) + C\left( \frac{u^{2} + v^{2} - 1}{u^{2} + v^{2} +
    1}\right) = D, \]
    which implies 
    \begin{equation}
      (C - D)u^{2} + 2Au + (C-D)v^{2} + 2Bv = D + C. 
      \label{2.3}
    \end{equation}
    Now, if $N \in \gamma$, then $C = D$, so~\eqref{2.3} becomes 
    \begin{equation}
      Au + Bv = C, 
      \label{2.4}
    \end{equation}
    which is the equation of a line in $\mathbb{C}$. If $N \notin C$, then $C \neq D$,
    so we can write~\eqref{2.3} as 
    \[ u^{2} + 2\left( \frac{A}{C - D}\right)u + v^{2} + 2\left( \frac{B}{C-D} \right)v = \frac{C + D}{C - D}, \]
    which, by completing the square, becomes 
    \begin{equation}
      \left( u - \frac{A}{D - C} \right)^{2} + \left( v - \frac{B}{D - C} \right)^{2} = \frac{A^2 + B^2 + C^2 - D^2}{(C - D)^{2}}, 
      \label{2.5}
    \end{equation}
    the equation of a circle in $\mathbb{C}$.
  \end{claimproof}

  \begin{claim}
    Any circle or line $\Gamma \subset \mathbb{C}$ is equal to $p(\gamma)$ for some circle $\gamma$ on $S$.
  \end{claim}
  \begin{claimproof}
    Suppose $\Gamma = \{(u,v) \in \mathbb{C} : (u - u_0)^{2} + (v - v_0)^{2} = R\}$ is a circle. Then take the circle $\gamma \in S$ given by the
    equation $Ax + By + Cz = D$, where $A := u_0$, $B := v_0$, and $C := D + 1$, and $D := R - 1 - u_{0}^{2} - v_{0}^{2}$. Then by~\eqref{2.5} 
    we see that $p(\gamma) = \Gamma$. Similarly, if $\Gamma = \{(u,v) \in \mathbb{C} : Fu + Gv = H\}$ is a line, then take $A := F$, $B := G$, and $C
    := D := H$. Then by~\eqref{2.4} we see that $p(\gamma) = \Gamma$.
  \end{claimproof}

\end{Proof}







\end{document}

