\documentclass[12pt]{article}
\usepackage{amsmath}
\usepackage{amsfonts}
\usepackage{parskip}
\usepackage{amsthm}
\usepackage{thmtools}
\usepackage[headheight=15pt]{geometry}
\geometry{a4paper, left=20mm, right=20mm, top=30mm, bottom=30mm}
\usepackage{graphicx}
\usepackage{bm} % for bold font in math mode - command is \bm{text}
\usepackage{enumitem}
\usepackage{fancyhdr}
\usepackage{amssymb} % for stacked arrows and other shit
\pagestyle{fancy}
\usepackage{changepage}
\usepackage{mathcomp}
\usepackage{tcolorbox}

\declaretheoremstyle[headfont=\normalfont]{normal}
\declaretheorem[style=normal]{Theorem}
\declaretheorem[style=normal]{Proposition}
\declaretheorem[style=normal]{Lemma}
\newcounter{ProofCounter}
\newcounter{ClaimCounter}[ProofCounter]
\newcounter{SubClaimCounter}[ClaimCounter]
\newenvironment{Proof}{\stepcounter{ProofCounter}\textsc{Proof.}}{\hfill$\square$}
\newenvironment{claim}[1]{\vspace{1mm}\stepcounter{ClaimCounter}\par\noindent\underline{\bf Claim \theClaimCounter:}\space#1}{}
\newenvironment{claimproof}[1]{\par\noindent\underline{Proof of claim \theClaimCounter:}\space#1}{\hfill $\blacksquare$ Claim \theClaimCounter}
\newenvironment{subclaim}[1]{\stepcounter{SubClaimCounter}\par\noindent\emph{Subclaim \theClaimCounter.\theSubClaimCounter:}\space#1}{}
% \newenvironment{subclaimproof}[1]{\begin{adjustwidth}{2em}{0pt}\par\noindent\emph{Proof of subclaim \theClaimCounter.\theSubClaimCounter:}\space#1}{\hfill
% $\blacksquare$ \emph{Subclaim \theClaimCounter.\theSubClaimCounter}\vspace{5mm}\end{adjustwidth}}
\newenvironment{subclaimproof}[1]{\par\noindent\emph{Proof of subclaim \theClaimCounter.\theSubClaimCounter:}\space#1}{\hfill
$\Diamond$ \emph{Subclaim \theClaimCounter.\theSubClaimCounter}}

\title{MATH 511: HW 2}
\author{Evan P. Walsh}
\makeatletter
\let\runauthor\@author
\let\runtitle\@title
\makeatother
\lhead{\runauthor}
\chead{\runtitle}
\rhead{\thepage}
\cfoot{}

\begin{document}
% \maketitle


% \def\IF{{\bf F}}
\def\IC{{\bf C}}
% \def\IR{{\bf R}}
% \def\IRd{\IR_{\downarrow}}
% \def\cH{{\cal H}}
% \def\cL{{\cal L}}
% \def\cB{{\cal B}}
% \def\cD{{\cal D}}
% \def\cO{{\cal O}}
% \def\cF{{\cal F}}
% \def\cP{{\cal P}}
% \def\cS{{\cal S}}
% \def\cT{{\cal T}}
% \def\cU{{\cal U}}
% \def\cV{{\cal V}}
% \def\ba{{\bf a}}
% \def\bb{{\bf b}}
% \def\bc{{\bf c}}
% \def\bd{{\bf d}}
% \def\bs{{\bf s}}
% \def\xx{{\bf x}}
% \def\yy{{\bf y}}
% \def\tt{\theta}
% \def\bV{{\bf V}}
% \def\dim{{\rm dim}\,}
% \def\diag{{\rm diag}\,}
% \def\span{{\rm span}\,}
% \def\Ra{{\Rightarrow}}
\def\({\left (}
\def\){\right )}
% \def\ra{{\rightarrow}}
% \def\Ra{{\ \Rightarrow\ }}
% \def\lra{{\leftrightarrow}}
% \def\Lra{{\ \Leftrightarrow\ }}
% \def\dfrac{\displaystyle\frac}
\def\zb{\overline z}
\def\dP{\partial}


\subsection*{1}
\begin{tcolorbox}
Evaluate
$\oint_\gamma \dfrac{\overline{z}}{8+z}\,dz$, where $\gamma$ is the
rectangle with vertices $\pm 3\pm i$ with counter-clockwise
direction.
\end{tcolorbox}
Let 
\begin{align*}
& \gamma_{1}(t) := 3 - i + it, \ 0 \leq t \leq 2 \\
& \gamma_{2}(t) := 3 + i - t,  \ 0 \leq t \leq 6 \\
& \gamma_{3}(t) := -3 - i - it, \ 0 \leq t \leq 2 \\
& \gamma_{4}(t) := -3 - i + t, \ 0 \leq t \leq 6.
\end{align*}
First we will evaluate the line integral along $\gamma_{1}$. We have 
\begin{align*}
\oint_{\gamma_{1}}\frac{\bar{z}}{8 + z}dz & = \int_{0}^{2}\frac{\overline{3 - i + it}}{8 + (3-i + it)}\cdot i\  dt \\
& = \int_{0}^{2} 
\left[ \frac{3i - (t-1)}{11i + t - 1}\right]\times \left[ \frac{(t-1) - 11i}{(t-1) - 11i}\right]dt \\
& = \int_{0}^{2}\frac{3(t-1)i - (t-1)^2 + (t-1)11i + 33 }{(t-1)^{2} + 121}dt \\
(u:= t - 1)\  \ & = \underbrace{\int_{-1}^{1}\frac{3iu}{u^{2} + 121}du}_{A} - \underbrace{\int_{-1}^{1}\frac{u^{2}}{u^{2} + 121}du}_{B} + 
\underbrace{\int_{-1}^{1}\frac{11iu}{u^{2}+121}du}_{C} +
\underbrace{\int_{-1}^{1}\frac{33}{u^{2} + 121}du}_{D}.
\end{align*}
Then $A = C = 0$ since the function $u / (u^{2} + 121)$ is symmetric. For $B$, we have 
\[ B = \int_{-1}^{1}\left[ 1 - \frac{121}{u^{2} + 121}\right]du = \left[ u - 11\tan^{-1}\left( \frac{u}{11} \right)\right]_{-1}^{1}  = 2 -
22\tan^{-1}\left( \frac{1}{11} \right), \]
and similarly for $D$,
\[ D = 33 \int_{-1}^{1}\frac{1}{u^{2} + 121}du = 33\left[ \frac{1}{11}\tan^{-1}\left( \frac{u}{11} \right)\right]_{-1}^{1} = 6\tan^{-1}\left( 
\frac{1}{11}\right). \]
Thus $\oint_{\gamma_{1}}\frac{\bar{z}}{8 + z}dz = -2 + 28\tan^{-1}\left( \frac{1}{11} \right)$. Similarly, we find 
\begin{align*}
& \oint_{\gamma_{2}}\frac{\bar{z}}{8 + z}dz = \int_{0}^{6}\frac{t - 3 + i}{11 + i - t}dt = -6 + (1-4i)\tan^{-1}\left( \frac{168}{775}\right) + (8 +
2i)\tanh^{-1}\left( \frac{24}{37} \right)\\
& \oint_{\gamma_{3}}\frac{\bar{z}}{8 + z}dz = \int_{0}^{2}\frac{1 + t - 3i}{5 - i(1 + t)}dt = 2i + 4\log\left( \frac{171}{169} - \frac{140i}{169} \right) \\
& \oint_{\gamma_{4}}\frac{\bar{z}}{8 + z}dz = \int_{0}^{6}\frac{-3 + t + i}{5 + t - i}dt =  6 + (4 - i)\log\left( \frac{775}{3721} - \frac{168i}{3721}
\right).  \\
\end{align*}
So $\oint_{\gamma}\frac{\bar{z}}{8 + z}dz = \sum_{i=1}^{4}\oint_{\gamma_{i}}\frac{\bar{z}}{8 + z}dz$.

\newpage
\subsection*{2}
\begin{tcolorbox}
Let $f(x+iy)=x^3y^2+i\,x^2y^3$. Find all the points $P$ in $\mathbb{C}$ where $f'(P)$ exists.
\end{tcolorbox}
Let $u := x^{3}y^{2}$ and $v := x^{2}y^{3}$. Since the derivates of $u$ and $v$ exist everywhere on $\mathbb{R}^{2}$, it suffices to find the points
where the Cauchy-Riemann equations hold. Thus, we must have
\begin{equation}
\frac{\dP u}{\dP x} = 3x^{2}y^{2} = \frac{\dP v}{\dP y} = 3x^{2}y^{2},
\label{2.1}
\end{equation}
and 
\begin{equation}
\frac{\dP u}{\dP y} = 2x^{3}y = -\frac{\dP v}{\dP x} = -2xy^{3}.
\label{2.2}
\end{equation}
Focusing on \eqref{2.2} since \eqref{2.1} is trivial, we see that we must have $x(x^{2} + y^{2})y = 0$, or $x^{2} + y^{2} = 0$. This is satisfied
whenever either $x$ or $y$ is 0.


\subsection*{3}
\begin{tcolorbox}
Let $U\subset \mathbb{C}$ be an open set  and  $f\in C^1(U)$. Let $z_0\in U$.
\begin{enumerate}[label=(\alph*),itemsep=1ex]
\item Suppose $|D_{e^{i\theta_1}}f(z_0)|=|D_{e^{i\theta_2}}f(z_0)|$  for all $\theta_1,\ \theta_2$.  Prove that either $f'(z_0)$ or $(\overline{f})'(z_0)$ exists. (Note: $(\overline{f})'\neq    \overline{(f')}$. )
\item Suppose $D_{e^{i\theta_1}}f(z_0) \overline{  D_{e^{i\theta_2}}f(z)}=|D_{e^{i\theta_1}}f(z_0)   D_{e^{i\theta_2}}f(z_0)|e^{i(\theta_1-\theta_2)}$ for all $\theta_1,\ \theta_2$. Prove that   $f'(z_0)$ exists.
\end{enumerate}
\end{tcolorbox}
Over the course problem assume $f = u + iv$. We will denote $u_x = \frac{\dP u}{\dP x}(z_{0})$ and similarly for $u_y$, $v_x$, and $v_y$. 
\begin{enumerate}[label=(\alph*),itemsep=1ex]
\item \begin{Proof}
Suppose $|D_{e^{i\theta_1}}f(z_0)|=|D_{e^{i\theta_2}}f(z_0)|$ for all $\theta_1$ and $\theta_2$. Then 
\begin{equation}
D_{e^{i\theta_1}}f(z_{0})\cdot \overline{ D_{e^{i\theta_1}}f(z_0)} = |D_{e^{i\theta_1}}f(z_0)|^{2} = |D_{e^{i\theta_2}}f(z_0)|^{2} =
D_{e^{i\theta_2}}f(z_{0})\cdot \overline{ D_{e^{i\theta_2}}f(z_0)}.
\label{3.1}
\end{equation}
The left side of \eqref{3.1} is then 
\begin{align*}
(u_x \cos \theta_1 + u_y \sin \theta_1)^2 + (v_x \cos\theta_1 + v_y\sin \theta_1)^2 & = (u_x\cos\theta_1)^2 + (u_y\sin\theta_1)^2 +
2u_xu_y\cos\theta_1\sin\theta_1 \\
& \ + (v_x\cos\theta_1)^2 + (v_y\sin\theta_1)^2 + 2v_x v_y\cos\theta_1\sin\theta_1 
\end{align*}
Similarly, the right side of \eqref{3.1} is
\begin{align*}
(u_x \cos \theta_2 + u_y \sin \theta_2)^2 + (v_x \cos\theta_2 + v_y\sin \theta_2)^2 & = (u_x\cos\theta_2)^2 + (u_y\sin\theta_2)^2 +
2u_xu_y\cos\theta_2\sin\theta_2 \\
& \ + (v_x\cos\theta_2)^2 + (v_y\sin\theta_2)^2 + 2v_x v_y\cos\theta_2\sin\theta_2 
\end{align*}
Then it's clear that in order for both sides to not depend on $\theta_i$, we need either 
\begin{equation}
u_x = v_y \qquad \text{ and } \qquad u_y = -v_x, 
\label{3.2}
\end{equation}
or 
\begin{equation}
u_x = -v_y \qquad \text{ and } \qquad u_y = v_x. 
\label{3.3}
\end{equation}
If either \eqref{3.2} or \eqref{3.3} holds, the left side of \eqref{3.1} now becomes 
\[ u_x(\cos^2\theta_1 + \sin^2\theta_1) + u_y^2(\cos^2\theta_1 + \sin^2\theta_1) = u_x^2 + u_y^2, \]
and the right side also becomes $u_x^2 + u_y^2$. But \eqref{3.2} implies that $f'(z_0)$ exists, while \eqref{3.3} implies that $(\bar{f})'(z_0)$ exists.
\end{Proof}
\item \begin{Proof}
Suppose 
\begin{equation}
D_{e^{i\theta_1}}f(z_0) \overline{  D_{e^{i\theta_2}}f(z)}=|D_{e^{i\theta_1}}f(z_0)   D_{e^{i\theta_2}}f(z_0)|e^{i(\theta_1-\theta_2)} 
\label{3.4}
\end{equation}
for all $\theta_1,\ \theta_2$. Then by squaring both sides of \eqref{3.4}, we have 
\[ (D_{e^{i\theta_1}}f(z_0))^2(\overline{D_{e^{i\theta_2}}f(z_0)})^2 = D_{e^{i\theta_1}}f(z_0)\overline{D_{e^{i\theta_1}}f(z_0)}\times D_{e^{i\theta_2}}f(z_0)
\overline{D_{e^{i\theta_2}}f(z_0)}\times e^{2i(\theta_1 - \theta_2)}. \]
But this implies 
\begin{equation}
\overline{ \overline{D_{e^{i\theta_1}}f(z_0)}D_{e^{i\theta_2}}f(z_0)e^{i(\theta_1-\theta_2)}} = 
\overline{D_{e^{i\theta_1}}f(z_0)}D_{e^{i\theta_2}}f(z_0)e^{i(\theta_1-\theta_2)}.
\label{3.5}
\end{equation}
But \eqref{3.5} holds if and only if both sides of \eqref{3.5} have no imaginary part. If we expand the left side, we get 
\begin{align*}
[(u_x\cos \theta_1 + u_y\sin\theta_1)-&i(v_x\cos\theta_1+v_y\sin\theta_1)][(u_x\cos\theta_2+u_y\sin\theta_2)+i(v_x\cos\theta_2+v_y\sin\theta_2)] \\
&\times [(\cos\theta_1\cos\theta_2 + \sin\theta_1\sin\theta_2) + i(\sin\theta_1\cos\theta_2 - \cos\theta_1\sin\theta_2)], 
\end{align*}
and the imaginary part is 
\begin{align*}
(\sin\theta_1\cos\theta_2 &- \cos\theta_1\sin\theta_2)\bigg[(u_x\cos\theta_1 + u_y\sin\theta_1)(u_x\cos\theta_2 + u_y\sin\theta_2) \\
& + (v_x\cos\theta_1 + v_y\sin\theta_1)(v_x\cos\theta_2 + v_y\sin\theta_2)\bigg] \\
& + (\cos\theta_1\cos\theta_2 + \sin\theta_1\sin\theta_2)\bigg[(u_x\cos\theta_1 + u_y\sin\theta_1)(v_x\cos\theta_2 + v_y\sin\theta_2) \\
& - (v_x\cos\theta_1 + v_y\sin\theta_1)(u_x\cos\theta_2 + u_y\sin\theta_2)\bigg]
\end{align*}
By expanding further we find the following terms:
\[ (u_xu_x - u_xv_y)\sin\theta_1\cos\theta_1\cos^2\theta_2 \qquad \text{and}\qquad (u_x u_y + v_xv_y) \cos\theta_1\sin\theta_2. \]
The first term will only cancel out if $u_x = v_y$. With that, the second term cancels out only if $u_y = -v_x$, i.e. the Cauchy-Riemann equations are satisfied. 
It follows that the rest of the terms in the imaginary part also cancel out. The implies $f'(z_0)$ exists.
\end{Proof}
\end{enumerate}


\newpage
\subsection*{4}
\begin{tcolorbox}
Let $u$ be a real-valued $C^1$ function on an open disc $U$ with center $0$. Assume that $u$ is harmonic on
$U\setminus\{0\}$. Prove that $u$ is the real part of a holomorphic
function on $U$.
\end{tcolorbox}
\begin{Proof}
Set $f := -\frac{\partial u}{\partial y}$ and $g := \frac{\dP u}{\partial x}$. Therefore $\frac{\dP f}{\dP y} = \frac{\dP g}{\dP x}$ and $f$ and $g$
are continuously differentiably on $U \setminus \left\{ 0 \right\}$. So by Theorem 2.3.2, there exists some $C^{1}$ function $v : U \rightarrow
\mathbb{R}$ such that 
\[ \frac{\dP v}{\dP x} = f \qquad \text{ and } \qquad \frac{\dP v}{\dP y} = g. \]
Hence $F := u + iv$ is $C^{1}$ and satisfies the Cauchy-Riemann equations.
\end{Proof}


\subsection*{5}
\begin{tcolorbox}
Evaluate
$\oint_\gamma \dfrac{\zeta^2+8i}{(\zeta+i)(\zeta-8)}\,d\zeta$,
where $\gamma$ is the circle with center $2+ i$ and radius $3$ with
counter-clockwise direction.
\end{tcolorbox}
Using the Cauchy Integral Formula,
\begin{align*}
\oint_{\gamma}\frac{\zeta^2 + 8i}{(\zeta + i)(\zeta - 8)}d\zeta = \oint_{\gamma} \frac{(\zeta^2 + 8i)/(\zeta - 8)}{\zeta + i}d\zeta & = 2\pi i
\left[ \frac{(-i)^2 + 8i}{-i - 8} \right] \\
& = 2\pi i \left[ \frac{-1 + 8i}{-i - 8} \right] = 2\pi.
\end{align*}

\subsection*{6}
\begin{tcolorbox}
Let $\gamma(t) = e^{it}$, $0\le t \le 2\pi$ and $\psi$ be a
complex continuous function on $\gamma$. Show that
$$\overline{\oint_\gamma\psi(z)\,dz}=-\oint_\gamma\overline{\psi(z)\,z^2}\,dz.$$
\end{tcolorbox}
Let $\psi(z) = u(z) + iv(z)$. Then
\begin{align*}
\overline{ \oint_{\gamma}\psi(z)\ dz} = \overline{ \int_{0}^{2\pi}\psi(e^{it})ie^{it}\ dt}& = \overline{\int_{0}^{2\pi}[u(e^{-it}) + iv(e^{it})][i\cos
t - \sin t]\ dt} \\
& = \overline{-\int_{0}^{2\pi}[u(e^{it})\sin t+v(e^{it})\cos t]\ dt + i\int_{0}^{2\pi}[u(e^{it})\cos t - v(e^{it})\sin t]\ dt} \\
& = -\int_{0}^{2\pi}[u(e^{it}) - iv(e^{it})][i\cos t + \sin t]\ dt \\
& = -\int_{0}^{2\pi}[u(e^{it}) - iv(e^{it})]i[\cos t - i\sin t]\ dt \\
& = -\int_{0}^{2\pi}[u(e^{it}) - iv(e^{it})]i\overline{e^{it}}\overline{e^{it}}e^{it}\ dt \\
& = -\int_{0}^{2\pi}\overline{\psi(e^{it})}\overline{e^{2it}}ie^{it}\ dt \\
& = -\oint_{\gamma}\overline{\psi(z)z^{2}}\ dz.
\end{align*}


\subsection*{7}
\begin{tcolorbox}
Let $f(z)=z^2$. Show that the  integral of $f$ around the
circle $\partial D(2,1)$ given by
$$\int_0^{2\pi}f\(2+e^{i\theta}\)\,d\theta$$ is not zero. Yet the
Cauchy integral theorem asserts that $$\oint_{\partial
D(2,1)}f(\zeta)\,d\,\zeta=0\,.$$ Give an explanation.
\end{tcolorbox}
Using the Cauchy Integral Theorem, we have 
\begin{equation}
\oint_{\dP D(2,1)}f(\zeta)\ d\zeta = \int_{0}^{2\pi}(2 + e^{i\theta})^{2}\cdot ie^{it}\ dt = 0. 
\label{7.1}
\end{equation}
Now,
\[
\int_{0}^{2\pi}f(2 + e^{i\theta})\ d\theta = \int_{0}^{2\pi}(4 + 4e^{i\theta} + e^{2i\theta})\ d\theta = 8\pi + 4\int_{0}^{2\pi}e^{i\theta}d\theta +
\int_{0}^{2\pi}e^{2i\theta}d\theta = 8\pi \neq 0.
\]
The reason the above integral does not evaluate to 0 is that there ends up being a constant term in the function being integrated. When we integrate
over the path as in \eqref{7.1}, this constant term gets multiplied by $\frac{\dP \gamma}{\dP t}$.


\subsection*{8}
\begin{tcolorbox}
{\bf a)} Evaluate the integral $\oint_{\partial
D(1+i,\,2)}(\zb+1)^{2}dz$ directly. {\bf b)}
Then evaluate it using the Cauchy integral formula and Cauchy integral theorem.
\end{tcolorbox}
\begin{enumerate}[label=(\alph*)]
\item 
\begin{align*}
\oint_{\dP D(1+i,2)}(\bar{z} + 1)^{2}dz & = \int_{0}^{2\pi}(\overline{1 + i + 2e^{it}} + 1)^{2}2ie^{it}\ dt \\
& = \int_{0}^{2\pi}(2-i + 2e^{-it})^{2}2ie^{it}\ dt \\
& = 2i(2-i)^2\int_{0}^{2\pi}e^{it}\ dt + 8i(2-i)\int_{0}^{2\pi}1\ dt + 8i \int_{0}^{2\pi}e^{-it}\ dt \\
& = 0 + 16\pi i(2-i) + 0 = 16\pi i(2-i).
\end{align*}
\item Let $\gamma^* := \dP D(0,2)$, i.e. $\gamma^{*}(t) = 2^{it}$ for $0 \leq t \leq 2\pi$. By a change of variables $\zeta := z - 1 - i$, and 
since $\bar{\zeta} = \frac{|\zeta|^2}{\zeta} = \frac{4}{\zeta}$ on $\dP (0,2)$,
\begin{align*}
\oint_{\gamma}(\bar{z} + 1)^{2}\ dz = \oint_{\gamma^*}(\overline{\zeta + 1 + i} + 1)^2\ d\zeta & = \oint_{\gamma^*}(2 - i + \bar{\zeta})^2\ d\zeta \\
& = \oint_{\gamma^*}[(2-i)^2 + 2(z-i)\bar{\zeta} + \bar{\zeta}^2]\ d\zeta \\
& = (2-i)^2 \oint_{\gamma^*}1\ d\zeta + 8(2-i)\oint_{\gamma^*}\frac{1}{\zeta}\ d\zeta + 16\oint_{\gamma^*}\frac{1}{\zeta^2}\ d\zeta \\
\text{(where $f \equiv 1$ on $\mathbb{C}$) }\ \  & = 0 + 8(2-i)2\pi i f(0) + 16 f'(0) \\
& = 16\pi i(2-i).
\end{align*}
\end{enumerate}

\subsection*{9}
\begin{tcolorbox}Let $\gamma$ be the unit circle equipped with {\bf clockwise}
orientation. For each real number $\lambda$, give an example of a
nonconstant holomorphic function $F$ on the annulus
$U := \{z:\frac12<|z|<2\}$ such that $$\dfrac{1}{2\pi
i}\oint_{\gamma}F(z)\,dz=\lambda\,.$$
\end{tcolorbox}
Let $\lambda \in \mathbb{R}$. Let $F : U \rightarrow \mathbb{C}$ be defined by $F(z) := \lambda / z$. Then by the Cauchy Integral Theorem
\[ \frac{1}{2\pi i}\oint_{\gamma}F(z)\ dz = \frac{1}{2\pi i}\oint \frac{\lambda}{\zeta - 0}\ d\zeta = f(0) = \lambda. \]


\subsection*{10}
\begin{tcolorbox}
Let $\gamma_1$ be the curve $\partial D(0,1)$ and let $\gamma_2$ be the curve $\partial
D(0,3)$, both equipped with counterclockwise direction. Evaluate
$$\dfrac{1}{2\pi
i}\oint_{\gamma_2}
\dfrac{\zeta^3-3\zeta-6}{\zeta(\zeta+2)(\zeta+4)}\,d\zeta-\dfrac{1}{2\pi
i}\oint_{\gamma_1}
\dfrac{\zeta^3-3\zeta-6}{\zeta(\zeta+2)(\zeta+4)}\,d\zeta$$
\end{tcolorbox}
Let $\gamma_3 := \dP D(-2,1)$. The expression involving the integral over $\gamma_{1}$ is given by 
\begin{equation}
\frac{1}{2\pi i}\oint_{\gamma_1}\frac{(\zeta^3 - 3\zeta - 6)(\zeta+2)^{-1}(\zeta+4)^{-1}}{\zeta}\ d\zeta = -\frac{6}{8} = -\frac{3}{4}, 
\label{10.1}
\end{equation}
and the integral over $\gamma_{2}$ is given by
\begin{align}
\frac{1}{2\pi i}\oint_{\gamma_2} \frac{\zeta^3 - 3\zeta - 6}{\zeta(\zeta + 2)(\zeta + 4)}\ d\zeta & = \frac{1}{2\pi i}\oint_{\gamma_1}\frac{(\zeta^3 -
3\zeta - 6)(\zeta + 4)^{-1}(\zeta+2)^{-1}}{\zeta}\ d\zeta \nonumber \\ 
& \qquad + \frac{1}{2\pi i}\oint_{\gamma_3}\frac{(\zeta^{3} - 3\zeta - 6)(\zeta+4)^{-1}\zeta^{-1}}{\zeta + 2}\ d\zeta \nonumber \\
& = -\frac{3}{4} + 2 = \frac{5}{4}. \label{10.2}
\end{align}
Thus, by \eqref{10.1} and \eqref{10.2}, the entire expression is equal to 2.



\end{document}

