\documentclass[12pt]{article}
\usepackage{amsmath}
\usepackage{amsfonts}
\usepackage{parskip}
\usepackage{amsthm}
\usepackage{thmtools}
\usepackage[headheight=15pt]{geometry}
\geometry{a4paper, left=20mm, right=20mm, top=30mm, bottom=30mm}
\usepackage{graphicx}
\usepackage{bm} % for bold font in math mode - command is \bm{text}
\usepackage{enumitem}
\usepackage{fancyhdr}
\usepackage{amssymb} % for stacked arrows and other shit
\pagestyle{fancy}
\usepackage{changepage}
\usepackage{mathcomp}
\usepackage{tcolorbox}

\declaretheoremstyle[headfont=\normalfont]{normal}
\declaretheorem[style=normal]{Theorem}
\declaretheorem[style=normal]{Proposition}
\declaretheorem[style=normal]{Lemma}
\newcounter{ProofCounter}
\newcounter{ClaimCounter}[ProofCounter]
\newcounter{SubClaimCounter}[ClaimCounter]
\newenvironment{Proof}{\stepcounter{ProofCounter}\textsc{Proof.}}{\hfill$\square$}
\newenvironment{claim}[1]{\vspace{1mm}\stepcounter{ClaimCounter}\par\noindent\underline{\bf Claim \theClaimCounter:}\space#1}{}
\newenvironment{claimproof}[1]{\par\noindent\underline{Proof of claim \theClaimCounter:}\space#1}{\hfill $\blacksquare$ Claim \theClaimCounter}
\newenvironment{subclaim}[1]{\stepcounter{SubClaimCounter}\par\noindent\emph{Subclaim \theClaimCounter.\theSubClaimCounter:}\space#1}{}
% \newenvironment{subclaimproof}[1]{\begin{adjustwidth}{2em}{0pt}\par\noindent\emph{Proof of subclaim \theClaimCounter.\theSubClaimCounter:}\space#1}{\hfill
% $\blacksquare$ \emph{Subclaim \theClaimCounter.\theSubClaimCounter}\vspace{5mm}\end{adjustwidth}}
\newenvironment{subclaimproof}[1]{\par\noindent\emph{Proof of subclaim \theClaimCounter.\theSubClaimCounter:}\space#1}{\hfill
$\Diamond$ \emph{Subclaim \theClaimCounter.\theSubClaimCounter}}

\title{MATH 511: HW 2}
\author{Evan P. Walsh}
\makeatletter
\let\runauthor\@author
\let\runtitle\@title
\makeatother
\lhead{\runauthor}
\chead{\runtitle}
\rhead{\thepage}
\cfoot{}

\begin{document}
\maketitle


% \def\IF{{\bf F}}
\def\IC{{\bf C}}
% \def\IR{{\bf R}}
% \def\IRd{\IR_{\downarrow}}
% \def\cH{{\cal H}}
% \def\cL{{\cal L}}
% \def\cB{{\cal B}}
% \def\cD{{\cal D}}
% \def\cO{{\cal O}}
% \def\cF{{\cal F}}
% \def\cP{{\cal P}}
% \def\cS{{\cal S}}
% \def\cT{{\cal T}}
% \def\cU{{\cal U}}
% \def\cV{{\cal V}}
% \def\ba{{\bf a}}
% \def\bb{{\bf b}}
% \def\bc{{\bf c}}
% \def\bd{{\bf d}}
% \def\bs{{\bf s}}
% \def\xx{{\bf x}}
% \def\yy{{\bf y}}
% \def\tt{\theta}
% \def\bV{{\bf V}}
% \def\dim{{\rm dim}\,}
% \def\diag{{\rm diag}\,}
% \def\span{{\rm span}\,}
% \def\Ra{{\Rightarrow}}
\def\({\left (}
\def\){\right )}
% \def\ra{{\rightarrow}}
% \def\Ra{{\ \Rightarrow\ }}
% \def\lra{{\leftrightarrow}}
% \def\Lra{{\ \Leftrightarrow\ }}
% \def\dfrac{\displaystyle\frac}
\def\zb{\overline z}
\def\dP{\partial}


\subsection*{1}
\begin{tcolorbox}
Evaluate
$\oint_\gamma \dfrac{\overline{z}}{8+z}\,dz$, where $\gamma$ is the
rectangle with vertices $\pm 3\pm i$ with counter-clockwise
direction.
\end{tcolorbox}


\subsection*{2}
\begin{tcolorbox}
Let $f(x+iy)=x^3y^2+i\,x^2y^3$. Find all the points $P$ in $\IC$ where $f'(P)$ exists.
\end{tcolorbox}
Let $u := x^{3}y^{2}$ and $v := x^{2}y^{3}$. Since the derivates of $u$ and $v$ exist everywhere on $\mathbb{R}^{2}$, it suffices to find the points
where the Cauchy-Riemann equations hold. Thus, we must have
\begin{equation}
\frac{\dP u}{\dP x} = 3x^{2}y^{2} = \frac{\dP v}{\dP y} = 3x^{2}y^{2},
\label{2.1}
\end{equation}
and 
\begin{equation}
\frac{\dP u}{\dP y} = 2x^{3}y = -\frac{\dP v}{\dP x} = -2xy^{3}.
\label{2.2}
\end{equation}
Focusing on \eqref{2.2} since \eqref{2.1} is trivial, we see that we must have $x(x^{2} + y^{2})y = 0$, or $x^{2} + y^{2} = 0$. The only point that
satisfies this is $(0,0)$.



\subsection*{3}
\begin{tcolorbox}
Let $U\subset \IC$ be an open set  and  $f\in C^1(U)$. Let $z_0\in U$.
\begin{enumerate}[label=(\alph*),itemsep=1ex]
\item Suppose $|D_{e^{i\theta_1}}f(z_0)|=|D_{e^{i\theta_2}}f(z_0)|$  for all $\theta_1,\ \theta_2$.  Prove that either $f'(z_0)$ or $(\overline{f})'(z_0)$ exists. (Note: $(\overline{f})'\neq    \overline{(f')}$. )
\item Suppose $ D_{e^{i\theta_1}}f(z_0) \overline{  D_{e^{i\theta_2}}f(z)}=|D_{e^{i\theta_1}}f(z_0)   D_{e^{i\theta_2}}f(z_0)|e^{i(\theta_1-\theta_2)}$ for all $\theta_1,\ \theta_2$. Prove that   $f'(z_0)$ exists.
\end{enumerate}
\end{tcolorbox}


\subsection*{4}
\begin{tcolorbox}
Let $u$ be a real-valued $C^1$ function on an open disc $U$ with center $0$. Assume that $u$ is harmonic on
$U\setminus\{0\}$. Prove that $u$ is the real part of a holomorphic
function on $U$.
\end{tcolorbox}
\begin{Proof}
Set $f := -\frac{\partial u}{\partial y}$ and $g := \frac{\dP u}{\partial x}$. Therefore $\frac{\dP f}{\dP y} = \frac{\dP g}{\dP x}$ and $f$ and $g$
are continuously differentiably on $U \setminus \left\{ 0 \right\}$. So by Theorem 2.3.2, there exists some $C^{1}$ function $v : U \rightarrow
\mathbb{R}$ such that 
\[ \frac{\dP v}{\dP x} = f \qquad \text{ and } \qquad \frac{\dP h}{\dP y} = g. \]
Hence $F := u + iv$ is $C^{1}$ and satisfies the Cauchy-Riemann equations.
\end{Proof}


\subsection*{5}
\begin{tcolorbox}
Evaluate
$\oint_\gamma \dfrac{\zeta^2+8i}{(\zeta+i)(\zeta-8)}\,d\zeta$,
where $\gamma$ is the circle with center $2+ i$ and radius $3$ with
counter-clockwise direction.
\end{tcolorbox}
Using the Cauchy Integral Formula,
\begin{align*}
\oint_{\gamma}\frac{\zeta^2 + 8i}{(\zeta + i)(\zeta - 8)}d\zeta = \oint_{\gamma} \frac{(\zeta^2 + 8i)/(\zeta - 8)}{\zeta + i}d\zeta & = 2\pi i
\left[ \frac{(-i)^2 + 8i}{(-i)^2 - 8} \right] \\
& = 2\pi i \left[ \frac{-1 + 8i}{-i - 8} \right] = 2\pi.
\end{align*}

\subsection*{6}
\begin{tcolorbox}
Let $\gamma(t) = e^{it}$, $0\le t \le 2\pi$ and $\psi$ be a
complex continuous function on $\gamma$. Show that
$$\overline{\oint_\gamma\psi(z)\,dz}=-\oint_\gamma\overline{\psi(z)\,z^2}\,dz.$$
\end{tcolorbox}


\subsection*{7}
\begin{tcolorbox}
Let $f(z)=z^2$. Show that the  integral of $f$ around the
circle $\partial D(2,1)$ given by
$$\int_0^{2\pi}f\(2+e^{i\theta}\)\,d\theta$$ is not zero. Yet the
Cauchy integral theorem asserts that $$\oint_{\partial
D(2,1)}f(\zeta)\,d\,\zeta=0\,.$$ Give an explanation.
\end{tcolorbox}


\subsection*{8}
\begin{tcolorbox}
{\bf a)} Evaluate the integral $\oint_{\partial
D(1+i,\,2)}(\zb+1)^{2}dz$ directly. {\bf b)}
Then evaluate it using the Cauchy integral formula and Cauchy integral theorem.
\end{tcolorbox}

\subsection*{9}
\begin{tcolorbox}Let $\gamma$ be the unit circle equipped with {\bf clockwise}
orientation. For each real number $\lambda$, give an example of a
nonconstant holomorphic function $F$ on the annulus
$\{z:\frac12<|z|<2\}$ such that $$\dfrac{1}{2\pi
i}\oint_{\gamma}F(z)\,dz=\lambda\,.$$
\end{tcolorbox}



\subsection*{10}
\begin{tcolorbox}
Let $\gamma_1$ be the curve $\partial D(0,1)$ and let $\gamma_2$ be the curve $\partial
D(0,3)$, both equipped with counterclockwise direction. Evaluate
$$\dfrac{1}{2\pi
i}\oint_{\gamma_2}
\dfrac{\zeta^3-3\zeta-6}{\zeta(\zeta+2)(\zeta+4)}\,d\zeta-\dfrac{1}{2\pi
i}\oint_{\gamma_1}
\dfrac{\zeta^3-3\zeta-6}{\zeta(\zeta+2)(\zeta+4)}\,d\zeta$$
\end{tcolorbox}


\end{document}

