\documentclass[12pt]{article}
\usepackage{amsmath}
\usepackage{amsfonts}
\usepackage{parskip}
\usepackage{amsthm}
\usepackage{thmtools}
\usepackage[headheight=15pt]{geometry}
\geometry{a4paper, left=20mm, right=20mm, top=30mm, bottom=30mm}
\usepackage{graphicx}
\usepackage{bm} % for bold font in math mode - command is \bm{text}
\usepackage{enumitem}
\usepackage{fancyhdr}
\usepackage{amssymb} % for stacked arrows and other shit
\pagestyle{fancy}
\usepackage{changepage}
\usepackage{mathcomp}
\usepackage{tcolorbox}

\declaretheoremstyle[headfont=\normalfont]{normal}
\declaretheorem[style=normal]{Theorem}
\declaretheorem[style=normal]{Proposition}
\declaretheorem[style=normal]{Lemma}
\newcounter{ProofCounter}
\newcounter{ClaimCounter}[ProofCounter]
\newcounter{SubClaimCounter}[ClaimCounter]
\newenvironment{Proof}{\stepcounter{ProofCounter}\textsc{Proof.}}{\hfill$\square$}
\newenvironment{claim}[1]{\vspace{1mm}\stepcounter{ClaimCounter}\par\noindent\underline{\bf Claim \theClaimCounter:}\space#1}{}
\newenvironment{claimproof}[1]{\par\noindent\underline{Proof of claim \theClaimCounter:}\space#1}{\hfill $\blacksquare$ Claim \theClaimCounter}
\newenvironment{subclaim}[1]{\stepcounter{SubClaimCounter}\par\noindent\emph{Subclaim \theClaimCounter.\theSubClaimCounter:}\space#1}{}
% \newenvironment{subclaimproof}[1]{\begin{adjustwidth}{2em}{0pt}\par\noindent\emph{Proof of subclaim \theClaimCounter.\theSubClaimCounter:}\space#1}{\hfill
% $\blacksquare$ \emph{Subclaim \theClaimCounter.\theSubClaimCounter}\vspace{5mm}\end{adjustwidth}}
\newenvironment{subclaimproof}[1]{\par\noindent\emph{Proof of subclaim \theClaimCounter.\theSubClaimCounter:}\space#1}{\hfill
$\Diamond$ \emph{Subclaim \theClaimCounter.\theSubClaimCounter}}

\title{MATH 511: HW 3}
\author{Evan P. Walsh}
\makeatletter
\let\runauthor\@author
\let\runtitle\@title
\makeatother
\lhead{\runauthor}
\chead{\runtitle}
\rhead{\thepage}
\cfoot{}

\begin{document}
\maketitle

\subsection*{1}
\begin{tcolorbox}
Suppose for $m,n \geq 1$, $|a_{m,n}| \leq 1$. Define 
\[ K(w,z) := \sum_{n=1}^{\infty}\sum_{n=1}^{\infty}a_{m,n}w^{m}z^{n}. \]
Prove that for every $w \in D(0,1)$, the function $f_{w}(z) := K(w,z)$ is well-defined and holomorphic on $D(0,1)$.
\end{tcolorbox}
\begin{Proof}
Let $w \in D(0,1)$. For each $m \geq 1$, let $f_{m}(z) := \sum_{n=1}^{\infty}a_{m,n}w^{m}z^{n}$. For each $m \geq 1$,
\[ \sum_{n=1}^{\infty}|a_{m,n}w^{m}z^{n}| \leq \sum_{n=1}^{\infty}|z|^{n} = \frac{|z|}{1 - |z|}, \]
which is well-defined for all $z \in D(0,1)$. Hence $f_{m}(z)$ is well-defined on $D(0,1)$, and by Lemma 3.2.10, $f_{m}(z)$ is also holomorphic on
$D(0,1)$. Now,
\[ \sum_{m=1}^{\infty}f_{m}(z) \leq \sum_{m=1}^{\infty}|w|^{m}\sum_{n=1}^{\infty}|a_{m,n}z^{n}| \leq \sum_{m=1}^{\infty}|w|^{m}\left( \frac{|z|}{1 -
|z|} \right) = \left( \frac{|w|}{1 - |w|} \right)\left( \frac{|z|}{1 - |z|} \right), \]
which is well-defined for all $z \in D(0,1)$, so $f_{w}(z)$ is well-defined on $D(0,1)$. To show that $f_{w}(z)$ is holomorphic, we will show that
$\sum_{m=1}^{\infty}f_{m}(z)$ converges uniformly on every compact $E \subseteq D(0,1)$. Well, if $E \subseteq D(0,1)$, then 
$r := \max\left\{ |\zeta| : \zeta \in E \right\}$ is well-defined and $0 \leq r < 1$. Therefore $|f_{m}(z)| \leq |w|^{m}\frac{r}{1-r}$ for all $m \geq
1$ and $z \in E$, and 
\[ \sum_{m=1}^{\infty}|w|^{m}\left( \frac{r}{1-r} \right) = \left( \frac{|w|}{1 - |w|} \right)\left( \frac{r}{1-r}\right) < \infty. \]
Thus, by the Weierstrass M-test, $\sum_{m=1}^{\infty}f_{m}(z)$ converges uniformly on $E$. Hence by Theorem 3.5.1, $f_{w}(z)$ is holomorphic on
$D(0,1)$.
\end{Proof}


\newpage
\subsection*{2}
\begin{tcolorbox}
Find the power series expansion of the following holomorphic functions at the given point and find the radius of convergence.

\begin{enumerate}[label=(\alph*),itemsep=5mm,topsep=4mm]
\item $f(z) := \dfrac{1}{z}$ at $z_{0} := 2 - i$.
\item $f(z) := \dfrac{z - \frac{1}{2}}{1 - \frac{z}{2}}$ at $z_{0} := 0$.
\end{enumerate}
\end{tcolorbox}
\begin{enumerate}[label=(\alph*),itemsep=5mm,topsep=4mm]
\item We have 
\begin{align*}
f(z) = \frac{1}{z} = \frac{1}{z - (2-i) + (2-i)} = \frac{1}{(2-i)\left[ 1 + \frac{z - (2-i)}{2-i} \right]} & = \frac{1}{(2-i)\left[ 1 - 
\frac{z - (z-i)}{i - 2} \right]} \\
& = \frac{1}{2-i}\sum_{n=0}^{\infty}\left[ \frac{z - (2-i)}{i-2} \right]^{n}.
\end{align*}
The radius of convergence is $r = |2 - i - 0| = \sqrt{5}$.
\item We have 
\begin{align*}
\frac{z - \frac{1}{2}}{1 - \frac{z}{2}} = \frac{z}{1 - \frac{z}{2}} - \frac{1}{2}\left( \frac{1}{1 - \frac{z}{2}} \right)  = 
2\left( \frac{\frac{z}{2}}{1 - \frac{z}{2}} \right) - \frac{1}{2}\left( \frac{1}{1 - \frac{z}{2}} \right) & = 
2\sum_{n=1}^{\infty}\left( \frac{z}{2} \right)^{n} - \frac{1}{2}\sum_{n=0}^{\infty}\left( \frac{z}{2} \right)^{n} \\
& = \sum_{n=1}^{\infty}\frac{z^{n}}{2^{n-1}} - \sum_{n=0}^{\infty}\frac{z^{n}}{2^{n+1}} \\
& = \frac{1}{2} + \sum_{n=1}^{\infty}z^{n}\left( \frac{1}{2^{n-1}} - \frac{1}{2^{n+1}} \right) \\
& = \frac{1}{2} + 3\sum_{n=1}^{\infty}\frac{1}{2^{n+1}}z^{n}.
\end{align*}
The radius of convergence is $r = 2$.
\end{enumerate}


\newpage 
\subsection*{3}
\begin{tcolorbox}
Suppose that $f : D(0,2) \rightarrow \mathbb{C}$ is holomorphic and that $|f(z)| \leq 9$ for all $z \in D(0,2)$. Prove that 
\[ \left| \frac{\partial^{3}}{\partial z^{3}}f\left( \frac{i}{2} \right)\right| \leq 16. \]
\end{tcolorbox}
\begin{Proof}
Let $\epsilon > 0$. Since $f$ is holomorphic on $D(0,2)$, $\frac{\partial^{3}f}{\partial z^{3}}$ is continuous on $D(0,2)$. Therefore there exists 
some $\delta > 0$ such that $\left|\frac{i}{2} - z\right| < \delta$ implies 
\begin{equation}
\left|\frac{\partial^{3}}{\partial z^{3}}f\left( \frac{i}{2} \right) - \frac{\partial^{3}}{\partial z}f(z)\right| < \epsilon.
\label{3.1}
\end{equation}
Let $P := \frac{i}{2} - \frac{\delta i}{2}$. Then $\overline{D(P,3/2)} \subset D(0,2)$. Thus, by Theorem 3.4.1,
\begin{equation}
\left| \frac{\partial^{3}}{\partial z^{3}}f\left( P \right)\right| \leq \frac{9(3!)}{(3/2)^{3}} = 16.
\label{3.2}
\end{equation}
Therefore by \eqref{3.1} and \eqref{3.2}, 
\[ \left| \frac{\partial^{3}}{\partial z^{3}}f\left( \frac{i}{2} \right)\right| < 16 + \epsilon. \]
But since we can make $\epsilon$ arbitrarily small, we have the result.
\end{Proof}


\newpage 
\subsection*{4}
\begin{tcolorbox}
Suppose for some $k \geq 0$, $a_{k} \geq a_{k+1} \geq a_{k+2} \geq \dots \geq a_{n} \geq \dots$ and $\lim_{n\rightarrow\infty}a_{n} = 0$. Show that 
$\sum_{n=0}^{\infty}a_{n}z^{n}$ converges for all $z$ with $|z| = 1$ and $z\neq 1$.
\end{tcolorbox}
\begin{Proof}
Suppose $|z| = 1$ and $z \neq 1$. Then there exists $t \in (0,2\pi)$ such that $z = e^{it}$. So the series becomes 
\[ \sum_{n=0}^{\infty}a_{n}z^{n} = \sum_{n=0}^{k-1}a_{n}z^{n} + \sum_{n=k}^{\infty}a_{n}e^{int} = \sum_{n=0}^{k-1}a_{n}z^{n} + \sum_{n=k}^{\infty}
a_{n}[\cos(nt) + i\sin(nt)], \]
which converges if and only if $S := \sum_{n=k}^{\infty}a_{n}[\cos(nt) + i\sin(nt)]$ converges. But $S$ converges 
if and only if the real and imaginary parts of the partial sums converge, i.e.
if $S_{1,N} := \sum_{n=k}^{N}a_{n}\cos(nt)$ and $S_{2,N} := \sum_{n=k}^{N}a_{n}\sin(nt)$ converge as $N \rightarrow \infty$. 
By the alternating series test, both $S_{1,N}$ and $S_{2,N}$ do, in fact, converge.
\end{Proof}


\newpage 
\subsection*{5}
\begin{tcolorbox}
Determine the radius of convergence of the series $\sum_{k=0}^{\infty}\frac{k}{k^{2} + 4}z^{k}$ and the points (including those on the boundary of the
disk of convergence) at which the series converges.
\end{tcolorbox}
Since 
\[
\lim_{k\rightarrow\infty}\left( \frac{k}{k^{2} + 4} \right)^{1/k} = \lim_{k\rightarrow\infty}\exp \left( \frac{\log\left( \frac{k}{k^{2} + 4}
\right)}{k} \right) = \exp(0) = 1,
\]
the radius of convergence is given by 
\[ r = \frac{1}{\limsup_{k\rightarrow\infty}\left( \frac{k}{k^{2} + 4} \right)^{1/k}} = 1. \]
Thus the series converges for all $z$ with $|z| < 1$. But by the result in question 4, the series also converges for $z$ with $|z| = 1$ as long as $z
\neq 1$.


\newpage 
\subsection*{6} 
\begin{tcolorbox}
Suppose $f \not\equiv 0$ is an entire function such that for some $B, K > 0$, $|f(z)| \leq B|z|^{K}$ for all $z \in \mathbb{C}$. Prove that $K$ is an
integer and $f(z) = Cz^{K}$ for some $C \in \mathbb{C}$ with $|C| \leq |B|$.
\end{tcolorbox}
\begin{Proof}

\begin{claim}
$\frac{\partial^{n}}{\partial z^{n}}f(0) = 0$ for all integers $n$ such that $n > K$.
\end{claim}
\begin{claimproof}
Let $n > K$ be an integer. Let $x := n - K$. By Theorem 3.4.1, for any $|z| = r > 0$,
\[ \left| \frac{\partial^{n}}{\partial z^{n}}f(0)\right| \leq \frac{B|z|^{K}n!}{r^{n}} = \frac{B|z|^{K}n!}{r^{K+x}} = \frac{Bn!}{r^{x}} 
\rightarrow 0 \ \ \text{ as } r \rightarrow \infty. \]
\end{claimproof}
\begin{claim}
$\frac{\partial^{n}}{\partial z^{n}}f(0) = 0$ for all integers $n$ such that $n < K$.
\end{claim}
\begin{claimproof}
Let $0 \leq n < K$ be an integer. Let $x := K - n$. By Theorem 3.4.1, for any $|z| = r > 0$,
\[ \left| \frac{\partial^{n}}{\partial z^{n}}f(0)\right| \leq \frac{B|z|^{K}n!}{r^{n}} = \frac{B|z|^{K}n!}{r^{K-x}} = B(n!)r^{x} 
\rightarrow 0 \ \ \text{ as } r \rightarrow 0. \]
\end{claimproof}

By claims 1 and 2, $\frac{\partial^{n}}{\partial z^{n}}f(0) = 0$ whenever $n$ is an integer and $n \neq K$. Therefore, by Theorem 3.3.1, $f \equiv 0$
if $K$ is not an integer. This is a contradiction. Hence $K$ is an integer and 
$f(z) = Cz^{K}$, where $C = \frac{\partial^{K}f / \partial z^{K}(0)}{K!}$.
\end{Proof}


\newpage 
\subsection*{7}
\begin{tcolorbox}
Suppose $f$ is a holomorphic function on $D(0,1)$ such that $f^{2}$ is a holomorphic polynomial on $D(0,1)$. Must $f$ be a holomorphic polynomial on
$D(0,1)$? Explain your answer.
\end{tcolorbox}


\newpage 
\subsection*{8}
\begin{tcolorbox}
Suppose $U\subseteq \mathbb{C}$ is open and $f : U \rightarrow \mathbb{C}$ is a function such that both $f^{2}$ and $f^{3}$ are holomorphic on $U$.
Prove that $f$ is holomorphic on $U$. 
\end{tcolorbox}


\newpage 
\subsection*{9}
\begin{tcolorbox}
Suppose $f$ is bounded and holomorphic on $\mathbb{C}\setminus \{0\}$. Prove that $f$ is constant on $\mathbb{C}\setminus\left\{ 0 \right\}$.
\end{tcolorbox}
\begin{Proof}
Let 
\[ g(z) := \left\{ \begin{array}{cl}
z^{2}f(z) & : z \neq 0, \\
0 & : z = 0. 
\end{array} \right. \]
Let $M := \sup\left\{ |f(z)| : z \in \mathbb{C} \right\}$. By assumption, $M < \infty$.
\begin{claim}
$g$ is entire.
\end{claim}
\begin{claimproof}
It suffices to show that the complex derivative of $g$ exists at $z = 0$. Well, 
\[ \lim_{z\rightarrow 0}\left|\frac{g(z) - g(0)}{z}\right|  = \lim_{z\rightarrow 0}\left|\frac{z^{2}f(z)}{z}\right| 
\leq \lim_{z\rightarrow 0} |z|M = 0. \]
Therefore $g'(0)$ exists and is equal to 0.
\end{claimproof}

Now, if $f \equiv 0$ then we are done. So suppose $f \not\equiv 0$. Note that $|g(z)| \leq |z|^{2}M$. Thus, by claim 1 and the result from question 6,
$g(z) = Cz^{2}$ for some $C \in \mathbb{C}$. But since $g(z) = f(z)z^{2}$ whenever $z \neq 0$, $f(z) \equiv C$ on $\mathbb{C}\setminus \left\{ 0
\right\}$.
\end{Proof}


\newpage 
\subsection*{10}
\begin{tcolorbox}
In each of the following cases, determine if there exists $f$ holomorphic on $D(0,1)$ satisfying the condition. If so, find $f$. If not, explain why.
\begin{enumerate}[label=(\alph*),itemsep=4mm,topsep=3mm]
\item $f\left( \dfrac{1}{2n + 1} \right) = \dfrac{1}{n}$.
\item $f\left( \dfrac{(-1)^{n}}{n} \right) = \dfrac{1}{n}$.
\end{enumerate}
\end{tcolorbox}


\end{document}

