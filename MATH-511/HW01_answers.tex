\documentclass[12pt]{article}
\usepackage{amsmath}
\usepackage{amsfonts}
\usepackage{parskip}
\usepackage{amsthm}
\usepackage{thmtools}
\usepackage[headheight=15pt]{geometry}
\geometry{a4paper, left=20mm, right=20mm, top=30mm, bottom=30mm}
\usepackage{graphicx}
\usepackage{bm} % for bold font in math mode - command is \bm{text}
\usepackage{enumitem}
\usepackage{fancyhdr}
\usepackage{amssymb} % for stacked arrows and other shit
\pagestyle{fancy}
\usepackage{changepage}
\usepackage{mathcomp}
\usepackage{tcolorbox}

\declaretheoremstyle[headfont=\normalfont]{normal}
\declaretheorem[style=normal]{Theorem}
\declaretheorem[style=normal]{Proposition}
\declaretheorem[style=normal]{Lemma}
\newcounter{ProofCounter}
\newcounter{ClaimCounter}[ProofCounter]
\newcounter{SubClaimCounter}[ClaimCounter]
\newenvironment{Proof}{\stepcounter{ProofCounter}\textit{Proof.}}{\hfill$\square$}
\newenvironment{claim}[1]{\vspace{1mm}\stepcounter{ClaimCounter}\par\noindent\underline{\bf Claim \theClaimCounter:}\space#1}{}
\newenvironment{claimproof}[1]{\par\noindent\underline{Proof of claim \theClaimCounter:}\space#1}{\hfill $\blacksquare$ Claim \theClaimCounter}
\newenvironment{subclaim}[1]{\stepcounter{SubClaimCounter}\par\noindent\emph{Subclaim \theClaimCounter.\theSubClaimCounter:}\space#1}{}
% \newenvironment{subclaimproof}[1]{\begin{adjustwidth}{2em}{0pt}\par\noindent\emph{Proof of subclaim \theClaimCounter.\theSubClaimCounter:}\space#1}{\hfill
% $\blacksquare$ \emph{Subclaim \theClaimCounter.\theSubClaimCounter}\vspace{5mm}\end{adjustwidth}}
\newenvironment{subclaimproof}[1]{\par\noindent\emph{Proof of subclaim \theClaimCounter.\theSubClaimCounter:}\space#1}{\hfill
$\Diamond$ \emph{Subclaim \theClaimCounter.\theSubClaimCounter}}

\title{MATH 511: HW 1}
\author{Evan P. Walsh}
\makeatletter
\let\runauthor\@author
\let\runtitle\@title
\makeatother
\lhead{\runauthor}
\chead{\runtitle}
\rhead{\thepage}
\cfoot{}

\begin{document}
\maketitle


\subsection*{1}
\begin{tcolorbox}
Show that a cubic equation $E_{1} : t^{3} + at^{2} + bt + c = 0$ can be transformed into an equation of the form $E_{2} : x^{3} + px = q$ by a
substitution of $t := x - a/3$. Suppose $\alpha^{3}$ and $\beta^{3}$ are the roots of the equation $z^{2} - qz - p^{3}/27 = 0$ with $\alpha\beta =
-p/3$. Show that $x = \alpha + \beta$ is a solution of $E_{2}$. Hence, give the solutions of $E_{1}$ in terms of $a$, $b$ and $c$.
\end{tcolorbox}

By substituting $t := x - a/3$ into $E_{1}$, we get 
\begin{align*}
& (x - a/3)^{3} + a(x-a/3)^{2} + b(x-a/3) + c = 0 \\
& (x -a/3)\left(x^{2} - \frac{2}{3}ax + \frac{a^{2}}{9}\right) + ax^{2} - \frac{2}{3}a^{2}x + \frac{a^{3}}{9} + bx - \frac{ab}{3} + c = 0 \\
& x^{3} + \frac{a^{2}}{3}x - \frac{2a^{2}}{3}x + bx - \frac{ab}{3} + c - \frac{a^{3}}{27} + \frac{a^{3}}{9} \\
& x^{3} + \left( b - \frac{a^{2}}{3} \right)x = \frac{ab}{3} - c - \frac{2a^{3}}{27}.
\end{align*}
Therefore set $p := b - \frac{a^{2}}{3}$ and $q := \frac{ab}{3} - c - \frac{2a^{2}}{27}$. So $E_{1}$ becomes $x^{3} + px = q$, which is $E_{2}$. Now
suppose $\alpha^{3}$ and $\beta^{3}$ are roots of the equation $z^{2} - qz - p^{3}/27 = 0$, with $\alpha\beta = -p/3$. Then note that 
\begin{equation}
\alpha^{3} = - \frac{p^{3}}{27\beta^{3}}, \qquad \alpha = - \frac{p}{3\beta}, \qquad \text{and} \qquad \beta^{6} - \beta^{3}q - \frac{p^{3}}{27} =
0. 
\label{1.1}
\end{equation}
Thus, plugging $\alpha + \beta$ into $E_{2}$, we have 
\begin{align*}
(\alpha + \beta)^{3} + p(\alpha + \beta) - q & = (\alpha + \beta)(\alpha^{2} + 2\alpha\beta + \beta^{2}) + p\alpha + p\beta - q \\
&= \alpha^{3} + \beta^{3} + 3\alpha^{2}\beta + 3\alpha\beta^{2} + p\alpha + p\beta - q \\
\text{(substitution by \eqref{1.1}) } \ & = -\frac{p^{3}}{27\beta^{3}} + \beta^{3} + \frac{p^{2}}{3\beta} - p\beta - \frac{p^{2}}{3\beta} + p\beta - q \\
\text{(by \eqref{1.1} again) }\ &= \beta^{6} - \beta^{3}q - \frac{p^{3}}{27} = 0.
\end{align*}
Therefore $\alpha + \beta$ is a solution of $E_{2}$. Now we will find the other solutions of $E_{2}$, starting by finding $u$ and $v$ in the following equation:
\begin{equation}
(x - (\alpha + \beta))(x^{2} + ux + v) = 0.
\label{1.2}
\end{equation}
But by expanding \eqref{1.2}, we have 
\[ x^{3} + ux^{2} + vx - (\alpha + \beta)x^{2} - u(\alpha + \beta)x - (\alpha + \beta)v = 0. \]
Which, by comparing to $E_{2}$ implies that $u = \alpha + \beta$ and $v = q / (\alpha + \beta)$. Therefore the other solutions of $E_{2}$ are 
\begin{align*}
& r_{1} := \frac{-u - \sqrt{u^{2} - 4v}}{2} = \frac{-(\alpha + \beta) - \sqrt{(\alpha+\beta)^{2} - 4q/(\alpha+\beta)}}{2} \\
& r_{2} := \frac{-u + \sqrt{u^{2} - 4v}}{2} = \frac{-(\alpha + \beta) + \sqrt{(\alpha+\beta)^{2} - 4q/(\alpha+\beta)}}{2}. \\
\end{align*}
Since $t = x - a / 3$, the solutions of $E_{1}$ are 
\begin{align*}
\text{Root 1: } & \alpha + \beta - a/3, \\
\text{Root 2: } & \frac{-(\alpha + \beta) - \sqrt{(\alpha+\beta)^{2} - 4q/(\alpha+\beta)}}{2} - a/3, \\
\text{Root 3: } & \frac{-(\alpha + \beta) + \sqrt{(\alpha+\beta)^{2} - 4q/(\alpha+\beta)}}{2} - a/3,
\end{align*}
where $\alpha$ and $\beta$ can be written as 
\[ \alpha := \left( \frac{q - \sqrt{q^{2} + 4p^{3}/27}}{2} \right)^{1/3}, \qquad \beta := \left( \frac{q + \sqrt{q^{2} + 4p^{3}/27}}{2} \right)^{1/3},
\]
and $p = b - \frac{a^{2}}{3}$, $q = \frac{ab}{3} - c - \frac{2a^{2}}{27}$ as before.






\subsection*{2}
\begin{tcolorbox}
Find all solutions of the equation $E_{3} : x^{3} - 15x - 4 = 0$ directly. Then compare them with the solutions using the formula found in 1. For each
solution, find the corresponding $\alpha$ and $\beta$ as given in 1.
\end{tcolorbox}

By trial and error we find that $x_{1} := 4$ is a solution of $E_{3}$. Next we proceed to find $a$ and $b$ in 
\[ 0 = (x - 4)(x^{2} + bx + c) = x^{3} + bx^{2} + cx - 4x^{2} - 4bx - 4c. \]
But by comparing the above equation to $E_{3}$, we see that we must have $b = 4$ and $c = 1$. Therefore we solve $x^{2} + 4x + 1 = 0$ using the
quadratic equation, and get the remaining two solutions to $E_{3}$:
\[ x_{2} := -2 - \sqrt{3} \qquad \text{and} \qquad x_{3} := -2 + \sqrt{3}. \]
If we were to solve $E_{3}$ using our formula in question 1, we start with $p = -15$ and $q = 4$. Then,
\[ \alpha = \left( \frac{4 - \sqrt{16 + 4(-15)^{3}/27}}{2} \right)^{1/3} = (2 - 11i)^{1/3} = 2 - i. \]
Similarly, $\beta = 2 + i$, and so $\alpha + \beta = 4$ coincides with $x_{1}$. For the other two roots, we have 
\begin{align*}
\text{Root 2: } & \frac{-(\alpha + \beta) - \sqrt{(\alpha+\beta)^{2} - 4q/(\alpha+\beta)}}{2} = \frac{-4 - \sqrt{16 - 4}}{2} = -2 - \sqrt{3}, \\
\text{Root 3: } & \frac{-(\alpha + \beta) + \sqrt{(\alpha+\beta)^{2} - 4q/(\alpha+\beta)}}{2} = \frac{-4 + \sqrt{16 - 4}}{2} = -2 + \sqrt{3},
\end{align*}
which coincide with $x_{2}$ and $x_{3}$, respectively.








\subsection*{3}
\begin{tcolorbox}
Let $S$ be the sphere of radius $1$, centered at the origin in $\mathbb{R}^{3}$ and $N = (0,0,1)$. For each point $P(x,y,z)$ on $S\setminus \{N\}$,
let the ray from $N$ to $P$ intersect the $X-Y$ plane at $Q(u,v,0)$. Express $(x,y,z)$ in terms of $(u,v)$ and $(u,v)$ in terms of $(x,y,z)$.
\end{tcolorbox}

Suppose $Q = (u,v,0)$ is known. The vector from of the line from $N$ to $Q$ is 
\[ \vec{r} := \langle 0,0,1\rangle + t\langle u,v,-1\rangle = \langle tu, tv, 1-t\rangle. \]
When $\vec{r}$ intersects $S$, we must have 
\[ t^{2}u^{2} + t^{2}v^{2} + (1-t)^{2} = 1 \qquad \Rightarrow \qquad (u^{2} + v^{2} + 1)t^{2} - 2t = 0. \]
Therefore
\[ t_{0} := \frac{2\pm \sqrt{4}}{2(u^{2} + v^{2} + 1)} \qquad \Rightarrow \qquad t_{0} = \frac{2}{u^{2} + v^{2} + 1}. \]
Thus,
\[ P = \langle x,y,z \rangle = \left\langle \frac{2u}{u^{2} + v^{2} + 1}, \frac{2v}{u^{2} + v^{2} + 1}, \frac{u^{2} + v^{2} - 1}{u^{2} + v^{2} + 1}\right\rangle.
\]
Conversely, suppose $P = (x,y,z)$ is known. Then the vector form of the line from $N$ to $P$ is 
\[ \vec{r} := \langle 0,0,1 \rangle + t\langle x,y,z-1 \rangle = \langle tx,ty,tz-t+1\rangle. \]
We know that $\vec{r}$ intersects the $X-Y$ plane when $tz - t + 1 = 0$. Solving for $t$, we have $t_{0} = \frac{1}{1-z}$. Thus,
\[ Q = \langle u,v,0 \rangle = \left\langle \frac{x}{1-z}, \frac{y}{1-z}, 0 \right\rangle. \]












\newpage
\subsection*{4}
\begin{tcolorbox}
Prove Lagranges Identity and use that to deduce the Cauchy-Schwars Inequality.
\end{tcolorbox}


\begin{Proof}
We will need the following claim.

\begin{claim}
Suppose $z_{i}, w_{i} \in \mathbb{C}$ for $i = 1, \hdots, n$. Then 
\[ \sum_{i\neq j}z_{i}\bar{z}_{j}w_{i}\bar{w}_{j} - \sum_{i\neq j}z_{i}\bar{z}_{i}w_{j}\bar{w}_{j} = - \sum_{i < j}|z_{i}\bar{w}_{j} -
\bar{w}_{i}z_{j}|^{2}. \]
\end{claim}
\begin{claimproof}
We will proceed by induction. For base case consider when $n = 2$. Then 
\begin{align*}
\sum_{i\neq j}z_{i}\bar{z}_{j}w_{i}\bar{w}_{j} - \sum_{i\neq j}z_{i}\bar{z}_{i}w_{j}\bar{w}_{j} & = z_{1}\bar{z}_{2}w_{1}\bar{w}_{2} +
z_{2}\bar{z}_{1}w_{2}\bar{w}_{2} - z_{1}\bar{z}_{1}w_{2}\bar{w}_{2} - z_{2}\bar{z}_{2}w_{1}\bar{w}_{1} \\
& = -(z_{1}\bar{w}_{2} - \bar{w}_{1}z_{2})(\bar{z}_{1}w_{2} - w_{1}\bar{z}_{2}) \\
& = -(z_{1}\bar{w}_{2} - \bar{w}_{1}z_{2})\overline{(z_{1}\bar{w}_{2} - \bar{w}_{1}z_{2})} \\
& = -|z_{1}\bar{w}_{2} - \bar{w}_{1}z_{2}|^{2}.
\end{align*}
Now, by way of induction let $n > 2$ and suppose claim 1 holds for $n - 1$. Then 
\begin{align*}
\sum_{i\neq j}z_{i}\bar{z}_{j}w_{i}\bar{w}_{j} - \sum_{i\neq j}z_{i}\bar{z}_{i}w_{j}\bar{w}_{j} & = 
\sum_{i\neq j: j,j < n}z_{i}\bar{z}_{j}w_{i}\bar{w}_{j} - \sum_{i\neq j: i,j < n}z_{i}\bar{z}_{i}w_{j}\bar{w}_{j} \\
& \qquad + \sum_{i < n}(z_{i}\bar{z}_{n}w_{i}\bar{w}_{n} + z_{n}\bar{z}_{i}w_{n}\bar{w}_{i}) - 
\sum_{i < n}(z_{i}\bar{z}_{i}w_{n}\bar{w}_{n} + z_{n}\bar{z}_{n}w_{i}\bar{w}_{i}) \\
\text{(by inductive assumption) } & = - \sum_{i < j: j < n}|z_{i}\bar{w}_{j} - \bar{w}_{i}z_{j}|^{2} \\
& \qquad + \sum_{i < n}(z_{i}\bar{z}_{n}w_{i}\bar{w}_{n} + z_{n}\bar{z}_{i}w_{n}\bar{w}_{i} - 
z_{i}\bar{z}_{i}w_{n}\bar{w}_{n} - z_{n}\bar{z}_{n}w_{i}\bar{w}_{i}) \\
\text{(by base case) } & = - \sum_{i < j: j < n}|z_{i}\bar{w}_{j} - \bar{w}_{i}z_{j}|^{2} - \sum_{i < n}|z_{i}\bar{w}_{n} - \bar{w}_{i}z_{n}|^{2} \\
& = - \sum_{i < j}|z_{i}\bar{w}_{j} - \bar{w}_{i}z_{j}|^{2}.
\end{align*}
\end{claimproof}

Now we will prove Lagranges Identity.
\begin{align*}
\left| \sum_{i=1}^{n}z_{i}w_{i}\right|^{2} = \left( \sum_{i=1}^{n}z_{i}w_{i} \right)\left( \sum_{i=1}^{n}\bar{z}_{i}\bar{w}_{i} \right) & =
\sum_{i=1}^{n}\sum_{j=1}^{n}z_{i}\bar{z}_{j}w_{i}\bar{w}_{j} \\
& = \sum_{i=1}^{n}|z_{i}|^{2}|w_{i}|^{2} + \sum_{i\neq j}z_{i}\bar{z}_{j}w_{i}\bar{w}_{j} \\
& = \sum_{i=1}^{n}|z_{i}|^{2}|w_{i}|^{2} + \sum_{i\neq j}|z_{i}|^{2}|w_{j}|^{2} - \sum_{i\neq j}|z_{i}|^{2}|w_{j}|^{2} + \sum_{i\neq j}z_{i}\bar{z}_{j}w_{i}\bar{w}_{j} \\
& = \left( \sum_{i=1}^{n}|z_{i}|^{2} \right)\left( \sum_{i=1}^{n}|w_{i}|^{2} \right) + \sum_{i\neq j}z_{i}\bar{z}_{j}w_{i}\bar{w}_{j} - \sum_{i\neq
j}z_{i}\bar{z}_{i}w_{j}\bar{w}_{j} \\
\text{(by claim 1) } & = \left( \sum_{i=1}^{n}|z_{i}|^{2} \right)\left( \sum_{i=1}^{n}|w_{i}|^{2} \right) - \sum_{i < j}|z_{i}\bar{w}_{j} - \bar{w}_{i}z_{j}|^{2}.
\end{align*}
Thus,
\[ \left|\sum_{i=1}^{n}z_{i}w_{i}\right|^{2} \leq \left( \sum_{i=1}^{n}|z_{i}|^{2} \right)\left( \sum_{i=1}^{n}|w_{i}|^{2} \right), \]
so by taking the square root of both sides we are left with the Cauchy-Schwarz Inequality.
\end{Proof}



\subsection*{5}
\begin{tcolorbox}
Compute the follwing derivatives:
\begin{enumerate}[label=(\alph*)]
\item $\frac{\partial^{3}}{\partial x^{2}\partial y}(3z^{2}\bar{z}^{4} - 2z^{3}\bar{z} + z^{4} - \bar{z}^{5})$
\item $\frac{\partial^{4}}{\partial z\partial \bar{z}^{3}}(xy^{2})$
\end{enumerate}
\end{tcolorbox}

See work attached.

(a) 
\[ \left( 24i - \frac{36}{i} \right)\bar{z}^{3} + 60i\bar{z}^{2} + \left( \frac{36}{i} - 72 \right)z\bar{z}^{2} - 72z^{2}\bar{z} + 12iz - 12i\bar{z}.
\]
(b) $0$.



\subsection*{6}
\begin{tcolorbox}
Find two real valued functions $u$ and $v$ on $\mathbb{C}$ such that $u\cdot v$ is not harmonic.
\end{tcolorbox}

Let $u(x,y) := x^{2} - y^{2}$ and $v(x,y) := x$. We see that 
\[ \frac{\partial^{2}u}{\partial x^{2}} = 2, \ \frac{\partial^{2}u}{\partial y^{2}} = -2, \qquad \text{and}\qquad 
\frac{\partial^{2}v}{\partial x^{2}} = 0, \ \frac{\partial^{2}v}{\partial y^{2}} = 0. \]
Therefore $u$ and $v$ are harmonic. However, with $u(x,y)\cdot v(x,y) = x^{3} - xy^{2}$, we see that
\[ \frac{\partial^{2}}{\partial x^{2}}u\cdot v = 6x \qquad \text{and}\qquad \frac{\partial^{2}}{\partial y^{2}}u\cdot v = 2x, \]
so $\triangle u\cdot v = 8x \neq 0$.



\subsection*{7}
\begin{tcolorbox}
Suppose $u,v : \mathbb{C} \rightarrow \mathbb{R}$ such that $u + iv$ is holomorphic. Prove that $u\cdot v$ is harmonic.
\end{tcolorbox}

\begin{Proof}
Since $u + iv$ is holomorphic, we have 
\begin{equation}
\frac{\partial u}{\partial x} = \frac{\partial v}{\partial y} \qquad \text{and} \qquad \frac{\partial u}{\partial y} = -\frac{\partial v}{\partial
x}.
\label{7.0}
\end{equation}
Thus,
\begin{align}
& \frac{\partial^{2}u}{\partial x^{2}} = \frac{\partial^{2}v}{\partial x\partial y} \label{7.1} \\
& \frac{\partial^{2}u}{\partial y\partial x} = \frac{\partial^{2}v}{\partial y^{2}} \label{7.2} \\
& \frac{\partial^{2}u}{\partial y^{2}} = -\frac{\partial^{2}v}{\partial y\partial x} \label{7.3} \\
& \frac{\partial^{2}u}{\partial y\partial x} = -\frac{\partial^{2}v}{\partial x^{2}}. \label{7.4}
\end{align}
Thus,
\begin{align*}
\frac{\partial^{2}}{\partial x^{2}}u\cdot v & = \left( \frac{\partial^{2}u}{\partial x^{2}} \right)\cdot v + \left( \frac{\partial u}{\partial x}
\right)\left( \frac{\partial v}{\partial x }\right) + u\cdot \left( \frac{\partial^{2}v}{\partial x^{2}} \right) + \left( \frac{\partial v}{\partial
x} \right)\left( \frac{\partial u}{\partial x} \right) \\
& =  \left( \frac{\partial^{2}u}{\partial x^{2}} \right)\cdot v  + u\cdot \left( \frac{\partial^{2}v}{\partial x^{2}} \right) + 2\left( \frac{\partial v}{\partial
x} \right)\left( \frac{\partial u}{\partial x} \right) \\
\text{(by \eqref{7.0}, \eqref{7.1}, \eqref{7.4}) } & = \left( \frac{\partial^{2}v}{\partial x\partial y} \right)\cdot v - u \left( \frac{\partial^{2}u}{\partial x \partial y} \right) + 2\left( 
\frac{\partial v}{\partial y}\right)\left( \frac{\partial v}{\partial x} \right),
\end{align*}
and similarly,
\begin{align*}
\frac{\partial^{2}}{\partial y^{2}}u\cdot v & = 
\left( \frac{\partial^{2}u}{\partial y^{2}} \right)\cdot v  + u\cdot \left( \frac{\partial^{2}v}{\partial y^{2}} \right) + 2\left( \frac{\partial v}{\partial
y} \right)\left( \frac{\partial u}{\partial y} \right) \\
\text{(by \eqref{7.0}, \eqref{7.2}, \eqref{7.3}) } & = - \left( \frac{\partial^{2}v}{\partial x\partial y} \right)\cdot v + 
u \left( \frac{\partial^{2}u}{\partial x \partial y} \right) - 2\left( \frac{\partial v}{\partial y}\right)\left( \frac{\partial v}{\partial x} \right),
\end{align*}
Hence $\triangle u\cdot v = 0$.

\end{Proof}




\subsection*{8} 
\begin{tcolorbox}
Let $U \subseteq \mathbb{C}$ be an open set. Let $z_{0} \in U$ and $r > 0$ and assume that $\left\{ z : |z - z_{0}| \leq r \right\} \subseteq U$. For
$j$ a positive integer, compute 
\[ \frac{1}{2\pi}\int_{0}^{2\pi}(z_{0} + re^{i\theta})^{j}\ d\theta \qquad \text{and}\qquad \frac{1}{2\pi}\int_{0}^{2\pi}\overline{(z_{0} +
re^{i\theta})^{j}}\ d\theta. \]
Use these results to prove that if $\mu$ is a harmonic polynomial on $U$, then 
\[ \frac{1}{2\pi}\int_{0}^{2\pi}u(z_{0} + re^{i\theta})\ d\theta = u(z_{0}). \]
\end{tcolorbox}

\begin{Proof}

\begin{claim}
For any positive integer $j$,
\[ \frac{1}{2\pi}\int_{0}^{2\pi}(z_{0} + re^{i\theta})^{j}\ d\theta = z_{0}^{j} \qquad \text{and}\qquad \frac{1}{2\pi}\int_{0}^{2\pi}\overline{(z_{0} +
re^{i\theta})^{j}}\ d\theta = \bar{z}_{0}^{j}. \]
\end{claim}
\begin{claimproof}
First note that for any $c \in \mathbb{C}$ and positive integer $k$,
\[ \int_{0}^{2\pi}ce^{i\theta k}\ d\theta = c\int_{0}^{2\pi}\cos(k\theta)\ d\theta + ci\int_{0}^{2\pi}\sin(k\theta)\ d\theta = 0, \]
and 
\[ \int_{0}^{2\pi}c\overline{e^{i\theta k}}\ d\theta = c\int_{0}^{2\pi}\cos(k\theta)\ d\theta - ci\int_{0}^{2\pi}\sin(k\theta)\ d\theta = 0. \]
Thus,
\[ \frac{1}{2\pi}\int_{0}^{2\pi}(z_{0} + re^{i\theta})\ d\theta = \frac{1}{2\pi}\int_{0}^{2\pi}\left[ z_{0}^{j} + c_{1}z_{0}^{j-1}e^{i\theta} + 
c_{1}z_{0}^{j-2}e^{i\theta 2} + \dots + c_{j}e^{i\theta j} \right]d\theta = z_{0}^{j}, \]
and 
\begin{align*}
\frac{1}{2\pi}\int_{0}^{2\pi}\overline{z_{0} + re^{i\theta})^{j}}\ d\theta & = \frac{1}{2\pi}\int_{0}^{2\pi}(\bar{z}_{0} +
r\overline{e^{i\theta}})^{j}\ d\theta \\
& = \frac{1}{2\pi}\int_{0}^{2\pi}\left[ \bar{z}_{0}^{j} + c_{1}\bar{z}_{0}^{j-1}\overline{e^{i\theta}} + c_{1}\bar{z}_{0}^{j-2}\overline{e^{i\theta 2}} + \dots + 
c_{j}\overline{e^{i\theta j}} \right]d\theta = \bar{z}_{0}^{j}.
\end{align*}
\end{claimproof}

\begin{claim}
For $u(z) = \sum_{l,m}c_{l,m}z^{l}\bar{z}^{m}$, we have $c_{l,m} = 0$ whenever both $l > 0$ and $m > 0$. In other words, there are no terms involving both $z$
and $\bar{z}$.
\end{claim}
\begin{claimproof}
By way of contradiction, suppose there exists $i,j > 0$ such that $c_{i,j} \neq 0$. But then 
\[ \triangle (c_{i,j}z^{i}\bar{z}^{j}) = 4\frac{\partial^{2}}{\partial z\partial \bar{z}}(c_{i,j}z^{i}\bar{z}^{j}) = 4c_{i,j}\frac{\partial}{\partial
z}(z^{i}j\bar{z}^{j-1})  = 4c_{i,j}(ijz^{i-1}\bar{z}^{j-1}) \neq 0,\]
so $u$ is not harmonic. This is a contradiction.
\end{claimproof}

By claims 1 and 2,
\begin{align*}
\frac{1}{2\pi}\int_{0}^{2\pi}u(z_{0} + re^{i\theta})\ d\theta & = \sum_{l}\frac{c_{l,0}}{2\pi}\int_{0}^{2\pi}(z_{0} + re^{i\theta})^{l}\ d\theta + 
\sum_{m}\frac{c_{0,m}}{2\pi}\int_{0}^{2\pi}\overline{(z_{0} + re^{i\theta})}^{m}\ d\theta \\
& = \sum_{l}c_{l,0}z_{0}^{l} + \sum_{m}c_{0,m}\bar{z}_{0}^{m} \\
& = u(z_{0}).
\end{align*}
\end{Proof}

\subsection*{9}
\begin{tcolorbox}
Prove that if $f$ is holomorphic on an open set $U \subseteq \mathbb{C}$ and $f$ is nonvanishing, then 
\[ \triangle(|f|^{p}) = p^{2}|f|^{p-2}\left| \frac{\partial f}{\partial z}\right|^{2}, \ \text{ for all } p > 0. \]
\end{tcolorbox}

\begin{align*}
\triangle(|f|^{p}) & = 4\frac{\partial^{2}}{\partial z\partial \bar{z}}(|f|^{2})^{p/2} \\
& = 4\frac{\partial^{2}}{\partial z\partial \bar{z}}(f\cdot \bar{f})^{p/2} \\
& = 4\frac{\partial}{\partial z}\left[ \left( \frac{p}{2} \right)\left( |f|^{2} \right)^{p/2-1}\left( f
\cdot \frac{\partial\bar{f}}{\partial \bar{z}}  + \bar{f}\cdot \frac{\partial f}{\partial \bar{z}} \right)\right] \\
& = 4\frac{\partial}{\partial z}\left[ \left( \frac{p}{2} \right)\left( |f|^{2} \right)^{p/2-1}\left( f
\cdot \frac{\partial\bar{f}}{\partial \bar{z}}\right)\right] \\
& = 4\left[ \left( \frac{p}{2} \right)\left( \frac{p}{2} - 1 \right)\left( |f|^{2} \right)^{p/2-2}\left( \bar{f}\cdot \frac{\partial f}{\partial z} \right)
\left( f \cdot \frac{\partial \bar{f}}{\partial \bar{z}} \right) + \left( \frac{p}{2}\right)\left( |f|^{2} \right)^{p/2-1} 
\left( \frac{\partial f}{\partial z}\cdot \frac{\partial \bar{f}}{\partial \bar{z}} + f\cdot \frac{\partial^{2}\bar{f}}{\partial z\partial \bar{z}}
 \right)\right] \\
& = 4\left[ \left( \frac{p}{2} \right)\left( \frac{p}{2} - 1 \right)\left( |f|^{2} \right)^{p/2-2}\cdot |f|^{2}\cdot \left( \frac{\partial f}{\partial z}
\cdot \frac{\partial \bar{f}}{\partial \bar{z}} \right) + \left( \frac{p}{2}\right)\left( |f|^{2} \right)^{p/2-1} 
\left( \frac{\partial f}{\partial z}\cdot \frac{\partial \bar{f}}{\partial \bar{z}} \right)\right] \\
& = 4\left[ \left( \frac{p}{2} \right)\left( \frac{p}{2} - 1 \right)\left( |f|^{2} \right)^{p/2-2}\cdot |f|^{2}\cdot \left( \frac{\partial f}{\partial z}
\right)\left(\overline{\frac{\partial f}{\partial z}} \right) + \left( \frac{p}{2}\right)\left( |f|^{2} \right)^{p/2-1} 
\left( \frac{\partial f}{\partial z}\right)\left(\overline{\frac{\partial f}{\partial z}} \right)\right] \\
& = 4\left[ \left( \frac{p}{2} \right)\left( \frac{p}{2} - 1 \right)\left( |f|^{2} \right)^{p/2-2}\cdot |f|^{2}\cdot \left| \frac{\partial f}{\partial z}
\right|^{2} + \left( \frac{p}{2}\right)\left( |f|^{2} \right)^{p/2-1} 
\left| \frac{\partial f}{\partial z}\right|^{2}\right] \\
& = 4\left[ \left( \frac{p}{2} \right)\left( \frac{p}{2} - 1 \right) + \left( \frac{p}{2} \right) \right]\cdot |f|^{p-2}\left| \frac{\partial
f}{\partial z}\right|^{2} \\
& = p^{2}|f|^{p-2}\left| \frac{\partial f}{\partial z}\right|^{2}.
\end{align*}





\newpage
\subsection*{10}
\begin{tcolorbox}
Suppose $U$ is star-shaped and $f,g \in C^{1}(U)$ satisfy $f_{y} = g_{x}$. Show that the function $h(x,y) := \int_{0}^{1}(f(r(t)),g(r(t)))\cdot
r'(t)\ dt$ satisfies $h_{x} = f$ and $h_{y} = g$, where $r(t) := (a + t(x-a), b + t(y-b))$.
\end{tcolorbox}


\begin{Proof}
First note that 
\begin{equation}
\frac{d}{dt}t f(r(t)) = f(r(t)) + t\big[ f_{x}(r(t))(x-a) + f_{y}(r(t))(y-b)\big],
\label{10.1}
\end{equation}
and 
\begin{equation}
\frac{d}{dt}t g(r(t)) = g(r(t)) + t\big[ g_{x}(r(t))(x-a) + g_{y}(r(t))(y-b)\big].
\label{10.2}
\end{equation}
Now,
\[ h(x,y) = \int_{0}^{1} f(r(t))(x-a)\ dt + \int_{0}^{1} g(r(t))(y-b)\ dt. \]
Further, 
\begin{equation}
\frac{\partial}{\partial x}\left[ \int_{0}^{1}f(r(t))(x-a)\ dt \right] = \int_{0}^{1}\left[ f_{x}(r(t))\cdot t \cdot (x-a) + f(r(t)) \right]dt,
\label{10.3}
\end{equation}
and 
\begin{equation}
\frac{\partial}{\partial x} \left[ \int_{0}^{1}g(r(t))(y-b)\ dt \right] = \int_{0}^{1} g_{x}(r(t))\cdot t \cdot (y-b)\ dt = \int_{0}^{1}
f_{y}(r(t))\cdot t \cdot (y-b)\ dt
\label{10.4}
\end{equation}
Thus we see that $h_{x}(x,y) = \eqref{10.3} + \eqref{10.4} = \int_{0}^{1} \eqref{10.1}\ dt = \int_{0}^{1} \frac{d( t f(r(t)))}{dt}\ dt = f(x,y)$, by the
fundamental theorem. On the other hand,
\begin{equation}
\frac{\partial}{\partial y}\left[ \int_{0}^{1}f(r(t))(x-a)\ dt \right] = \int_{0}^{1}f_{y}(r(t))\cdot t \cdot (x-a)\ dt = \int_{0}^{1}g_{x}(r(t))\cdot
t \cdot (x-a)\ dt,
\label{10.5}
\end{equation}
and 
\begin{equation}
\frac{\partial}{\partial y}\left[ \int_{0}^{1}g(r(t))(y-b)\ dt \right] = \int_{0}^{1}\left[ g_{y}(r(t))\cdot t \cdot (y-b) + g(r(t)) \right]dt.
\label{10.6}
\end{equation}
Therefore $h_{y}(x,y) = \eqref{10.5} + \eqref{10.6} = \int_{0}^{1} \eqref{10.2}\ dt = \int_{0}^{1}\frac{d(t g(r(t)))}{dt}\ dt = g(x,y)$, by the
fundamental theorem once again.
\end{Proof}

















\end{document}

