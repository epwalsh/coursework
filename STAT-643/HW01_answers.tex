\documentclass[12pt]{article}
\usepackage{amsmath}
\usepackage{amsfonts}
\usepackage{parskip}
\usepackage{amsthm}
\usepackage{thmtools}
\usepackage[headheight=15pt]{geometry}
\geometry{a4paper, left=20mm, right=20mm, top=30mm, bottom=30mm}
\usepackage{graphicx}
\usepackage{bm} % for bold font in math mode - command is \bm{text}
\usepackage{enumitem}
\usepackage{fancyhdr}
\usepackage{amssymb} % for stacked arrows and other shit
\pagestyle{fancy}
\usepackage{changepage}
\usepackage{mathcomp}
\usepackage{tcolorbox}

\declaretheoremstyle[headfont=\normalfont]{normal}
\declaretheorem[style=normal]{Theorem}
\declaretheorem[style=normal]{Proposition}
\declaretheorem[style=normal]{Lemma}
\newcounter{ProofCounter}
\newcounter{ClaimCounter}[ProofCounter]
\newcounter{SubClaimCounter}[ClaimCounter]
\newenvironment{Proof}{\stepcounter{ProofCounter}\textsc{Proof.}}{\hfill$\square$}
\newenvironment{claim}[1]{\vspace{1mm}\stepcounter{ClaimCounter}\par\noindent\underline{\bf Claim \theClaimCounter:}\space#1}{}
\newenvironment{claimproof}[1]{\par\noindent\underline{Proof of claim \theClaimCounter:}\space#1}{\hfill $\blacksquare$ Claim \theClaimCounter}
\newenvironment{subclaim}[1]{\stepcounter{SubClaimCounter}\par\noindent\emph{Subclaim \theClaimCounter.\theSubClaimCounter:}\space#1}{}
% \newenvironment{subclaimproof}[1]{\begin{adjustwidth}{2em}{0pt}\par\noindent\emph{Proof of subclaim \theClaimCounter.\theSubClaimCounter:}\space#1}{\hfill
% $\blacksquare$ \emph{Subclaim \theClaimCounter.\theSubClaimCounter}\vspace{5mm}\end{adjustwidth}}
\newenvironment{subclaimproof}[1]{\par\noindent\emph{Proof of subclaim \theClaimCounter.\theSubClaimCounter:}\space#1}{\hfill
$\Diamond$ \emph{Subclaim \theClaimCounter.\theSubClaimCounter}}

\allowdisplaybreaks{}

% chktex-file 3

\title{STAT 643: HW 1}
\author{Evan P. Walsh}
\makeatletter
\makeatother
\lhead{Evan P. Walsh}
\chead{}
\rhead{\thepage}
\cfoot{}

\begin{document}
\maketitle

\subsection*{1}
\begin{tcolorbox}

\end{tcolorbox}

{\bf Solution:}

We need to show that 
\begin{enumerate}[label=\roman*.]
  \item $h(x,y)$ is measurable with respect to $(\mathbb{R}^{2}, \sigma\langle Y\rangle )$,\footnote{By this we mean $\langle
      (\mathbb{R}^{2}, \sigma \langle Y\rangle), (\mathbb{R}, \mathcal{B}(\mathbb{R}))\rangle$-measurable.} \\
    \item and for any $G \in \sigma\langle Y\rangle$, 
      \begin{equation}
        \int_{G}h(x,y)\ dP(x,y) = \int_{G} P(X\in B | Y)\ dP.
        \label{1.1}
      \end{equation}
\end{enumerate}

To show i., let $g_{B}(y) := \int_{B}f(t,y)d\mu_{1}(t)$. Note that by Fubini's Theorem $g_{B}(y)$ and $f_{Y}(y)$ are measurable with respect to 
$(\mathbb{B}, \mathcal{B}(\mathbb{R}))$ as functions of just $y$. Hence $g_{B}(y)$ and $f_{Y}(y)$ are measurable with respect to $\sigma\langle
Y\rangle$ as functions of $(x,y)$. It follows easily that $h(x,y)$ is measurable with respect to $(\mathbb{R}^{2}, \mathcal{B}(\mathbb{R}^{2}))$.

To show ii., let $G = \mathbb{R} \times A \in \sigma \langle Y \rangle$. Note that 
\begin{equation}
  \int_{G}P(X\in B | Y)\ dP = \int_{G}E(I_{B\times \mathbb{R}}|Y)\ dP = \int_{G}I_{B\times \mathbb{R}}\ dP = P(\mathbb{R}\times A \cap B \times
  \mathbb{R}) = P(B\times A).
  \label{1.2}
\end{equation}
Further, note that $h(x,y) \equiv \frac{1}{f_{Y}(y)}\int_{B}f(t,y)d\mu_{1}(t)$ almost surely. Hence,
\begin{align*}
  \int_{G}h(x,y)\ dP(x,y) & = \int_{G}\frac{1}{f_{Y}(y)}\int_{B}f(t,y)d\mu_{1}(t)\ dP(x,y) \\
  & = \int_{G}\frac{1}{f_{Y}(y)}\int_{B}f(t,y)d\mu_{1}(t) \cdot f(x,y)\ d(\mu_{1}\times \mu_{2})(x,y) \\
  & \qquad \text{(since $f$ is the R-N derivative)} \\
  & = \int_{A}\int_{\mathbb{R}}\frac{1}{f_{Y}(y)}\int_{B}f(t,y)d\mu_{1}(t)\cdot f(x,y)\ d\mu_{1}(x)d\mu_{2}(y) \ \ \text{(by Tonelli's)} \\
  & = \int_{A}\int_{B}f(t,y)\ d\mu_{1}(t)d\mu_{2}(y) \\
  & = \int_{B\times A}f(t,y)\ d(\mu_{1}\times \mu_{2})(t,y) \ \ \text{(by Tonelli's again)} \\
  & = P(B\times A) = \int_{G}P(X \in B| Y) \ dP \ \ \text{by \eqref{1.2}.}
\end{align*}
Thus \eqref{1.1} is satisfied.



\end{document}

