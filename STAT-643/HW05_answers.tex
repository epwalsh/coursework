\documentclass[12pt]{article}
\usepackage{amsmath}
\usepackage{amsfonts}
\usepackage{parskip}
\usepackage{amsthm}
\usepackage{thmtools}
\usepackage[headheight=15pt]{geometry}
\geometry{a4paper, left=20mm, right=20mm, top=30mm, bottom=30mm}
\usepackage{graphicx}
\usepackage{bm} % for bold font in math mode - command is \bm{text}
\usepackage{enumitem}
\usepackage{fancyhdr}
\usepackage{amssymb} % for stacked arrows and other shit
\pagestyle{fancy}
\usepackage{changepage}
\usepackage{mathcomp}
\usepackage{tcolorbox}

\declaretheoremstyle[headfont=\normalfont]{normal}
\declaretheorem[style=normal]{Theorem}
\declaretheorem[style=normal]{Proposition}
\declaretheorem[style=normal]{Lemma}
\newcounter{ProofCounter}
\newcounter{ClaimCounter}[ProofCounter]
\newcounter{SubClaimCounter}[ClaimCounter]
\newenvironment{Proof}{\stepcounter{ProofCounter}\textsc{Proof.}}{\hfill$\square$}
\newenvironment{Solution}{\stepcounter{ProofCounter}\textbf{Solution:}}{\hfill$\square$}
\newenvironment{claim}[1]{\vspace{1mm}\stepcounter{ClaimCounter}\par\noindent\underline{\bf Claim \theClaimCounter:}\space#1}{}
\newenvironment{claimproof}[1]{\par\noindent\underline{Proof of claim \theClaimCounter:}\space#1}{\hfill $\blacksquare$ Claim \theClaimCounter}
\newenvironment{subclaim}[1]{\stepcounter{SubClaimCounter}\par\noindent\emph{Subclaim \theClaimCounter.\theSubClaimCounter:}\space#1}{}
% \newenvironment{subclaimproof}[1]{\begin{adjustwidth}{2em}{0pt}\par\noindent\emph{Proof of subclaim \theClaimCounter.\theSubClaimCounter:}\space#1}{\hfill
% $\blacksquare$ \emph{Subclaim \theClaimCounter.\theSubClaimCounter}\vspace{5mm}\end{adjustwidth}}
\newenvironment{subclaimproof}[1]{\par\noindent\emph{Proof of subclaim \theClaimCounter.\theSubClaimCounter:}\space#1}{\hfill
$\Diamond$ \emph{Subclaim \theClaimCounter.\theSubClaimCounter}}

\newcommand{\E}{\mathrm{E}}
\renewcommand{\baselinestretch}{1}

\allowdisplaybreaks{}

% chktex-file 3

\title{STAT 643: HW 5}
\author{Evan P. Walsh}
\makeatletter
\makeatother
\lhead{Evan P. Walsh}
\chead{STAT 643: HW 5}
\rhead{\thepage}
\cfoot{}

\begin{document}
\maketitle


\newpage
\subsection*{3}
\begin{tcolorbox}
  Consider a decision problem where $P_\theta$ is the Normal$(\theta,1)$ distribution on $\mathcal{X}=\mathbb{R}$, $\mathcal{A}=\{0,1\}$, and $L(\theta,a)=I[a=0]I[\theta>5] + I[a=1]I[\theta \leq 5]$.  If $\Theta= (-\infty,5] \cup [6,\infty)$, guess what prior is least favorable, find the corresponding Bayes decision rule, and prove that it is minimax.
\end{tcolorbox}

\begin{Solution}
  Let $f_{\theta}(x)$ be the density of a $N(\theta, 1)$ distribution with respect the Lebesgue measure $\mu$ on $\mathbb{R}$.
  Following the example on page 38 of section 4.6 in our notes, it makes sense that the prior $G$ defined by $G(\left\{ 5 \right\}) = G(\left\{ 6
  \right\}) = 1/2$ is least favorable. Under $G$, the posterior distribution is given by 
  \[
    \pi^{\theta | X}(C) := \int_{C} \frac{f_{5}(x)I(\theta = 5) + f_{6}(x)I(\theta = 6)}{f_{5}(x) + f_{6}(x)} dc(\theta)
  \]
  where $c$ is the counting measure. Thus, by the result from Question 6 (b) on homework 5 or the remarks on page 33 of section 4.5, the Bayes rule with respect to $G$ is given by 
  \[
    \delta(x) := I\left\{ \pi^{\theta|X}(\{6\}) > \pi^{\theta |X}(\{5\}) \right\} = I\{f_{6}(x) > f_{5}(x)\} = I\{x > 5.5\}.
  \]
  \begin{claim}
    $\delta$ is minimax.
  \end{claim}
  \begin{claimproof}
    First note that for all $\theta \in \Theta$,
    \begin{equation}
      R(\theta, \delta) = E_{\theta}L(\theta, \delta) = I(\theta \leq 5)P_{\theta}(X > 5.5) + I(\theta > 5)P_{\theta}(X \leq 5.5).
      \label{3.0}
    \end{equation}
    Further, the Bayes risk of $G$ is given by 
    \begin{align}
      BR(G) = BR(G, \delta) = \int_{\Theta}R(\theta, \delta)dG(\theta) & = \frac{1}{2}\left[ R(5, \delta) + R(6,\delta)  \right] \nonumber \\
      & = \frac{1}{2}\left[ E_{\theta=5}L(5, \delta(X)) + E_{\theta=6}L(6,\delta(X)) \right] \nonumber\\
      & = \frac{1}{2}\left[ P_{5}(X > 5.5) + P_{6}(X \leq 5.5) \right]\nonumber \\
      & = P_{5}(X > 5.5) \label{3.1} \\
      & = P_{6}(X \leq 5.5) \label{3.2}
    \end{align}
    Thus, if $\theta \leq 5$, then by \eqref{3.0} and \eqref{3.1}, 
    \[
      R(\theta, \delta) = P_{\theta}(X > 5.5) \leq P_{5}(X > 5.5) = BR(G),
    \]
    and if $\theta \geq 6$, then by \eqref{3.0} and \eqref{3.2},
    \[
      R(\theta, \delta) = P_{\theta}(X \leq 5.5) \leq P_{6}(X \leq 5.5) = BR(G).
    \]
    Hence $R(\theta, \delta) \leq BR(G)$ for all $\theta \in \Theta$. Thus $\delta$ is minimax by Corollary 87.
  \end{claimproof}

\end{Solution}


\end{document}



% \begin{enumerate} \itemsep .3cm
% \item Suppose that $X \sim$Binomial$(n,p)$ and that one wishes to estimate $p\in(0,1)=\Theta$. Suppose first that $L(p,a)=p^{-1}(1-p)^{-1}(p-a)^2$ where $a\in[0,1]=\mathcal{A}$

% \begin{enumerate}
% \item Show that $X/n$ is Bayes with respect to the uniform prior on $(0,1)$.\\
% Note: Technically, be careful in considering posterior risk (expected loss) when $x=0$ or $n$ because the posterior risk diverges to $\infty$ unless the action $a$ is chosen appropriately.
 % \item Show that $X/n$ is admissible.
 % \item Show that $X/n$ is minimax.
% \item Now consider the ordinary squared error loss, $L^*(\theta,a)=(p-a)^2$.  By applying Problem~6 of Homework~3,
% prove that
 % $X/n$ is admissible under this loss function as well.
         % \end{enumerate}
         % \item Suppose that $X\sim$Bernoulli$(p)$ and that one wishes to estimate $p\in[0,1]=\Theta=\mathcal{A}$ with the loss function $L(p,a)=|p-a|$.  Consider the estimator $\delta$ with $\delta(0)=\frac{1}{4}$ and $\delta(0)=\frac{3}{4}$.


% \begin{enumerate}
% \item Write out the risk function of $\delta$ and show that $R(p,\delta)\leq \frac{1}{4}$ for all $p\in[0,1]$.
% \item Show that there exists a prior distribution placing all its mass on $\{0,\frac{1}{2},1\}$ with respect to which $\delta$ is Bayes.\\
% Hint: Consider finding   prior probabilities on $\{0,\frac{1}{2},1\}$ so that $\delta$ is a median (in the STAT 642 sense) of the posterior distribution on $\{0,\frac{1}{2},1\}$.
% \item Prove that $\delta$ is minimax in this problem and identify a least favorable prior.
         % \end{enumerate}

             % \item Consider a decision problem where $P_\theta$ is the Normal$(\theta,1)$ distribution on $\mathcal{X}=\mathbb{R}$, $\mathcal{A}=\{0,1\}$, and $L(\theta,a)=I[a=0]I[\theta>5] + I[a=1]I[\theta \leq 5]$.  If $\Theta= (-\infty,5] \cup [6,\infty)$, guess what prior is least favorable, find the corresponding Bayes decision rule, and prove that it is minimax.

             % \item Consdier estimation of $\lambda$, the mean of a Poisson distribution, with the (inverse-mean) weighted squared error loss.
% That is, let $\Lambda=(0,\infty)$ be the parameter space, $\mathcal{A}=[0,\infty)$ be the action space, $P_\lambda$ be the Poisson distribution on $\mathcal{X}=\{0,1,2,3,\ldots\}$ and $L(\lambda,a)=
% \lambda^{-1}(\lambda-a)^2$.  Let $\delta(X)=X$.
% \begin{enumerate}
% \item Show that $\delta$ is an equalizer rule.
% \item Show that $\delta$ is generalized Bayes with respect to the Lebesgue measure on $\Lambda$.
 % \item Find the Bayes estimator with respect to a gamma$(\alpha,\beta)$ prior on $\Lambda$ which has a density given by
 % \[
 % g(\lambda|\alpha,\beta) = \frac{\beta^\alpha}{\Gamma(\alpha)} \lambda^{\alpha-1} e^{-\beta \lambda},\quad \lambda>0,
 % \]
 % for $\alpha,\beta>0$ and $\Gamma(\alpha) = \int_0^\infty t^{\alpha-1} e^{-t}dt$.\\[.2cm]
 % Note: Different parameterizations of the gamma distribution exist and the above one has mean $\alpha/\beta$.
 % \item Prove that $\delta$ is minimax for this problem.
         % \end{enumerate}

         % \item Let $X_1,\ldots,X_n$ be iid random variables, each having a distribution y7u$P_\lambda$, $\lambda>0$, with a density \[
         % f_\lambda(x) = \frac{d P_\lambda}{d \mu}(x) =  \lambda\exp(-\lambda x)I[x\neq0] +(1-e^{-\lambda})I[x=0]  =  \left(\lambda\exp(-\lambda x)\right)^{I[x\neq0]} \left((1-e^{-\lambda})\right)^{I[x=0]}
         % \]
         % for $\mu=m+\nu$ with $m$ being the Lebesgue measure on $[1,\infty)$ and $\nu$ being the point mass measure at $0$. Let $\delta_i= I[X_i=0]$, $i=1,\ldots,n$ and $M_n = n- \sum_{i=1}^n \delta_i$.
         % \begin{enumerate}
% \item Show that there is no maximum likelihood estimate (MLE) of $\lambda$ when $M_n=n$, but there is a MLE of $\lambda$ when $M_n<n$.
% \item  Show that, for any $\lambda>0$, with $\lambda$-probability tending to 1 as $n\rightarrow \infty$, the MLE of $\lambda$ exists.
% \item Give  as a simple estimator of $\lambda$ based on $M_n$ alone and prove that this estimator is consistent.
% \item Write down an explicit one-step Newton improvement of the estimator from (c) using the likelihood function from (a). (No need to simplify the expression).
         % \end{enumerate}
         % \item Suppose that $X_1,\ldots,X_n$ are iid with the distribution $P_\theta$ for $\theta \in \mathbb{R}$, where $P_\theta$ has the RN derivative with respect to the counting measure $\nu$ on $\mathcal{X}=\{0,1,2\}$ given by
             % \[
             % f_\theta(x)= \frac{\exp(x\theta)}{1+\exp(\theta)+\exp(2\theta)}.
             % \]
          % Find an estimator $T_n$ of $\theta$ based on $n_0\equiv \sum_{i=1}^n I[X_i=0]$ such that $\sqrt{n}(T_n-\theta)$ has a limiting normal distribution (implying that $\sqrt{n}(T_n-\theta)$  is bounded in probability or tight and so that $T_n$ is $\sqrt{n}$-consistent).\\
          
          % Note:  It may be helpful to recall the delta method.  Let  $T_n$ be a statistic such that $\sqrt{n}(T_n-c)\stackrel{d}{\rightarrow} N(0,\sigma^2)$ as $n\rightarrow \infty$ for some $c\in \mathbb{R}$ and variance $\sigma^2>0$.  Take a function any function $g$ where $g(t)$ is differentiable at $t=c$ and $g^\prime(c)\neq 0$ (the derivative of $g(t)$ at $t=c$ is not zero).  Then, as $n\rightarrow \infty$,
% $\sqrt{n}(g(T_n)-g(c))\stackrel{d}{\rightarrow} N(0,[g^\prime(c)]^2\sigma^2)$. 
% \end{enumerate}
