\documentclass[11pt]{article}

\usepackage{epsfig}
\usepackage{times}
\usepackage{latexsym}
\usepackage{amssymb}
\usepackage{amsmath}

\usepackage{bm}
\usepackage{color}


\setlength{\textwidth}{7.0in} \setlength{\oddsidemargin}{-.5in}
\setlength{\evensidemargin}{-.5in} \setlength{\textheight}{9in}
\setlength{\topmargin}{-0.5in}
\newcommand{\E}{\mathrm{E}}
\renewcommand{\baselinestretch}{1}

\begin{document}

\vspace{6cm}
\begin{center}\Large \textbf{Homework 4 -- STAT 643} \\
\normalsize \textbf{Due Friday, November 11 (in class)}
\end{center}

\begin{enumerate} \itemsep .3cm
\item Consider the two-state decision problem with $\Theta=\{1,2\}$ and $P_1$ as the Bernoulli$(1/4)$ distribution and
$P_2$ as the Bernoulli$(1/2)$ distribution. Let $\mathcal{A}=\Theta$ and $L(\theta,a)=I(\theta \neq a)$.
\begin{enumerate}
\item Find $\mathcal{S}^0$, the set of risk vectors (risk points in $\mathbb{R}^2$) for the four possible non-randomized decision rules here.  Plot these risk points in the plane and then sketch $\mathcal{S}$, the set of all randomized risk vectors.
    \item Identify $A(\mathcal{S})$, the set of all admissible risk vectors.  Is there a minimal complete class for this decision problem? If there is one, what is it?  (Note: It can be shown through direct, but tedious, calculation that any element of $\mathcal{D}^*$ has a corresponding element of $\mathcal{D}_*$ with the same risk vector, and vice versa.  You may use this fact without proof. In the finite geometry decision framework, we used randomized rules $\mathcal{D}_*$ and, in discussing complete classes, we used behavioral rules $\mathcal{D}^*$; but this difference generally doesn't matter, in this problem in particular.)

         \end{enumerate}
         \item Consider the two-state decision problem with $\Theta=\{1,2\}$, where the observable $X=(X_1,X_2)$ has iid Bernoulli$(1/4)$
         components when $\theta=1$ and  has iid Bernoulli$(1/2)$
         components when $\theta=2$. Let $\mathcal{A}=\Theta$ and $L(\theta,a)=I(\theta \neq a)$.  Consider the behavioral decision rule $\phi$, with $\phi_x\equiv \phi_{(x_1,x_2)}$, defined as
         \begin{eqnarray*}
          \phi_x(\{1\})=1 &\mbox{if $x_1=0$}\\
          \phi_x(\{1\})=1/2 &\mbox{if $x_1=1$}
         \end{eqnarray*}

\begin{enumerate}
\item Show that $\phi$ is inadmissible by finding a decision rule with a better risk function.

    \item Find a behavioral decision rule that is a function of the sufficient statistic $T(X)=X_1+X_2$ and is risk equivalent to $\phi$.

\end{enumerate}
\item Let $X_1,\ldots,X_n$ be iid binary random variables with $P_\theta(X_1=1)=\theta=1-P_\theta(X_1=0)$ for $\theta \in(0,1)$.  Consider estimating $\theta$ based on squared error loss and $X=(X_1,\ldots,X_n)$.  Derive the risk functions of the following estimators:
    \begin{enumerate}
    \item the non-randomized rules $\bar{X}$ (sample mean of $X$) and
    \[
    T_0(X) = \left\{\begin{array}{lcl}
    0 && \mbox{if more than half of the $X_i$'s are 0}\\
    1 && \mbox{if more than half of the $X_i$'s are 1}\\
    0.5 && \mbox{if exactly half of the $X_i$'s are 0}
    \end{array}\right.
    \]
    It suffices to express risks for $T_0(X)$ in terms of probabilities involving $\bar{X}$.
    \item the decision rules (involving forms of randomization) given by
      \[
    T_1(X) = \left\{\begin{array}{lcl}
    \bar{X} && \mbox{with probability $0.5$}\\
    T_0(X) &&  \mbox{with probability $0.5$}
    \end{array}\right. \quad \& \quad  T_2(X) = \left\{\begin{array}{lcl}
    \bar{X} && \mbox{with probability $\bar{X}$}\\
    0.5 &&  \mbox{with probability $1-\bar{X}$}
    \end{array}\right.
    \]
    It again suffices to express risks for $T_1(X)$ in terms of probabilities involving $\bar{X}$.  The risk for $T_2(X)$ should be
    \[
    \frac{\theta (1-\theta)(1+\theta (n-2))}{n^2}+ (\theta-0.5)^2(1-\theta).
    \]

      \end{enumerate}

\item Find decision rules that are better than $T_1(X)$ and $T_2(X)$, respectively, in Problem~3 for $n\geq 3$.  (Hint: Consider Lemma 51.)
\item Can any of the rules in Problem~4 be improved by using the Rao-Blackwell theorem?  Explain why or why not.
\item   Consider the two-state decision problem with $\Theta=\{1,2\}=\mathcal{A}$, where  $P_0$ and $P_1$ have densities $f_0(x)$ and $f_1(x)$
with respect to a dominating $\sigma$-finite measure $\mu$.  Let the loss function be $L(\theta,a)$ for $a,\theta \in\{0,1\}$.

\begin{enumerate}
\item For an arbitrary prior distribution $G$, find a formal Bayes rule with respect to $G$.
\item Specialize your result in (a) to the case where $L(\theta,a)=I(\theta\neq a)$.  What connection does the form of these Bayes rules have to the theory of simple-versus-simple hypothesis testing (i.e., $H_0:\theta=0$ vs $H_1:\theta=1$)?
\end{enumerate}
\end{enumerate}
\end{document}
