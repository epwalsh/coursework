\documentclass[12pt]{article}
\usepackage{amsmath}
\usepackage{amsfonts}
\usepackage{parskip}
\usepackage{amsthm}
\usepackage{thmtools}
\usepackage[headheight=15pt]{geometry}
\geometry{a4paper, left=20mm, right=20mm, top=30mm, bottom=30mm}
\usepackage{graphicx}
\usepackage{bm} % for bold font in math mode - command is \bm{text}
\usepackage{enumitem}
\usepackage{fancyhdr}
\usepackage{amssymb} % for stacked arrows and other shit
\pagestyle{fancy}

\declaretheoremstyle[headfont=\normalfont]{normal}
\declaretheorem[style=normal]{Theorem}
\declaretheorem[style=normal]{Proposition}
\declaretheorem[style=normal]{Lemma}
\newenvironment{claimproof}[1]{\par\noindent\underline{Proof:}\space#1}{\hfill $\blacksquare$}

% \begin{itemize}[label={},leftmargin=4mm, itemsep=1em, parsep=1em]

\title{MATH 515: HW 3}
\author{Evan ``Pete'' Walsh}
\makeatletter
\let\runauthor\@author
\let\runtitle\@title
\makeatother
\lhead{\runauthor}
\chead{\runtitle}
\rhead{\thepage}
\cfoot{}

\begin{document}
\maketitle

{\bf (1) [RF 2.11]} If a $\sigma$-algebra of $\mathbb{R}$ contains intervals of the form $(a, \infty)$, then it contains all intervals.

{\bf Solution:}

\begin{proof}
Let $\mathcal{F} \subseteq \mathcal{P}(\mathbb{R})$ be a $\sigma$-algebra. Assume $\mathcal{F}$ contains all intervals of the form $(x, \infty)$. We
need to show that $\mathcal{F}$ contains all sets of the following forms:

\begin{itemize}[label={},leftmargin=4mm, itemsep=1em, parsep=0em]
\item (i) $(-\infty, a)$,
\item (ii) $(-\infty, a]$, 
\item (iii) $[a, \infty)$,
\item (iv) $[a, b)$,
\item (v) $[a,b]$,
\item (vi) $(a, b]$, and 
\item (vii) $(a, b)$, where $a, b \in \mathbb{R}$ and $a < b$.
\end{itemize}

So, let $a, b \in \mathbb{R}$ with $a < b$.

(i) Define $I_{n} = (a - 1/n, \infty)$ for each $n \in \mathbb{N}$. By assumption, $I_{n} \in \mathcal{F}$. Thus $(I_{n})^{c} = (-\infty, a - 1/n] \in
\mathcal{F}$ since $\mathcal{F}$ is closed under complimentation by definition of a $\sigma$-algebra. Further, 
\[ \cup_{n\in\mathbb{N}}(I_{n})^{c} = \cup_{n\in\mathbb{N}}(-\infty, a - 1/n] = (-\infty, a) \in \mathcal{F}, \]
since $\mathcal{F}$ is closed under countable unions. Hence sets of form $(i)$ are in $\mathcal{F}$.

(ii) Next, we know $(a, \infty) \in \mathcal{F}$ by assumption, so $(a, \infty)^{c} = (-\infty, a] \in \mathcal{F}$ by closure of compliments.

(iii) Since $(-\infty, a) \in \mathcal{F}$ by $(i)$, $(-\infty, a)^{c} = [a, \infty) \in \mathcal{F}$ by closure of compliments.

(iv) By (iii), $[b, \infty) \in \mathcal{F}$ and $[a,\infty) \in \mathcal{F}$, so $[a,\infty)\setminus [b,\infty) = [a,b] \in \mathcal{F}$ by Lemma
(II)(2)(xvi).

(v) By (iii), $[a, \infty) \in \mathcal{F}$ and $(b,\infty) \in \mathcal{F}$ by assumption, so $[a,\infty) \setminus (b,\infty) = [a,b] \in
\mathcal{F}$ by Lemma (II)(2)(xvi).

(vi) By (ii), $(-\infty, b] \in \mathcal{F}$ and $(-\infty, a] \in \mathcal{F}$, so $(-\infty, b] \setminus (-\infty, a] = (a,b] \in \mathcal{F}$ by
Lemma (II)(2)(xvi).

(vii) By (i), $(-\infty, b) \in \mathcal{F}$ and by (ii) $(-\infty, a] \in \mathcal{F}$. So $(-\infty, b) \setminus (-\infty, a] = (a,b) \in
\mathcal{F}$.
\end{proof}



{\bf (2) [RF 2.14]} If a set $E$ has positive outer measure, then there is a bounded subset of $E$ that also has positive outer measure.

{\bf Solution:}

\begin{proof}
Let $E$ be a set with positive outer measure. There are two cases to consider: either $E$ is bounded or $E$ is not bounded. If $E$ is bounded then the
proof is trivial since $E \subseteq E$ is bounded with positive outer measure by assumption. Now assume that $E$ is unbounded. We will do a proof by
contradiction. So assume that there is no subset of $E$ that has positive outer measure. Since outer measure is always non-negative, this means that
every subset of $E$ has measure 0. Define $A_{n} = E \cap [-n, n]$ for every $n \in \mathbb{N}$. Note that each $A_{n}$ is a subset of $E$ and is bounded since $|x| \leq n$
for every $x \in A_{n}$. Therefore $\mu^{*}(A_{n}) = 0$ for all $n \in \mathbb{N}$ by assumption. Further, $E = \cup_{n=1}^{\infty}A_{n}$, so by the countable subadditivity of outer measure,
\[ \mu^{*}(E) = \mu^{*}\left( \cup_{n=1}^{\infty}A_{n} \right) \leq \sum_{n=1}^{\infty}\mu^{*}(A_{n}) = \sum_{n=1}^{\infty} 0 = 0. \]
This is a contradiction since $\mu^{*}(E) > 0$.  
\end{proof}

{\bf (3) [RF 2.15]} If $E$ has finite measure and $\epsilon > 0$, then $E$ is the disjoint union of a finite number of measurable sets, each of which
has measure at most $\epsilon$.

{\bf Solution:} First we will have to prove the following lemma.

\begin{proof}
Note from question (3) on homework 2, the definition from the book of outer measure is equivalent to our definition. For the convenience of this
proof, we will use the definition from the book. So, assume $E \subset \mathbb{R}$ has finite measure, i.e. $E$ is measurable and $m^{*}(E) <
\infty$, and let $\epsilon > 0$ and $\delta > 0$. It follows from
the definition of [outer] measure that we can choose a collection of non-empty, bounded, open intervals $\left\{ J_{k} \right\}_{k\in\mathbb{N}}$ such that 
\[ \sum_{k=1}^{\infty}\ell(J_{k}) \leq m(E) + \delta. \]
Since $m(E) + \delta < \infty$, $\sum_{k=1}^{\infty}\ell(J_{k})$ converges. Thus, there exists some $N \in \mathbb{N}$ such that
$\sum_{k=N}^{\infty}\ell(J_{k}) < \epsilon$. Now, define $E_{1} = E\cap \left( \cup_{k=N}^{\infty}J_{k} \right)$. Since $\cup_{k=N}^{\infty}J_{k}$ is
a countable union of open intervals, each of which is measurable by RF Proposition 2.8, $\cup_{k=N}^{\infty}J_{k}$ is also measurable since the measurable
sets form a $\sigma$-algebra by Theorem (II)(2)(xviii) which, by definition, is closed under countable unions. Hence, by Corollary (II)(2)(xii), $E_{1}$ 
is measurable because it is the intersection of two measurable sets. So, by monotonicity and countable subadditivity,
\begin{equation}
m(E_{1}) \leq m\left( \cup_{k=N}^{\infty}J_{k} \right) \leq \sum_{k=N}^{\infty}\ell(J_{k}) < \epsilon.
\label{eqn:3-0}
\end{equation}
Now, since
$\cup_{k=1}^{N-1}J_{k}$ is a finite union of bounded intervals, $\cup_{k=1}^{N-1}J_{k}$ is bounded as well. Thus there exists some $M \in \mathbb{R}$ 
such that $2 < M < \infty$ and
$|x| < M/2 - 1/2$ for all $x \in \cup_{k=1}^{N-1}J_{k}$. Now, let $b = \sup\left\{ x : x \in \cup_{k=1}^{N-1}J_{k} \right\}$ and $a = \inf\left\{
x : x \in \cup_{k=1}^{N-1}J_{k} \right\}$. Then,
\[ |b -a | \leq |b| + |a| \leq \left( \frac{M}{2} - \frac{1}{2} \right) + \left( \frac{M}{2} - \frac{1}{2} \right) = M - 1 < M. \]
So $|b - a|$ is bounded. Hence we can choose a finite sequence $(a_{1}, a_{2}, \dots, a_{p})$ in $\mathbb{R}$ such that $a = a_{1} < a_{2} < \dots <
a_{p} = b$, where $a_{k} - a_{k-1} < \epsilon$ for all $k \in \left\{ 2, 3, \dots, p \right\}$. Now define $E_{k} = \left( E\setminus E_{1}
\right)\cap [a_{k-1}, a_{k})$ for $k \in \left\{ 2, \dots, p \right\}$. Using Corollary (II)(2)(xii) again, each $E_{k}$ is measurable, and by monotonicity 
% Further, 
\begin{equation}
m(E_{k}) \leq m([a_{k-1}, a_{k}))  = a_{k} - a_{k-1} < \epsilon,
\label{eqn:3-1}
\end{equation}
since $E_{k} \subseteq [a_{k-1}, a_{k})$ for each $k \in \left\{ 2, \dots, p \right\}$. 
Further, 
\[ E \setminus E_{1} = E \setminus \left( E\cap \bigcup_{k=N}^{\infty}J_{k} \right) \subseteq \cup_{k=1}^{N-1}J_{k} \subseteq (a, b) \subseteq [a,b)
 = \cup_{k=2}^{p}[a_{k-1}, a_{k}), \]
so 
\begin{equation}
E\setminus E_{1} = E\setminus E_{1} \cap \left( \cup_{k=2}^{p}[a_{k-1},a_{k}) \right) = \bigcup_{k=2}^{p}\bigg((E\setminus E_{1})\cap [a_{k-1},
a_{k})\bigg) = \cup_{k=2}^{p}E_{k}. 
\label{eqn:3-2}
\end{equation}
In addition, it follows from the fact that $[a_{k-1}, a_{k}) \cap [a_{j-1}, a_{j}) = \emptyset$ when $j,k \in \left\{ 2,\dots, p \right\}$ with $j
\neq p$, that each $E_{k}$ is disjoint. Thus, by construction $\left\{ E_{k} \right\}_{k=1}^{p}$ is a pairwise disjoint collection of measurable sets such that 
\[ \cup_{k=1}^{p}E_{k} = E_{1}\cup \left( \cup_{k=1}^{p}E_{k} \right) \stackrel{(\ref{eqn:3-2})}{=} E_{1} \cup \left( E\setminus
E_{1} \right) = E\cup E_{1} = E, \]
and $m(E_{k}) < \epsilon$ for each $1 \leq k \leq p$ by equations \ref{eqn:3-0} and \ref{eqn:3-1}.
\end{proof}

{\bf (4)} 

{\bf Solution}


{\bf (5)}


\end{document}

