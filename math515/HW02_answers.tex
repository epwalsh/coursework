\documentclass[12pt]{article}
\usepackage{amsmath}
\usepackage{amsfonts}
\usepackage{parskip}
\usepackage{amsthm}
\usepackage{thmtools}
\usepackage[headheight=15pt]{geometry}
\geometry{a4paper, left=20mm, right=20mm, top=30mm, bottom=30mm}
\usepackage{graphicx}
\usepackage{bm} % for bold font in math mode - command is \bm{text}
\usepackage{enumitem}
\usepackage{fancyhdr}
\usepackage{amssymb} % for stacked arrows
\pagestyle{fancy}

\declaretheoremstyle[headfont=\normalfont]{normal}
\declaretheorem[style=normal]{Theorem}
\declaretheorem[style=normal]{Proposition}
\declaretheorem[style=normal]{Lemma}

\newenvironment{claim}[1]{\par\noindent\underline{Claim:}\space#1}{}
\newenvironment{claimproof}[1]{\par\noindent\underline{Proof:}\space#1}{\hfill $\blacksquare$}

\title{MATH 515: HW 2}
\author{Evan ``Pete'' Walsh}
\makeatletter
\let\runauthor\@author
\let\runtitle\@title
\makeatother
\lhead{\runauthor}
\chead{\runtitle}
\rhead{\thepage}
\cfoot{}

\begin{document}
\maketitle

{\bf (1) [Lemma (II)(1)(xvi)]} If $J_{0}, \dots, J_{n}$ are open intervals, and
if $(J_{0}, \dots, J_{n})$ is a simple chain, then $\cup_{k=1}^{n}J_{k}$ is an
open interval.

{\bf Solution:}

\begin{proof} Assume $J_{0}, \dots, J_{n}$ are open intervals and $(J_{0},
\dots, J_{n})$ is a simple chain. First we will show that $\cup_{k=0}^{n}J_{k}$
is open. Let $x \in \cup_{k=0}^{n}J_{k}$. Then $x\in J_{i}$ for some $i \in
\{0,\dots, n\}$. Since $x\in J_{i}$ and $J_{i}$ is open, there exists an
$\epsilon > 0$ such that $(x-\epsilon, x+\epsilon) \subseteq J_{i}$. Therefore
$(x-\epsilon, x+\epsilon) \subseteq \cup_{k=0}^{n}J_{k}$, so
$\cup_{k=0}^{n}J_{k}$ is open. Now we will show that $\cup_{k=0}^{n}J_{k}$ is an
interval. We will do a proof by contradiction. That is, assume that
$\cup_{k=0}^{n}J_{k}$ is not an interval. Then there exists $x_{0}, x_{1} \in
\cup_{k=0}^{n}J_{k}$ and $y \in (\mathbb{R}\setminus \cup_{k=0}^{n}J_{k})$ such that
$x_{0} < y < x_{1}$. There are two cases to consider: either $x_{0}$ and $x_{1}$
are members of the same $J_{k}$ or they are not.

{\bf Case 1:} Assume $x_{0}, x_{1} \in J_{i'}$ for some $i' \in \{0, \dots,
n\}$. Well, since $J_{i'}$ is an interval, $y \in J_{i'}$ for every $y \in
(x_{0}, x_{1})$. This contradicts our assumption.

{\bf Case 2:} Assume $x_{0} \in J_{i_{0}}$ and $x_{1} \in J_{i_{1}}$ where
$i_{0}, i_{1} \in \{0, \dots, n\}$ and $i_{0} \neq i_{1}$. We will also assume
that $x_{1} \notin J_{i_{0}}$ and $x_{0} \notin J_{i_{1}}$, as the case where $x_{0}$ and $x_{1}$ are both members of the same $J_{i}$ has
already been considered in case 1. Now, let 

\[ I_{0} = \{i : \exists x \in J_{i} \text{ s.t. } x < y\}, \text{ and } I_{1} =
\{ i : \exists x \in J_{i} \text{ s.t. } x > y\}. \] 

We know that $I_{0}$ and $I_{1}$ are not empty because $I_{0}$ contains $i_{0}$
and $I_{1}$ contains $i_{1}$ by assumption. Further, $I_{0} \cap I_{1} =
\emptyset$, for if this was not the case and $I_{0} \cap I_{1} \neq \emptyset$,
then there would exist $x_{0}'$ and $x_{1}'$ such that $x_{0}' < y < x_{1}'$,
where $x_{0}'$ and $x_{1}'$ are both in the same $J_{i}$. But we have already
argued in case 1 that this cannot happen. Thus $I_{0} \cap I_{1} = \emptyset$.
Also, $I_{0} \cup I_{1} = \{0, \dots, n\}$ since for every $x \in
\cup_{k=0}^{n}J_{k}$, either $x < y$ or $x > y$ because $x$ cannot be equal to
$y$. So every $i \in \{0, \dots, n\}$ must fall into either $I_{0}$ or $I_{1}$.
Hence, $n \in I_{0}$ or $n \in I_{1}$. Without loss of generality, assume that
$n \in I_{1}$ from now on. Then, set $i^{*} = n$ and recursively check if $i^{*}
- 1 \in I_{1}$. If $i^{*} - 1 \in I_{1}$, then set $i^{*} = i^{*} - 1$ and repeat. 
Since $i_{0} \notin I_{1}$, at some point $i^{*} - 1 \notin I_{1}$ for $i^{*}
\geq i_{0} + 1$. When this occurs, we have $i^{*} \in I_{1}$ and $i^{*} - 1 \in
I_{0}$. So $x > y$ for every $x \in J_{i^{*}}$ and $x < y$ for every $x \in
J_{i^{*} - 1}$. Thus, since $J_{i^{*}}$ and $J_{i^{*}-1}$ are intervals and
$y\notin J_{i^{*}} \cup J_{i^{*}-1}$, $J_{i^{*}}\cap J_{i^{*}-1} = \emptyset$.
However, since $J_{i^{*}}$ and $J_{i^{*}-1}$ are in the simple chain $(J_{0},
\dots, J_{n})$ and $|i^{*} - (i^{*}-1)| = 1 \leq 1$, $J_{i^{*}} \cap J_{i^{*}-1}
\neq \emptyset$. This is a contradiction.

Since both cases lead to a contradiction, no such $y$ can exist. Hence, for all
$x_{0}, x_{1} \in \cup_{k=0}^{n}J_{k}$, with $x_{0} < x_{1}$, there is no $y\in
(\mathbb{R}\setminus \cup_{k=0}^{n}J_{k})$ such that $x_{0} < y < x_{1}$. So
$\cup_{k=0}^{n}J_{k}$ is an open interval.
\end{proof}

{\bf (2) [Lemma (II)(1)(xviii)]} If $J_{0}, \dots, J_{n}$ are open intervals,
then $\ell(\cup_{k=0}^{n}J_{k}) \leq \sum_{k=0}^{n}\ell(J_{k})$.

{\bf Solution:}

\begin{proof} 
Assume $J_{0}, \dots, J_{n}$ are open intervals. There are two cases to
consider. Either none of the $J_{i}$'s are unbounded or at least one of them is.

{\bf Case 1:} Assume that $J_{i^{*}}$ is unbounded for some $i^{*} \in \{0,\dots, n\}$.

Let $M > 0$. Then there exists some $a$ and $b$ such that $a < b$ and $(a,b)
\subseteq J_{i^{*}}$, where $b - a > M$. Thus,
\[ M < \ell\left( (a,b) \right) < \ell(J_{i^{*}}) < \sum_{k=0}^{n}\ell(J_{k}).
\]
Since $M$ was arbitrary, $\sum_{k=0}^{n}\ell(J_{k}) = \infty$. Thus 
\[ \ell\left( \cup_{k=0}^{n}J_{k} \right) \leq \sum_{k=0}^{n}\ell(J_{k}), \]
and we are done.

{\bf Case 2:} Now consider the case where $J_{i}$ is bounded for every $i \in \{0, \dots, n\}$.

We will consider two subcases separately: either all of the $J_{i}$'s are
pairwise disjoint, or not all $J_{i}$'s are pairwise disjoint.
\begin{itemize}[label={},leftmargin=8mm, itemsep=1em, parsep=1em]
\item {\bf Case 2 (i):} First assume that all $J_{i}$'s are pairwise disjoint.

Then, by definition, 
\[ \ell\left(\cup_{k=0}^{n}J_{k}\right) = \sum_{k=0}^{n}\ell(J_{k}), \]
and equality is established, so we are done.

\item {\bf Case 2 (ii):} Now assume that not all $J_{i}$'s are disjoint.

Define the relation $R$ on $\{0, \dots, n\}$ by $i_{0}\  R\  i_{1}$ if and only if
for some $x_{0} \in J_{i_{0}}$ and $x_{1} \in J_{i_{1}}$, there exists a simple
chain $(J_{k_{0}}, \dots, J_{k_{m}})$ from $x_{0}$ to $x_{1}$, where $k_{i} \in
\{0, \dots, n\}$ for $0\leq i \leq m \leq n$.

\underline{Claim 1:} $R$ is an equivalence relation on $\{0,\dots, n\}$.

\begin{claimproof}
To prove that $R$ is an equivalence relation, we need to show that $R$ is
symmetric, reflexive, and transitive.
\begin{itemize}[label={},leftmargin=4mm, itemsep=1em, parsep=1em]
\item {\bf Symmetry:} If $x_{0}, x_{1} \in J_{i_{0}}$, then $(J_{i_{0}})$ is a
simple chain from $x_{0}$ to $x_{1}$, so $i_{0}\ R\ i_{0}$.

\item {\bf Reflexivity:} If $i_{0}\ R\ i_{1}$, then there is a simple chain from
$x_{0}\in J_{i_{0}}$ to $x_{1} \in J_{i_{1}}$. But by definition of simple chain
from $a$ to $b$, this is also a simple chain from $x_{1}$ to $x_{0}$. Thus
$i_{1}\ R\ i_{0}$.

\item {\bf Transitivity:} Assume $i_{0}\ R\ i_{1}$ and $i_{1}\ R\ i_{2}$, where
$i_{0}\neq i_{1}, i_{0} \neq i_{2}, i_{1}\neq i_{2}$. Let $x_{0} \in J_{i_{0}}$, $x_{1} \in J_{i_{1}}$,
and $x_{2} \in J_{i_{2}}$. Further, let $a_{0}, a_{1}$ and $a_{2}$ denote the
left endpoints of $J_{i_{0}}, J_{i_{1}}$, and $J_{i_{2}}$ respectively. Without
loss of generality we can assume that $a_{0} \leq a_{1}$. Then there are three
possibilities: either (a) $a_{2} < a_{0}$, (b) $a_{0} \leq a_{2} \leq a_{1}$, or
(c) $a_{2} > a_{1}$.

\begin{itemize}[label={},leftmargin=4mm, itemsep=1em, parsep=1em]
\item {\bf Case 2 (ii) (a):} Assume $a_{2} < a_{0}$.

Since $i_{0}\ R\ i_{1}$, $i_{2}\ R\ i_{1}$ by reflexivity. Thus, there is a
simple chain $(J_{i_{2}} = J_{k_{0}}, \dots, J_{i_{1}} = J_{k_{m}})$ from some
$x_{2} \in J_{i_{2}}$ to some $x_{1} \in J_{i_{1}}$. But since $a_{2} < a_{0}
\leq a_{1}$, there exists some $x_{0} \in
\left(\cup_{l=0}^{m}J_{k_{l}}\right)\cap \left(J_{i_{0}}\right)$. So $x_{0}$ is
in the simple chain from $x_{2}$ to $x_{1}$, and therefore there is a simple
chain from $x_{2}$ to $x_{0}$, so $i_{2}\ R\ i_{0}$, and thus $i_{0}\ R\ i_{2}$.

\item {\bf Case 2 (ii) (b):} Assume $a_{0} \leq a_{2} \leq a_{1}$.

Since there is a simple chain $(J_{i_{0}} = J_{k_{0}}, \dots, J_{i_{1}} =
J_{k_{m}})$ from some $x_{0} \in J_{i_{0}}$ to some $x_{1} \in J_{i_{1}}$, and
$a_{0} \leq a_{2} \leq a_{1}$, there exists some $x_{2} \in
\left(\cup_{l=0}^{n}J_{k_{l}}\right)\cap\left(J_{i_{2}}\right)$. So $x_{2}$ is
in the simple chain from $x_{0}$ to $x_{1}$, and therefore there is a simple
chain from $x_{0}$ to $x_{2}$. So $i_{0}\ R\ i_{2}$.

\item {\bf Case 2 (ii) (c):} Assume $a_{2} > a_{1}$.

By assumption, there are simple chains $(J_{i_{0}} = J_{k_{0}}, J_{k_{1}}, \dots, J_{k_{m}} = J_{i_{1}})$ from $x_{0} \in J_{i_{0}}$ to $x_{1} \in J_{i_{1}}$ and
$(J_{i_{1}} = J_{k_{0}}', J_{k_{1}}', \dots, J_{k_{l}}' = J_{i_{2}})$ from $x_{1}$ to $x_{2}
\in J_{i_{2}}$. Now, let 
\[ m^{*} = \min\left\{i : 0\leq i\leq m \text{ and }
J_{k_{i}}\cap\left(\cup_{j=0}^{l}J_{k_{j}}'\right)\neq \emptyset\right\}, \footnote{We
know that this set is non-empty since $J_{k_{m}} \cap J_{k_{1}}' = J_{i_{1}}
\cap J_{k_{1}}' \neq \emptyset$.} \text{ and } \]
\[ p^{*} = \max\left\{ i : 0 \leq i \leq l \text{ and } J_{k_{i}}' \cap J_{k_{m^{*}}} \neq
\emptyset\right\}. \]
Thus $(J_{i_{0}} = J_{k_{0}}, \dots, J_{k_{m^{*}}}, J_{k_{p^{*}}}',
J_{k_{p^{*}+1}}', \dots, J_{k_{l}}' = J_{i_{2}})$ is a simple chain from $x_{0}$
to $x_{2}$, so $i_{0}\ R\ i_{2}$.
\end{itemize}
\end{itemize}
\end{claimproof}

Since $R$ is an equivalence relation on $\{0, \dots, n\}$, the collection of
equivalence classes is a disjoint collection of non-empty subsets of $\{0,\dots,
n\}$ whose union is $\{0, \dots, n\}$. Let's say there are $q + 1$ equivalence
classes denoted by $E_{0}, \dots, E_{q}$. It will be clear from the next claim
and the assumption of case 2 that $q < n$. 

\underline{Claim 2:} $\left(\cup_{k\in E_{i}}J_{k}\right) \cap \left( \cup_{k\in
E_{j}} J_{k}\right) = \emptyset$ for all $i,j \in \{0, \dots, q\}$ where $i \neq
j$.

\begin{claimproof}
We will prove this by contradiction. So assume that $\left(\cup_{k \in
E_{i}}J_{k}\right)\cap \left(\cup_{k=\in E_{j}}J_{k}\right) \neq \emptyset$ for
some $i,j \in \{0, \dots, q\}$ with $i\neq j$. Then $J_{k} \cap J_{l} \neq
\emptyset$ for some $k \in E_{i}$ and $l \in E_{j}$. Thus $(J_{k}, J_{l})$ is a
simple chain from any $x_{0} \in J_{k}$ to $x_{1} \in J_{l}$. So $k\ R\ l$, and
thus $l \in E_{i}$. This is a contradiction since equivalence classes are disjoint.
\end{claimproof}

By claim 2 and our definition of length,
\begin{equation}
\ell\left(\bigcup_{k=0}^{n}J_{k}\right) = \sum_{i=0}^{q}\ell\left(\bigcup_{k\in
E_{i}}J_{k}\right).
\end{equation}
Now, for each $E_{i}$, $0 \leq i \leq q$, let $i_{\min}\in E_{i}$ be the index
of the set whose left endpoint $a_{i_{\min}}$, is the smallest out of all sets
whose index is in $E_{i}$. Similarly, let $i_{\max}$ be the index of the set
with the largest right endpoint, $b_{i_{\max}}$, out of all sets whose index is
in $E_{i}$. Then,
\begin{equation}
\ell\left(\bigcup_{k\in E_{i}}J_{k}\right) = b_{i_{\max}} - a_{i_{\min}} =
|a_{i_{\min}} - b_{i_{\max}}|.
\end{equation}
Further, since $i_{\min}\ R\ i_{\max}$, there is a simple chain $(J_{i_{\min}} =
J_{i_{0}}, J_{i_{1}}, \dots, J_{i_{p}} = J_{i_{\max}})$, with $p \leq |E_{i}|$,
from $x_{0} \in J_{i_{\min}}$ to $x_{1} \in J_{i_{\max}}$. Since this is a
simple chain and each $J_{i}$ is an open interval, $a_{i_{j+1}} < b_{i_{j}}$,
where $a_{i_{j+1}}$ is the left endpoint of $J_{i_{j+1}}$ and $b_{i_{j}}$ is the
right endpoint for $J_{i_{j}}$, for $0 \leq j < p$ (assuming $p > 0$). Thus,
\begin{align*}
\ell\left(\bigcup_{k\in E_{i}}J_{k}\right) \stackrel{(2)}{=} |a_{i_{\min}} -
b_{i_{\max}}| & = |a_{i_{0}} - b_{i_{p}}| \\
& = |a_{i_{0}} - a_{i_{1}} + a_{i_{1}} - a_{i_{2}} + a_{i_{2}} - \dots +
a_{i_{p}} - b_{i_{p}}| \\
& \leq |a_{i_{0}} - a_{i_{1}}| + |a_{i_{1}} - a_{i_{2}}| + \dots + |a_{i_{p}} -
b_{i_{p}}| \\
& \leq |a_{i_{0}} - b_{i_{0}}| + |a_{i_{1}} - b_{i_{1}}| + \dots + |a_{i_{p}} -
b_{i_{p}}| \\
& = \sum_{j=0}^{p}\ell(J_{i_{0}}) \\
& \leq \sum_{k\in E_{i}}\ell(J_{k}).
\end{align*}
So,
\begin{equation}
\ell\left(\bigcup_{k\in E_{i}}J_{k}\right) \leq \sum_{k\in E_{i}}\ell(J_{k}).
\end{equation}
Thus,
\[ \ell\left(\bigcup_{k=0}^{n}J_{k}\right) \stackrel{(1)}{=}
\sum_{i=0}^{q}\ell\left(\bigcup_{k\in E_{i}}J_{k}\right) \stackrel{(3)}{\leq}
\sum_{i=0}^{q}\sum_{k\in E_{i}}\ell(J_{k}) = \sum_{k=0}^{n}\ell(J_{k}). \]

\end{itemize}
\end{proof}

{\bf (3)} Consider the two definitions of outer measure for a set $A \subseteq
\mathbb{R}$:
\[ \mu^{*}(A) = \inf\{\ell(U) : A \subseteq U \subseteq \mathbb{R}, U \neq
\emptyset, U \text{ open}\}, \text{ and } \]
\[ m^{*}(A) = \inf\left\{\sum_{k=1}^{\infty}\ell(I_{k}) : A \subseteq
\cup_{k=1}^{\infty}I_{k} \subseteq \mathbb{R}, I_{k} \neq \emptyset, I_{k}
\text{ an open interval}\right\}. \]
Show that these two definitions are equivalent.

{\bf Solution:}

\begin{proof}
Let $A \subseteq \mathbb{R}$. 

$(\Rightarrow)$ First we will show that $m^{*}(A) \leq \mu^{*}(A)$.

For each $U \in \{U:A \subseteq U \neq \emptyset, U \text{ open}\}$, we have $U = \cup_{k=1}^{n}E_{k}$, where
the $E_{k}$'s are non-empty, disjoint, open intervals. Therefore
$\cup_{k=1}^{n}E_{k}
\subseteq \cup_{k=1}^{\infty}E_{k}$ where, for $k > n$, $E_{k}$ is an arbitrary
non-empty open interval such that $\ell(E_{k}) \leq \frac{\epsilon}{2^{k-n}}$
for some $\epsilon > 0$.
Denote $\left\{\cup_{k=1}^{\infty}E_{k}:A\subseteq U = \cup_{k=1}^{n}E_{k}\right\}$ as the collection of all
such sets that satisfy these conditions for every $U \in \{U:A \subseteq U \neq \emptyset, U \text{ open}\}$.
Thus, 
\[ \left\{ \cup_{k=1}^{\infty}E_{k}:A\subseteq U = \cup_{k=1}^{n}E_{k}\right\} \subseteq
\left\{\cup_{k=1}^{\infty}I_{k}: A \subseteq
\cup_{k=1}^{\infty}I_{k}, I_{k} \neq \emptyset, I_{k}
\text{ an open interval}\right\}, \]
 and so 
\[ \left\{\sum_{k=1}^{\infty}\ell(E_{k}):A\subseteq U = \cup_{k=1}^{n}E_{k}\right\} \subseteq
\left\{\sum_{k=1}^{\infty}\ell(I_{k}): A \subseteq
\cup_{k=1}^{\infty}I_{k}, I_{k}\neq \emptyset, I_{k}\text{ open interval}\right\}, \]
since there is a one-to-one (injective) correspondence between an interval and its length. Hence, 
\begin{align*}
\inf\left\{\sum_{k=1}^{\infty}\ell(I_{k}): A \subseteq
\cup_{k=1}^{\infty}I_{k}\right\} & \leq
\inf\left\{\sum_{k=1}^{\infty}\ell(E_{k}):A\subseteq U = \cup_{k=1}^{n}E_{k}\right\} \\
& = \inf\bigg\{\ell(U) +
\sum_{k=1}^{\infty}\ell(E_{k}):A \subseteq U \neq \emptyset, U \text{ open}, \\
& \qquad E_{k}\text{ open interval s.t. } \ell(E_{k}) \leq \epsilon / 2^{k-n}\bigg\} \\
& = \inf\left\{\ell(U) +
\sum_{K=1}^{\infty}\frac{\epsilon}{2^{k-n}}:A \subseteq U \neq \emptyset, U \text{ open}\right\} \\
& = \inf\left\{\ell(U) + \epsilon :A \subseteq U \neq \emptyset, U \text{ open}\right\} \\
& = \inf\left\{\ell(U):A \subseteq U \neq \emptyset, U \text{ open}\right\} + \epsilon. 
\end{align*}
Since $\epsilon > 0$ was arbitrary,
$\inf\left\{\sum_{k=1}^{\infty}\ell(I_{k}): A \subseteq
\cup_{k=1}^{\infty}I_{k}\right\} \leq \inf\{\ell(U):A
\subseteq U \neq \emptyset, U \text{ open}\}$.
Therefore $m^{*}(A) \leq \mu^{*}(A)$.

$(\Leftarrow)$ Now we will show that $m^{*}(A) \geq \mu^{*}(A)$.

First note that each $\cup_{k=1}^{\infty}I_{k}$ in
$\{\cup_{k=1}^{\infty}I_{k}: A \subseteq
\cup_{k=1}^{\infty}I_{k}, I_{k}\text{ open non-empty interval}\}$ is open and non-empty, and so 
\[ \{\cup_{k=1}^{\infty}I_{k}: A \subseteq \cup_{k=1}^{\infty}I_{k}\}
\subseteq \{U:A \subseteq U \neq \emptyset, U \text{ open}\},\]
since $\{U:A \subseteq U \neq \emptyset, U \text{ open}\}$ is more generally the collection of all
non-empty open sets that cover $A$. Therefore,
\[ \left\{\ell\left(\cup_{k=1}^{\infty}I_{k}\right): A \subseteq
\cup_{k=1}^{\infty}I_{k}\right\} \subseteq \{\ell(U):A \subseteq U \neq \emptyset, U \text{ open}\}.\]
Hence,
\begin{align*}
\inf\{\ell(U):A \subseteq U \neq \emptyset, U \text{ open}\} & \leq 
\inf\left\{\ell\left(\cup_{k=1}^{\infty}I_{k}\right): A \subseteq
\cup_{k=1}^{\infty}I_{k}\right\} \\
\text{Lemma
(II)(1)(xviii)}\Rightarrow\qquad & \leq \inf\left\{ \sum_{k=1}^{\infty}\ell(I_{k}) : A \subseteq
\cup_{k=1}^{\infty}I_{k}\right\}. 
\end{align*}
So $\mu^{*}(A) \leq m^{*}(A)$.

Thus we have shown $\mu^{*}(A) = m^{*}(A)$, and therefore the definitions are equivalent.
\end{proof}

{\bf (4) [RF 2.7]} A set of real numbers is said to be a $G_{\delta}$ set
provided it is the intersection of a countable collection of open sets. Show
that for any bounded set $E$, there is a $G_{\delta}$ set $G$ for which 
\[ E \subseteq G \text{ and } m^{*}(G) = m^{*}(E). \]

{\bf Solution:}

\begin{proof}
Let $E \subset \mathbb{R}$ be bounded. Let $\epsilon > 0$. For each $n \in
\mathbb{N}$, let $G_{n}$ be an open set such that 

\begin{itemize}[label={},leftmargin=8mm, itemsep=1em, parsep=0em]
\item (i) $G_{n} = \cup_{k=1}^{\infty}I_{n,k}$, where $I_{n,k}$ is an open,
non-empty interval for each $k \in \mathbb{N}$,
\item (ii) $E \subseteq G_{n}$, and 
\item (iii) $\sum_{k=1}^{\infty}\ell(I_{n,k}) \leq m^{*}(E) +
\frac{\epsilon}{n}$.
\end{itemize}
Now define $G = \cap_{k=1}^{\infty}G_{n}$. Then $G$ is a $G_{\delta}$ set by
contruction and, for each $n\in\mathbb{N}$,
\begin{align*}
m^{*}(G) \stackrel{\text{monotonicity}}{\leq} m^{*}(G_{n}) & =
m^{*}\left(\cup_{k=1}^{\infty}I_{n,k}\right) \\
\text{(subadditivity)} \qquad & \leq \sum_{k=1}^{\infty}\ell(I_{n,k}) \\
& \leq m^{*}(E) + \frac{\epsilon}{n}.
\end{align*}
So $m^{*}(G) \leq m^{*}(E) + \epsilon / n$, and since the left side of the
inequality does not depend on $n$, 
\begin{equation}
m^{*}(G) \leq m^{*}(E).
\end{equation}

Now, since $E \subseteq G_{n}$ for every $n \in \mathbb{N}$, $E \subseteq G =
\cap_{n=1}^{\infty}G_{n}$. Therefore, 
\begin{equation}
m^{*}(E) \leq m^{*}(G)
\end{equation} 
by monotonicity. Therefore $m^{*}(E) = m^{*}(G)$ by (4) and (5).
\end{proof}

{\bf (5) [RF 2.9]} If $m^{*}(A) = 0$, then $m^{*}(A\cup B) = m^{*}(B)$.

{\bf Solution:}

\begin{proof}
Let $A$ and $B$ be subsets of $\mathbb{R}$ such that $m^{*}(A) = 0$. Now denote
$A_{1} = A$, $A_{2} = B$, and $A_{n} = \emptyset$ for all $n > 2$. Thus,
\[ m^{*}(A\cup B) = m^{*}\left( \cup_{n=1}^{\infty}A_{n} \right)
\stackrel{\text{Thm (II)(1)(xxiv)}}{\leq} \sum_{n=1}^{\infty}m^{*}(A_{n}) =
m^{*}(A) + m^{*}(B) + \sum_{n=2}^{\infty} 0, \]
so 
\begin{equation}
m^{*}(A \cup B) \leq m^{*}(B).
\end{equation}
Further, since $B \subseteq A\cup B$, 
\begin{equation}
m^{*}(B) \leq m^{*}(A\cup B),
\end{equation}
by monotonicity. Therefore (6) and (7) imply that $m^{*}(A\cup B) = m^{*}(B)$.
\end{proof}


{\bf (6) [Claim 1 in the proof of Thm (II)(1)(xxiv)]} $\cup_{(n,j')\in
G_{j}}I_{n,j'} = I_{j}$.

{\bf Solution:}

\begin{proof}
$(\Rightarrow)$ Let $x \in \cup_{(n,j')\in G_{j}}I_{n,j'}$. Then $x \in I_{n_{0},j_{0}}$ for
some $(n_{0},j_{0}) \in G_{j}$. Thus $x \in I_{j}$ since $I_{n_{0},j_{0}}
\subseteq I_{j}$. So,
\begin{equation}
\cup_{(n,j')\in G_{j}}I_{n,j'} \subseteq I_{j}.
\end{equation}
$(\Leftarrow)$ We now need to show that $I_{j} \subseteq \cup_{(n,j')\in
G_{j}}I_{n,j'}$. We will show by contradiction. Thus, assume $I_{j} \supset
\cup_{(n,j')\in G_{j}}I_{n,j'}$. Then there exists some $x \in I_{j}$ such that
$x \notin \cup_{(n,j')\in G_{j}}I_{n,j'}$. But since 
\begin{equation}
\cup_{j\in F}I_{j} = V = \cup_{n=0}^{\infty}U_{n} =
\cup_{n=0}^{\infty}\left(\cup_{j\in F_{n}}I_{n,j}\right),
\end{equation}
$x\in I_{n_{0},j_{0}}$ for some $n_{0} \geq 0$ and $j_{0} \in F_{n_{0}}$, where
$(n_{0}, j_{0}) \notin G_{j}$. Thus $I_{n_{0},j_{0}} \nsubseteq I_{j}$, so 
\begin{equation}
I_{n_{0},j_{0}} \cap \left(\cup_{j^{*}\in F\setminus \{j\}}I_{j^{*}}\right) \neq
\emptyset.
\end{equation}
Further, since $x \in I_{n_{0},j_{0}}$ and $x\in I_{j}$, $x \in
I_{n_{0},j_{0}}\cap I_{j}$, so $I_{n_{0},j_{0}}\cap I_{j}\neq \emptyset$.
Therefore, since $I_{n_{0},j_{0}} \subseteq \cup_{j\in F}I_{j}$, we have
\begin{equation}
I_{n_{0},j_{0}} = \left( I_{n_{0},j_{0}} \cap I_{j} \right) \cup \left(
I_{n_{0},j_{0}} \cap \bigcup_{j^{*}\in F\setminus \{j\}}I_{j^{*}}
\right).
\end{equation}
So $I_{n_{0},j_{0}}$ must be disconnected since $I_{j} \cap \left(
\cup_{j^{*}\in F\setminus \{j\}}I_{j^{*}}
\right) \neq \emptyset$. But this is a contradiction since $I_{n_{0},j_{0}}$ is
an interval. Thus 
\begin{equation}
I_{j} \subseteq \cup_{(n,j')\in G_{j}}I_{n,j'}.
\end{equation}
So (8) and (12) imply that $\cup_{(n,j')\in G_{j}}I_{n,j'} = I_{j}$.
\end{proof}





\end{document}

