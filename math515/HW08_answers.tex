\documentclass[12pt]{article}
\usepackage{amsmath}
\usepackage{amsfonts}
\usepackage{parskip}
\usepackage{amsthm}
\usepackage{thmtools}
\usepackage[headheight=15pt]{geometry}
\geometry{a4paper, left=20mm, right=20mm, top=30mm, bottom=30mm}
\usepackage{graphicx}
\usepackage{bm} % for bold font in math mode - command is \bm{text}
\usepackage{enumitem}
\usepackage{fancyhdr}
\usepackage{amssymb} % for stacked arrows and other shit
\pagestyle{fancy}

\declaretheoremstyle[headfont=\normalfont]{normal}
\declaretheorem[style=normal]{Theorem}
\declaretheorem[style=normal]{Proposition}
\declaretheorem[style=normal]{Lemma}
\newenvironment{claimproof}[1]{\par\noindent\underline{Proof of claim:}\space#1}{\hfill $\blacksquare$\vspace{5mm}}

\title{MATH 515: HW 8}
\author{Evan ``Pete'' Walsh}
\makeatletter
\let\runauthor\@author
\let\runtitle\@title
\makeatother
\lhead{\runauthor}
\chead{\runtitle}
\rhead{\thepage}
\cfoot{}

\begin{document}
%\maketitle

\section*{1 [Example (IV)(1)(ii)(a)]}
If $s$ is a step function on $[a,b]$, then 
\[ \int_{a}^{b} s = \int_{[a,b]}sd\mu. \]

\subsection*{Solution}
\begin{proof}
Let $s : [a,b] \rightarrow \mathbb{R}$ be a step function. Let $(x_{0}, \hdots, x_{n})$ be a partition of $[a,b]$ such that $f$ is constant with value
$c_j$ on each subinterval $(x_{j-1},x_{j})$ for $j = 1,\hdots, n$. By HW 6 Question 1, $s$ is also a simple function on $[a,b]$. Let $\text{ran}(s) = \left\{
a_{1}, \hdots, a_{m} \right\}$. Now, for $i = 1,\hdots, m$, let $F_{i} = \left\{ j : c_{j} = a_{i} \right\}$. 

\underline{Claim 1:} $\left\{ F_{i} \right\}_{i=1}^{m}$ is a pairwise disjoint sequence of sets such that $\cup_{i=1}^{m}F_{i} = \left\{ 1,\hdots,
n \right\}$.
\begin{claimproof}
Suppose $i_{0},i_{1} \in \left\{ 1,\hdots, m \right\}$ such that $i_{0}\neq i_{1}$. Since $a_{i_{0}} \neq a_{i_{1}}$,
\[ \left\{ j : c_{j} = a_{i_{0}} \right\} \cap \left\{ j : c_{j} = a_{i_{1}} \right\} = \emptyset. \]
Thus $\left\{ F_{i} \right\}_{i=1}^{m}$ is pairwise disjoint. Now, clearly $\cup_{i=1}^{m}F_{i} \subseteq \left\{ 1,\hdots, n \right\}$ by construction. Further,
for all $1\leq j \leq n$, $c_{j} = s(x)$ for all $x \in (x_{j-1},x_{j})$. Therefore $c_{j} \in ran(s)$, and so $c_{j} = a_{i'}$ for some $i' \in 
\left\{ 1,\hdots, m \right\}$. Therefore $j \in F_{i'}$. Hence $\left\{ 1,\hdots, n \right\} \subseteq \cup_{i=1}^{m}F_{i}$.
\end{claimproof}

By Proposition (I)(1)(xi) and claim 1,
\begin{equation}
\int_{a}^{b}s = \sum_{j=1}^{n}c_{j}(x_{j} - x_{j-1}) = \sum_{i=1}^{m}\sum_{j\in F_{i}}c_{j}(x_{j} - x_{j-1}) = \sum_{i=1}^{m}\sum_{j\in
F_{i}}a_{i}\mu\left( (x_{j-1},x_{j}) \right).
\label{1.1}
\end{equation}
Further, by the construction of $F_{i}$,
\begin{equation}
\bigcup_{j\in F_{i}}(x_{j-1},x_{j}) = s^{-1}[\{a_{i}\}] - s^{-1}[\{a_{i}\}]\cap \bigcup_{j\in F_{j}}\{x_{j-1},x_{j}\},
\label{1.3}
\end{equation}
for each $i \in \left\{ 1,\hdots, m \right\}$. Thus, using finite additivity and the excision property,
\begin{align}
\sum_{i=1}^{m}a_{i}\sum_{j\in F_{i}} \mu\left( (x_{j-1},x_{j}) \right) & = \sum_{i=1}^{m}a_{i}\mu\left( \bigcup_{j\in F_{i}}(x_{j-1},x_{j}) \right) \nonumber \\
& \stackrel{\eqref{1.3}}{=} \sum_{i=1}^{m}a_{i}\mu\left( s^{-1}[\{a_{i}\}] - s^{-1}[\{a_{i}\}]\cap \bigcup_{j\in F_{i}}\{x_{j-1},x_{j}\} \right) \nonumber \\
& = \sum_{i=1}^{m}a_{i}\left[ \mu\left( s^{-1}[\{a_{i}\}] \right)- \mu\left( s^{-1}[\{a_{i}\}]\cap\bigcup_{j\in F_{i}}\{x_{j-1},x_{j}\} \right)  \right] \nonumber \\
& = \sum_{i=1}^{m}a_{i}\mu\left( s^{-1}[\{a_{i}\}] \right) = \int_{[a,b]}sd\mu. \label{1.2}
\end{align}

Thus by \eqref{1.1} and \eqref{1.2} we have the equality.
\end{proof}


\newpage 
\section*{2 [RF 4.16]}
Let $f$ be a non-negatvie, bounded, measurable function on a set of finite measure $E$.\footnote{We actually don't need the assumptions that $f$ is
bounded and $\mu(E) < \infty$.} Assume 
\[ \int_{E}fd\mu = 0.\] 
Show that $f = 0$ a.e. on $E$.

\subsection*{Solution}
\begin{proof}
Let $A = \left\{ x \in E : f(x) > 0 \right\}$. We want to show that $\mu(A) = 0$. Let $n \in \mathbb{N}$. Define $A_{n} = \left\{ x \in E : f(x) >
2^{-n} \right\}$. Since $f$ is measurable, $A_{n}$ is measurable. Thus,
\[ 0 = \int_{E}fd\mu \geq \int_{E}f\chi_{A_{n}}d\mu = \int_{A_{n}}fd\mu \geq \int_{A_{n}}2^{-n}d\mu = 2^{-n}\mu(A_{n}), \]
which implies $\mu(A_{n}) = 0$. Now, 
\[ A = \bigcup_{n=0}^{\infty}A_{n}, \]
so by the continuity of measure, $\mu(A) = \lim_{n\rightarrow\infty}\mu(A_{n}) = 0$. Therefore $f = 0$ a.e. on $E$.
\end{proof}


\newpage 
\section*{3 [RF 4.17]}
Let $E$ be a set of measure zero and define $f \equiv \infty$ on $E$. Show that $\int_{E}fd\mu = 0$.

\subsection*{Solution}
\begin{proof}
For each $n \in \mathbb{N}$, let $s_{n} : E \rightarrow [0,\infty)$ be defined by $s_{n}(x) = n$ for all $x \in E$. Then 
\[ \int_{E} s_{n}d\mu = n\cdot \mu(E) = 0, \ \text{ for all } n \in \mathbb{N}. \]
Now, since $\left\{ s_{n} \right\}_{n \in \mathbb{N}}$ is a sequence of measurable increasing functions and $s_{n}\rightarrow f$ on $E$,
\[ \int_{E}fd\mu = \lim_{n\rightarrow\infty}\int_{E}s_{n}d\mu = 0, \]
by the MCT.
\end{proof}



\newpage 
\section*{4 [RF 4.25]}
Let $\left\{ f_{n} \right\}_{n\in\mathbb{N}}$ be a sequence of non-negative, measurable functions on $E$ that converges pointwise on $E$ to $f$.
Suppose $f_{n} \leq f$ on $E$ for each $n \in \mathbb{N}$. Show that 
\[ \lim_{n\rightarrow\infty}\int_{E}f_{n}d\mu = \int_{E}fd\mu. \]

\subsection*{Solution} 
\begin{proof}
By Fatou's Lemma,
\begin{equation}
\int_{E}fd\mu = \int_{E}\liminf_{n\rightarrow\infty}f_{n}d\mu \leq \liminf_{n\rightarrow\infty}\int_{E}f_{n}d\mu.
\label{4.1}
\end{equation}
Further, since $f_{n} \leq f$ for each $n \in \mathbb{N}$,
\[ \int_{E}f_{n}d\mu \leq \int_{E}fd\mu,\]
by Proposition (IV)(2)(vi). Thus,
\begin{equation}
\limsup_{n\rightarrow\infty}\int_{E}f_{n}d\mu \leq \int_{E}fd\mu.
\label{4.2}
\end{equation}
So by \eqref{4.1} and \eqref{4.2} we have
\[ \lim_{n\rightarrow\infty}\int_{E}f_{n}d\mu = \int_{E}fd\mu. \]
\end{proof}












\end{document}

