\documentclass[12pt]{article}
\usepackage{amsmath}
\usepackage{amsfonts}
\usepackage{parskip}
\usepackage{amsthm}
\usepackage{thmtools}
\usepackage[headheight=15pt]{geometry}
\geometry{a4paper, left=20mm, right=20mm, top=30mm, bottom=30mm}
\usepackage{graphicx}
\usepackage{bm} % for bold font in math mode - command is \bm{text}
\usepackage{enumitem}
\usepackage{fancyhdr}
\usepackage{amssymb} % for stacked arrows and other shit
\pagestyle{fancy}
\usepackage{changepage}
\usepackage{mathcomp}

\declaretheoremstyle[headfont=\normalfont]{normal}
\declaretheorem[style=normal]{Theorem}
\declaretheorem[style=normal]{Proposition}
\declaretheorem[style=normal]{Lemma}
\newcounter{ProofCounter}
\newcounter{ClaimCounter}[ProofCounter]
\newcounter{SubClaimCounter}[ClaimCounter]
\newenvironment{Proof}{\stepcounter{ProofCounter}\textit{Proof.}}{\hfill$\square$}
\newenvironment{claim}[1]{\vspace{3mm}\stepcounter{ClaimCounter}\par\noindent\underline{\bf Claim \theClaimCounter:}\space#1}{}
\newenvironment{claimproof}[1]{\par\noindent\underline{Proof of claim \theClaimCounter:}\space#1}{\hfill $\blacksquare$ Claim \theClaimCounter}
\newenvironment{subclaim}[1]{\stepcounter{SubClaimCounter}\par\noindent\emph{Subclaim \theClaimCounter.\theSubClaimCounter:}\space#1}{}
% \newenvironment{subclaimproof}[1]{\begin{adjustwidth}{2em}{0pt}\par\noindent\emph{Proof of subclaim \theClaimCounter.\theSubClaimCounter:}\space#1}{\hfill
% $\blacksquare$ \emph{Subclaim \theClaimCounter.\theSubClaimCounter}\vspace{5mm}\end{adjustwidth}}
\newenvironment{subclaimproof}[1]{\par\noindent\emph{Proof of subclaim \theClaimCounter.\theSubClaimCounter:}\space#1}{\hfill
$\Diamond$ \emph{Subclaim \theClaimCounter.\theSubClaimCounter}}

\title{Spring 2001 Analysis Qualifier}
\author{Evan P. Walsh}
\makeatletter
\let\runauthor\@author
\let\runtitle\@title
\makeatother
\lhead{\runauthor}
\chead{\runtitle}
\rhead{\thepage}
\cfoot{}

\begin{document}
% \maketitle

\section*{Part I: Real Analysis}

\subsection*{1}
Let $(X, \mathcal{A}, \mu)$ be an arbitrary measure space with $\mu$ a positive measure. Prove that $(X, \mathcal{A}, \mu)$ is $\sigma$-finite if and
only if there exists a strictly positive function $f \in L^{1}(\mu)$.

{\bf Solution:}

\begin{Proof}
$(\Rightarrow)$ Suppose $(X, \mathcal{A}, \mu)$ is $\sigma$-finite. Then there exists sets $\left\{ A_{n} \right\}_{n=0}^{\infty} \subseteq
\mathcal{A}$ such that $\mu(A_{n}) < \infty$ for all $n \in \mathbb{N}$ and $X = \cup_{n=0}^{\infty}A_{n}$. Without loss of generality assume 
$\left\{ A_{n} \right\}_{n=0}^{\infty}$ is pairwise disjoint and $A_{n} \neq \emptyset$ for all $n \in \mathbb{N}$. Let $f : X \rightarrow \mathbb{R}$
be defined 
\[ f(x) := \sum_{n=0}^{\infty}\frac{2^{-n}}{\mu(A_{n})}\chi_{A_{n}}(x). \] 
Then $f$ is strictly positive and 
\[ \|f\|_{1} = \sum_{n=0}^{\infty}2^{-n} = 2 < \infty. \]
$(\Leftarrow)$ Suppose there exists a strictly positive function $f \in L^{1}(\mu)$. By way of contradiction assume $(X, \mathcal{A}, \mu)$ is not
$\sigma$-finite. For each $n \in \mathbb{N}$, let $A_{n} := \left\{ x \in X : f(x) > 2^{-n} \right\}$. Since $f$ is measurable, $A_{n}$ is measurable
for all $n$. Also, since $\cup_{n=0}^{\infty}A_{n} = X$, there must exist some $n_{0} \in \mathbb{N}$ such that $\mu(A_{n_{0}}) = \infty$. Therefore,
\[ \|f\|_{1} = \int_{X} f\ d\mu \geq \int_{A_{n_{0}}}f \ d\mu \geq \int_{A_{n_{0}}} 2^{-n_{0}} \ d\mu = \infty. \]
This contradicts the assumption that $f \in L^{1}(\mu)$.
\end{Proof}

\newpage
\subsection*{2}
Give an example of each of the following:
\begin{enumerate}[label=(\alph*)]
\item A function $f$ which is unbounded but Lebesgue integrable on $(0,\infty)$.
\item A function $f$ which is Lipschitz continuous but not differentiable everywhere.
\item A function $f$ which is absolutely continuous but not Lipschitz continuous on $[0,1]$.
\item A sequence $\left\{ f_{n} \right\}$ of functions that converges to $0$ pointwise on $[0,1]$ but not in $L^{1}([0,1])$.
\end{enumerate}

{\bf Solution:}
\begin{enumerate}[label=(\alph*)]
\item Let $f : (0,\infty) \rightarrow \mathbb{R}$ be defined by 
\[ f(x) := \sum_{n=0}^{\infty}2^{-n}\chi_{(n,n+1]}(x). \]
\item Let $f(x) := |x|$.
\item Let $f(x) := \sqrt{x}$.
\item For each $n \in \mathbb{N}$, let $f_{n}(x) := 2^{-n}\chi_{(0,2^{-n})}(x)$.
\end{enumerate}


\newpage
\subsection*{3}
Let $p, q > 1$ be conjugate and $\Omega \subset \mathbb{R}^{N}$.
\begin{enumerate}[label=(\alph*)]
\item Show that if $f_{n} \rightarrow f$ in $L^{p}(\Omega)$ and $g_{n} \rightarrow g$ in $L^{q}(\Omega)$, then $f_{n}g_{n} \rightarrow fg$ in
$L^{1}(\Omega)$.
\item Explain carefully what is meant by the statement that $L^{q}(\Omega)$ is the dual space of $L^{p}(\Omega)$.
\end{enumerate}

{\bf Solution:}
\begin{enumerate}[label=(\alph*)]
\item 
\begin{Proof}
Since $f_{n} \rightarrow f$ in $L^{p}(\Omega)$, there exists $N \in \mathbb{N}$ such that $n \geq N$ implies $\|f_{n} - f\|_{p} < 1$. Thus, for $n
\geq N$,
\[ \bigg| \|f_{n}\|_{p} - \|f\|_{p}\bigg| \leq \|f_{n} - f\|_{p} < 1, \]
so $\left\{ \|f_{n}\|_{p} \right\}_{n\geq N}$ is bounded.
But since $\left\{ \|f_{n}\|_{p} \right\}_{n<N}$ is a finite collection of reals, $\left\{ \|f_{n}\|_{p} \right\}$ is bounded. 
Thus there exists $M > 0$ such that $\|f_{n}\|_{p} < M$ for all $n$. Therefore, by Minkowski's and multiple use of H\"{o}lder's Inequality,
\begin{align*}
\|f_{n}g_{n} - fg\|_{1} = \|f_{n}g_{n} - f_{n}g + f_{n}g - fg\|_{1} & \leq \|f_{n}(g_{n} - g)\|_{1} + \|g(f_{n} - f)\|_{1} \\
& \leq \|f_{n}\|_{p}\cdot\|g_{n} - g\|_{q} + \|g\|_{q}\cdot\|f_{n} - f\|_{p} \\
& \leq M\cdot\|g_{n} - g\|_{q} + \|g\|_{q}\cdot\|f_{n} - f\|_{p} \stackrel{n\rightarrow\infty}{\longrightarrow} 0.
\end{align*}
\end{Proof}

\item To say that $L^{q}(\Omega)$ is the dual space of $L^{p}(\Omega)$ means that for every bounded linear function $T$ on $L^{p}(\Omega)$, there exists 
$g \in L^{q}(\Omega)$ such that
\[ T(f) = \int_{\Omega} f\cdot g\ d\mu, \]
for all $f \in L^{p}(\Omega)$.
\end{enumerate}


\newpage
\subsection*{4}
If $f \in L^{q}(\mathbb{R}^{N})$ for some $q < \infty$, show that 
\[ \lim_{p\rightarrow\infty} \|f\|_{p} = \|f\|_{\infty}. \]
Also show by example that the conclusion may be false without the assumption that $f \in L^{q}(\mathbb{R}^{N})$.

{\bf Solution:}

\begin{Proof}
Without loss of generality assume $\|f\|_{\infty} < \infty$. Let $0 < \delta < \|f\|_{\infty}$ and let $E := \left\{ x : |f(x)| \geq \|f\|_{\infty} - \delta \right\}$. 
Since $f \in L^{q}(\mathbb{R}^{N})$, $\mu(E) < \infty$. Now for any $0 < p < \infty$,
\[ \|f\|_{p} \geq \left( \int_{E}(\|f\|_{\infty} - \delta)^{p} d\mu\right)^{1/p} = (\|f\|_{\infty} - \delta)\cdot \mu(E)^{1/p}, \]
so $\liminf_{p \rightarrow \infty}\|f\|_{p} \geq \lim_{p\rightarrow\infty}\|f\|_{\infty} - \delta$. Thus,
\begin{equation}
\liminf_{p \rightarrow \infty} \|f\|_{p} \geq \|f\|_{\infty}.
\label{1.4.1}
\end{equation}
Further, for any $0 < p < \infty$,
\[ \|f\|_{p} = \left( \int |f|^{p-q}|f|^{q}\ d\mu \right)^{1/p} \leq \|f\|_{\infty}^{\frac{p-q}{p}}\|f\|_{q}^{q/p}. \]
Hence 
\begin{equation}
\limsup_{p\rightarrow\infty}\|f\|_{p} \leq \|f\|_{\infty}.
\label{1.4.2}
\end{equation}
So by \eqref{1.4.1} and \eqref{1.4.2} we have equality.
\end{Proof}

For an illustration of how this relationship fails when there does not exist a $q < \infty$ such that $f \in L^{q}(\mathbb{R}^{N})$, consider any
constant function $f(x) \equiv c$. Then $\|f\|_{\infty} = c$ but $\|f\|_{p} = \infty$ for all $0 < p < \infty$.


\end{document}

