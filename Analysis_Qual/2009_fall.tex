\documentclass[12pt]{article}
\usepackage{amsmath}
\usepackage{amsfonts}
\usepackage{parskip}
\usepackage{amsthm}
\usepackage{thmtools}
\usepackage[headheight=15pt]{geometry}
\geometry{a4paper, left=20mm, right=20mm, top=30mm, bottom=30mm}
\usepackage{graphicx}
\usepackage{bm} % for bold font in math mode - command is \bm{text}
\usepackage{enumitem}
\usepackage{fancyhdr}
\usepackage{amssymb} % for stacked arrows and other shit
\pagestyle{fancy}
\usepackage{changepage}
\usepackage{mathcomp}

\declaretheoremstyle[headfont=\normalfont]{normal}
\declaretheorem[style=normal]{Theorem}
\declaretheorem[style=normal]{Proposition}
\declaretheorem[style=normal]{Lemma}
\newcounter{ProofCounter}
\newcounter{ClaimCounter}[ProofCounter]
\newcounter{SubClaimCounter}[ClaimCounter]
\newenvironment{Proof}{\stepcounter{ProofCounter}\textit{Proof.}}{\hfill$\square$}
\newenvironment{claim}[1]{\vspace{3mm}\stepcounter{ClaimCounter}\par\noindent\underline{\bf Claim \theClaimCounter:}\space#1}{}
\newenvironment{claimproof}[1]{\par\noindent\underline{Proof of claim \theClaimCounter:}\space#1}{\hfill $\blacksquare$ Claim \theClaimCounter}
\newenvironment{subclaim}[1]{\stepcounter{SubClaimCounter}\par\noindent\emph{Subclaim \theClaimCounter.\theSubClaimCounter:}\space#1}{}
% \newenvironment{subclaimproof}[1]{\begin{adjustwidth}{2em}{0pt}\par\noindent\emph{Proof of subclaim \theClaimCounter.\theSubClaimCounter:}\space#1}{\hfill
% $\blacksquare$ \emph{Subclaim \theClaimCounter.\theSubClaimCounter}\vspace{5mm}\end{adjustwidth}}
\newenvironment{subclaimproof}[1]{\par\noindent\emph{Proof of subclaim \theClaimCounter.\theSubClaimCounter:}\space#1}{\hfill
$\Diamond$ \emph{Subclaim \theClaimCounter.\theSubClaimCounter}}

\title{Fall 2009 Analysis Qualifier}
\author{Evan P. Walsh}
\makeatletter
\let\runauthor\@author
\let\runtitle\@title
\makeatother
\lhead{\runauthor}
\chead{\runtitle}
\rhead{\thepage}
\cfoot{}

\begin{document}
% \maketitle

\section*{Part II: Real Analysis}

\subsection*{1}
Let $f$ be a Borel measurable function which is Lebesgue integrable on $\mathbb{R}$. Suppose for every $a < b$ that 
$\int_{a}^{b}f\ d\mu = 0$. Show that $f = 0$ a.e.

{\bf Solution:}
\begin{Proof}
Let $E := \left\{ x \in \mathbb{R} : f(x) > 0 \right\}$. We will show that $\mu(E) = 0$. The proof that $\mu\left( \left\{ x : f(x) < 0 \right\}
\right) = 0$ follows by the same argument. Since $f$ is measurable, $E$ is measurable. Let $E_{n} := E \cap (-n,-n+1]\cup[n-1,n)$ for all $n \geq 1$.
Thus $E = \cup_{n=1}^{\infty}E_{n}$ and each $E_{n}$ is measurable, and $\left\{ E_{n} \right\}_{n=1}^{\infty}$ is pairwise disjoint. 
Thus, for each $n$ there exists a $G_{\delta}$ set $G_{n}$ such that $G_{n}\supseteq
E_{n}$ and 
\begin{equation}
\mu(G_{n} - E_{n}) = 0. 
\label{2.1.1}
\end{equation}
Without loss of generality assume $G_{n} = \cap_{k=1}^{\infty}O_{n,k}$ where $\left\{ O_{n,k}
\right\}_{k=1}^{\infty}$ is a descending sequence of bounded, open sets. But then $O_{n,k} = \cup_{j\in \mathcal{F}_{n,k}}I_{j}$, where $I_{j}$ is a bounded,
open interval for every $j \in \mathcal{F}_{n,k}$, $k,n \geq 1$, and therefore 
\[ \int_{I_{j}} f\ d\mu = 0 \]
by assumption. Thus,
\begin{equation}
\int_{O_{n,k}}f\ d\mu = \sum_{j\in\mathcal{F}_{n,k}}\int_{I_{j}}f\ d\mu = 0.
\label{2.1.2}
\end{equation}
Hence, using the continuity of integration,
\[ \int_{E}f\ d\mu = \sum_{n=1}^{\infty}\int_{E_{n}}f\ d\mu \stackrel{\eqref{2.1.1}}{=} \sum_{n=1}^{\infty}\int_{G_{n}}f\ d\mu  
= \sum_{n=1}^{\infty}\left[ \lim_{k\rightarrow\infty}\int_{O_{n,k}}f\ d\mu \right] \stackrel{\eqref{2.1.2}}{=} 0. \]
Since $\int_{E}f\ d\mu = 0$ and $f > 0$ on $E$, $\mu(E) = 0$.
\end{Proof}


\subsection*{2}
This is just the Borel-Cantelli Lemma.

\newpage
\subsection*{3}
Let $f$ be continuous on $\mathbb{R}$ and let 
\[ F_{n}(x) := \frac{1}{n}\sum_{k=0}^{n-1}f\left( x + \frac{k}{n} \right). \]
Prove that $F_{n}$ converges uniformly on every finite interval $[a,b]$.

{\bf Solution:}

\begin{Proof}
Let $-\infty < a < b < \infty$.

\begin{claim}
$F_{n}(x)$ converges pointwise to $\int_{x}^{x+1}f\ d\mu$ for all $x \in [a,b]$.
\end{claim}
\begin{claimproof}
Let $x \in [a,b]$. For each $n \geq 1$ and $0 \leq k < n$, let 
\[ E_{n,k} := \left\{ t : x + \sum_{i=0}^{k-1}\frac{i}{n} \leq t < x + \sum_{i=0}^{k}\frac{i}{n} \right\}. \]
Now, since $f$ is contiuous on $[x,x+1]$, there exists $M > 0$ such that $|f(t)| < M$ for all $t \in [x,x+1]$. Therefore $\sum_{k=0}^{n-1}f\left(
x + \frac{k}{n} \right)\chi_{E_{n,k}}(t) \leq M$ for all $n \in \mathbb{N}, t \in [x,x+1]$. Thus, by the Dominated Convergence Theorem,
\[ \lim_{n\rightarrow\infty}F_{n}(x) = \lim_{n\rightarrow\infty}\frac{1}{n}\sum_{k=0}^{n-1}f\left( x + \frac{k}{n} \right) = \lim_{n\rightarrow\infty}
\int_{x}^{x+1} \sum_{k=0}^{n-1}f\left( x + \frac{k}{n} \right)\chi_{E_{n,k}}(t)\ d\mu(t) = \int_{x}^{x+1}f\ d\mu. \]
\end{claimproof}

Now we will show that the convergence of $\left\{ F_{n} \right\}$ is uniform. Let $\epsilon > 0$. Since $f$ is continuous on $[a,b+1]$, $f$ is
uniformly contiuous on $[a,b+1]$. Therefore there exists $\delta > 0$ such that $|f(t_{0}) - f(t_{1})| < \epsilon/2$ whenever $|t_{0} - t_{1}| <
\delta$. Choose $N \geq 1$ such that $N^{-1} < \delta$. Thus, for $n \geq N$ and $x \in [a,b]$,
\begin{align*}
\left| F_{n}(x) - \int_{x}^{x+1}f\ d\mu \right| & = \left| \sum_{k=0}^{n-1} \int_{E_{n,k}}f\left( x + \frac{k}{n} \right)d\mu - \int_{x}^{x+1}f\
d\mu\right| \\
& = \left| \sum_{k=0}^{n-1}\int_{E_{n,k}}\left[f\left( x + \frac{k}{n} \right) - f(x)\right] d\mu \right| \\
& \leq \sum_{k=0}^{n-1}\int_{E_{n,k}}\left|f\left( x + \frac{k}{n} \right) - f(x)\right|d\mu \\
& \leq \sum_{k=0}^{n-1} \int_{E_{n,k}} \frac{\epsilon}{2}\ d\mu \\
& = \sum_{k=0}^{n-1}\frac{\epsilon}{2n} = \frac{\epsilon}{2} < \epsilon.
\end{align*}
\end{Proof}

\newpage
\subsection*{4}
Let $\left\{ f_{n} \right\}$, $f$ be measurable functions on $[0,1]$. Show that $f_{n} \rightarrow f$ is measure on $[0,1]$ if and only if 
\[ \lim_{n\rightarrow\infty} \int_{0}^{1} \frac{|f_{n} - f|}{1 + |f_{n} - f|} \ d\mu = 0. \]

{\bf Solution:}

\begin{Proof}
Let $g(x) = \frac{|x|}{1+|x|}$, $x \geq 0$.

$(\Rightarrow)$ First suppose $f_{n} \rightarrow f$ in measure on $[0,1]$. Let $\epsilon > 0$. Then
\begin{align*}
\int_{0}^{1}g\circ f\ d\mu  = \int_{\{|f_{n} - f| > \epsilon\}}g\circ f\ d\mu + \int_{\{|f_{n} - f| \leq \epsilon\}}g\circ f\ d\mu & \leq
\int_{\{|f_{n} - f|>\epsilon\}}1\ d\mu + \int_{0}^{1}\epsilon \ d\mu \\
& = \mu\left( \left\{ |f_{n} - f| > \epsilon \right\} \right) + \epsilon.
\end{align*}
Therefore $0 \leq \limsup_{n\rightarrow\infty} \int_{0}^{1} g\circ f \ d\mu \leq \epsilon$, so $\lim_{n\rightarrow\infty}\int_{0}^{1}g\circ f\ d\mu =
0$.

$(\Leftarrow)$ Now suppose $\lim_{n\rightarrow\infty}\int_{0}^{1}g\circ f\ d\mu = 0$. Let $\epsilon > 0$. Then
\begin{align*}
\mu\left( \left\{ x \in [0,1] : |f_{n}(x) - f(x)| > \epsilon \right\} \right) = \mu\left( \left\{ x\in [0,1] : g(f(x)) > g(\epsilon) \right\} \right) & \leq
g(\epsilon) \int_{\{|f_{n} - f| > \epsilon\}}g\circ f\ d\mu  \\
& \leq \int_{0}^{1} g\circ f \ d\mu.
\end{align*}
Therefore 
\[ 0 \leq \limsup_{n\rightarrow\infty}\mu\left( \left\{ |f_{n} - f| > \epsilon \right\} \right) \leq \limsup_{n\rightarrow\infty}\int_{0}^{1}g\circ f\ d\mu = 0. \]
\end{Proof}

\newpage
\subsection*{5}
Suppose $f : \mathbb{R} \rightarrow [0, \infty)$ is measurable and $\int f\ d\mu = c$, where $0 < c < \infty$. Prove 
\[ \lim_{n\rightarrow\infty} \int n \log\left[ 1 + \left( \frac{f(x)}{n} \right)^{\alpha} \right]d\mu(x) = \left\{ \begin{array}{cl}
\infty & \text{ if } 0 < \alpha < 1 \\
c & \text{ if } \alpha = 1 \\
0 & \text{ if } \alpha > 1.
\end{array}\right. \]
Hint: Show when $\alpha \geq 1$ that the integrand is dominated by $\alpha f(x)$.

{\bf Solution:}

\begin{Proof}
Let $g\circ f := n\log\left[ 1 + \left( \frac{f}{n} \right)^{\alpha} \right]$. Then for all $x \in \mathbb{R}$,
\begin{equation}
\lim_{n\rightarrow\infty}g(f(x)) = \lim_{n\rightarrow\infty}\log\left[ 1 + \left( \frac{f(x)^{\alpha}}{n^{\alpha - 1}} \right)\left( \frac{1}{n} \right)
 \right]^{n} = \left\{ \begin{array}{cl}
\infty & \text{ if } 0 < \alpha < 1 \\
f(x) & \text{ if } \alpha = 1 \\
0 & \text{ if } \alpha > 1.
\end{array} \right.
\label{2.5.1}
\end{equation}
Now, for $\varphi(x) \geq 0$ and $\alpha \geq 1$, we have $1 + \varphi(x)^{\alpha} \leq (1 + \varphi(x))^{\alpha}$. Thus 
\[ \log(1 + \varphi(x))^{\alpha} \leq \alpha \log(1 + \varphi(x)) \leq \alpha \varphi(x). \]
Hence, for $\alpha \geq 1$,
\[ g(f(x)) \leq n\alpha\frac{f(x)}{n} = \alpha f(x). \]
So by applying the Dominated Convergence Theorem we have the result for $\alpha = 1$ and $\alpha > 1$. For $0 < \alpha < 1$ we can apply Fatou's Lemma.
\end{Proof}


\newpage
\subsection*{6}
Let $\left\{ f_{n} \right\}_{n=0}^{\infty}$ be a sequence of increasing, continuously differentiable functions on the interval $[a,b]$ such that, for
all $x \in [a,b]$, $s(x) := \sum_{n=0}^{\infty}|f_{n}(x)| < \infty$.
Show that 
\[ s'(x) := \sum_{n=0}^{\infty}f_{n}'(x)\ \ \text{a.e.} \]

{\bf Solution:} Note that this is essentially Fubini's theorem on term-by-term differentiation and the assumption that $f_{n}'$ is continuous is not
needed.

\begin{Proof}
Clearly $s$ is increasing so $s'$ exists a.e. on $(a,b)$. Now suppose let $x \in (a,b)$ such that $s'(x)$ exists. Then
\begin{align*}
s'(x) = \lim_{k\rightarrow\infty}\text{Diff}_{2^{-k}}(s)(x) = \lim_{k\rightarrow\infty}\frac{\sum_{n=0}^{\infty}f_{n}(x + 2^{-k}) - \sum_{n=0}^{\infty}
f_{n}(x)}{2^{-k}} 
& = \lim_{k\rightarrow\infty}\sum_{n=0}^{\infty}\frac{f_{n}(x + 2^{-k}) - f_{n}(x)}{2^{-k}} \\
& = \lim_{k\rightarrow\infty}\sum_{n=0}^{\infty}\text{Diff}_{2^{-k}}(f_{n})(x) \\
\text{(Fatou's) } & \geq \sum_{n=0}^{\infty} \liminf_{k\rightarrow \infty}\text{Diff}_{2^{-k}}(f_{n})(x).
\end{align*}
So 
\begin{equation}
s'(x) \geq \sum_{n=0}^{\infty}f_{n}'(x).
\label{2.6.1}
\end{equation}
Now let for each $k \in \mathbb{N}$, let $\alpha_{k}(x) := \sum_{n=k+1}^{\infty}f_{n}(x)$. Clearly each $\alpha_{k}$ is an increasing function of $x$.
Thus
\[0 \leq \int_{a}^{b} \alpha_{k}'\ d\mu \leq \alpha_{k}(b) - \alpha_{k}(a) = \sum_{n=k+1}^{\infty}[f_{n}(b) - f_{n}(a)], \]
and since $\sum_{n=k+1}^{\infty}[f_{n}(b) - f_{n}(a)]$ converges for all $k \in \mathbb{N}$, 
\begin{equation}
\lim_{k\rightarrow\infty} \int_{a}^{b}\alpha_{k}'\ d\mu = 0.
\label{2.6.2}
\end{equation}
Therefore 
\[ \int_{a}^{b}s'\ d\mu = \int_{a}^{b}\left( \sum_{n=0}^{k}f_{n} \right)' d\mu + \int_{a}^{b}\alpha_{k}'\ d\mu 
= \int_{a}^{b}\sum_{n=0}^{k}f_{n}'\ d\mu + \int_{a}^{b}\alpha_{k}'\ d\mu 
\leq \int_{a}^{b} \sum_{n=0}^{\infty}f_{n}'\ d\mu + \int_{a}^{b}\alpha_{k}'\ d\mu. \]
Hence by \eqref{2.6.2}
\begin{equation}
\int_{a}^{b}s'\ d\mu \leq \int_{a}^{b}\sum_{n=0}^{\infty}f_{n}'\ d\mu + \lim_{k\rightarrow\infty}\int_{a}^{b}\alpha_{k}'\ d\mu =
\int_{a}^{b}\sum_{n=0}^{\infty}f_{n}'\ d\mu.
\label{2.6.3}
\end{equation}
However, the only way that \eqref{2.6.1} and \eqref{2.6.3} can be consisted is if $s'(x) = \sum_{n=0}^{\infty}f_{n}'(x)$ a.e.
\end{Proof}


\end{document}

