\documentclass[12pt]{article}
\usepackage{amsmath}
\usepackage{amsfonts}
\usepackage{parskip}
\usepackage{amsthm}
\usepackage{thmtools}
\usepackage[headheight=15pt]{geometry}
\geometry{a4paper, left=20mm, right=20mm, top=30mm, bottom=30mm}
\usepackage{graphicx}
\usepackage{bm} % for bold font in math mode - command is \bm{text}
\usepackage{enumitem}
\usepackage{fancyhdr}
\usepackage{amssymb} % for stacked arrows and other shit
\pagestyle{fancy}
\usepackage{changepage}
\usepackage{mathcomp}

\declaretheoremstyle[headfont=\normalfont]{normal}
\declaretheorem[style=normal]{Theorem}
\declaretheorem[style=normal]{Proposition}
\declaretheorem[style=normal]{Lemma}
\newcounter{ProofCounter}
\newcounter{ClaimCounter}[ProofCounter]
\newcounter{SubClaimCounter}[ClaimCounter]
\newenvironment{Proof}{\stepcounter{ProofCounter}\textit{Proof.}}{\hfill$\square$}
\newenvironment{claim}[1]{\vspace{3mm}\stepcounter{ClaimCounter}\par\noindent\underline{\bf Claim \theClaimCounter:}\space#1}{}
\newenvironment{claimproof}[1]{\par\noindent\underline{Proof of claim \theClaimCounter:}\space#1}{\hfill $\blacksquare$ Claim \theClaimCounter}
\newenvironment{subclaim}[1]{\stepcounter{SubClaimCounter}\par\noindent\emph{Subclaim \theClaimCounter.\theSubClaimCounter:}\space#1}{}
% \newenvironment{subclaimproof}[1]{\begin{adjustwidth}{2em}{0pt}\par\noindent\emph{Proof of subclaim \theClaimCounter.\theSubClaimCounter:}\space#1}{\hfill
% $\blacksquare$ \emph{Subclaim \theClaimCounter.\theSubClaimCounter}\vspace{5mm}\end{adjustwidth}}
\newenvironment{subclaimproof}[1]{\par\noindent\emph{Proof of subclaim \theClaimCounter.\theSubClaimCounter:}\space#1}{\hfill
$\Diamond$ \emph{Subclaim \theClaimCounter.\theSubClaimCounter}}

\title{Spring 2015 Analysis Qualifier}
\author{Evan P. Walsh}
\makeatletter
\let\runauthor\@author
\let\runtitle\@title
\makeatother
\lhead{\runauthor}
\chead{\runtitle}
\rhead{\thepage}
\cfoot{}

\begin{document}
% \maketitle

\section*{Part II: Real Analysis}

\subsection*{3}
Let $1 < p < \infty$ and $q$ be conjugate to $p$. Fix a real-valued function $g \in L^{q}(\mathbb{R})$ and define the functional $F :
L^{p}(\mathbb{R}) \rightarrow \mathbb{R}$ by 
\[ F(f) := \int_{\mathbb{R}}f\cdot g \ d\mu, \]
for all $f \in L^{p}(\mathbb{R})$. Show that 
\begin{enumerate}[label=(\roman*)]
\item $F$ is in fact a linear functional on $L^{p}(\mathbb{R})$.
\item $||F|| = ||g||_{q}$ where $||F|| := \sup\left\{ |F(f)| : ||f||_{p} = 1 \right\}$.
\end{enumerate}

{\bf Solution:}

\begin{Proof}
Let $g \in L^{q}(\mathbb{R})$.

\begin{claim}
$F$ is a linear functional.
\end{claim}
\begin{claimproof}
Let $f, h \in L^{p}(\mathbb{R}), \alpha, \beta \in \mathbb{R}$. Then 
\[ F(\alpha f + \beta h) = \int_{\mathbb{R}}(\alpha f + \beta h) g \ d\mu = \alpha \int_{\mathbb{R}}f\cdot g\ d\mu + \beta \int_{\mathbb{R}}h\cdot g\
d\mu = \alpha F(f) + \beta F(h). \]
\end{claimproof}

Now note that for any $f \in L^{p}(\mathbb{R})$,
\begin{equation}
|F(f)| = \left| \int_{\mathbb{R}}f\cdot g\ d\mu \right| \leq \int_{\mathbb{R}}|f\cdot g|\ d\mu \leq ||f||_{p}\cdot ||g||_{q} = ||g||_{q},
\label{2.3.1}
\end{equation}
by H\"{o}lder's Inequality. Now define $f^{*} := ||g||_{q}^{1 - q}\text{sign}(g)|g|^{q-1}$.
\begin{claim}
$f^{*} \in L^{p}(\mathbb{R})$.
\end{claim}
\begin{claimproof}
We have 
\[ ||f^{*}||_{p}^{p} = \int_{\mathbb{R}}|f^{*}|^{p}\ d\mu = ||g||_{q}^{1-q}\int_{\mathbb{R}}|g|^{qp - p}\ d\mu = ||g||_{q} < \infty. \]
\end{claimproof}

\begin{claim}
$||F|| = ||g||_{q}$.
\end{claim}
\begin{claimproof}
Note that
\begin{equation}
|F(f^{*})| = \left| \int_{\mathbb{R}} ||g||_{q}^{1-q} |g|^{q-1}|g|\ d\mu \right| = ||g||_{q}^{1-q}\cdot ||g||_{q}^{q} = ||g||_{q}.
\label{2.3.2}
\end{equation}
Thus, by \eqref{2.3.1} and \eqref{2.3.2} we are done.
\end{claimproof}

\end{Proof}


\subsection*{4}
Let $\left\{ f_{n} \right\}_{n=0}^{\infty}$ be a sequence in $L^{2}(\mathbb{R})$ such that $\sum_{n=0}^{\infty}||f_{n}||_{2} < \infty$ and so that
$\sum_{n=0}^{\infty}f_{n}(x) = 0$ for almost all $x \in \mathbb{R}$. Prove that for each $g \in L^{2}(\mathbb{R})$,
\[ \sum_{n=0}^{\infty}\int_{\mathbb{R}}f_{n}(x)g(x)\ d\mu(x) = 0. \]

{\bf Solution:}

\begin{Proof}
Let $g \in L^{2}(\mathbb{R})$. For each $k \in \mathbb{N}$, define $h_{k} := \sum_{n=0}^{\infty}f_{n}\cdot g$. Clearly
\begin{equation}
|h_{k}| = \left| \sum_{n=0}^{k}f_{n}\cdot g \right| \leq \sum_{n=0}^{k}|f_{n}\cdot g| \leq \sum_{n=0}^{\infty}|f_{n}\cdot g|,
\label{2.4.1}
\end{equation}
and 
\begin{equation}
\int_{\mathbb{R}} \sum_{n=0}^{\infty}|f_{n}\cdot g|\ d\mu = \sum_{n=0}^{\infty} \int_{\mathbb{R}}|f_{n}\cdot g|\ d\mu \leq ||g||_{2}\cdot
\sum_{n=0}^{\infty}||f_{n}||_{2} < \infty,
\label{2.4.2}
\end{equation}
by the Cauchy-Schwarz Inequality. So by \eqref{2.4.1} and \eqref{2.4.2} we can apply the Dominated Convergence Theorem to $\left\{ h_{k}
\right\}_{k=0}^{\infty}$. Thus,
\[ \sum_{n=0}^{\infty} \int_{\mathbb{R}} f_{n}\cdot g\ d\mu = \lim_{k\rightarrow \infty} \int_{\mathbb{R}} h_{k}\ d\mu =
\int_{\mathbb{R}}\lim_{k\rightarrow\infty} h_{k}\ d\mu = \int_{\mathbb{R}}g\sum_{n=0}^{\infty} f_{n} \ d\mu = 0. \]
\end{Proof}

\section*{5}
Suppose $f : [0,1] \rightarrow \mathbb{R}$ is absolutely continuous.
\begin{enumerate}[label=(\roman*)]
\item Prove that $f$ is Lipschitz continuous on $[0,1]$ if and only if $\sup_{x \in [0,1]}|f'(x)| < \infty$.
\item Does (i) still hold if we assume $f$ has bounded variation instead?
\end{enumerate}

{\bf Solution:}
\begin{enumerate}[label=(\roman*)]
\item 
\begin{Proof}

$(\Rightarrow)$ Suppose $f$ is Lipschitz continuous on $[0,1]$. Then there exists $c > 0$ such that $|f(a) - f(b)| < c|a - b|$ whenever $a,b \in
[0,1]$. Therefore 
\[ |\text{Diff}_{h}(f)(x)| = \left| \frac{f(x + h) - f(x)}{(x + h) - x}\right| < c, \]
for all $h > 0, x \in [0,1]$. Hence $|f'(x)| \leq c$ for all $x \in [0,1]$, and so $\sup_{x \in [0,1]}|f'(x)| \leq c$.

$(\Leftarrow)$ Now suppose $\sup_{x \in [0,1]}|f'(x)| < \infty$. Then there exists $c > 0$ such that $|f'| < c$. Now let $a,b \in [0,1]$. Without loss
of generality, suppose $a < b$. Since $f$ is absolutely continuous, $\int_{a}^{b}f'd\mu = f(b) - f(a)$. Thus,
\[ |f(b) - f(a)| = \left| \int_{a}^{b}f'd\mu \right| \leq \int_{a}^{b}|f'|d\mu \leq \int_{a}^{b}cd\mu = c(b - a). \]
Hence $f$ is Lipschitz continuous.
\end{Proof}

\item No, (i) does not remain true. For example, consider the function $f : [0,1] \rightarrow \mathbb{R}$ defined by
\[ f(x) := \left\{ \begin{array}{cl}
0 & : x < 1/2, \\
1 & : x \geq 1/2. \\ 
\end{array} \right. \]
\end{enumerate}


\end{document}

