\documentclass[12pt]{article}
\usepackage{amsmath}
\usepackage{amsfonts}
\usepackage{parskip}
\usepackage{amsthm}
\usepackage{thmtools}
\usepackage[headheight=15pt]{geometry}
\geometry{a4paper, left=20mm, right=20mm, top=30mm, bottom=30mm}
\usepackage{graphicx}
\usepackage{bm} % for bold font in math mode - command is \bm{text}
\usepackage{enumitem}
\usepackage{fancyhdr}
\usepackage{amssymb} % for stacked arrows and other shit
\pagestyle{fancy}

\declaretheoremstyle[headfont=\normalfont]{normal}
\declaretheorem[style=normal]{Theorem}
\declaretheorem[style=normal]{Proposition}
\declaretheorem[style=normal]{Lemma}
\newenvironment{claimproof}[1]{\par\noindent\underline{Proof of claim:}\space#1}{\hfill $\blacksquare$\vspace{5mm}}

\title{MATH 515: HW 7}
\author{Evan ``Pete'' Walsh}
\makeatletter
\let\runauthor\@author
\let\runtitle\@title
\makeatother
\lhead{\runauthor}
\chead{\runtitle}
\rhead{\thepage}
\cfoot{}

\begin{document}
\maketitle

\section*{1 [Proposition (iii)(5)iv]}
For each $X\subseteq \mathbb{N}$, let $\bm{c}(X) = \sum_{n\in X}\frac{2}{3^{n+1}}$. Prove that $\bm{c}$ is a bijection of $\mathcal{P}(\mathbb{N})$ onto $\mathcal{C}$.

\subsection*{Solution}
\begin{proof} We will show that $\bm{c}$ is surjective and injective.

\underline{Claim 1:} $\bm{c}$ is surjective.

\begin{claimproof}
Let $y \in \mathcal{C}$. Then $y = \sum_{n=0}^{\infty}\frac{a_{n}}{3^{n+1}}$, where $a_{n} \in \left\{ 0, 2 \right\}$ for all $n \in \mathbb{N}$. Let
$X = \left\{ n \in \mathbb{N} : a_{n} = 2 \right\}$. Then,
\[ \bm{c}(X) = \sum_{n\in X}\frac{2}{3^{n+1}} = \sum_{n\in X}\frac{2}{3^{n+1}} + \sum_{n \in \mathbb{N}\setminus X}\frac{0}{3^{n+1}} = \sum_{n\in
X}\frac{a_{n}}{3^{n+1}} + \sum_{n\in \mathbb{N}\setminus X}\frac{a_{n}}{3^{n+1}} = \sum_{n=0}^{\infty}\frac{a_{n}}{3^{n+1}} = y. \]
Thus $\bm{c}$ is surjective.
\end{claimproof}

\underline{Claim 2:} $\bm{c}$ is injective.

\begin{claimproof}
Let $X_{1}, X_{2} \subseteq \mathbb{N}$ such that $X_{1} \triangle X_{2} \neq \emptyset$. Let $N = \min\left\{ n \in X_{1} \triangle X_{2} \right\}$.
Without loss of generality, assume $N \in X_{1} \setminus X_{2}$. Then,
\begin{align*}
\bm{c}(X_{1}) - \bm{c}(X_{2}) = \sum_{n \in X_{1}}\frac{2}{3^{n+1}} - \sum_{n \in X_{2}}\frac{2}{3^{n+1}} & = \sum_{n \in X_{1}: n \geq
N}\frac{2}{3^{n+1}} - \sum_{n\in X_{2} : n > N}\frac{2}{3^{n+1}} \\
& \geq \frac{2}{3^{N+1}} - \sum_{n = N+1}^{\infty}\frac{2}{3^{n+1}} \\
\text{let } k = n - N \Rightarrow \qquad & = \frac{2}{3^{N+1}} - \frac{2}{3^{N+1}}\sum_{k=1}^{\infty}\frac{1}{3^{k}} \\
& = \frac{2}{3^{N+1}} - \frac{2}{3^{N+1}}\left( \frac{1}{2} \right) = \frac{1}{3^{N+1}} > 0.
\end{align*}
So $\bm{c}(X_{1}) \neq \bm{c}(X_{2})$. Thus $\bm{c}$ is injective.
\end{claimproof}

\vspace{-5mm}
\end{proof}


\section*{2 [Proposition (III)(5)viii]}
Define $\mathcal{C}_{0} = [0,1]$ and $\mathcal{C}_{n+1} = \left( \frac{1}{3}\mathcal{C}_{n} \right) \cup \left( \frac{2}{3} +
\frac{1}{3}\mathcal{C}_{n} \right)$ for all $n \in \mathbb{N}$. Prove that $\mathcal{C} = \bigcap_{n=0}^{\infty} \mathcal{C}_{n}$.

\subsection*{Solution}

\begin{proof} The result follows easily from the following claim.

\underline{Claim 1:} For all $n \geq 1$, $x \in \mathcal{C}_{n}$ if and only if $x$ can be expressed 
\[ x = \sum_{k=0}^{\infty}\frac{a_{k}}{3^{k+1}} = \sum_{k=0}^{n-1}\frac{a_{k}}{3^{n+1}} + \sum_{k=n}^{\infty}\frac{a_{k}}{3^{n+1}}, \]
where $a_{k} \in \left\{ 0, 2 \right\}$ if $k \in \left\{ 0, \hdots, n-1 \right\}$ and $a_{k} \in \left\{ 0,1,2 \right\}$ if $k \geq n$. In other
words, $x \in \mathcal{C}_{n}$ if and only if it has a ternary expansion where the first $n$ digits are either $0$ or $2$.

\begin{claimproof}
We will proceed by induction.

Suppose $n = 1$. Let $x = \sum_{k=0}^{\infty}\frac{a_{k}}{3^{k+1}}$, where $a_{k} \in \left\{ 0,2,3 \right\}$ for all $k \in \mathbb{N}$. Then it is
clear that if $a_{0} = 0$, then $x \in \left[0,\frac{1}{3}\right]$, and if $a_{0} = 2$ then $x \in \left[ \frac{2}{3}, 1 \right]$. Thus $x \in
\mathcal{C}_{1} = \left[ 0, \frac{1}{3} \right] \cup \left[ \frac{2}{3}, 1 \right]$. Conversely, if $x \in \left[ 0, \frac{1}{3} \right]$, then $a_{0}
= 0$. If $x \in \big( \frac{2}{3},1 \big]$, then $a_{0} = 2$. However, if $x = \frac{2}{3}$, then $x = 0.2000\hdots_{\text{base 3}}$ or $x =
0.1222\hdots_{\text{base 3}}$. Both representations are equivalent, so we can choose the former, which implies $a_{0} = 2$ and $a_{k} = 0$ for all $k \geq
1$. Therefore the claim holds for $n = 1$.

Now suppose $n > 1$. By way of induction, suppose that $x \in \mathcal{C}_{n-1}$ if and only if $x$ can be expressed as 
\begin{equation}
x = \sum_{k=0}^{n-2}\frac{a_{k}}{3^{k+1}} + \sum_{k=n-1}^{\infty}\frac{a_{k}}{3^{k+1}},
\label{2.1}
\end{equation}
where $a_{k} \in \left\{ 0,2 \right\}$ if $k \in \left\{ 0,\hdots, n-2 \right\}$ and $a_{k} \in \left\{ 0,1,2 \right\}$ if $k \geq n-1$. By
definition, $y \in \mathcal{C}_{n}$ if and only if $y = \frac{1}{3}x'$ or $y = \frac{2}{3} + \frac{1}{3}x'$ for some $x' \in
\mathcal{C}_{n-1}$. Let $\left\{ a_{k}' \right\}_{k=0}^{\infty}$ be the terms in the ternary expansion of $x'$ as in (\ref{2.1}), where
$a_{k}' \in \left\{ 0,2 \right\}$ for all $k \in \left\{ 0, \hdots, n-2 \right\}$.
\begin{description}
\item[Case 1:] $y = \frac{1}{3}x'$.

Note that $y = \frac{1}{3}x'$ if and only if
\[ y = \frac{1}{3}\sum_{k=0}^{\infty}\frac{a_{k}'}{3^{k+1}} = \frac{0}{3} + \sum_{k=0}^{n-2}\frac{a_{k}'}{3^{k+2}} +
\sum_{k=n-1}^{\infty}\frac{a_{k}'}{3^{k+2}} = \frac{0}{3} + \sum_{j=1}^{n-1}\frac{a_{j-1}'}{3^{j+1}} + \sum_{j=n}^{\infty}\frac{a_{j-1}'}{3^{j+1}}, \]
where $j = k + 1$, if and only if 
\[ y = \sum_{k=0}^{\infty}\frac{b_{k}}{3^{k+1}}, \]
where $b_{0} = 0$ and $b_{k} = a_{k-1}'$ for all $k \geq 1$. By the inductive hypothesis, $b_{k} \in \left\{ 0,2\right\}$ for all $k \in \left\{ 0,
\hdots, n-1 \right\}$.

\item[Case 2:] $y = \frac{2}{3} + \frac{1}{3}x'$.

Similarly, $y = \frac{2}{3} + \frac{1}{3}x'$ if and only if 
\[ y = \frac{2}{3} + \sum_{k=0}^{n-2}\frac{a_{k}'}{3^{k+2}} + \sum_{k=n-1}^{\infty}\frac{a_{k}'}{3^{k+2}} \stackrel{j=k+1}{=} \frac{2}{3} +
\sum_{j=1}^{n-1}\frac{a_{j-1}'}{3^{j+1}} + \sum_{j=n}^{\infty}\frac{a_{j-1}'}{3^{j+1}} = \sum_{k=0}^{\infty}\frac{b_{k}}{3^{k+1}}, \]
where $b_{0} = 2$ and $b_{k} = a_{k-1}'$ for all $k \geq 1$. By the inductive hypothesis, $b_{k} \in \left\{ 0,2 \right\}$ for all $k \in \left\{
0,\hdots, n-1 \right\}$.
\end{description}
\end{claimproof}

\underline{Claim 2:} $\mathcal{C} = \bigcap_{n=0}^{\infty}\mathcal{C}_{n}$.

\begin{claimproof}
\[ x \in \bigcap_{n=0}^{\infty}\mathcal{C}_{n}\ \  \Leftrightarrow \ \ x \in \bigcap_{n=1}^{\infty}\mathcal{C}_{n}\ \  \Leftrightarrow \ \ x\in \mathcal{C}_{n}
\ \forall\ n \geq 1, \] 
if and only if for all $n \geq 1$, $x$ can be expressed as 
\[ x = \sum_{k=0}^{\infty}\frac{a_{k}}{3^{k+1}}, \]
where $a_{k} \in \left\{ 0,2 \right\}$ for all $k \in \left\{ 0,\hdots,n-1 \right\}$ and $a_{k} \in \left\{ 0,1,2 \right\}$ for $k \geq n-1$, if and
only if $x$ can be expressed as 
\[ x = \sum_{n=0}^{\infty}\frac{a_{n}}{3^{n+1}}, \]
where $a_{n} \in \left\{ 0,2 \right\}$ for all $n \in \mathbb{N}$, if and only if $x \in \mathcal{C}$.
\end{claimproof}

\vspace{-5mm}
\end{proof}

\newpage 
\section*{3 [Proposition (III)(5)xvi]}
Suppose $n_{0} \in \mathbb{N}$. Then $|\phi(x_{0}) - \phi(x_{1})| < 2^{-n_{0}}$ whenever $x_{0}, x_{1} \in \mathcal{C}$ and $|x_{0} - x_{1}| < 3^{-n_{0}}$.

\subsection*{Solution}
\begin{proof}
Let $n_{0} \in \mathbb{N}$. Let $x_{0}, x_{1} \in \mathcal{C}$ such that $|x_{0} - x_{1}| < 3^{-n_{0}}$. WLOG, assume $x_{1} > x_{0}$. If $n_{0} = 0$, then $x_{1} - x_{0} < 1$
implies $x_{0}, x_{1} \in (0,1)$. Therefore $|\phi(x_{0}) - \phi(x_{1})| = \phi(x_{1}) - \phi(x_{0}) < 1$ since $\phi$ is strictly increasing and
$\phi(0) = 0$ and $\phi(1) = 1$. Now suppose $n_{0} > 0$. Since $x_{0}, x_{1} \in \mathcal{C}$, 
\[ x_{0} = \sum_{n=0}^{\infty}\frac{a_{n}}{3^{n+1}} \qquad \text{and} \qquad x_{1} = \sum_{n=0}^{\infty}\frac{b_{n}}{3^{n+1}}, \]
where $a_{n}, b_{n} \in \left\{ 0,2 \right\}$ for all $n \in \mathbb{N}$. Therefore 
\begin{equation}
\sum_{n=0}^{\infty}\frac{b_{n}}{3^{n+1}} - \sum_{n=0}^{\infty}\frac{a_{n}}{3^{n+1}} < \frac{1}{3^{n_{0}}}.
\label{3.1}
\end{equation}
\underline{Claim 1:} $b_{n} = a_{n}$ for all $0 \leq n \leq n_{0} - 1$.
\begin{claimproof}
By way of contradiction, assume there exists $0 \leq n\leq n_{0}-1$ such that $b_{n} - a_{n} \neq 0$. Let $n' = \min\left\{ 0 \leq n \leq n_0 -1 :
b_{n} \neq a_{n} \right\}$. Since $b_{n'}, a_{n'} \in \left\{ 0,2 \right\}$ and $x_{1} > x_{0}$ by assumption, $b_{n'} = 2$ and $a_{n'} = 0$. Thus,
\begin{align*}
|x_{0} - x_{1}| = x_{1} - x_{0} = \sum_{n=0}^{\infty}\frac{b_{n}}{3^{n+1}} - \sum_{n=0}^{\infty}\frac{a_{n}}{3^{n+1}} 
& = \frac{2}{3^{n'+1}} + \sum_{n=n'+1}^{\infty}\frac{b_{n}}{3^{n+1}} - \sum_{n=n'+1}^{\infty}\frac{a_{n}}{3^{n+1}} \\
& \geq \frac{2}{3^{n'+1}} - \sum_{n=n'+1}^{\infty}\frac{2}{3^{n+1}} \\
& = \frac{2}{3^{n'+1}} - \frac{1}{3^{n'+1}} = \frac{1}{3^{n'+1}} \geq \frac{1}{3^{n_{0}}}.
\end{align*}
This contradicts \eqref{3.1}.
\end{claimproof}

\vspace{-4mm}
\underline{Claim 2:} There exists $n \geq n_{0}$ such that either $b_{n} = 0$ or $a_{n} = 2$.
\begin{claimproof}
By way of contradiction, assume $b_{n} = 2$ and $a_{n} = 0$ for all $n\geq n_{0}$. Then 
\[ x_{1} - x_{0} = \sum_{n=n_{0}}^{\infty}\frac{2}{3^{n+1}} = \frac{1}{3^{n_0}}. \]
This contradicts \eqref{3.1}.
\end{claimproof}

\vspace{-4mm}
Thus, by the above two claims and since $\phi$ is increasing,
\begin{align*}
|\phi(x_{0}) - \phi(x_{1})| = \phi(x_{1}) - \phi(x_{0}) = \sum_{n=n_{0}}^{\infty}\frac{b_{n}/2}{2^{n+1}} -
\sum_{n=n_{0}}^{\infty}\frac{a_{n}/2}{2^{n+1}} & \stackrel{\text{claim 2}}{<} \sum_{n=n_{0}}^{\infty}\frac{2/2}{2^{n+1}} - \sum_{n=n_{0}}^{\infty}\frac{0/2}{2^{n+1}} \\
& = \sum_{n=n_0}^{\infty}\frac{1}{2^{n+1}} = \frac{1}{2^{n_{0}}}.
\end{align*}
\end{proof}


\newpage 
\section*{4 [Proposition (III)(5)xx]}
If $(a,b) \subseteq [0,1] - \mathcal{C}$, then $\phi$ is constant on $(a,b)$ and $\phi(a) = \phi(b)$.

\subsection*{Solution}
\begin{proof}
Let $(a,b) \subseteq [0,1] - \mathcal{C}$. 

\underline{Claim 1:} $\phi$ is constant over $[a,b)$.
\begin{claimproof}
\begin{description}
\item[Case 1:] $a \in \mathcal{C}$.

Then $a = \max\left\{ x' \in \mathcal{C} : x' < x \right\}$ for all $x \in (a,b)$. Therefore, by the definition of $\phi$, $\phi(x) = \phi(a)$ for all
$x \in (a,b)$.

\item[Case 2:] $a \in [0,1] - \mathcal{C}$.

Let $x_{0} = \max\left\{ x' \in \mathcal{C} : x' < a \right\}$. By definition, $\phi(a) = \phi(x_{0})$. Further, 
\[ x_{0} = \max\left\{ x' \in \mathcal{C} : x' < x \right\}, \ \text{ for all $x \in (a,b)$.} \]
Therefore $\phi(x) = \phi(x_{0}) = \phi(a)$ for all $x \in (a,b)$.
\end{description}
\end{claimproof}

\underline{Claim 2:} $\phi(b) = \phi(a)$.
\begin{claimproof}
\begin{description}
\item[Case 1:] $b \in [0,1] - \mathcal{C}$.

Then $\max\left\{ x' \in \mathcal{C} : x' < b \right\} = \max\left\{ x' \in \mathcal{C} : x' < x \right\}$, for all $x \in (a,b)$. Thus, by claim 1
and the definition of $\phi$, $\phi(b) = \phi(a)$.

\item[Case 2:] $b \in \mathcal{C}$.

Let $x_{0} = \max\left\{ x' \in \mathcal{C} : x' < b \right\} = \max\left\{ x' \in \mathcal{C} : x' < x, \text{ for some }x \in (a,b) \right\}$.
Conversely, 
\begin{equation}
b = \min\left\{ x' \in \mathcal{C} : x' > x_{0} \right\}.
\label{4.1}
\end{equation} 
By claim 1, $\phi(a) = \phi(x_{0})$. By definition of $\mathcal{C}$, we can express $x_{0}$ and $b$ as 
\[ x_{0} = \sum_{n=0}^{\infty}\frac{a_{n}}{3^{n+1}} \qquad \text{and} \qquad b = \sum_{n=0}^{\infty}\frac{b_{n}}{3^{n+1}}, \]
where $a_{n},b_{n} \in \left\{ 0,2 \right\}$ for all $n \in \mathbb{N}$. Let $n_{0}$ be the minimum $n \in \mathbb{N}$ such that $a_{n} \neq b_{n}$.
Since $b > x_{0}$, $b_{n_{0}} = 2$ and $a_{n_{0}} = 0$. Further, by \eqref{4.1} and the definition of $x_{0}$, $b_{n} = 0$ and $a_{n} = 2$ for all $n
> n_{0}$. Thus,
\begin{align*}
\phi(b) - \phi(a) = \phi(b) - \phi(x_{0}) & = \sum_{n=0}^{\infty}\frac{b_{n}/2}{2^{n+1}} - \sum_{n=0}^{\infty}\frac{a_n/2}{2^{n+1}} \\
& = \frac{2/2}{2^{n_{0}+1}} - \frac{0/2}{2^{n_{0}+1}} + \sum_{n=n_{0}+1}^{\infty}\frac{0/2}{2^{n+1}} - \sum_{n=n_0+1}^{\infty}\frac{2/2}{3^{n+1}} \\
& = \frac{1}{2^{n_0+1}} - \sum_{n=n_0+1}^{\infty}\frac{1}{2^{n+1}} = \frac{1}{2^{n_{0}+1}} - \frac{1}{2^{n_{0}+1}} = 0.
\end{align*}
So $\phi(b) = \phi(a)$.
\end{description}
\end{claimproof}

By claims 1 and 2, $\phi$ is constant over $[a,b]$.
\end{proof}


\newpage 
\section*{5}
Prove that increasing functions defined over the reals map Borel sets to Borel sets. 

{\small Note: I am proving the more general case where we don't assume continuity. I realize it would be a lot easier if we assumed continuity, but I had
already finished writing up this solution when I heard of the correction. So here it is.}

\subsection*{Solution}
\begin{proof}
Let $f : \mathbb{R} \rightarrow \mathbb{R}$ be an increasing function. First we will show that $f$ can have at most countably many discontinuities. We
will use that fact to establish that $f$ maps open sets to Borel sets. Then we will show that the collection of sets in $\mathbb{R}$ for which $f$ maps
to Borel sets is a $\sigma$-algebra which contains the open sets.

\underline{Claim 1:} $f$ has at most countably many discontinuities.
\begin{claimproof}
Since $f$ is increasing, the only possible type of discontinuity is a jump discontinuity. That is, $f$ is not continuous at $x \in \mathbb{R}$ if and
only if $f(x+) - f(x-) > 0$, where $f(x-) = \sup_{y<x}f(y)$ and $f(x+) = \inf_{y>x}f(y)$. Therefore, we can express the entire set of discontinuities
of $f$ as 
\[ \bigcup_{k=1}^{\infty}\bigcup_{n=0}^{\infty}A_{k,n}, \]
where $A_{k,n} = \left\{ x \in \mathbb{R} : |x| < k \wedge f(x+) - f(x-) > 2^{-n} \right\}$. Now, each $A_{k,n}$ has to be finite, otherwise $f$ would
go to infinity on $(-k, k)$ which would mean $f(k) = \infty$ since $f$ is increasing. This contradicts the fact that $f$ is real-valued. Therefore the entire
set of discontinuities is at most countable.
\end{claimproof}

\underline{Claim 2:} $f$ maps open sets to Borel sets.
\begin{claimproof}
Let $U \subseteq \mathbb{R}$ be open. Then $U = \cup_{j\in F}I_{j}$, where $\left\{ I_{j} \right\}_{j\in F}$ is a countable, pairwise disjoint
collection of open intervals. For each $j \in F$, let $X_{j} = \left\{ x\in I_{j} : f \text{ discontinuous at }x \right\}$. By claim 1, $X_{j}$ is at
most countable. Therefore 
\[ I_{j} - X_{j} = \bigcup_{i \in G_{j}}E_{j,i},\] 
where $\left\{ E_{j,i} \right\}_{i\in G_{j}}$ is a countable collection of pairwise disjoint, open intervals. For each $j \in F$, $i \in G_{j}$, $f$ is
continuous and increasing over $E_{j,i}$. Therefore $f(E_{j,i})$ is an open interval. Hence,
\begin{align*} 
f(U) = f\left( \bigcup_{j\in F}I_{j} \right) & = f\left( \bigcup_{j\in F}[I_{j} - X_{j}]\cup [I_{j}\cap X_{j}] \right) \\
& = f\left( \bigcup_{j\in F}[\cup_{i \in G_{j}}E_{j,i}]\cup [I_{j}\cap X_{j}] \right) \\
& = \bigcup_{j \in F}\left( [ \cup_{i\in G_{j}}f(E_{j,i}) ] \cup \left[ f(I_{j} \cap X_{j}) \right] \right),
\end{align*}
which is the countable union of open intervals and a countable collection of points. Thus $f(U)$ is Borel.
\end{claimproof}

Now let $\mathcal{F} = \left\{ B \subseteq \mathbb{R} : f(B) \in \mathcal{B}(\mathbb{R}) \right\}$.

\underline{Claim 3:} $\mathcal{F}$ is a $\sigma$-algebra.
\begin{claimproof}
\begin{description}
\item[Subclaim (a)] $\mathbb{R} \in \mathcal{F}$.

Since $\mathbb{R}$ is open, $\mathbb{R} \in \mathcal{F}$ by claim 2. 

\item[Subclaim (b)] If $B \in \mathcal{F}$, then $B^{c} \in \mathcal{F}$.

Let $B \in \mathcal{F}$. Then $f(B)$ is Borel. Thus, 
\[ f(B^{c}) = f(\mathbb{R} - B) = f(\mathbb{R}) - f(B) \in \mathcal{B}(\mathbb{R}) \text{ by subclaim (a)}. \]
Thus $B^{c} \in \mathcal{F}$.

\item[Subclaim (c)] If $\left\{ B_{n} \right\}_{n\in\mathbb{N}} \subseteq \mathcal{F}$, then $\cup_{n=0}^{\infty}B_{n} \in \mathcal{F}$.

Let $\left\{ B_{n} \right\}_{n\in\mathbb{N}} \subseteq \mathcal{F}$. Then $f(B_{n})$ is Borel for all $n \in \mathbb{N}$. Hence,
\[ f\left( \bigcup_{n=0}^{\infty} B_{n}\right) = \bigcup_{n=0}^{\infty} f(B_{n}) \in \mathcal{B}(\mathbb{R}). \]
Thus $\cup_{n=0}^{\infty}B_{n} \in \mathcal{F}$.
\end{description}
\end{claimproof}

By claims 2 and 3, $\mathcal{F}$ is $\sigma$-algebra containing the open sets. Therefore $\mathcal{B}(\mathbb{R}) \subseteq \mathcal{F}$.
\end{proof}


\newpage 
\section*{6}
Suppose $f : E \rightarrow [-\infty, \infty]$ is measurable. Does it follow that there is a non-decreasing sequence of simple functions on $E$
$\left\{ \sigma_{n} \right\}_{n=0}^{\infty}$ so that $\sigma_{n}\rightarrow f$ a.e. and $|\sigma_{n}| \leq |f|$ for all $n$? If yes, give a proof.
Otherwise give a counterexample.

\subsection*{Solution}
No. Let $f : \mathbb{R} \rightarrow [-\infty, \infty]$ be defined by $f(x) = -\infty$ for all $x \in \mathbb{R}$.



\end{document}

