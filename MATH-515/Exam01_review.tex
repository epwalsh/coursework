\documentclass[12pt]{article}
\usepackage{amsmath}
\usepackage{amsfonts}
\usepackage{parskip}
\usepackage{amsthm}
\usepackage{thmtools}
\usepackage[headheight=15pt]{geometry}
\geometry{a4paper, left=20mm, right=20mm, top=30mm, bottom=30mm}
\usepackage{graphicx}
\usepackage{bm} % for bold font in math mode - command is \bm{text}
\usepackage{enumitem}
\usepackage{fancyhdr}
\usepackage{amssymb} % for stacked arrows and other shit
\pagestyle{fancy}
\usepackage{changepage}
\usepackage{mathcomp}

\declaretheoremstyle[headfont=\normalfont]{normal}
\declaretheorem[style=normal]{Theorem}
\declaretheorem[style=normal]{Proposition}
\declaretheorem[style=normal]{Lemma}
\newcounter{ProofCounter}
\newcounter{ClaimCounter}[ProofCounter]
\newcounter{SubClaimCounter}[ClaimCounter]
\newenvironment{Proof}{\stepcounter{ProofCounter}\textit{Proof.}}{\hfill$\square$}
\newenvironment{claim}[1]{\vspace{3mm}\stepcounter{ClaimCounter}\par\noindent\underline{\bf Claim \theClaimCounter:}\space#1}{}
\newenvironment{claimproof}[1]{\par\noindent\underline{Proof of claim \theClaimCounter:}\space#1}{\hfill $\blacksquare$ Claim \theClaimCounter}
\newenvironment{subclaim}[1]{\stepcounter{SubClaimCounter}\par\noindent\emph{Subclaim \theClaimCounter.\theSubClaimCounter:}\space#1}{}
% \newenvironment{subclaimproof}[1]{\begin{adjustwidth}{2em}{0pt}\par\noindent\emph{Proof of subclaim \theClaimCounter.\theSubClaimCounter:}\space#1}{\hfill
% $\blacksquare$ \emph{Subclaim \theClaimCounter.\theSubClaimCounter}\vspace{5mm}\end{adjustwidth}}
\newenvironment{subclaimproof}[1]{\par\noindent\emph{Proof of subclaim \theClaimCounter.\theSubClaimCounter:}\space#1}{\hfill
$\Diamond$ \emph{Subclaim \theClaimCounter.\theSubClaimCounter}}

% This rule is for making an integral with a bar above it or below it.
\makeatletter
\newcommand\tint{\mathop{\mathpalette\tb@int{t}}\!\int}
\newcommand\bint{\mathop{\mathpalette\tb@int{b}}\!\int}
\newcommand\tb@int[2]{%
  \sbox\z@{$\m@th#1\int$}%
  \if#2t%
    \rlap{\hbox to\wd\z@{%
      \hfil
      \vrule width .35em height \dimexpr\ht\z@+1.4pt\relax depth -\dimexpr\ht\z@+1pt\relax
      \kern.05em % a small correction on the top
    }}
  \else
    \rlap{\hbox to\wd\z@{%
      \vrule width .35em height -\dimexpr\dp\z@+1pt\relax depth \dimexpr\dp\z@+1.4pt\relax
      \hfil
    }}
  \fi
}
\makeatother

\title{MATH 515: Exam 1 Review}
\author{Evan P. Walsh}
\makeatletter
\let\runauthor\@author
\let\runtitle\@title
\makeatother
\lhead{\runauthor}
\chead{\runtitle}
\rhead{\thepage}
\cfoot{}

\begin{document}
% \maketitle


\section*{1.1 [Royden/Fitzpatrick (RF) 1.31]} Show that if a set $E$ consists only of
isolated points then it is countable.


\section*{1.2 [RF 1.32]} 
Show that 
\begin{enumerate}[label=(\roman*)]
\item $E$ is open iff $E = \text{int} E$.
\item $E$ is dense iff $\text{int} (\mathbb{R} - E) = \emptyset$.
\end{enumerate}

\section*{1.3 [RF 1.52]} Show that a non-empty set of real numbers $E$ is closed and
bounded iff every continuous real-valued function on $E$ takes a maximum value.


\section*{1.4 [RF 1.59]} Let $\{f_{n}\}$ be a sequence of real-valued continuous
functions defined on a set $E$. Show that if $\{f_{n}\}$ converges uniformly to $f$ on
$E$, then $f$ is continuous.


\section*{1.5 [Proposition (I).(1).(xi)]} 
Suppose $f$ is a step function on $[a,b]$.
Then $f$ is integrable. Furthermore, if $(x_{0}, \dots, x_{n})$ is a partition
of $[a,b]$ so that $f$ is constant on each open subinterval $(x_{0}, x_{1}),
\dots, (x_{n-1}, x_{n})$, and if $c_{j}$ is the value of $f$ on $(x_{j-1},
x_{j})$ whenever $1 \leq j \leq n$, then 
\[ \int_{a}^{b} f = \sum_{j=1}^{n}c_{j}(x_{j} - x_{j-1}). \]


\section*{1.6 [RF 2.6]} 
Let $A$ be the set of irrational numbers in $[0,1]$. Show
that $m^{*}(A) = 1$.


\section*{1.7 [RF 2.10]} 
Let $A,B$ be bounded sets for which there is an $\alpha > 0$
such that $|a-b| \geq \alpha$ for every $a\in A$ and $b \in B$. Show that
$m^{*}(A\cup B) = m^{*}(A) + m^{*}(B)$.



\section*{2.1 [Lemma (II)(1)(xvi)]} 
If $J_{0}, \dots, J_{n}$ are open intervals, and
if $(J_{0}, \dots, J_{n})$ is a simple chain, then $\cup_{k=1}^{n}J_{k}$ is an
open interval.


\section*{2.2 [Lemma (II)(1)(xviii)]} 
If $J_{0}, \dots, J_{n}$ are open intervals, and $(J_{0}, \hdots, J_{n})$ is a simple chain, then $\ell(\cup_{k=0}^{n}J_{k}) \leq \sum_{k=0}^{n}\ell(J_{k})$.


\section*{2.3} 
Consider the two definitions of outer measure for a set $A \subseteq
\mathbb{R}$:
\[ \mu^{*}(A) = \inf\{\ell(U) : A \subseteq U \subseteq \mathbb{R}, U \neq
\emptyset, U \text{ open}\}, \text{ and } \]
\[ m^{*}(A) = \inf\left\{\sum_{k=1}^{\infty}\ell(I_{k}) : A \subseteq
\cup_{k=1}^{\infty}I_{k} \subseteq \mathbb{R}, I_{k} \neq \emptyset, I_{k}
\text{ an open bounded interval}\right\}. \]
Show that these two definitions are equivalent.


\section*{2.4 [RF 2.7]} A set of real numbers is said to be a $G_{\delta}$ set
provided it is the intersection of a countable collection of open sets. Show
that for any bounded set $E$, there is a $G_{\delta}$ set $G$ for which 
\[ E \subseteq G \text{ and } m^{*}(G) = m^{*}(E). \]


\section*{2.5 [RF 2.9]} 
If $m^{*}(A) = 0$, then $m^{*}(A\cup B) = m^{*}(B)$.


\section*{2.6 [Claim 1 in the proof of Thm (II)(1)(xxiv)]} 
Show that $\cup_{(n,j')\in G_{j}}I_{n,j'} = I_{j}$.



\section*{3.1 [RF 2.11]} 
If a $\sigma$-algebra of $\mathbb{R}$ contains intervals of the form $(a, \infty)$, then it contains all intervals.


\section*{3.2 [RF 2.14]} 
If a set $E$ has positive outer measure, then there is a bounded subset of $E$ that also has positive outer measure.


\section*{3.3} 
If $E$ has finite measure and $\epsilon > 0$, then $E$ is the disjoint union of a finite number of measurable sets, each of which has measure at most
$\epsilon$.


\section*{3.4} 
Prove that there is not a countably infinite $\sigma$-algebra over the reals.


\section*{3.5} 
Let $f: [a,b] \rightarrow \mathbb{R}$ be a function. Royden and Fitzpatrick's definition of the lower Riemann integral of $f$ over $[a,b]$ is 
\[ (R)\bint_{a}^{b} f = \sup\left\{ L^{*}(f,P) : P\text{ is a partition of }[a,b] \right\}, \]
where $L^{*}(f,P) = \sum_{i=1}^{n}m_{i}(x_{i} - x_{i-1})$, for $m_{i} = \inf\left\{ f(x) : x_{i-1} < x < x_{i} \right\}$, is a lower Riemann sum. Similary,
definition (I)(1)(v)(a), from our notes, of lower Riemann integral is 
\[ \bint_{a}^{b}f = \sup\left\{ L(f, P, V) : P\text{ is a partition of }[a,b], V = (v_{1}, \dots, v_{n})\text{ with }v_{i} \leq f(x)\forall x_{i-1} <
x < x_{i} \right\}, \]
where $L(f,P,V) = \sum_{i=1}^{n}v_{i}(x_{i} - x_{i-1})$, is also considered a lower Riemann sum. 
Show that these two definitions of lower Riemann integral are equivalent.



\section*{4.1}
Give an example of a Borel set of reals that is neither $F_{\sigma}$ or $G_{\delta}$.



\section*{4.2 [RF 2.12]}
Show that every interval is a Borel set.



\section*{4.3 [RF 2.13]}
Show that (i) the translate of an $F_{\sigma}$ is also $F_{\sigma}$, (ii) the translate of a $G_{\delta}$ is also $G_{\delta}$, and (iii) the
translate of a set of measure zero also has measure zero.


\section*{4.4 [RF 2.17]}
Show that a set $E$ is measurable if and only if for each $\epsilon > 0$, there is a closed set $F$ and open set $\mathcal{O}$ for which $F \subseteq E\subseteq
\mathcal{O}$ and $\mu^{*}(\mathcal{O}\setminus F) < \epsilon$.



\section*{4.5 [RF 2.20]}
(Legesgue) Let $E$ have finite outer measure. Show that $E$ is measurable if and only if for each open, bounded interval $(a,b)$,
\[ b - a = \mu^{*}\left( (a,b) \cap E \right) + \mu^{*}\left( (a,b) \setminus E \right). \]







\end{document}

