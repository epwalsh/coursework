\documentclass[12pt]{article}
\usepackage{amsmath}
\usepackage{amsfonts}
\usepackage{parskip}
\usepackage{amsthm}
\usepackage{thmtools}
\usepackage[headheight=15pt]{geometry}
\geometry{a4paper, left=20mm, right=20mm, top=30mm, bottom=30mm}
\usepackage{graphicx}
\usepackage{bm} % for bold font in math mode - command is \bm{text}
\usepackage{enumitem}
\usepackage{fancyhdr}
\usepackage{amssymb} % for stacked arrows
\pagestyle{fancy}

% This rule is for making an integral with a bar above it or below it.
\makeatletter
\newcommand\tint{\mathop{\mathpalette\tb@int{t}}\!\int}
\newcommand\bint{\mathop{\mathpalette\tb@int{b}}\!\int}
\newcommand\tb@int[2]{%
  \sbox\z@{$\m@th#1\int$}%
  \if#2t%
    \rlap{\hbox to\wd\z@{%
      \hfil
      \vrule width .35em height \dimexpr\ht\z@+1.4pt\relax depth -\dimexpr\ht\z@+1pt\relax
      \kern.05em % a small correction on the top
    }}
  \else
    \rlap{\hbox to\wd\z@{%
      \vrule width .35em height -\dimexpr\dp\z@+1pt\relax depth \dimexpr\dp\z@+1.4pt\relax
      \hfil
    }}
  \fi
}
\makeatother

\declaretheoremstyle[headfont=\normalfont]{normal}
\declaretheorem[style=normal]{Theorem}
\declaretheorem[style=normal]{Proposition}
\declaretheorem[style=normal]{Lemma}

\title{MATH 515: HW 1}
\author{Evan ``Pete'' Walsh}
\makeatletter
\let\runauthor\@author
\let\runtitle\@title
\makeatother
\lhead{\runauthor}
\chead{\runtitle}
\rhead{\thepage}
\cfoot{}

\begin{document}
\maketitle

{\bf (1) [Royden/Fitzpatrick (RF) 1.31]} Show that if a set $E$ consists only of
isolated points then it is countable.

{\bf Solution:}

\begin{proof}
  By construction, for every point $x \in E$, there exists an $r > 0$ such that
  $(x-r, x+r) \cap E = \{x\}$. Now, since $\mathbb{Q}$ is dense in $\mathbb{R}$,
  there exists some $y \in \mathbb{Q}$ such that $x-r < y < x+r$. Therefore
  there is a natural bijection between a subset of $\mathbb{Q}$ and $E$ by
  choosing such a $y \in \mathbb{Q}$ for every $x \in E$. Further, since any
  subset of $\mathbb{Q}$ is countable, $E$ must be countable.
\end{proof}

{\bf (2) [RF 1.32]} Show that 
\begin{itemize}[label={},leftmargin=4mm, itemsep=1em, parsep=1em]
  \item (i) $E$ is open iff $E = \text{int} E$.

  {\bf Solution:}
  \begin{proof}
    $(\Rightarrow)$ Assume $E$ is open. Let $x\in E$. Since $E$ is open, there
    exists some $r > 0$ such that $(x-r, x+r) \subseteq E$. Thus $x$ is an
    interior point of $E$ by definition, so $E \subseteq \text{int} E$. Clearly
    any point $y \in \text{int} E$ is also in $E$, so $\text{int} E \subseteq
    E$. Hence $E = \text{int}E$.

    $(\Leftarrow)$ Assume $E = \text{int}E$. Let $x \in E$. Then $x \in
    \text{int} E$, so there exists some $r > 0$ such that $(x-r, x+r) \subseteq
    E$. Thus $E$ is open by definition.
  \end{proof}

  \item (ii) $E$ is dense iff $\text{int} (\mathbb{R} \sim E) = \emptyset$.

  {\bf Solution:}
  \begin{proof}
    $(\Rightarrow)$ Assume $E$ is dense. If $\text{int}(\mathbb{R} \sim E) \neq
    \emptyset$, then there exists some point $x\in \mathbb{R} \sim E$ and some
    $r > 0$ such that $(x-r, x+r) \cap E = \emptyset$. But this is a
    contradiction since $E$ being dense implies that there exists some $y \in E$
    such that $x-r < y < x+r$, for any $x \in \mathbb{R}$, $r > 0$. Hence
    $\text{int}(\mathbb{R} \sim E) = \emptyset$.

    $(\Leftarrow)$ Assume $\text{int}(\mathbb{R}\sim E) = \emptyset$. Then for
    all $x \in \mathbb{R}$ and $r > 0$, $(x - r, x+r) \cap E \neq \emptyset$.
    So, let $a, b \in \mathbb{R}$ with $a < b$. Choose some $x \in \mathbb{R}$
    and $r > 0$ such that $a < x-r < x+r < b$. Since $(x-r, x+r)\cap E \neq
    \emptyset$, there exists some $y \in E$ such that $a < x-r < y < x+r < b$.
    Therefore $a < y < b$, so $E$ is dense.
  \end{proof}
\end{itemize}

{\bf (3) [RF 1.52]} Show that a non-empty set of real numbers $E$ is closed and
bounded iff every continuous real-valued function on $E$ takes a maximum value.

{\bf Solution:}

\begin{proof}
$(\Rightarrow)$ Assume $E$ is closed and bounded. Then by the Extreme Value
Theorem, every continuous real-valued function defined on $E$ must take a
maximum value.

$(\Leftarrow)$ Assume that every continuous real-valued function defined on
a non-empty set of reals $E$ takes on a maximum value. We will show by
contradiction that $E$ must be closed and bounded. First, assume that $E$ is
unbounded. Let $f : E \rightarrow \mathbb{R}$ be defined by $f(x) = |x|$. Then $f$ is continuous and since $E$ is unbounded,
for every $M > 0$, there exists some $x' \in E$ such that $|x'| > M$. Therefore
$f(x') > M$. Since $M$ was arbitrary, $f$ has no maximum value. This is a
contradiction, thus $E$ must be bounded.

Now, assume that $E$ is bounded but not closed. Then there exists some point
$x_{0} \in \mathbb{R} \sim E$ such that $x_{0}$ is a limit point of $E$. Define
$f : E \rightarrow \mathbb{R}$ by
\[ f(x) = \frac{1}{|x_{0} - x|}. \]
Then $f$ is continuous and if $(x_{n})_{n\in\mathbb{N}}\subset E$ is a sequence
in $E$ such that $\lim_{n\rightarrow \infty}x_{n} = x_{0}$, then 
\[ \lim_{n\rightarrow\infty} f(x_{n}) = \lim_{n\rightarrow\infty}
\frac{1}{|x_{0} - x_{n}|} = +\infty. \]
Therefore $f$ has no maximum, which is a contradiction. Thus $E$ must be closed
and bounded.
\end{proof}

{\bf (4) [RF 1.59]} Let $\{f_{n}\}$ be a sequence of real-valued continuous
functions defined on a set $E$. Show that if $\{f_{n}\}$ converges uniformly to $f$ on
$E$, then $f$ is continuous.

{\bf Solution:}

\begin{proof} Let $\epsilon > 0$ and $x \in E$. We need to show that there
  exists a $\delta > 0$ such that $y \in E$ and $|x-y| < \delta$ implies $|f(x)
  - f(y)| < \epsilon$. We first note that by repeated use of the Triangle Inequality,
  \[ |f(x) - f(y)| \leq |f(x) - f_{n}(x)| + |f_{n}(x) - f(y)| \leq |f(x) -
  f_{n}(x)| + |f_{n}(x) - f_{n}(y)| + |f_{n}(y) - f(y)|, \]
  for any $y \in E$. Now, since $\{f_{n}\}$ converges uniformly to $f$, we
  can choose some $N\in\mathbb{N}$ such that $n > N$ implies $|f_{n}(z) - f(z)|
  < \epsilon / 3$ for every $z \in E$. Further, for any particular choice of
  $n > N$, there exists a $\delta_{n} > 0$ such that $|x-y| < \delta_{n}$ implies
  $|f_{n}(x) - f_{n}(y)| < \epsilon / 3$ since $f_{n}$ is
  continuous. Therefore $|x - y| < \delta_{n}$ implies 
  \[ |f(x) - f(y)| \leq |f(x) - f_{n}(x)| + |f_{n}(x) - f_{n}(y)| +
    |f_{n}(y) - f(y)| < \frac{\epsilon}{3}+\frac{\epsilon}{3} +
  \frac{\epsilon}{3} = \epsilon. \]
  Thus $f$ is continuous.
\end{proof}

{\bf (5) [Proposition (I).(1).(xi)]} Suppose $f$ is a step function on $[a,b]$.
Then $f$ is integrable. Furthermore, if $(x_{0}, \dots, x_{n})$ is a partition
of $[a,b]$ so that $f$ is constant on each open subinterval $(x_{0}, x_{1}),
\dots, (x_{n-1}, x_{n})$, and if $c_{j}$ is the value of $f$ on $(x_{j-1},
x_{j})$ whenever $1 \leq j \leq n$, then 
\[ \int_{a}^{b} f = \sum_{j=1}^{n}c_{j}(x_{j} - x_{j-1}). \]

{\bf Solution:}

\begin{proof}
  By definition $\sum_{j=1}^{n}c_{j}(x_{j}- x_{j-1})$ is a lower Reimann
  (Darboux) sum since $c_{j} \leq f(x)$ for each $x \in (x_{j-1}, x_{j})$.
  Similary, $\sum_{j=1}^{n}c_{j}(x_{j} - x_{j-1})$ is also an upper Riemann sum.
  Thus, 
  \[ \bint_{a}^{b} = \sup\{L(f,P) | P \text{ a partition of
  }[a,b]\} \geq \sum_{j=1}^{n}c_{j}(x_{j} - x_{j-1}), \]
  and
  \[ \tint_{a}^{b} = \inf\{U(f,P) | P \text{ a partition of }[a,b]\}
\leq \sum_{j=1}^{n}c_{j}(x_{j} - x_{j-1}). \]
  We claim that $\bint_{a}^{b}f = \tint_{a}^{b}f =
  \sum_{j=1}^{n}c_{j}(x_{j} - x_{j-1})$. If we assume that this equality does
  not hold, say that $\bint_{a}^{b}f >
  \sum_{j=1}^{n}c_{j}(x_{j}-x_{j-1})$, then that means that there exists a lower
  Riemann sum that is greater than an upper Riemann sum, which is impossible.
  Similarly, $\tint_{a}^{b} f$ cannot be less than
  $\sum_{j=1}^{n}c_{j}(x_{j}- x_{j-1})$. Therefore the equality holds. Hence,
  since the lower and upper Riemann integrals are equal, $f$ is integrable and 
  \[ \int_{a}^{b}f = \sum_{j=1}^{n}c_{j}(x_{j} - x_{j-1}). \]
\end{proof}

{\bf (6) [RF 2.6]} Let $A$ be the set of irrational numbers in $[0,1]$. Show
that $m^{*}(A) = 1$.

{\bf Solution:}

Before proving the result we will have to prove one lemma.
\begin{Lemma}
If $A \subset \mathbb{R}$ is countably infinite, then $m^{*}(A) = 0$.
\end{Lemma}
\begin{proof}
Let $A = \{a_{n}\}_{n\in\mathbb{N}} \subset \mathbb{R}$ be a countably infinite
set. Since outer measure is always non-negative, $m^{*}(A) \geq 0$. Thus it
suffices to show that $m^{*}(A) \leq 0$. Now, from a theorem proved in class, we
know that $m^{*}(\{a_{n}\}) = 0$ for every $n \in \mathbb{N}$. Thus, by the
countable additivity of outer measure (RF Proposition 2.3), we have 
\[ m^{*}(A) = m^{*}(\cup_{n=1}^{\infty}\{a_{n}\}) \leq
\sum_{n=1}^{\infty}m^{*}(\{a_{n}\}) = \sum_{n=1}^{\infty}0 = 0.\]
Thus $m^{*}(A) = 0$.
\end{proof}

\begin{proof}
  Let $B$ be the set of rationals in $[0,1]$. So $A\cup B = [0,1]$. By RF
  Proposition 2.1, $m^{*}([0,1]) = 1$, and $m^{*}(B) = 0$ since $B$ is countably
infinite (Lemma 1). Thus, by the subadditivity of outer
  measure,
  \[ 1 = m^{*}(A\cup B) \leq m^{*}(A) + m^{*}(B) = m^{*}(A) + 0 = m^{*}(A). \]
  So $m^{*}(A) \geq 1$. However, since $A\subset A \cup B$, $m^{*}(A) \leq
  m^{*}(A\cup B) = 1$ by the monotonicity of outer measure. Thus $m^{*}(A) = 1$.
\end{proof}

{\bf (7) [RF 2.10]} Let $A,B$ be bounded sets for which there is an $\alpha > 0$
such that $|a-b| \geq \alpha$ for every $a\in A$ and $b \in B$. Show that
$m^{*}(A\cup B) = m^{*}(A) + m^{*}(B)$.

{\bf Solution:}

\begin{proof}
Let $A,B$ be bounded sets for which there is an $\alpha > 0$
such that $|a-b| \geq \alpha$ for every $a\in A$ and $b \in B$.
First note that by finite subadditivity, $m^{*}(A\cup B) \leq m^{*}(A) +
m^{*}(B)$. Thus, to show equality it remains to argue that $m^{*}(A) +
m^{*}(B) \leq m^{*}(A\cup B)$. Now, let $\{E_{k}\}_{k\in\mathbb{N}}$ be a
countable collection of bounded, open sets such that $A\cup B \subseteq
\cup_{k=1}^{\infty}E_{k}$ and 
\begin{equation}
\sum_{k=1}^{\infty}l(E_{k}) < m^{*}(A\cup B) + \epsilon.
\end{equation}
Further, since $|a-b| \geq \alpha$ for every $a \in A$ and $b \in B$, we can
assure that each $E_{k}$ contains either $A$ or $B$, but not both, by
stipulating that $l(E_{k}) < \alpha$ for every $k \in \mathbb{N}$. In other
words, if $A \cap E_{k} \neq \emptyset$, then $B\cap E_{k} = \emptyset$, and if
$B\cap E_{k} \neq \emptyset$, then $A\cap E_{k} = \emptyset$ for every $k \in
\mathbb{N}$. Now, let $I_{A}$ be the set of indices of
$\{E_{k}\}_{k\in\mathbb{N}}$ for which $A\cap E_{k} \neq \emptyset$, and $I_{B}$
the set of indices for which $B\cap E_{k} \neq \emptyset$. Then 
$A \subseteq \cup_{i\in I_{A}} E_{i} \subseteq \mathbb{R}$, $B \subseteq \cup_{j\in I_{B}}
E_{j} \subseteq \mathbb{R}$. Futher, since $\cup_{i \in I_{A}}E_{i}, \cup_{j\in I_{B}}E_{j} \neq \emptyset$ and open,
\begin{equation}
m^{*}(A) + m^{*}(B) \leq \sum_{i \in I_{A}}l(E_{i}) + \sum_{j\in I_{B}}l(E_{j})
\end{equation}
by the definition of outer measure. Also,
\begin{equation}
\sum_{i \in I_{A}}l(E_{i}) + \sum_{j\in I_{B}}l(E_{j}) \leq
\sum_{k=1}^{\infty}l(E_{k}) \stackrel{(1)}{<} m^{*}(A\cup B) + \epsilon.
\end{equation}
Putting (2) and (3) together, we have $m^{*}(A) + m^{*}(B) < m^{*}(A\cup B) +
\epsilon$. But since $\epsilon > 0$ was completely arbitrary, 
\[ m^{*}(A) + m^{*}(B) \leq m^{*}(A\cup B). \]
Therefore $m^{*}(A) + m^{*}(B) = m^{*}(A\cup B)$.
\end{proof}

\end{document}

