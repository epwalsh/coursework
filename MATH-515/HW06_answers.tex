\documentclass[12pt]{article}
\usepackage{amsmath}
\usepackage{amsfonts}
\usepackage{parskip}
\usepackage{amsthm}
\usepackage{thmtools}
\usepackage[headheight=15pt]{geometry}
\geometry{a4paper, left=20mm, right=20mm, top=30mm, bottom=30mm}
\usepackage{graphicx}
\usepackage{bm} % for bold font in math mode - command is \bm{text}
\usepackage{enumitem}
\usepackage{fancyhdr}
\usepackage{amssymb} % for stacked arrows and other shit
\pagestyle{fancy}

\declaretheoremstyle[headfont=\normalfont]{normal}
\declaretheorem[style=normal]{Theorem}
\declaretheorem[style=normal]{Proposition}
\declaretheorem[style=normal]{Lemma}
\newenvironment{claimproof}[1]{\par\noindent\underline{Proof of claim:}\space#1}{\hfill $\blacksquare$\vspace{5mm}}

\title{MATH 515: HW 6}
\author{Evan ``Pete'' Walsh}
\makeatletter
\let\runauthor\@author
\let\runtitle\@title
\makeatother
\lhead{\runauthor}
\chead{\runtitle}
\rhead{\thepage}
\cfoot{}

\begin{document}
\maketitle

\section*{1}
Prove that every step function is a simple function.

\subsection*{Solution}
\begin{proof}
Let $f : [a,b] \rightarrow \mathbb{R}$ be a step function. Then there is a partition $(x_{0}, \cdots, x_{n})$ of $[a,b]$ such that $f$ is constant on
each open subinterval $(x_{0}, x_{1}), \cdots, (x_{n-1}, x_{n})$. Let $E_{i} = (x_{i-1},x_{i})$ for $i = 1,\cdots, n$ and $c_{i}$ be the value of $f$
over each interval $E_{i}$. Further, for $i = n+1, \cdots, 2n+1$, let $E_{i} = \{x_{i-n-1}\}$ and $c_{i} = f(x_{i-n-1})$. Let 
\[ f' = \sum_{i=1}^{2n+1} c_{i}\cdot \chi_{E_{i}}. \]
Since $E_{i}$ is either an interval or a single point, $\left\{ E_{i} \right\}_{i=1}^{2n+1}$ is a collection of measurable sets. Thus, by Proposition
(III)(3)(iii), $f'$ is a simple function and by construction, $f(x) = f'(x)$ for all $x \in [a,b]$. Hence 
\[ f = f' \big|_{[a,b]}. \]

\underline{Claim:} $f$ is a simple function.
\begin{claimproof}
Clearly $f$ is real-valued with a finite number of elements in ran($f$). So we just need to show that $f$ is measurable. Let $U \subseteq [-\infty,
\infty]$ be open. Then,
\begin{align*}
f^{-1}(U) = \left\{ x \in [a,b] : f(x) \in U \right\} & = \left\{ x \in [a,b] : f'(x) \in U \right\} \\
& = \left\{ x \in \mathbb{R} : f'(x) \in U \wedge x \in [a,b] \right\} \\
& = (f')^{-1}[U] \cap [a,b],
\end{align*}
which is measurable since $f'$ is measurable.
\end{claimproof}

\end{proof}


\newpage 
\section*{2 [RF 3.1]}
Suppose $f$ and $g$ are continuous over $[a,b]$. If $f = g$ a.e. on $[a,b]$, then $f = g$ on $[a,b]$.

\subsection*{Solution}
\begin{proof}
We will proceed by contradiction. So, assume that there exists $x' \in [a,b]$ such that $f(x') \neq g(x')$. By the Extreme Value Theorem, $|f(x')|,
|g(x')| < \infty$. Therefore $f(x') - g(x') \neq 0$. Thus, there exists $\epsilon > 0$ such that 
\begin{equation}
|f(x') - g(x')| = \epsilon.
\label{2.1}
\end{equation} 
Now, since $f$ is
continuous over $[a,b]$, there exists $\delta_{1} > 0$ such that for all $y \in [a,b]$ so that $|y - x'| < \delta_{1}$, 
\[ |f(y) - f(x')| < \epsilon /2. \]
Similarly, there exists $\delta_{2} > 0$ such that for all $y \in [a,b]$ so that $|y - x'| < \delta_{2}$,
\[ |g(y) - g(x')| < \epsilon / 2. \]
Let $\delta = \min\left\{ \delta_{1}, \delta_{2} \right\}$. Now, since $(x'  - \delta, x' + \delta)$ and $[a,b]$ are intervals, 
\[ I = (x' -\delta, x' + \delta) \cap [a,b] \]
is an interval. Further, since $a \leq x' \leq b$, $0 < \ell(I) = \mu(I)$. Hence, since $f = g$ a.e. on $[a,b]$ and $I$ has positive measure, there exists $x_{0} \in I$ such that
$f(x_{0}) = g(x_{0})$. Since $x_{0} \in [a,b]$ and $|x_{0} - x'| < \ell(I) / 2 = \delta \leq \delta_{1}, \delta_{2}$,
\[ |f(x_{0}) - f(x')| < \epsilon / 2 \qquad \text{and} \qquad |g(x_{0}) - g(x')| < \epsilon / 2. \]
Therefore,
\begin{align*}
|f(x') - g(x')| & \leq |f(x') - f(x_{0})| + |f(x_{0}) - g(x')| \\
& \leq |f(x') - f(x_{0})| + |f(x_{0}) - g(x_{0})| + |g(x_{0}) - g(x')| \\
& < \epsilon / 2 + 0  + \epsilon / 2 = \epsilon.
\end{align*}
This contradicts equation \ref{2.1}. Thus $f = g$ on $[a,b]$.
\end{proof}
If $[a,b]$ is replaced by a general measurable set $E$, the assertion does not remain true. For example, let $E = \mathbb{N}$ and let $f$ and $g$ be
real-valued functions on $E$ such that 
\[ f(n) = n \qquad \text{ and } \qquad g(n) = n+1, \]
for all $n \in E$. $f$ and $g$ are continuous since for any open $U \subseteq [-\infty, \infty]$,
\[ f^{-1}(U) = \mathbb{N} \cap U = E \cap U, \text{ and } g^{-1}(U) = \mathbb{N} \cap (U - 1) = E \cap (U - 1). \]
Further $\mu\left( \{n \in E : f(n) \neq g(n)\} \right) = \mu(E) = \mu(\mathbb{N}) = 0$, so $f = g$ a.e. on $E$. But $f(n) \neq g(n)$ for every $n \in E$.


\newpage 
\section*{3}
For every positive integer $n$, let $S_{n}(x)$ denote the number of 5's in the first $n$ digits of the non-terminating decimal expansion of $x$.
Prove that $\left\{ x \in [0,1] : \lim_{n\rightarrow\infty}\frac{S_{n}(x)}{n}\text{ exists} \right\}$ is measurable.

\subsection*{Solution}

\begin{proof}
Let $n \in \mathbb{N}$.

\underline{Claim 1:} $S_{n} : [0,1] \rightarrow [-\infty, \infty]$ is measurable.
\begin{claimproof}
Since $[0,1]$ is measurable, it remains to show that $S_{n}^{-1}(U)$ is measurable for any open $U \subseteq [-\infty, \infty]$. 

For each positive integer $j$, let $X_{j}$ be defined as in Question 3 of HW 5. Now, for each 
$k \in \left\{ 0, \dots, n \right\}$, if $S_{n}(x) = k$, then there are $\binom{n}{k}$ possibilities for the positions of the $k$ 5's in the first $n$
digits of the non-terminating decimal expansion of $x$. Let $c_{k} = \binom{n}{k}$. Let $\left\{ F_{k,i} \right\}_{k=1}^{c_{k}}$, where $F_{k,i} =
\left\{ j_{1}, \dots, j_{k} \right\}$, denote the $c_{k}$ unique choices for the $k$ 5's in the first $n$ digits of the non-terminating decimal
expansion of $x$. So,
\begin{align*}
S_{n}^{-1}(k) & = \left\{ x \in [0,1] : S_{n}(x) = k \right\} \\
& = \bigcup_{i=1}^{c_{k}}\left\{ x \in [0,1] : x\text{ has a 5 only at the positions }j \in F_{k,i} \right\} \\
& = \bigcup_{i=1}^{c_{k}}\left[  \left(\bigcap_{j\in F_{k,i}}X_{j}\right) - \bigcup_{j \notin F_{k,i}, 1 \leq j \leq n} X_{j} \right]
\end{align*}
which is measurable since $X_{j}$ is measurable for all $j \geq 1$ by HW 5 Question 3. Now, let $U \subseteq [-\infty, \infty]$ be open. Then,
\begin{align*}
S_{n}^{-1}(U) = \left\{ x \in [0,1] : S_{n}(x) \in U \right\} & = \left\{ x \in [0,1] : S_{n}(x) \in U \cap \mathbb{N} \right\} \\
& = \bigcup_{k \in U \cap \mathbb{N}} S_{n}^{-1}(k)
\end{align*}
is measurable. So $S_{n}$ is measurable.
\end{claimproof}

It follows by Theorem (III)(2)(vii) that $f_{n}(x) = \frac{S_{n}(x)}{n}$ is measurable. Since $n \geq 1$ was arbitrary, $\left\{ f_{n}
\right\}_{n=1}^{\infty}$ is a sequence of measurable functions with a common domain. So, by Question 4 (the next question in this homework), the set 
\[ \left\{ x \in [0,1] : \lim_{n\rightarrow\infty}f_{n}(x)\text{ exists} \right\} = \left\{ x \in [0,1] : 
\lim_{n\rightarrow\infty}\frac{S_{n}(x)}{n} \text{ exists}\right\} \]
is measurable.
\end{proof}


\newpage
\section*{4 [RF 3.9]}
Let $\left\{ f_{n} \right\}$ be a sequence of measurable functions defined on a measurable set $E$. Define $E_{0}$ as the set of all points $x \in E$
such that $\left\{ f_{n}(x) \right\}$ converges. We will show that $E_{0}$ is measurable.

\subsection*{Solution}
\underline{Lemma:} If $f : E \rightarrow [-\infty, \infty]$ is measurable, then 
\[ f^{*}(x) = f(x) \cdot \chi_{\mathbb{R}}[f(x)] = \left\{ \begin{array}{cl}
f(x) & \text{ if } x \in \mathbb{R} \\
0 & \text{ if } x \notin \mathbb{R} 
\end{array} \right. \]
is measurable.

\begin{proof}
Let $a \in \mathbb{R}$. If $a \leq 0$, then 
\begin{align*}
(f^{*})^{-1}\left( [a,\infty] \right) & = \left\{ x \in E : a \leq f^{*}(x) \leq \infty \right\} \\
& = \left\{ x \in E : a \leq f(x) \leq \infty \vee f(x) = -\infty \right\} \\
& = \left\{ x \in E : a \leq f(x) \leq \infty \right\} \cup \left\{ x \in E : f(x) = -\infty \right\} \\
& = f^{-1}\left( [a,\infty] \right) \cup f^{-1}(\left\{ -\infty \right\}),
\end{align*}
which is measurable. If $a > 0$, then 
\begin{align*}
(f^{*})^{-1}\left( [a,\infty] \right) & = \left\{ x \in E : a \leq f^{*}(x) \leq \infty \right\} \\
& = \left\{ x \in E : a \leq f(x) < \infty \right\} \\
& = f^{-1}\left( [a,\infty) \right),
\end{align*}
which is measurable. Thus, by Proposition (III)(1)(vii), $f^{*}$ is measurable.
\end{proof}

We will now prove the main result.

\begin{proof}
Let $\left\{ f_{n} \right\}$ be a sequence of measurable functions defined on a measurable set $E$. Define $E_{0}$ as the set of all points $x \in E$
such that $\left\{ f_{n}(x) \right\}$ converges. Note that for each $x \in E$, $\left\{ f_{n}(x) \right\}$ converges if and only if 
\[ -\infty < \lim\inf f_{n}(x) = \lim\sup f_{n}(x) < \infty. \]
Let $g_{0} = \lim\inf f_{n}$ and $g_{1} = \lim\sup f_{n}$. By Question 5 on this homework, $g_{0}$ and $g_{1}$ are measurable. Now define 
\[ g_{0}^{*} = g_{0}\cdot \chi_{\mathbb{R}}[g_{0}] \qquad \text{ and } \qquad g_{1}^{*} = g_{1}\cdot \chi_{\mathbb{R}}[g_{1}]. \]
By construction, $g_{i}^{*}$ is finite, and by the lemma, $g_{i}^{*}$ is measurable, for $i = 1,2$. Hence, by Theorem (III)(2)(vii), $g = g_{0}^{*} - g_{1}^{*}$ is measurable. Thus,
\begin{align*}
E_{0} & = \left\{ x \in E : g(x) = 0 \wedge g_{0} \in \mathbb{R} \wedge g_{1} \in \mathbb{R} \right\} \\
& = \left\{ x \in E : g(x) = 0 \right\} \cap \left\{ x \in E : g_{0}(x) \in \mathbb{R} \right\} \cap \left\{ x \in E : g_{1}(x) \in \mathbb{R}
\right\} \\ 
& = g^{-1}\left[ \left\{ 0 \right\} \right] \cap (g_{0})^{-1}\left[ \mathbb{R} \right] \cap \left( g_{1} \right)^{-1}\left[ \mathbb{R}
\right], 
\end{align*}
which is measurable.
\end{proof}


\newpage
\section*{5 [RF 3.21]}
Let $\left\{ f_{n} \right\}_{n=0}^{\infty}$ be a sequence of measurable functions with common domain $E$. Show that each of the following functions is
measurable: $\inf\left\{ f_{n} \right\}$, $\sup\left\{ f_{n} \right\}$, $\lim\inf\left\{ f_{n} \right\}$, and $\lim\sup\left\{ f_{n} \right\}$.

\subsection*{Solution}
\begin{description}
\item[(i)] $h_{1} \equiv \inf\left\{ f_{n} \right\}$ is measurable. 

\begin{proof}
Let $a \in \mathbb{R}$. Then 
\begin{align*}
h_{1}^{-1}\left( [a,\infty] \right) = \left\{ x \in E : h(x) \geq a \right\} & = \left\{ x \in E : \inf\left\{ f_{n}(x) \right\} \geq a \right\} \\
& = \left\{ x \in E : f_{n}(x) \geq a \ \forall\ n \in \mathbb{N} \right\} \\
& = \bigcap_{n=0}^{\infty}\left\{ x \in E : f_{n}(x) \geq a \right\} \\
& = \bigcap_{n=0}^{\infty}f_{n}^{-1}\left( [a,\infty] \right),
\end{align*}
which is measurable by Proposition (3)(1)(vii). Therefore, by Proposition (3)(1)(vii) again, $h_{1}$ is measurable.
\end{proof}

\item[(ii)] $h_{2} \equiv \sup\left\{ f_{n} \right\}$ is measurable.

\begin{proof}
Let $a \in \mathbb{R}$. Then 
\begin{align*}
h_{2}^{-1}\left( [-\infty, a] \right) = \left\{ x \in E : h_{2}(x) \leq a \right\} & = \left\{ x \in E : \sup\left\{ f_{n}(x) \right\} \leq a \right\}
\\
& = \left\{ x \in E : f_{n}(x) \leq a\ \forall\ n \in \mathbb{N} \right\} \\
& = \bigcap_{n=0}^{\infty}\left\{ x \in E : f_{n}(x) \leq a \right\} \\
& = \bigcap_{n=0}^{\infty}f_{n}^{-1}\left( [-\infty, a] \right),
\end{align*}
which is measurable. Thus $h_{2}$ is measurable.
\end{proof}

\item[(iii)] $h_{3} \equiv \lim\inf\left\{ f_{n} \right\}$ is measurable.

\begin{proof}
Let $a \in \mathbb{R}$. Then, 
\begin{align*}
h_{3}^{-1}\left( [a,\infty] \right) = \left\{ x \in E : h(x) \geq a \right\} & = \left\{ x \in E : \lim\inf f_{n}(x) \geq a \right\} \\
& = \left\{ x \in E: \exists\ n \in \mathbb{N} \text{ such that } k \geq n \text{ implies } f_{k}(x) \geq a \right\} \\
& = \bigcup_{n=0}^{\infty}\bigcap_{k\geq n}\left\{ x \in E : f_{k}(x) \geq a \right\} \\
& = \bigcup_{n=0}^{\infty}\bigcap_{k\geq n}f_{k}^{-1}\left( [a,\infty] \right),
\end{align*}
which is measurable. Thus $h_{3}$ is measurable.
\end{proof}

\item[(iv)] $h_{4} \equiv \lim\sup\left\{ f_{n} \right\}$ is measurable.

\begin{proof}
Let $a \in \mathbb{R}$. Then,
\begin{align*}
h_{4}^{-1}\left( [-\infty,a] \right) = \left\{ x \in E : h(x) \leq a \right\} & = \left\{ x \in E : \lim\sup f_{n}(x) \leq a \right\} \\
& = \left\{ x \in E : \exists\ n \in \mathbb{N} \text{ such that } k \geq n \text{ implies } f_{k}(x) \leq a \right\} \\
& = \bigcup_{n=0}^{\infty}\bigcap_{k\geq n} \left\{ x \in E : f_{k}(x) \leq a \right\} \\
& = \bigcup_{n=0}^{\infty}\bigcap_{k\geq n} f_{k}^{-1}\left( [-\infty, a] \right),
\end{align*}
which is measurable. Thus $h_{4}$ is measurable.
\end{proof}
\end{description}


















\end{document}

