\documentclass[12pt]{article}
\usepackage{amsmath}
\usepackage{amsfonts}
\usepackage{parskip}
\usepackage{amsthm}
\usepackage{thmtools}
\usepackage[headheight=15pt]{geometry}
\geometry{a4paper, left=20mm, right=20mm, top=30mm, bottom=30mm}
\usepackage{graphicx}
\usepackage{bm} % for bold font in math mode - command is \bm{text}
\usepackage{enumitem}
\usepackage{fancyhdr}
\usepackage{amssymb} % for stacked arrows and other shit
\pagestyle{fancy}

% This rule is for making an integral with a bar above it or below it.
\makeatletter
\newcommand\tint{\mathop{\mathpalette\tb@int{t}}\!\int}
\newcommand\bint{\mathop{\mathpalette\tb@int{b}}\!\int}
\newcommand\tb@int[2]{%
  \sbox\z@{$\m@th#1\int$}%
  \if#2t%
    \rlap{\hbox to\wd\z@{%
      \hfil
      \vrule width .35em height \dimexpr\ht\z@+1.4pt\relax depth -\dimexpr\ht\z@+1pt\relax
      \kern.05em % a small correction on the top
    }}
  \else
    \rlap{\hbox to\wd\z@{%
      \vrule width .35em height -\dimexpr\dp\z@+1pt\relax depth \dimexpr\dp\z@+1.4pt\relax
      \hfil
    }}
  \fi
}
\makeatother

\declaretheoremstyle[headfont=\normalfont]{normal}
\declaretheorem[style=normal]{Theorem}
\declaretheorem[style=normal]{Proposition}
\declaretheorem[style=normal]{Lemma}
\newenvironment{claimproof}[1]{\par\noindent\underline{Proof:}\space#1}{\hfill $\blacksquare$}

% \begin{itemize}[label={},leftmargin=4mm, itemsep=1em, parsep=1em]

\title{MATH 515: HW 3}
\author{Evan ``Pete'' Walsh}
\makeatletter
\let\runauthor\@author
\let\runtitle\@title
\makeatother
\lhead{\runauthor}
\chead{\runtitle}
\rhead{\thepage}
\cfoot{}

\begin{document}
\maketitle

{\bf (1) [RF 2.11]} If a $\sigma$-algebra of $\mathbb{R}$ contains intervals of the form $(a, \infty)$, then it contains all intervals.

{\bf Solution:}

\begin{proof}
Let $\mathcal{F} \subseteq \mathcal{P}(\mathbb{R})$ be a $\sigma$-algebra. Assume $\mathcal{F}$ contains all intervals of the form $(x, \infty)$. We
need to show that $\mathcal{F}$ contains all sets of the following forms:

\begin{itemize}[label={},leftmargin=4mm, itemsep=1em, parsep=0em]
\item (i) $(-\infty, a)$,
\item (ii) $(-\infty, a]$, 
\item (iii) $[a, \infty)$,
\item (iv) $[a, b)$,
\item (v) $[a,b]$,
\item (vi) $(a, b]$, and 
\item (vii) $(a, b)$, where $a, b \in \mathbb{R}$ and $a < b$.
\end{itemize}

So, let $a, b \in \mathbb{R}$ with $a < b$.

(i) Define $I_{n} = (a - 1/n, \infty)$ for each $n \in \mathbb{N}$. By assumption, $I_{n} \in \mathcal{F}$. Thus $(I_{n})^{c} = (-\infty, a - 1/n] \in
\mathcal{F}$ since $\mathcal{F}$ is closed under complimentation by definition of a $\sigma$-algebra. Further, 
\[ \cup_{n\in\mathbb{N}}(I_{n})^{c} = \cup_{n\in\mathbb{N}}(-\infty, a - 1/n] = (-\infty, a) \in \mathcal{F}, \]
since $\mathcal{F}$ is closed under countable unions. Hence sets of form $(i)$ are in $\mathcal{F}$.

(ii) Next, we know $(a, \infty) \in \mathcal{F}$ by assumption, so $(a, \infty)^{c} = (-\infty, a] \in \mathcal{F}$ by closure of compliments.

(iii) Since $(-\infty, a) \in \mathcal{F}$ by $(i)$, $(-\infty, a)^{c} = [a, \infty) \in \mathcal{F}$ by closure of compliments.

(iv) By (iii), $[b, \infty) \in \mathcal{F}$ and $[a,\infty) \in \mathcal{F}$, so $[a,\infty)\setminus [b,\infty) = [a,b] \in \mathcal{F}$ by Lemma
(II)(2)(xvi).

(v) By (iii), $[a, \infty) \in \mathcal{F}$ and $(b,\infty) \in \mathcal{F}$ by assumption, so $[a,\infty) \setminus (b,\infty) = [a,b] \in
\mathcal{F}$ by Lemma (II)(2)(xvi).

(vi) By (ii), $(-\infty, b] \in \mathcal{F}$ and $(-\infty, a] \in \mathcal{F}$, so $(-\infty, b] \setminus (-\infty, a] = (a,b] \in \mathcal{F}$ by
Lemma (II)(2)(xvi).

(vii) By (i), $(-\infty, b) \in \mathcal{F}$ and by (ii) $(-\infty, a] \in \mathcal{F}$. So $(-\infty, b) \setminus (-\infty, a] = (a,b) \in
\mathcal{F}$.
\end{proof}



{\bf (2) [RF 2.14]} If a set $E$ has positive outer measure, then there is a bounded subset of $E$ that also has positive outer measure.

{\bf Solution:}

\begin{proof}
Let $E$ be a set with positive outer measure. There are two cases to consider: either $E$ is bounded or $E$ is not bounded. If $E$ is bounded then the
proof is trivial since $E \subseteq E$ is bounded with positive outer measure by assumption. Now assume that $E$ is unbounded. We will do a proof by
contradiction. So assume that there is no subset of $E$ that has positive outer measure. Since outer measure is always non-negative, this means that
every subset of $E$ has measure 0. Define $A_{n} = E \cap [-n, n]$ for every $n \in \mathbb{N}$. Note that each $A_{n}$ is a subset of $E$ and is bounded since $|x| \leq n$
for every $x \in A_{n}$. Therefore $\mu^{*}(A_{n}) = 0$ for all $n \in \mathbb{N}$ by assumption. Further, $E = \cup_{n=1}^{\infty}A_{n}$, so by the countable subadditivity of outer measure,
\[ \mu^{*}(E) = \mu^{*}\left( \cup_{n=1}^{\infty}A_{n} \right) \leq \sum_{n=1}^{\infty}\mu^{*}(A_{n}) = \sum_{n=1}^{\infty} 0 = 0. \]
This is a contradiction since $\mu^{*}(E) > 0$.  
\end{proof}

{\bf (3) [RF 2.15]} If $E$ has finite measure and $\epsilon > 0$, then $E$ is the disjoint union of a finite number of measurable sets, each of which
has measure at most $\epsilon$.

{\bf Solution:} 

\begin{proof}
Note from question (3) on homework 2, the definition from the book of outer measure is equivalent to our definition. For the convenience of this
proof, we will use the definition from the book. So, assume $E \subset \mathbb{R}$ has finite measure, i.e. $E$ is measurable and $m^{*}(E) <
\infty$, and let $\epsilon > 0$ and $\delta > 0$. It follows from
the definition of [outer] measure that we can choose a collection of non-empty, bounded, open intervals $\left\{ J_{k} \right\}_{k\in\mathbb{N}}$ such that 
\[ \sum_{k=1}^{\infty}\ell(J_{k}) \leq m(E) + \delta. \]
Since $m(E) + \delta < \infty$, $\sum_{k=1}^{\infty}\ell(J_{k})$ converges. Thus, there exists some $N \in \mathbb{N}$ such that
$\sum_{k=N}^{\infty}\ell(J_{k}) < \epsilon$. Now, define $E_{1} = E\cap \left( \cup_{k=N}^{\infty}J_{k} \right)$. Since $\cup_{k=N}^{\infty}J_{k}$ is
a countable union of open intervals, each of which is measurable by RF Proposition 2.8, $\cup_{k=N}^{\infty}J_{k}$ is also measurable since the measurable
sets form a $\sigma$-algebra by Theorem (II)(2)(xviii) which, by definition, is closed under countable unions. Hence, by Corollary (II)(2)(xii), $E_{1}$ 
is measurable because it is the intersection of two measurable sets. So, by monotonicity and countable subadditivity,
\begin{equation}
m(E_{1}) \leq m\left( \cup_{k=N}^{\infty}J_{k} \right) \leq \sum_{k=N}^{\infty}\ell(J_{k}) < \epsilon.
\label{eqn:3-0}
\end{equation}
Now, since
$\cup_{k=1}^{N-1}J_{k}$ is a finite union of bounded intervals, $\cup_{k=1}^{N-1}J_{k}$ is bounded as well. Thus there exists some $M \in \mathbb{R}$ 
such that $2 < M < \infty$ and
$|x| < M/2 - 1/2$ for all $x \in \cup_{k=1}^{N-1}J_{k}$. Now, let $b = \sup\left\{ x : x \in \cup_{k=1}^{N-1}J_{k} \right\}$ and $a = \inf\left\{
x : x \in \cup_{k=1}^{N-1}J_{k} \right\}$. Then,
\[ |b -a | \leq |b| + |a| \leq \left( \frac{M}{2} - \frac{1}{2} \right) + \left( \frac{M}{2} - \frac{1}{2} \right) = M - 1 < M. \]
So $|b - a|$ is bounded. Hence we can choose a finite sequence $(a_{1}, a_{2}, \dots, a_{p})$ in $\mathbb{R}$ such that $a = a_{1} < a_{2} < \dots <
a_{p} = b$, where $a_{k} - a_{k-1} < \epsilon$ for all $k \in \left\{ 2, 3, \dots, p \right\}$. Now define $E_{k} = \left( E\setminus E_{1}
\right)\cap [a_{k-1}, a_{k})$ for $k \in \left\{ 2, \dots, p \right\}$. Using Corollary (II)(2)(xii) again, each $E_{k}$ is measurable, and by monotonicity 
% Further, 
\begin{equation}
m(E_{k}) \leq m([a_{k-1}, a_{k}))  = a_{k} - a_{k-1} < \epsilon,
\label{eqn:3-1}
\end{equation}
since $E_{k} \subseteq [a_{k-1}, a_{k})$ for each $k \in \left\{ 2, \dots, p \right\}$. 
Further, 
\[ E \setminus E_{1} = E \setminus \left( E\cap \bigcup_{k=N}^{\infty}J_{k} \right) \subseteq \cup_{k=1}^{N-1}J_{k} \subseteq (a, b) \subseteq [a,b)
 = \cup_{k=2}^{p}[a_{k-1}, a_{k}), \]
so 
\begin{equation}
E\setminus E_{1} = E\setminus E_{1} \cap \left( \cup_{k=2}^{p}[a_{k-1},a_{k}) \right) = \bigcup_{k=2}^{p}\bigg((E\setminus E_{1})\cap [a_{k-1},
a_{k})\bigg) = \cup_{k=2}^{p}E_{k}. 
\label{eqn:3-2}
\end{equation}
In addition, it follows from the fact that $[a_{k-1}, a_{k}) \cap [a_{j-1}, a_{j}) = \emptyset$ when $j,k \in \left\{ 2,\dots, p \right\}$ with $j
\neq p$, that each $E_{k}$ is disjoint. Thus, by construction $\left\{ E_{k} \right\}_{k=1}^{p}$ is a pairwise disjoint collection of measurable sets such that 
\[ \cup_{k=1}^{p}E_{k} = E_{1}\cup \left( \cup_{k=1}^{p}E_{k} \right) \stackrel{(\ref{eqn:3-2})}{=} E_{1} \cup \left( E\setminus
E_{1} \right) = E\cup E_{1} = E, \]
and $m(E_{k}) < \epsilon$ for each $1 \leq k \leq p$ by equations \ref{eqn:3-0} and \ref{eqn:3-1}.
\end{proof}

{\bf (4)} We will show that there is not a countably infinite $\sigma$-algebra over the reals.

{\bf Solution:}

\begin{proof}
We will proceed using a proof by contradiction. So, assume that there exists a countably infinite $\sigma$-algebra over $\mathbb{R}$, call it
$\mathcal{F}$. The objective right now will be to construct a countably infinite disjoint collection of sets in $\mathcal{F}$ such that every set in
$\mathcal{F}$ can be expressed as the unique union of a subset of that collection. Once such a set is constructed, a contradiction easily follows by
using Cantor's diagonal argument.

To accomplish this, we first define the following funtion $f : \mathbb{R} \rightarrow \mathcal{F}$, by 
\[ f(x) = \bigcap_{\{E : E\in \mathcal{F}, x \in E\}}E, \]
for all $x \in \mathbb{R}$.

\underline{Claim 1:} $\mathcal{F}$ is well-defined.

\begin{claimproof}
We need to show that for each $x \in \mathbb{R}$, the image of $x$ under $f$ is in $\mathcal{F}$. So, let $x \in \mathbb{R}$. By assumption, $\mathcal{F}$ is countable, and
so $\left\{ E : E \in \mathcal{F}, x \in E \right\} \subseteq \mathcal{F}$ is countable. Therefore 
\[ \bigcap_{\left\{ E: E\in\mathcal{F}, x \in E \right\}}E \in \mathcal{F}, \]
since $\mathcal{F}$ is closed under countable intersections.
\end{claimproof}

\underline{Claim 2:} The image of $f$, $\left\{ A : A = f(x), x \in \mathbb{R} \right\}$ is a disjoint collection of sets.

\begin{claimproof}
Let $x, y \in \mathbb{R}$. We need to show that $f(x) = f(y)$ or $f(x) \cap f(y) = \emptyset$. In order to derive a contradiction, assume that $f(x)
\neq f(y)$ and $f(x) \cap f(y) \neq \emptyset$. Then there are two cases to consider. Either (i) $x \in f(x) \setminus f(y)$, or (ii) $x \in f(x) \cap
f(y)$.
\begin{itemize}[label={},leftmargin=4mm, itemsep=1em, parsep=1em]
\item (i) Assume $x \in f(x) \setminus f(y)$.

Since $f(x), f(y) \in \mathcal{F}$ by claim 1, $f(x) \setminus f(y) \in \mathcal{F}$ by Lemma (II)(2)(xvi). So, $x \in f(x) \setminus f(y) \subset
f(x)$, which is a contradiction since $f(x)$ should be a subset of all other sets in $\mathcal{F}$ that contain $x$.

\item (ii) Now assume that $x \in f(x) \cap f(y)$.

Well, $f(x) \cap f(y) \in \mathcal{F}$ and $f(x) \cap f(y) \subset f(x)$, which brings us to a contradiction again since $f(x)$ should be a subset of
all other sets in $\mathcal{F}$ that contain $x$.
\end{itemize}
Since both cases lead to a contradiction, the image of $f$ must be a disjoint collection of sets.
\end{claimproof}

\underline{Claim 3:} Every set $B \in \mathcal{F}$ can be expressed as the countable union of a subset of the image of $f$. Further, this
representation of $B$ is unique in the sense that there is only one collection of sets in $\left\{ A : A = f(x), x \in \mathbb{R} \right\}$ such that
$B$ is their union.

\begin{claimproof}
Let $B \in \mathcal{F}$, and let $x \in B$. Clearly $x \in f(x) \subseteq \cup_{\left\{ f(y) : y \in B \right\}}f(y)$. So $B \subseteq \cup_{\left\{
f(y) : y \in B \right\}}f(y)$. Further, $f(x) \subseteq B$ because if that was not true, then $f(x) \cap B$ would be a ``smaller'' set in
$\mathcal{F}$ that contains $x$, which is a contradiction. So $f(x) \subseteq B$ for each $x \in B$. Therefore $\cup_{\left\{f(y) : y \in B
\right\}}f(y) \subseteq B$. Hence $B = \cap_{\{f(y) : y \in B\}}f(y)$.

The fact that $\left\{ f(y) : y \in B \right\}$ is the unique collection of sets in $\left\{ A : A = f(x), x \in \mathbb{R} \right\}$ such that 
$B$ is their union follows directly as a
result of claim 2, that the image of $f$, $\left\{ A : A = f(x), x \in \mathbb{R} \right\}$, is a pairwise disjoint collection of sets.
\end{claimproof}

\underline{Claim 4:} The image of $f$, $\left\{ A : A = f(x), x \in \mathbb{R} \right\}$, is countably infinite.

\begin{claimproof}
Since $\left\{ A : A = f(x), x \in \mathbb{R} \right\}$ is a subset of $\mathcal{F}$, it cannot be uncountable. On the other hand, if $\left\{ A : A = f(x), x \in \mathbb{R} \right\}$
was finite, then there could only be a finite number of unique collections of sets in $\left\{ A : A = f(x), x \in \mathbb{R} \right\}$, i.e. its power
set would be finite. Thus, by claim 3, $\mathcal{F}$ would be finite since every set in $\mathcal{F}$ can be expressed as the unique union of a subset
of $\left\{ A : A = f(x), x \in \mathbb{R} \right\}$. This is a contradiction. Hence $\left\{ A : A = f(x), x \in \mathbb{R} \right\}$ is countably infinite.
\end{claimproof}

As a result of claim 4, we can enumerate $\left\{ A : A = f(x), x \in \mathbb{R} \right\}$ as 
\[ \left\{ A : A = f(x), x \in \mathbb{R} \right\} = \left\{ A_{n} \right\}_{n\in\mathbb{N}}. \]
Now, let $\left\{ 0,1 \right\}^{\mathbb{N}}$ be the set of all countably infinite sequences of 0's and 1's. Define 
\[ g: \left\{ 0,1 \right\}^{\mathbb{N}} \longrightarrow \mathcal{F} \]
by 
\[ g\left( \left( a_{n} \right)_{n\in\mathbb{N}} \right) = \bigcup_{\{n : a_{n} = 1\}}A_{n}, \text{ for each } \left( a_{n} \right)_{n\in\mathbb{N}}
\in \left\{ 0,1 \right\}^{\mathbb{N}}. \]
Note that $g$ is well-defined since $\cup_{\{n:a_{n}=1\}}A_{n}$ is a countable union of sets in $\mathcal{F}$, and $\mathcal{F}$ is closed under
countable unions.

\underline{Claim 5:} $g$ is a bijection.

\begin{claimproof}
By claim 2, $\left\{ A : A = f(x), x \in \mathbb{R} \right\} = \left\{ A_{n} \right\}_{n\in\mathbb{N}}$ is a disjoint collection of sets, so the union
of any collection of sets in $\left\{ A_{n} \right\}_{n\in\mathbb{N}}$ is unique. Thus $g$ is one-to-one. Further, it follows as a corollary to claim
3 that $g$ is onto, since any set $B \in \mathcal{F}$ can be expressed as the union of sets in $\left\{ A_{n} \right\}_{n\in\mathbb{N}}$. Hence $g$ is a bijection.
\end{claimproof}

To wrap up, Cantor's diagonal argument for the uncountability of the interval $[0,1]$ establishes that $\left\{ 0,1 \right\}^{\mathbb{N}}$ is uncountable.
This is where the contradiction arises, since by claim 5 $\mathcal{F}$ has to be uncountable.

Therefore there does not exist a countably infinite $\sigma$-algebra over the reals.
\end{proof}

As a side note, I think that the above argument would be valid for ANY $\sigma$-algebra, not just $\sigma$-algebras over the reals. So there must not
exist any countably infinite $\sigma$-algebra in general.

{\bf (5)} Let $f: [a,b] \rightarrow \mathbb{R}$ be a function. Royden and Fitzpatrick's definition of the lower Riemann integral of $f$ over $[a,b]$
is 
\[ (R)\bint_{a}^{b} f = \sup\left\{ L^{*}(f,P) : P\text{ is a partition of }[a,b] \right\}, \]
where $L^{*}(f,P) = \sum_{i=1}^{n}m_{i}(x_{i} - x_{i-1})$, for $m_{i} = \inf\left\{ f(x) : x_{i-1} < x < x_{i} \right\}$, is a lower Riemann sum. Similary,
definition (I)(1)(v)(a), from our notes, of lower Riemann integral is 
\[ \bint_{a}^{b}f = \sup\left\{ L(f, P, V) : P\text{ is a partition of }[a,b], V = (v_{1}, \dots, v_{n})\text{ with }v_{i} \leq f(x)\forall x_{i-1} <
x < x_{i} \right\}, \]
where $L(f,P,V) = \sum_{i=1}^{n}v_{i}(x_{i} - x_{i-1})$, is also considered a lower Riemann sum. For notational simplicity, let 
\[ \sup_{P}\left\{ L^{*}(f,P) \right\} =\sup\left\{ L^{*}(f,P) : P\text{ is a partition of }[a,b] \right\}, \text{ and } \]
\begin{align*}
\sup_{P, V}\left\{ L(f,P,V) \right\} = \sup\big\{ L(f, P, V) :& P\text{ is a partition of }[a,b], \\
\qquad & V = (v_{1}, \dots, v_{n})\text{ with }v_{i} \leq f(x)\forall x_{i-1} < x < x_{i} \big\}.
\end{align*}
We will show that these two definitions of lower Riemann integral are equivalent.

{\bf Solution:}

\begin{proof}
The two definitions only differ in the way they define a lower Riemann SUM. Thus, we need to show that the supremums of both sets of lower Riemann
sums, defined in their own way, are equal. That is, we need to show $\sup_{P}\left\{ L^{*}(f,P) \right\} = \sup_{P,V}\left\{ L(f,P,V) \right\}$.

$(\Rightarrow)$ First we will show that $\sup_{P}\left\{ L^{*}(f,P) \right\} \leq \sup_{P,V}\left\{ L(f,P,V) \right\}$. 

Let $P' = (x_{0}, \dots, x_{n})$ be a partition of $[a,b]$. By definition,
\[ m_{i} = \inf\left\{ f(x) : x_{i-1} < x < x_{i} \right\} \leq f(x)\ \forall\ x_{i-1} < x < x_{i}, i \in \left\{ 1, \dots, n \right\}. \]
Thus, $(m_{1}, \dots, m_{n}) \in \left\{ V : V = (v_{1}, \dots, v_{n}), v_{i} \leq f(x)\ \forall\ x_{i-1} < x < x_{i} \right\}$. So,
\begin{align*}
L^{*}(f, P') = \sum_{i=1}^{n}m_{i}(x_{i}-x_{i-1}) \in& \left\{ \sum_{i=1}^{n}v_{i}(x_{i}-x_{i-1}) : v_{i} \leq f(x)\ \forall\ x_{i-1} < x < x_{i}
\right\} \\
&\qquad = \left\{ L(f,P',V) : v_{i} \leq f(x) \ \forall\ x_{i-1} < x < x_{i} \right\}. 
\end{align*}
Hence, $L^{*}(f,P') \leq \sup_{V}\left\{ L(f,P',V) \right\}$. But since this holds for any partition $P$ of $[a,b]$,
\begin{equation}
\sup_{P}\left\{ L^{*}(f,P) \right\} \leq \sup_{P,V}\left\{ L(f,P,V) \right\}.
\label{eqn:5-1}
\end{equation}

$(\Leftarrow)$ We will now show that $\sup_{P}\left\{ L^{*}(f,P) \right\} \geq \sup_{P,V}\left\{ L(f,P,V) \right\}$. 

Again, let $P' = (x_{0}, \dots, x_{n})$ be any partition of $[a,b]$, and let $V' = (v_{1}, \dots, v_{n})$ such that $v_{i} \leq f(x)$ for each
$x_{i-1} < x < x_{i}$, $i \in \{1, \dots, n\}$. By definition, $v_{i}$ is a lower bound of $f$ on $(x_{i-1}, x_{i})$ for each $i \in \left\{ 1, \dots,
n \right\}$. Thus, 
\[ m_{i} = \inf\left\{ f(x) : x_{i-1} < x < x_{i} \right\} \geq v_{i}, \text{ for all } i \in \left\{ 1,\dots,n \right\}.\]
Therefore,
\[ L^{*}(f,P') = \sum_{i=1}^{n}m_{i}(x_{i}-x_{i-1}) \geq \sum_{i=1}^{n}v_{i}(x_{i}-x_{i-1}) = L(f,P',V'). \]
Since this holds for any $V' = (v_{1}, \dots, v_{n})$ such that $v_{i} \leq f(x)$ for each $x_{i-1} < x < x_{i}$, $i \in \left\{ 1,\dots, n \right\}$, 
\[ L^{*}(f, P') \geq \sup_{V}\left\{ L(f,P',V) \right\}. \]
But the above inequality holds for any partition $P$ of $[a,b]$, so 
\begin{equation}
\sup_{P}\left\{ L^{*}(f,P) \right\} \geq \sup_{P,V}\left\{ L(f,P,V) \right\}.
\label{eqn:5-2}
\end{equation}

From equations \ref{eqn:5-1} and \ref{eqn:5-2}, 
\[ \sup_{P}\left\{ L^{*}(f,P) \right\} = \sup_{P,V}\left\{ L(f,P,V) \right\}. \]
Hence the definitions of lower Riemann integral are equivalent.
\end{proof}


\end{document}

