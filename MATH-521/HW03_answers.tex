\documentclass[12pt]{article}
\usepackage{amsmath}
\usepackage{amsfonts}
\usepackage{parskip}
\usepackage{amsthm}
\usepackage{thmtools}
\usepackage[headheight=15pt]{geometry}
\geometry{a4paper, left=20mm, right=20mm, top=30mm, bottom=30mm}
\usepackage{graphicx}
\usepackage{bm} % for bold font in math mode - command is \bm{text}
\usepackage{enumitem}
\usepackage{fancyhdr}
\usepackage{amssymb} % for stacked arrows and other shit
\pagestyle{fancy}
\usepackage{changepage}
\usepackage{mathcomp}
\usepackage{tcolorbox}

\declaretheoremstyle[headfont=\normalfont]{normal}
\declaretheorem[style=normal]{Theorem}
\declaretheorem[style=normal]{Proposition}
\declaretheorem[style=normal]{Lemma}
\newcounter{ProofCounter}
\newcounter{ClaimCounter}[ProofCounter]
\newcounter{SubClaimCounter}[ClaimCounter]
\newenvironment{Proof}{\stepcounter{ProofCounter}\textsc{Proof.}}{\hfill$\square$}
\newenvironment{Solution}{\stepcounter{ProofCounter}\textbf{Solution:}}{\hfill$\square$}
\newenvironment{claim}[1]{\vspace{1mm}\stepcounter{ClaimCounter}\par\noindent\underline{\bf Claim \theClaimCounter:}\space#1}{}
\newenvironment{claimproof}[1]{\par\noindent\underline{Proof of claim \theClaimCounter:}\space#1}{\hfill $\blacksquare$ Claim \theClaimCounter}
\newenvironment{subclaim}[1]{\stepcounter{SubClaimCounter}\par\noindent\emph{Subclaim \theClaimCounter.\theSubClaimCounter:}\space#1}{}
% \newenvironment{subclaimproof}[1]{\begin{adjustwidth}{2em}{0pt}\par\noindent\emph{Proof of subclaim \theClaimCounter.\theSubClaimCounter:}\space#1}{\hfill
% $\blacksquare$ \emph{Subclaim \theClaimCounter.\theSubClaimCounter}\vspace{5mm}\end{adjustwidth}}
\newenvironment{subclaimproof}[1]{\par\noindent\emph{Proof of subclaim \theClaimCounter.\theSubClaimCounter:}\space#1}{\hfill
$\Diamond$ \emph{Subclaim \theClaimCounter.\theSubClaimCounter}}

\allowdisplaybreaks{}

% chktex-file 3

\lhead{Evan P. Walsh}
\chead{MATH 521}
\rhead{\thepage}
\cfoot{}

% Custom commands.
\newcommand\toinfty{\rightarrow\infty}
\newcommand\toinf{\rightarrow\infty}
\newcommand{\sinf}[1]{\sum_{#1=0}^{\infty}}
\newcommand{\linf}[1]{\lim_{#1\rightarrow\infty}}

\begin{document}\thispagestyle{empty}
\begin{center}
  \Large \textsc{math 521 -- ASSIGNMENT III -- fall 2018} \\ 
  \vspace{5mm}
  \large Evan Pete Walsh
\end{center}

%------------------------------------------------------------------------------------------------------------------%
% Question 1
%------------------------------------------------------------------------------------------------------------------%

\subsection*{1}
\begin{tcolorbox}
  Suppose $X_n, n\geq 0$ are defined on the same probability space $(\Omega, \mathcal{F}, P)$. Consider the following conditions:
  \begin{enumerate}[label=(\roman*)]
    \item $X_n$ converges to $X$ a.s., as $n \rightarrow \infty$.
    \item $X_n$ converges to $X$ in probability, as $n \toinf$.
    \item $X_n$ converges to $X$ in $L^p$, as $n \toinf$.
    \item $X_n$ converges to $X$ in distribution, as $n \toinf$.
  \end{enumerate}
  For each ordered pair of conditions, determine whether the first condition in the pair implies the other.
\end{tcolorbox}
\begin{Solution}
  \begin{claim}
    (i) implies (ii).
  \end{claim}
  \begin{claimproof}
    Suppose (i). Let $A$ be the subset of $\Omega$ on which $X_n$ does not converge to $X$. Equivelently, $A$ is the set of all $\omega \in \Omega$ such that there exists $k \geq 0$, where for all $n \geq 0$ there exists $j \geq n$ such that $|X_j(\omega) - X(\omega)| > 2^{-k}$, i.e.
    \[
      A = \bigcup_{k \geq 0} \bigcap_{n \geq 0} \bigcup_{j \geq n} A_{k, j},
    \]
    where $A_{k,j} := \{ \omega \in \Omega : |X_j(\omega) - X(\omega)| > 2^{-k} \}$. By (i), $P(A) = 0$, so clearly
    \[
      \bigcap_{n \geq 0} \bigcup_{j \geq n} A_{k, j} = 0 \ \forall k \geq 0.
    \]
    Now let $B_{k, n} := \bigcup_{j \geq n} A_{k, j}$ for each $k, n \geq 0$. So for fixed $k$, $B_{k, n} \downarrow \cap_{n \geq 0} \cup_{j \geq n} A_{k, j}$, and so, by the continuity of measure from above, $P(B_{k, n}) \rightarrow 0$ for all $k$.

    Hence, if we fix $\epsilon > 0$, we can choose $k \geq 0$ such that $2^{-k} < \epsilon$.  Therefore,
    \[
      P(|X_n - X| > \epsilon) \leq P(|X_n - X| > 2^{-k}) \leq P(B_{k, n}) \longrightarrow 0
    \]
    as $n \rightarrow \infty$.
  \end{claimproof}

  \begin{claim}
    (i) does not imply (iii).
  \end{claim}
  \begin{claimproof}
    Consider the Borel measure space on $[0, 1]$, and let $X \equiv 0$ and 
    \[
      X_n := 2^{n} * \bm{1}_{[0, 2^{-n}]}.
    \]
    Then $X_n(\omega) \rightarrow X$ for all $\omega \in (0, 1]$, but $\| X_n - X \|_{1} = EX_n \equiv 1$ for all $n$.
  \end{claimproof}

\end{Solution}

%------------------------------------------------------------------------------------------------------------------%
% Question 2
%------------------------------------------------------------------------------------------------------------------%

\subsection*{2}
\begin{tcolorbox}
  Suppose that $X_n, n \geq 0$ and $X$ are defined on the same probability space $(\Omega, \mathcal{F}, P)$. For each condition (ii)-(iv) in Problem 1, determine whether the DCT and MCT hold if we replace the assumption (i) with the new assumption. Give a proof or counter-example for each one.
\end{tcolorbox}

\begin{Solution}
    We will use the following 2 facts from real analysis.

    \textbf{Fact 1:} If $f, f_0, f_1, \dots$ are measurable functions on a measure space $(\Omega, \mathcal{F}, \mu)$ and $f_n \rightarrow f$ in measure, then there exists a subsequence $\{ f_{n_k} \}_{k=0}^{\infty}$ of $\{ f_n \}_{n=0}^{\infty}$ that converges to $f$ almost surely.

    \textbf{Fact 2:} If $\{ x_n \}_{n=0}^{\infty}$ is a sequence of real numbers such that every subsequence has a further subsequence that converges to the same real number $x$, then $x_n \rightarrow x$.

  \begin{claim}
    The DCT holds with assumption (ii).
  \end{claim}
  \begin{claimproof}
    Suppose $X_n$ converges to $X$ in probability and $|X_n| \leq |Y|$ for $n \in \mathbb{N}$, where $Y$ is integrable. We need to show that $X$ is integrable and $EX_n \rightarrow EX$.

    Besides showing that $X$ is in fact integrable, Fact 2 says that it suffices to show that for any subsequence $\{ X_{n_k} \}_{k=0}^{\infty}$, there exists a further subsequence $\{ X_{n_{k_j}} \}_{j=0}^{\infty}$ such that $EX_{n_{k_j}} \rightarrow EX$.

    So let $\{ X_{n_k} \}_{k=0}^{\infty}$ be any subsequence. Since $X_n \rightarrow X$ in probability, clearly $X_{n_k} \rightarrow X$ in probability as well. Hence by Fact 1, there exists a further subsequence $\{ X_{n_{k_j}} \}_{j=0}^{\infty}$ of $\{ X_{n_k} \}_{k=0}^{\infty}$ such that $X_{n_{k_j}} \rightarrow X$ almost surely.
    But since the assumptions of DCT hold for $\{ X_{n_{k_j}} \}_{j=0}^{\infty}$ (treated as its own sequence) with respect to $X$ and $Y$, we have that $X$ is integrable and $EX_{n_{k_j}} \rightarrow EX$.
  \end{claimproof}

\end{Solution}

\end{document}
