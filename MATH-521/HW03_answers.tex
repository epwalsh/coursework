\documentclass[12pt]{article}
\usepackage{amsmath}
\usepackage{amsfonts}
\usepackage{parskip}
\usepackage{amsthm}
\usepackage{thmtools}
\usepackage[headheight=15pt]{geometry}
\geometry{a4paper, left=20mm, right=20mm, top=30mm, bottom=30mm}
\usepackage{graphicx}
\usepackage{bm} % for bold font in math mode - command is \bm{text}
\usepackage{enumitem}
\usepackage{fancyhdr}
\usepackage{amssymb} % for stacked arrows and other shit
\pagestyle{fancy}
\usepackage{changepage}
\usepackage{mathcomp}
\usepackage{tcolorbox}

\declaretheoremstyle[headfont=\normalfont]{normal}
\declaretheorem[style=normal]{Theorem}
\declaretheorem[style=normal]{Proposition}
\declaretheorem[style=normal]{Lemma}
\newcounter{ProofCounter}
\newcounter{ClaimCounter}[ProofCounter]
\newcounter{SubClaimCounter}[ClaimCounter]
\newenvironment{Proof}{\stepcounter{ProofCounter}\textsc{Proof.}}{\hfill$\square$}
\newenvironment{Solution}{\stepcounter{ProofCounter}\textbf{Solution:}}{\hfill$\square$}
\newenvironment{claim}[1]{\vspace{1mm}\stepcounter{ClaimCounter}\par\noindent\underline{\bf Claim \theClaimCounter:}\space#1}{}
\newenvironment{claimproof}[1]{\par\noindent\underline{Proof of claim \theClaimCounter:}\space#1}{\hfill $\blacksquare$ Claim \theClaimCounter}
\newenvironment{subclaim}[1]{\stepcounter{SubClaimCounter}\par\noindent\emph{Subclaim \theClaimCounter.\theSubClaimCounter:}\space#1}{}
% \newenvironment{subclaimproof}[1]{\begin{adjustwidth}{2em}{0pt}\par\noindent\emph{Proof of subclaim \theClaimCounter.\theSubClaimCounter:}\space#1}{\hfill
% $\blacksquare$ \emph{Subclaim \theClaimCounter.\theSubClaimCounter}\vspace{5mm}\end{adjustwidth}}
\newenvironment{subclaimproof}[1]{\par\noindent\emph{Proof of subclaim \theClaimCounter.\theSubClaimCounter:}\space#1}{\hfill
$\Diamond$ \emph{Subclaim \theClaimCounter.\theSubClaimCounter}}

\allowdisplaybreaks{}

% chktex-file 3

\lhead{Evan P. Walsh}
\chead{MATH 521}
\rhead{\thepage}
\cfoot{}

% Custom commands.
\newcommand\toinfty{\rightarrow\infty}
\newcommand\toinf{\rightarrow\infty}
\newcommand{\sinf}[1]{\sum_{#1=0}^{\infty}}
\newcommand{\linf}[1]{\lim_{#1\rightarrow\infty}}

\begin{document}\thispagestyle{empty}
\begin{center}
  \Large \textsc{math 521 -- ASSIGNMENT III -- fall 2018} \\ 
  \vspace{5mm}
  \large Evan Pete Walsh
\end{center}


%------------------------------------------------------------------------------------------------------------------%
% Question 1
%------------------------------------------------------------------------------------------------------------------%

\subsection*{1}
\begin{tcolorbox}
  Show that the Dominated Convergence Theorem (DCT) still holds if we replace almost sure convergence with convergence in probability.
\end{tcolorbox}
\begin{Solution}
  We will use the following 2 facts from real analysis.

  \textbf{Fact 1:} If $f, f_0, f_1, \dots$ are measurable functions on a measure space $(\Omega, \mathcal{F}, \mu)$ and $f_n \rightarrow f$ in measure, then there exists a subsequence $\{ f_{n_k} \}_{k=0}^{\infty}$ of $\{ f_n \}_{n=0}^{\infty}$ that converges to $f$ almost surely.

  \textbf{Fact 2:} If $\{ x_n \}_{n=0}^{\infty}$ is a sequence of real numbers such that every subsequence has a further subsequence that converges to the same real number $x$, then $x_n \rightarrow x$.

  Now suppose $Y, X, X_0, X_1, X_2, \dots$ are random variables on a probability space $(\Omega, \mathcal{F}, P)$ such that $X_n$ converges to $X$ in probability and $|X_n| \leq |Y|$ for $n \in \mathbb{N}$, where $Y$ is integrable. We need to show that $X$ is integrable and $EX_n \rightarrow EX$.
  
  Besides showing that $X$ is in fact integrable, Fact 2 says that it suffices to show that for any subsequence $\{ X_{n_k} \}_{k=0}^{\infty}$, there exists a further subsequence $\{ X_{n_{k_j}} \}_{j=0}^{\infty}$ such that $EX_{n_{k_j}} \rightarrow EX$.

  So let $\{ X_{n_k} \}_{k=0}^{\infty}$ be any subsequence. Since $X_n \rightarrow X$ in probability, clearly $X_{n_k} \rightarrow X$ in probability as well. Hence by Fact 1, there exists a further subsequence $\{ X_{n_{k_j}} \}_{j=0}^{\infty}$ of $\{ X_{n_k} \}_{k=0}^{\infty}$ such that $X_{n_{k_j}} \rightarrow X$ almost surely.
  But since the assumptions of DCT hold for $\{ X_{n_{k_j}} \}_{j=0}^{\infty}$ (treated as its own sequence) with respect to $X$ and $Y$, we have that $X$ is integrable and $EX_{n_{k_j}} \rightarrow EX$.
\end{Solution}

\end{document}
