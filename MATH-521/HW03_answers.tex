\documentclass[12pt]{article}
\usepackage{amsmath}
\usepackage{amsfonts}
\usepackage{parskip}
\usepackage{amsthm}
\usepackage{thmtools}
\usepackage[headheight=15pt]{geometry}
\geometry{a4paper, left=20mm, right=20mm, top=30mm, bottom=30mm}
\usepackage{graphicx}
\usepackage{bm} % for bold font in math mode - command is \bm{text}
\usepackage{enumitem}
\usepackage{fancyhdr}
\usepackage{amssymb} % for stacked arrows and other shit
\pagestyle{fancy}
\usepackage{changepage}
\usepackage{mathcomp}
\usepackage{tcolorbox}

\declaretheoremstyle[headfont=\normalfont]{normal}
\declaretheorem[style=normal]{Theorem}
\declaretheorem[style=normal]{Proposition}
\declaretheorem[style=normal]{Lemma}
\newcounter{ProofCounter}
\newcounter{ClaimCounter}[ProofCounter]
\newcounter{SubClaimCounter}[ClaimCounter]
\newenvironment{Proof}{\stepcounter{ProofCounter}\textsc{Proof.}}{\hfill$\square$}
\newenvironment{Solution}{\stepcounter{ProofCounter}\textbf{Solution:}}{\hfill$\square$}
\newenvironment{claim}[1]{\vspace{1mm}\stepcounter{ClaimCounter}\par\noindent\underline{\bf Claim \theClaimCounter:}\space#1}{}
\newenvironment{claimproof}[1]{\par\noindent\underline{Proof of claim \theClaimCounter:}\space#1}{\hfill $\blacksquare$ Claim \theClaimCounter}
\newenvironment{subclaim}[1]{\stepcounter{SubClaimCounter}\par\noindent\emph{Subclaim \theClaimCounter.\theSubClaimCounter:}\space#1}{}
% \newenvironment{subclaimproof}[1]{\begin{adjustwidth}{2em}{0pt}\par\noindent\emph{Proof of subclaim \theClaimCounter.\theSubClaimCounter:}\space#1}{\hfill
% $\blacksquare$ \emph{Subclaim \theClaimCounter.\theSubClaimCounter}\vspace{5mm}\end{adjustwidth}}
\newenvironment{subclaimproof}[1]{\par\noindent\emph{Proof of subclaim \theClaimCounter.\theSubClaimCounter:}\space#1}{\hfill
$\Diamond$ \emph{Subclaim \theClaimCounter.\theSubClaimCounter}}

\allowdisplaybreaks{}

% chktex-file 3

\lhead{Evan P. Walsh}
\chead{MATH 521}
\rhead{\thepage}
\cfoot{}

% Custom commands.
\newcommand\toinfty{\rightarrow\infty}
\newcommand\toinf{\rightarrow\infty}
\newcommand{\sinf}[1]{\sum_{#1=0}^{\infty}}
\newcommand{\linf}[1]{\lim_{#1\rightarrow\infty}}

\begin{document}\thispagestyle{empty}
\begin{center}
  \Large \textsc{math 521 -- ASSIGNMENT III -- fall 2018} \\ 
  \vspace{5mm}
  \large Evan Pete Walsh
\end{center}

%------------------------------------------------------------------------------------------------------------------%
% Question 1
%------------------------------------------------------------------------------------------------------------------%

\subsection*{1}
\begin{tcolorbox}
  Suppose $X_n, n\geq 0$ are defined on the same probability space $(\Omega, \mathcal{F}, P)$. Consider the following conditions:
  \begin{enumerate}[label=(\roman*)]
    \item $X_n$ converges to $X$ a.s., as $n \rightarrow \infty$.
    \item $X_n$ converges to $X$ in probability, as $n \toinf$.
    \item $X_n$ converges to $X$ in $L^p$, as $n \toinf$.
    \item $X_n$ converges to $X$ in distribution, as $n \toinf$.
  \end{enumerate}
  For each ordered pair of conditions, determine whether the first condition in the pair implies the other.
\end{tcolorbox}
\begin{Solution}
  \begin{claim}
    (i) implies (ii).
  \end{claim}
  \begin{claimproof}
    Suppose (i). Let $A$ be the subset of $\Omega$ on which $X_n$ does not converge to $X$. Equivelently, $A$ is the set of all $\omega \in \Omega$ such that there exists $k \geq 0$, where for all $n \geq 0$ there exists $j \geq n$ such that $|X_j(\omega) - X(\omega)| > 2^{-k}$, i.e.
    \[
      A = \bigcup_{k \geq 0} \bigcap_{n \geq 0} \bigcup_{j \geq n} A_{k, j},
    \]
    where $A_{k,j} := \{ \omega \in \Omega : |X_j(\omega) - X(\omega)| > 2^{-k} \}$. By (i), $P(A) = 0$, so clearly
    \[
      \bigcap_{n \geq 0} \bigcup_{j \geq n} A_{k, j} = 0 \ \forall k \geq 0.
    \]
    Now let $B_{k, n} := \bigcup_{j \geq n} A_{k, j}$ for each $k, n \geq 0$. So for fixed $k$, $B_{k, n} \downarrow \cap_{n \geq 0} \cup_{j \geq n} A_{k, j}$, and so, by the continuity of measure from above, $P(B_{k, n}) \rightarrow 0$ for all $k$.

    Hence, if we fix $\epsilon > 0$, we can choose $k \geq 0$ such that $2^{-k} < \epsilon$.  Therefore,
    \[
      P(|X_n - X| > \epsilon) \leq P(|X_n - X| > 2^{-k}) \leq P(B_{k, n}) \longrightarrow 0
    \]
    as $n \rightarrow \infty$.
  \end{claimproof}

  \begin{claim}
    (i) does not imply (iii).
  \end{claim}
  \begin{claimproof}
    Consider the Borel measure space on $[0, 1]$, and let $X \equiv 0$ and 
    \[
      X_n := 2^{n} * \bm{1}_{[0, 2^{-n}]}.
    \]
    Then $X_n(\omega) \rightarrow X$ for all $\omega \in (0, 1]$, but $\| X_n - X \|_{1} = EX_n \equiv 1$ for all $n$.
  \end{claimproof}

\end{Solution}

%------------------------------------------------------------------------------------------------------------------%
% Question 2
%------------------------------------------------------------------------------------------------------------------%

\subsection*{2}
\begin{tcolorbox}
  Suppose that $X_n, n \geq 0$ and $X$ are defined on the same probability space $(\Omega, \mathcal{F}, P)$. For each condition (ii)-(iv) in Problem 1, determine whether the DCT and MCT hold if we replace the assumption (i) with the new assumption. Give a proof or counter-example for each one.
\end{tcolorbox}

\begin{Solution}

  %------------------------------------------------------------------------------------------------------------------%
  % (ii) => MCT/DCT
  %------------------------------------------------------------------------------------------------------------------%

  \begin{claim}
    The MCT and DCT hold with assumption (ii) instead of assumption (i).
  \end{claim}
  \begin{claimproof}
    We will use the following 2 facts from real analysis.

    \textbf{Fact 1:} If $f, f_0, f_1, \dots$ are measurable functions on a measure space $(\Omega, \mathcal{F}, \mu)$ and $f_n \rightarrow f$ in measure, then there exists a subsequence $\{ f_{n_k} \}_{k=0}^{\infty}$ of $\{ f_n \}_{n=0}^{\infty}$ that converges to $f$ almost surely.

    \textbf{Fact 2:} If $\{ x_n \}_{n=0}^{\infty}$ is a sequence of real numbers such that every subsequence has a further subsequence that converges to the same (extended) real number $x$, then $x_n \rightarrow x$.

    \begin{subclaim}
      The MCT holds.
    \end{subclaim}
    \begin{subclaimproof}
      Suppose $0 \leq X_0 \leq X_1 \leq \dots \leq X$ a.s. and $X_n$ converges to $X$ in distribution. We need to show that $EX_n \rightarrow EX$. Since $EX_n$ is just a sequence of real numbers, it suffices to show that for any subsequence $\{ X_{n_k} \}_{k=0}^{\infty}$, there exists a further subsequence $\{ X_{n_{k_j}} \}_{j=0}^{\infty}$ so that $EX_{n_{k_j}} \rightarrow EX$.

      To that end, let $\{ X_{n_k} \}_{k=0}^{\infty}$ be any subsequence. Since $X_n \rightarrow X$ in probability, clearly $X_{n_k} \rightarrow X$ in probability as well. Hence by Fact 1, there exists a further subsequence $\{ X_{n_{k_j}} \}_{j=0}^{\infty}$ of $\{ X_{n_k} \}_{k=0}^{\infty}$ such that $X_{n_{k_j}} \rightarrow X$ almost surely.
      So since the assumptions of MCT hold for $\{ X_{n_{k_j}} \}_{j=0}^{\infty}$ (treated as its own sequence) with respect to $X$, we have that $EX_{n_{k_j}} \rightarrow EX$.
    \end{subclaimproof}

    \begin{subclaim}
      The DCT holds.
    \end{subclaim}
    \begin{subclaimproof}
      Suppose $X_n$ converges to $X$ in probability and $|X_n| \leq Y$ for $n \in \mathbb{N}$, where $Y \geq 0$ is integrable. We need to show that $X$ is integrable and $EX_n \rightarrow EX$.

      Besides showing that $X$ is in fact integrable, similar to Subclaim 1.1, it suffices to show that for any subsequence $\{ X_{n_k} \}_{k=0}^{\infty}$, there exists a further subsequence $\{ X_{n_{k_j}} \}_{j=0}^{\infty}$ such that $EX_{n_{k_j}} \rightarrow EX$.

      So, as in Subclaim 1.1, let $\{ X_{n_k} \}_{k=0}^{\infty}$ be any subsequence. Since $X_n \rightarrow X$ in probability, $X_{n_k} \rightarrow X$ in probability as well. Thus, by Fact 1, there exists a further subsequence $\{ X_{n_{k_j}} \}_{j=0}^{\infty}$ of $\{ X_{n_k} \}_{k=0}^{\infty}$ such that $X_{n_{k_j}} \rightarrow X$ almost surely.
      But since the assumptions of DCT hold for $\{ X_{n_{k_j}} \}_{j=0}^{\infty}$ with respect to $X$ and $Y$, we have that $X$ is integrable and $EX_{n_{k_j}} \rightarrow EX$.

    \end{subclaimproof}

  \end{claimproof}

  %------------------------------------------------------------------------------------------------------------------%
  % (iii) => MCT/DCT
  %------------------------------------------------------------------------------------------------------------------%<`0`>

  \begin{claim}
    The MCT and DCT hold with assumption (iii) instead of assumption (i).
  \end{claim}
  \begin{claimproof}
    Suppose that for some $p \in [1, \infty)$, $X_n$ converges to $X$ in $\mathcal{L}^{p}$. Then of course $X_n$ converges to $X$ in $L^{1}$, i.e.
    \begin{equation}
      E|X_n - X| \rightarrow 0, \ n \rightarrow \infty.
      \label{2.1}
    \end{equation}
    \begin{subclaim}
      The MCT holds.
    \end{subclaim}
    \begin{subclaimproof}
      Suppose $0 \leq X_0 \leq X_1 \leq \dots \leq X$ almost surely. Consider first the case where all $X_n$ are integrable. Then for any $n \in \mathbb{N}$,
      \[
        E|X| = E|X - X_n + X_n| \leq E|X - X_n| + E|X_n| < \infty.
      \]
      Hence $X$ is integrable, and so
      \[
        \limsup_{n\rightarrow \infty} \bigg| EX_n - EX \bigg| \leq \limsup_{n\rightarrow\infty} E|X_n - X| = 0,
      \]
      that is, $\linf{n} EX_n = EX$. For the other case, suppose for some $n' \in \mathbb{N}$ that $X_{n'}$ is not integrable. By monotonicity of integration,
      $X$ and $X_k$, for $k \geq n'$, are therefor not integrable as well. Then
      \[
        \linf{n} EX_n = \infty = EX.
      \]
      So in both cases the MCT holds.
    \end{subclaimproof}

    \begin{subclaim}
      The DCT holds.
    \end{subclaim}
    \begin{subclaimproof}
      Suppose there exists an integrable $Y \geq 0$ such that $|X_n| \leq Y$ a.s. for all $n \in \mathbb{N}$. Then,
      \[
        E|X| = E|X - X_n + X_n| \leq E|X - X_n| + E|X_n| \leq E|X - X_n| + E|Y| < \infty.
      \]
      Hence $X$ is integrable, and so
      \[
        \limsup_{n\rightarrow \infty} \bigg| EX_n - EX \bigg| \leq \limsup_{n\rightarrow\infty} E|X_n - X| = 0.
      \]
      Therefore $\linf{n} EX_n = EX$.
    \end{subclaimproof}

  \end{claimproof}

  %------------------------------------------------------------------------------------------------------------------%
  % (iv) => MCT/DCT
  %------------------------------------------------------------------------------------------------------------------%<`0`>

  \begin{claim}
    The MCT and DCT hold with assumption (iv) instead of assumption (i).
  \end{claim}
  \begin{claimproof}
    We will make use of Skorohod's embedding theorem, which says that if $X_n \rightarrow X$ in distribution, then there exists random variables $X_0', X_1', \dots$ and $X'$ defined on a common probability space such that each $X_n'$ has the same distribution as $X_n$, $X'$ has the same distribution as $X$, and $X_n' \rightarrow X'$ almost surely.

    \begin{subclaim}
      The MCT holds.
    \end{subclaim}
    \begin{subclaimproof}
      Suppose $0 \leq X_0 \leq X_1 \leq \dots \leq X$ almost surely. By the monotonicity of integration, $EX_n \leq EX$ for all $n \in \mathbb{N}$.
      Hence,
      \[
        \limsup_{n\rightarrow \infty} EX_n \leq EX.
      \]
      Further, by Fatou's Lemma,
      \[
        \liminf_{n\rightarrow \infty} EX_n  = \liminf_{n\rightarrow \infty} EX_n' \geq E[ \liminf_{n\rightarrow \infty} X_n' ] = EX' = EX.
      \]
      Thus $\linf{n} EX_n = EX$.
    \end{subclaimproof}

    \begin{subclaim}
      The DCT holds.
    \end{subclaim}
    \begin{subclaimproof}
      Suppose there exists an integrable $Y \geq 0$ such that $|X_n| \leq Y$ almost surely for all $n \in \mathbb{N}$. Then by Fatou's Lemma,
      \[
        E|X| = E|X'| \leq \liminf_{n\rightarrow \infty} E|X_n'| = \liminf_{n\rightarrow \infty} E|X_n| \leq E|Y| < \infty,
      \]
      so $X$ and $X'$ are integrable. Thus, since
      \[
        |EX_n - EX| = |EX_n' - EX'| \leq E|X_n' - X'|,
      \]
      it suffices to show that $E|X_n' - X'| \rightarrow 0$ as $n \rightarrow \infty$. To that end, note that
      \begin{align*}
        \lim_{t \rightarrow \infty} \sup_{n} E\left( |X_n' - X'| \cdot \bm{1}_{|X_n' - X'| > t} \right) & \leq \lim_{t \rightarrow \infty} \sup_{n} E\left( Y \cdot \bm{1}_{Y > t}\right) \\
          & = \lim_{t \rightarrow \infty} E\left( Y \cdot \bm{1}_{Y > t}\right) \\
          & = 0
      \end{align*}
      by the DCT, since $Y \geq Y \cdot \bm{1}_{Y > t} \rightarrow 0$ almost surely. Now let $\epsilon > 0$. Then the inequality above tells us that there exists $t > 0$ such that
      \begin{equation}
        \sup_{n}E\left( |X_n' - X'| \cdot \bm{1}_{|X_n' - X'| > t} \right) \leq \epsilon.
        \label{2.2}
      \end{equation}
      Further, by the DCT again,
      \begin{equation}
        \lim_{n\rightarrow\infty} E\left(|X_n' - X'| \cdot \bm{1}_{|X_n' - X'| \leq t} \right) = 0
        \label{2.3}
      \end{equation}
      since $|X_n' - X'| \cdot \bm{1}_{|X_n' - X'| \leq t}$ is bounded above by $t$.
      Therefore, we have
      \begin{align*}
        \limsup_{n\toinf} E|X_n' - X'| & = \limsup_{n\toinf} \left[ E\left(|X_n' - X'| \cdot \bm{1}_{|X_n' - X'| > t} \right) + E\left(|X_n' - X'| \cdot \bm{1}_{|X_n' - X'| \leq t} \right)\right] \\
        & \leq \sup_{n}E\left( |X_n' - X'| \cdot \bm{1}_{|X_n' - X'| > t} \right) + \limsup_{n\toinf}E\left(|X_n' - X'| \cdot \bm{1}_{|X_n' - X'| \leq t} \right) \\
        & \leq \epsilon.
      \end{align*}
      Since $\epsilon > 0$ was arbitrary,
      \[
        \limsup_{n\toinf} E|X_n' - X'| = 0.
      \]
    \end{subclaimproof}
    
  \end{claimproof}

\end{Solution}

\end{document}
