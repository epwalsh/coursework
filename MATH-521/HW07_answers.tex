\documentclass[12pt]{article}
\usepackage{amsmath}
\usepackage{amsfonts}
\usepackage{parskip}
\usepackage{amsthm}
\usepackage{thmtools}
\usepackage[headheight=15pt]{geometry}
\geometry{a4paper, left=20mm, right=20mm, top=30mm, bottom=30mm}
\usepackage{graphicx}
\usepackage{bm} % for bold font in math mode - command is \bm{text}
\usepackage{enumitem}
\usepackage{fancyhdr}
\usepackage{amssymb} % for stacked arrows and other shit
\pagestyle{fancy}
\usepackage{changepage}
\usepackage{mathcomp}
\usepackage{tcolorbox}

\declaretheoremstyle[]{normal}
\declaretheorem[style=normal]{Theorem}
\declaretheorem[style=normal]{Proposition}
\declaretheorem[style=normal]{Lemma}
\newcounter{ProofCounter}
\newcounter{ClaimCounter}[ProofCounter]
\newcounter{SubClaimCounter}[ClaimCounter]
\newenvironment{Proof}{\stepcounter{ProofCounter}\textsc{Proof.}}{\hfill$\square$}
\newenvironment{Solution}{\stepcounter{ProofCounter}\textbf{Solution:}}{\hfill$\square$}
\newenvironment{claim}[1]{\vspace{1mm}\stepcounter{ClaimCounter}\par\noindent\underline{\bf Claim \theClaimCounter:}\space#1}{}
\newenvironment{claimproof}[1]{\par\noindent\underline{Proof of claim \theClaimCounter:}\space#1}{\hfill $\blacksquare$ Claim \theClaimCounter \\}
\newenvironment{subclaim}[1]{\stepcounter{SubClaimCounter}\par\noindent\emph{Subclaim \theClaimCounter.\theSubClaimCounter:}\space#1}{}
% \newenvironment{subclaimproof}[1]{\begin{adjustwidth}{2em}{0pt}\par\noindent\emph{Proof of subclaim \theClaimCounter.\theSubClaimCounter:}\space#1}{\hfill
% $\blacksquare$ \emph{Subclaim \theClaimCounter.\theSubClaimCounter}\vspace{5mm}\end{adjustwidth}}
\newenvironment{subclaimproof}[1]{\par\noindent\emph{Proof of subclaim \theClaimCounter.\theSubClaimCounter:}\space#1}{\hfill
$\Diamond$ \emph{Subclaim \theClaimCounter.\theSubClaimCounter \\}}

\allowdisplaybreaks{}

% chktex-file 3

\lhead{Evan P. Walsh}
\chead{MATH 521}
\rhead{\thepage}
\cfoot{}

% Custom commands.
\newcommand\toinfty{\rightarrow\infty}
\newcommand\toinf{\rightarrow\infty}
\newcommand{\sinf}[1]{\sum_{#1=0}^{\infty}}
\newcommand{\linf}[1]{\lim_{#1\rightarrow\infty}}

\begin{document}\thispagestyle{empty}
\begin{center}
  \Large \textsc{math 521 -- ASSIGNMENT VII -- fall 2018} \\ 
  \vspace{5mm}
  \large Evan Pete Walsh
\end{center}

%------------------------------------------------------------------------------------------------------------------%
% Question 1
%------------------------------------------------------------------------------------------------------------------%

\subsection*{1}
\begin{tcolorbox}
  Suppose $\{ B_t, t \geq 0\}$ is Brownian motion. Fix $a > 0$. Let $T_a := \inf\{ s \geq 0 : B_s > a\}$. Find an explicit formula for the density of $T_a$.
\end{tcolorbox}
\begin{Solution}
  Let $\Phi$ and $\varphi$ denote the distribution and density functions of a standard normal random variable, respectively. By the lemma from class, we have
  \begin{equation}
    P(T_a \leq t) = 2P(B_t > a), \ t > 0.
    \label{1.1}
  \end{equation}
  Therefore
  \begin{align*}
    P(T_a \leq t) = 2[1 - P(B_t \leq a)] & = 2 \left[ 1 - P \left( \frac{B_t}{\sqrt{t}} \leq \frac{a}{\sqrt{t}} \right) \right] = 2[1 - \Phi(a/\sqrt{t})].
  \end{align*}
  Thus,
  \begin{align*}
    \frac{d}{dt} P(T_a \leq t) = \frac{d}{dt} 2[1 - \Phi(a/\sqrt{t})] & = \varphi(a/\sqrt{t}) \frac{a}{t^{3/2}} = \frac{a}{\sqrt{2 \pi t^3}} e^{-a^2 / 2t}.
  \end{align*}
  So the density of $T_a$ is
  \[
    \frac{a}{\sqrt{2 \pi t^3}} e^{-a^2 / 2t}, \ t \geq 0.
  \]
\end{Solution}

%------------------------------------------------------------------------------------------------------------------%
% Question 2
%------------------------------------------------------------------------------------------------------------------%

\subsection*{2}
\begin{tcolorbox}
  \begin{enumerate}[label=(\roman*)]
      \item Prove that the local Law of Iterated Logarithms holds at every fixed $s \geq 0$, that is, for every fixed $s \geq 0$,
      \[
        \limsup_{h\downarrow 0} \frac{B(s + h) - B(s)}{\sqrt{2h\log|\log h|}} = 1, \ \ \text{a.s.}
      \]
    \item Prove that for almost every $\omega$, the set of times $s \geq 0$ such that
      \[
        \limsup_{h\downarrow 0} \frac{B(s + h) - B(s)}{\sqrt{2h\log|\log h|}} \neq 1
      \]
      has Lebesgue measure 0.
    \item Prove that for almost every $\omega$, the set of times $s \geq 0$ such that
      \[
        \limsup_{h\downarrow 0} \frac{B(s + h) - B(s)}{\sqrt{2h\log|\log h|}} \neq 1
      \]
      is dense in $[0, \infty)$.
  \end{enumerate}
\end{tcolorbox}
\begin{Solution}

  (i) Fix $s \geq 0$. Then let $Y(t) = B(s + t) - B(s)$, $t \geq 0$. So $ \left\{ Y_t, t \geq 0 \right\} $ is BM. Then let $X(t) = tY(1/t)$. So by invariance under time inversion, $ \left\{ X_t, t \geq 0 \right\}$ is also BM. Thus,
  \begin{align*}
  \limsup_{h\downarrow 0} \frac{B(s + h) - B(s)}{\sqrt{2h \log|\log h|}} = \limsup_{h\downarrow 0} \frac{Y(h)}{\sqrt{2h\log\log(1/h)}} & = \limsup_{t\rightarrow\infty} \frac{Y(1/t)}{\sqrt{2(1/t)\log\log t}} \\
  & = \limsup_{t\rightarrow\infty} \frac{tY(1/t)}{\sqrt{2t\log\log t}} \\
  & = \limsup_{t\toinf} \frac{X(t)}{\sqrt{2t\log\log t}} \\
  & = 1,
  \end{align*}
  by the Law of Iterated Logarithms.

  (ii) Let
  \[
    f_s(\omega) = \begin{cases}{}
      1 & \text{ if } \limsup_{h\downarrow 0} \frac{B(s + h) - B(s)}{\sqrt{2h\log|\log h|}} \neq 1 \\
      0 & \text{ otherwise }
    \end{cases} \ \ \  \ \ \text{for} \ s \geq 0, \omega \in \Omega.
  \]
  For fixed $s \geq 0$, $f_s$ is measurable with respect to $(\Omega, \mathcal{F}, P)$ since it based on the limit of measurable functions. Similarly, for fixed $\omega \in \Omega$, $f_s(\omega)$ is measurable with respect to Lebesgue measure since it based on the limit of continuous functions. Thus $f_s(\omega)$ is measurable with respect to the product measure of $P$ and Lebesgue measure on $[0, \infty)$. Hence all of the integrals below are well-defined.

  Now, by part (i), $f_s = 0$ almost surely $(P)$ for each $s \geq 0$. Therefore $Ef_s = 0$, $s \geq 0$. Hence
  \[
    \int_{0}^{\infty} (Ef_s) ds = 0.
  \]
  So by Fubini's / Tonelli's theorem,
  \[
    E \left( \int_{0}^{\infty} f_s ds \right) = 0.
  \]
  But this implies that for almost all $\omega \in \Omega$,
  \[
    \int_{0}^{\infty} f_s(\omega) ds = 0,
  \]
  since $f_s$ is non-negative. Thus, for almost all $\omega \in \Omega$ (with respect to $P$), $f_s = 0$ for almost all $s \geq 0$ (with respect to the Lebesgue measure).
  In other words, for almost every $\omega$, the set of times $s \geq 0$ such that
  \begin{equation}
    \limsup_{h\downarrow 0} \frac{B(s + h) - B(s)}{\sqrt{2h\log|\log h|}} \neq 1
    \label{2.1}
  \end{equation}
  has Lebesgue measure 0.

  (iii) As a Corollary to part (ii), we can say that for almost every $\omega$, the set of times $s \geq 0$ such that
  \begin{equation}
    \liminf_{h\downarrow 0} \frac{B(s + h) - B(s)}{\sqrt{2h\log|\log h|}} \neq -1
    \label{2.2}
  \end{equation}
  has Lebesgue measure 0. Now let $E$ be the set of all $\omega$ such that the set of times $s \geq 0$ for which both \eqref{2.1} or \eqref{2.2} holds has Lebesgue measure 0. So $E$ has probability 1, and for all $\omega \in E$, the local LIL for both the $\limsup$ and $\liminf$ hold at almost every time. We will use this to show that for all $\omega \in E$, the set of times such that \eqref{2.1} holds is dense.

  Let $\omega \in E$ and $0 \leq s_1 < s_2$. It suffices to show that we can find $s^* \in (s_1, s_2)$ such that \eqref{2.1} holds (with respect to the given $\omega$). Well, by our construction of $E$, we can find $s_0 \in (s_1, s_2)$ such that
  \[
    \limsup_{h\downarrow 0} \frac{B(s_0 + h) - B(s_0)}{\sqrt{2h\log|\log h|}} = 1 \ \text{ and } \ \liminf_{h\downarrow 0} \frac{B(s_0 + h) - B(s_0)}{\sqrt{2h\log|\log h|}} = -1.
  \]
  So at $s_0$, the local BM process $\{ B(s_0 + h) - B(s_0), h \geq 0 \}$ crosses 0 infinitely often as $h \rightarrow 0$. Then let $h_0 > 0$ such that $s_0 + h_0 < s_2$ and $B(s_0 + h_0) - B(s_0) = 0$, and define
  \[
    s^* := s_0 + \text{argmax}_{0\leq h \leq h_0} \{ B(s_0 + h) - B(s_0) \}.
  \]
  That is, $s^*$ is the maximal point of $B(t)$ for $t \in [s_0, s_0 + h_0]$, and $B(s^*) > B(s_0 + h_0)$. Hence, for small $h$ such that $s^* + h < s_0 + h_0$, we have
  $B(s^*) \geq B(s^* + h)$, and so
  \[
    \limsup_{h\downarrow 0} \frac{B(s^* + h) - B(s^*)}{\sqrt{2h\log|\log h|}} \leq 0.
  \]
  Therefore \eqref{2.1} holds at $s^* \in (s_1, s_2)$, so we are done.
\end{Solution}

\end{document}
