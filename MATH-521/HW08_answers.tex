\documentclass[12pt]{article}
\usepackage{amsmath}
\usepackage{amsfonts}
\usepackage{parskip}
\usepackage{amsthm}
\usepackage{thmtools}
\usepackage[headheight=15pt]{geometry}
\geometry{a4paper, left=20mm, right=20mm, top=30mm, bottom=30mm}
\usepackage{graphicx}
\usepackage{bm} % for bold font in math mode - command is \bm{text}
\usepackage{enumitem}
\usepackage{fancyhdr}
\usepackage{amssymb} % for stacked arrows and other shit
\pagestyle{fancy}
\usepackage{changepage}
\usepackage{mathcomp}
\usepackage{tcolorbox}

\declaretheoremstyle[]{normal}
\declaretheorem[style=normal]{Theorem}
\declaretheorem[style=normal]{Proposition}
\declaretheorem[style=normal]{Lemma}
\newcounter{ProofCounter}
\newcounter{ClaimCounter}[ProofCounter]
\newcounter{SubClaimCounter}[ClaimCounter]
\newenvironment{Proof}{\stepcounter{ProofCounter}\textsc{Proof.}}{\hfill$\square$}
\newenvironment{Solution}{\stepcounter{ProofCounter}\textbf{Solution:}}{\hfill$\square$}
\newenvironment{claim}[1]{\vspace{1mm}\stepcounter{ClaimCounter}\par\noindent\underline{\bf Claim \theClaimCounter:}\space#1}{}
\newenvironment{claimproof}[1]{\par\noindent\underline{Proof of claim \theClaimCounter:}\space#1}{\hfill $\blacksquare$ Claim \theClaimCounter \\}
\newenvironment{subclaim}[1]{\stepcounter{SubClaimCounter}\par\noindent\emph{Subclaim \theClaimCounter.\theSubClaimCounter:}\space#1}{}
% \newenvironment{subclaimproof}[1]{\begin{adjustwidth}{2em}{0pt}\par\noindent\emph{Proof of subclaim \theClaimCounter.\theSubClaimCounter:}\space#1}{\hfill
% $\blacksquare$ \emph{Subclaim \theClaimCounter.\theSubClaimCounter}\vspace{5mm}\end{adjustwidth}}
\newenvironment{subclaimproof}[1]{\par\noindent\emph{Proof of subclaim \theClaimCounter.\theSubClaimCounter:}\space#1}{\hfill
$\Diamond$ \emph{Subclaim \theClaimCounter.\theSubClaimCounter \\}}

\allowdisplaybreaks{}

% chktex-file 3

\lhead{Evan P. Walsh}
\chead{MATH 521}
\rhead{\thepage}
\cfoot{}

% Custom commands.
\newcommand\toinfty{\rightarrow\infty}
\newcommand\toinf{\rightarrow\infty}
\newcommand{\sinf}[1]{\sum_{#1=0}^{\infty}}
\newcommand{\linf}[1]{\lim_{#1\rightarrow\infty}}

\begin{document}\thispagestyle{empty}
\begin{center}
  \Large \textsc{math 521 -- ASSIGNMENT VII -- fall 2018} \\ 
  \vspace{5mm}
  \large Evan Pete Walsh
\end{center}

%------------------------------------------------------------------------------------------------------------------%
% Question 1
%------------------------------------------------------------------------------------------------------------------%

\subsection*{1}
\begin{tcolorbox}
  Suppose $ \left\{ B_t, t \geq 0 \right\} $ is Brownian motion. Let
  \[
    T := \inf \left\{ t \geq 0 : |B_t| \geq 1 \right\}.
  \]
  Find $c  < \infty$ and $q < 1$ such that $P(T \geq t) \leq cq^t$, for all $t \geq 0$.
\end{tcolorbox}
\begin{Solution}

  \textbf{Claim:} $P(T \geq t) \leq 2 q^t$, where
  \[
    q := \frac{2}{\sqrt{2\pi}} \int_{1}^{\infty}  s^{-3/2} e^{-2/s}ds.
  \]

  Let $t \geq 0$. For simplicity, first assume $t = n$ for some integer $n \geq 1$. Then
  \begin{align}
    P(T \geq n) & = \int_{-1}^{1} P(T \geq n, B(n-1) = x, T \geq n-1) dx \nonumber \\
    & = \int_{-1}^{1} P(T \geq n | B(n-1) = x, T \geq n-1) P(B(n-1) = x, T \geq n-1) dx.
    \label{1.1}
  \end{align}
  Now, by the strong Markov property,
  \begin{equation}
    P(T \geq n | B(n-1) = x, T \geq n-1) = P(T \geq n | B(n-1) = x), \ \ x \in (-1, 1).
    \label{1.2}
  \end{equation}
  Further, if we let $T_a := \inf \left\{ t \geq 0 : B_t = a \right\}$, then for $x \in (-1, 1)$,
  \begin{align}
    P(T \geq n | B(n-1) = x) & = P( \text{min}\{T_{-1}, T_1 \} \geq n | B(n-1) = x)  \nonumber \\
    & = P(T_{-1} \geq n, T_1 \geq n | B(n-1) = x) \nonumber \\
    & \leq P(T_1 \geq n | B(n-1) = x) \nonumber \\
    & \leq P(T_1 \geq n | B(n-1) = -1) \nonumber \\
    & = P(T_2 \geq n | B(n-1) = 0) \nonumber \\
    & = P(T_2 \geq 1).
    \label{1.3}
  \end{align}
  So by \eqref{1.1}, \eqref{1.2}, and \eqref{1.3}, we have
  \begin{align}
    P(T \geq n) & = \int_{-1}^{1} P(T_2 \geq 1) P(B(n-1) = x, T \geq n-1) dx \nonumber \\
    & = P(T_2 \geq 1) \int_{-1}^{1} P(B(n-1) = x, T \geq n-1) dx \nonumber \\
    & = P(T_2 \geq 1) P(T \geq n-1).
    \label{1.4}
  \end{align}
  Therefore, by induction, we have
  \begin{equation}
    P(T \geq n) = [P(T_2 \geq 1)]^n.
    \label{1.5}
  \end{equation}
  However, by the last homework, $P(T_2 \geq 1) = q$. So all together, we have
  \begin{equation}
    P(T \geq n) = q^n.
    \label{1.5b}
  \end{equation}
  Now take any $t \geq 0$. Let $u = t - \lfloor t \rfloor$. By similar steps to above, we can write
  \begin{align}
    P(T \geq t) & = \int_{-1}^{1} P(T \geq t | B(\lfloor t \rfloor) = x ) P(B(\lfloor t \rfloor ) = x, T \geq \lfloor t \rfloor ) dx \nonumber \\
    & \leq P(T_2 \geq u) \int_{-1}^{1} P(B(\lfloor t \rfloor ) = x, T \geq \lfloor t \rfloor ) dx \nonumber \\
    & = P(T_2 \geq u) P(T \geq \lfloor t \rfloor ) \nonumber \\
    \text{(by \eqref{1.5b})} \ \ \ & = P(T_2 \geq u) q^{\lfloor t \rfloor}.
    \label{1.6}
  \end{align}
  Further, by computational methods we can see that $q \approx 0.95$, and so it certainly holds that
  \begin{equation}
    P(T_2 \geq u) \leq 1 < 2 q^u.
    \label{1.7}
  \end{equation}
  So by \eqref{1.6} and \eqref{1.7}, we have
  \begin{equation}
    P(T \geq  t) \leq 2q^{u}q^{\lfloor t \rfloor} = 2q^{t}.
    \label{1.8}
  \end{equation}
  Hence we are done.
\end{Solution}

\end{document}
