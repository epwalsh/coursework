\documentclass[12pt]{article}
\usepackage{amsmath}
\usepackage{amsfonts}
\usepackage{parskip}
\usepackage{amsthm}
\usepackage{thmtools}
\usepackage[headheight=15pt]{geometry}
\geometry{a4paper, left=20mm, right=20mm, top=30mm, bottom=30mm}
\usepackage{graphicx}
\usepackage{bm} % for bold font in math mode - command is \bm{text}
\usepackage{enumitem}
\usepackage{fancyhdr}
\usepackage{amssymb} % for stacked arrows and other shit
\pagestyle{fancy}
\usepackage{changepage}
\usepackage{mathcomp}
\usepackage{tcolorbox}

\declaretheoremstyle[headfont=\normalfont]{normal}
\declaretheorem[style=normal]{Theorem}
\declaretheorem[style=normal]{Proposition}
\declaretheorem[style=normal]{Lemma}
\newcounter{ProofCounter}
\newcounter{ClaimCounter}[ProofCounter]
\newcounter{SubClaimCounter}[ClaimCounter]
\newenvironment{Proof}{\stepcounter{ProofCounter}\textsc{Proof.}}{\hfill$\square$}
\newenvironment{Solution}{\stepcounter{ProofCounter}\textbf{Solution:}}{\hfill$\square$}
\newenvironment{claim}[1]{\vspace{1mm}\stepcounter{ClaimCounter}\par\noindent\underline{\bf Claim \theClaimCounter:}\space#1}{}
\newenvironment{claimproof}[1]{\par\noindent\underline{Proof of claim \theClaimCounter:}\space#1}{\hfill $\blacksquare$ Claim \theClaimCounter}
\newenvironment{subclaim}[1]{\stepcounter{SubClaimCounter}\par\noindent\emph{Subclaim \theClaimCounter.\theSubClaimCounter:}\space#1}{}
% \newenvironment{subclaimproof}[1]{\begin{adjustwidth}{2em}{0pt}\par\noindent\emph{Proof of subclaim \theClaimCounter.\theSubClaimCounter:}\space#1}{\hfill
% $\blacksquare$ \emph{Subclaim \theClaimCounter.\theSubClaimCounter}\vspace{5mm}\end{adjustwidth}}
\newenvironment{subclaimproof}[1]{\par\noindent\emph{Proof of subclaim \theClaimCounter.\theSubClaimCounter:}\space#1}{\hfill
$\Diamond$ \emph{Subclaim \theClaimCounter.\theSubClaimCounter}}

\allowdisplaybreaks{}

% chktex-file 3

\lhead{Evan P. Walsh}
\chead{MATH 521}
\rhead{\thepage}
\cfoot{}

% Custom commands.
\newcommand\toinfty{\rightarrow\infty}
\newcommand\toinf{\rightarrow\infty}
\newcommand{\sinf}[1]{\sum_{#1=0}^{\infty}}
\newcommand{\linf}[1]{\lim_{#1\rightarrow\infty}}

\begin{document}\thispagestyle{empty}
\begin{center}
  \Large \textsc{math 521 -- ASSIGNMENT IV -- fall 2018} \\ 
  \vspace{5mm}
  \large Evan Pete Walsh
\end{center}

%------------------------------------------------------------------------------------------------------------------%
% Question 1
%------------------------------------------------------------------------------------------------------------------%

\subsection*{1}
\begin{tcolorbox}
  Find events A, B, and C which are pairwise independent but not jointly independent.
\end{tcolorbox}
\begin{Solution}
  Suppose $X_1, X_2$ and $X_3$ are independent random variables such that $P(X_i = 1) = P(X_i = 0) = \frac{1}{2}$ for $i = 1, 2, 3$. Define the events $A = \{ \omega : X_1(\omega) = X_2(\omega) \}$, $B = \{ \omega : X_1(\omega) = X_3(\omega) \}$, and $C = \{ \omega : X_2(\omega) = X_3(\omega) \}$. So
  \begin{align*}
    P(A) = P(B) = P(C) = P(X_i = X_j) & = P(X_i = X_j = 0) + P(X_i = X_j = 1) \\
    & = \left( \frac{1}{2} \right)^2 + \left( \frac{1}{2} \right)^2 = \frac{1}{2},
  \end{align*}
  for $i \neq j \in \{ 1, 2, 3 \}$. Now,
  \begin{align*}
    P(A \cap B) = P((X_1 = X_2) \cap (X_1 = X_3)) & = P(X_1 = X_2 = X_3) \\
    & = P(X_1 = X_2 = X_3 = 0) + P(X_1 = X_2 = X_3 = 1) \\
    & = \left( \frac{1}{2} \right)^3 + \left( \frac{1}{2} \right)^3 \\
    & = \frac{1}{4} \\
    & = P(A)P(B).
  \end{align*}
  Similarly, $P(A \cap C) = \frac{1}{4} = P(A)P(C)$, and $P(B\cap C) = \frac{1}{4} = P(B)P(C)$. Thus, the events A, B, and C are pairwise independent. However,
  \[
    P(A \cap B \cap C) = P(X_1 = X_2 = X_3) = \frac{1}{4} \neq \frac{1}{8} = P(A)P(B)P(C).
  \]
  Hence the events are not jointly independent.
\end{Solution}

%------------------------------------------------------------------------------------------------------------------%
% Question 2
%------------------------------------------------------------------------------------------------------------------%

\subsection*{2}
\begin{tcolorbox}
  For each $n \geq 3$, find a family of random variables $X_1, X_2, \dots, X_n$ such that every subfamily of size $n - 1$ or smaller is jointly independent, but the whole family is not.
\end{tcolorbox}
\begin{Solution}
  Let $n\geq 3$ and $X_1, X_2, \dots, X_{n-1}$ be jointly independent random variables on some probability space such that $P(X_i = -1) = P(X_i = 1) = \frac{1}{2}$ for eac h $ i \in \{1, \dots, n-1 \}$, and define $X_n := \prod_{i=1}^{n-1}X_i$.

  \begin{claim}
    $P(X_n = -1) = P(X_n = 1) = \frac{1}{2}$.
  \end{claim}
  \begin{claimproof}
    We will proceed by induction on $n$. If $n = 1$, it is trivial. Now suppose the claim holds for $k \geq 1$. Let $n = k + 1$. By the inductive hypothesis,
    \[
      P \left( \prod_{i=1}^{n-2}X_i = -1 \right) = P \left( \prod_{i=1}^{n-2}X_i = 1 \right) = \frac{1}{2}.
    \]
    Therefore,
    \begin{align*}
      P(X_n = -1) & = P \left( \prod_{i=1}^{n-2}X_i = -1, X_{n-1} = 1 \right) + P \left( \prod_{i=1}^{n-2}X_i = 1, X_{n-1} = -1 \right) \\
      & = P \left( \prod_{i=1}^{n-2}X_i = -1 \right)P(X_{n-1} = 1) + P \left( \prod_{i=1}^{n-2}X_i = 1 \right)P(X_{n-1} = -1) \\
      & = \frac{1}{2}.
    \end{align*}
    By symmetry, $P(X_n = 1) = \frac{1}{2}$ as well.
  \end{claimproof}

  \begin{claim}
    Any subfamily $\{ X_{i_1}, \dots, X_{i_{n-1}} \}$ of size $n - 1$ is jointly independent.
  \end{claim}
  \begin{claimproof}
    Let $\{ X_{i_1}, \dots, X_{i_{n-1}} \}$ be any subfamily of size $n - 1$. Note that if $X_n$ is not a member of the subfamily, then the claim is trivial. So without loss of generality assume the subfamily is $\{ X_2, \dots, X_{n-1}, X_n \}$. Then it suffices to show that for any $\{ a_2, \dots, a_{n} \} \in \{0,1\}^{n-1}$,
    \begin{equation}
      P(X_2 = a_2, \dots, X_{n-1} = a_{n-1}, X_n = a_n) = P(X_2 = a_2) \cdots P(X_{n-1} = a_{n-1}) P(X_n = a_n).
      \label{2.1}
    \end{equation}
    Well, if $\text{sign}(\prod_{i=2}^{n-1}a_i) = \text{sign}(a_n)$, let $a_1 = 1$ and otherwise let $a_1 = -1$. Then 
    \[
      P(X_n = a_n | X_2 = a_2, \dots , X_{n-1} = a_{n-1}) = P(X_1 = a_1),
    \]
    so
    \begin{align*}
      P(X_2 = a_2, \dots, X_{n-1} = a_{n-1}, X_n = a_n) & = P(X_n = a_n | X_2 = a_2, \dots , X_{n-1} = a_{n-1})\cdot \\
      & \ \ \ \ \ \ \ \ \ P(X_2 = a_2, \dots , X_{n-1} = a_{n-1}) \\
      & = P(X_1 = a_1)P(X_2 = a_2, \dots, X_{n-1} = a_{n-1}) \\
      & = P(X_1 = a_1)P(X_2 = a_2) \dots P(X_{n-1} = a_{n-1}) \\
      & = \left(\frac{1}{2}\right)^{n-1} \\
      & = P(X_2 = a_2) \cdots P(X_{n-1} = a_{n-1}) P(X_n = a_n),
    \end{align*}
    by the joint independence of $X_2, \dots, X_{n-1}$ and Claim 1. Thus \eqref{2.1} holds.
  \end{claimproof}

  \begin{claim}
    The family $\{ X_1, \dots, X_n \}$ is not jointly independent.
  \end{claim}
  \begin{claimproof}
    Note that
    \[
      P(X_1 = 1, \dots, X_{n-1} = 1, X_n = -1) = 0 \neq \left(\frac{1}{2}\right)^{n} = P(X_1 = 1) \dots P(X_{n-1}=1)P(X_n = -1).
    \]
  \end{claimproof}

\end{Solution}

\end{document}
