\documentclass[12pt]{article}
\usepackage{amsmath}
\usepackage{amsfonts}
\usepackage{parskip}
\usepackage{amsthm}
\usepackage{thmtools}
\usepackage[headheight=15pt]{geometry}
\geometry{a4paper, left=20mm, right=20mm, top=30mm, bottom=30mm}
\usepackage{graphicx}
\usepackage{bm} % for bold font in math mode - command is \bm{text}
\usepackage{enumitem}
\usepackage{fancyhdr}
\usepackage{amssymb} % for stacked arrows and other shit
\pagestyle{fancy}
\usepackage{changepage}
\usepackage{mathcomp}
\usepackage{tcolorbox}

\declaretheoremstyle[headfont=\normalfont]{normal}
\declaretheorem[style=normal]{Theorem}
\declaretheorem[style=normal]{Proposition}
\declaretheorem[style=normal]{Lemma}
\newcounter{ProofCounter}
\newcounter{ClaimCounter}[ProofCounter]
\newcounter{SubClaimCounter}[ClaimCounter]
\newenvironment{Proof}{\stepcounter{ProofCounter}\textsc{Proof.}}{\hfill$\square$}
\newenvironment{Solution}{\stepcounter{ProofCounter}\textbf{Solution:}}{\hfill$\square$}
\newenvironment{claim}[1]{\vspace{1mm}\stepcounter{ClaimCounter}\par\noindent\underline{\bf Claim \theClaimCounter:}\space#1}{}
\newenvironment{claimproof}[1]{\par\noindent\underline{Proof of claim \theClaimCounter:}\space#1}{\hfill $\blacksquare$ Claim \theClaimCounter}
\newenvironment{subclaim}[1]{\stepcounter{SubClaimCounter}\par\noindent\emph{Subclaim \theClaimCounter.\theSubClaimCounter:}\space#1}{}
% \newenvironment{subclaimproof}[1]{\begin{adjustwidth}{2em}{0pt}\par\noindent\emph{Proof of subclaim \theClaimCounter.\theSubClaimCounter:}\space#1}{\hfill
% $\blacksquare$ \emph{Subclaim \theClaimCounter.\theSubClaimCounter}\vspace{5mm}\end{adjustwidth}}
\newenvironment{subclaimproof}[1]{\par\noindent\emph{Proof of subclaim \theClaimCounter.\theSubClaimCounter:}\space#1}{\hfill
$\Diamond$ \emph{Subclaim \theClaimCounter.\theSubClaimCounter}}

\allowdisplaybreaks{}

% chktex-file 3

\lhead{Evan P. Walsh}
\chead{MATH 521}
\rhead{\thepage}
\cfoot{}

% Custom commands.
\newcommand\toinfty{\rightarrow\infty}
\newcommand\toinf{\rightarrow\infty}
\newcommand{\sinf}[1]{\sum_{#1=0}^{\infty}}
\newcommand{\linf}[1]{\lim_{#1\rightarrow\infty}}

\begin{document}\thispagestyle{empty}
\begin{center}
  \Large \textsc{math 521 -- ASSIGNMENT IV -- fall 2018} \\ 
  \vspace{5mm}
  \large Evan Pete Walsh
\end{center}

%------------------------------------------------------------------------------------------------------------------%
% Question 1
%------------------------------------------------------------------------------------------------------------------%

\subsection*{1}
\begin{tcolorbox}
  Find events A, B, and C which are pairwise independent but not jointly independent.
\end{tcolorbox}
\begin{Solution}
  Suppose $X_1, X_2$ and $X_3$ are independent random variables such that $P(X_i = 1) = P(X_i = 0) = \frac{1}{2}$ for $i = 1, 2, 3$. Define the events $A = \{ \omega : X_1(\omega) = X_2(\omega) \}$, $B = \{ \omega : X_1(\omega) = X_3(\omega) \}$, and $C = \{ \omega : X_2(\omega) = X_3(\omega) \}$. So
  \begin{align*}
    P(A) = P(B) = P(C) = P(X_i = X_j) & = P(X_i = X_j = 0) + P(X_i = X_j = 1) \\
    & = \left( \frac{1}{2} \right)^2 + \left( \frac{1}{2} \right)^2 = \frac{1}{2},
  \end{align*}
  for $i \neq j \in \{ 1, 2, 3 \}$. Now,
  \begin{align*}
    P(A \cap B) = P((X_1 = X_2) \cap (X_1 = X_3)) & = P(X_1 = X_2 = X_3) \\
    & = P(X_1 = X_2 = X_3 = 0) + P(X_1 = X_2 = X_3 = 1) \\
    & = \left( \frac{1}{2} \right)^3 + \left( \frac{1}{2} \right)^3 \\
    & = \frac{1}{4} \\
    & = P(A)P(B).
  \end{align*}
  Similarly, $P(A \cap C) = \frac{1}{4} = P(A)P(C)$, and $P(B\cap C) = \frac{1}{4} = P(B)P(C)$. Thus, the events A, B, and C are pairwise independent. However,
  \[
    P(A \cap B \cap C) = P(X_1 = X_2 = X_3) = \frac{1}{4} \neq \frac{1}{8} = P(A)P(B)P(C).
  \]
  Hence the events are not jointly independent.
\end{Solution}

%------------------------------------------------------------------------------------------------------------------%
% Question 2
%------------------------------------------------------------------------------------------------------------------%

\subsection*{2}
\begin{tcolorbox}
  For each $n \geq 3$, find a family of random variables $X_1, X_2, \dots, X_n$ such that every subfamily of size $n - 1$ or smaller is jointly independent, but the whole family is not.
\end{tcolorbox}
\begin{Solution}
  Let $n\geq 3$ and $X_1, X_2, \dots, X_{n-1}$ be jointly independent random variables on some probability space such that $P(X_i = -1) = P(X_i = 1) = \frac{1}{2}$ for eac h $ i \in \{1, \dots, n-1 \}$, and define $X_n := \prod_{i=1}^{n-1}X_i$.

  \begin{claim}
    $P(X_n = -1) = P(X_n = 1) = \frac{1}{2}$.
  \end{claim}
  \begin{claimproof}
    We will proceed by induction on $n$. If $n = 2$, it is trivial. Now suppose the claim holds for $k \geq 2$. Let $n = k + 1$. By the inductive hypothesis,
    \[
      P \left( \prod_{i=1}^{n-2}X_i = -1 \right) = P \left( \prod_{i=1}^{k-1}X_i = -1 \right) = P \left( \prod_{i=1}^{k-1}X_i = 1 \right) = P \left( \prod_{i=1}^{n-2}X_i = 1 \right) = \frac{1}{2}.
    \]
    Therefore,
    \begin{align*}
      P(X_n = -1) & = P \left( \prod_{i=1}^{n-2}X_i = -1, X_{n-1} = 1 \right) + P \left( \prod_{i=1}^{n-2}X_i = 1, X_{n-1} = -1 \right) \\
      & = P \left( \prod_{i=1}^{n-2}X_i = -1 \right)P(X_{n-1} = 1) + P \left( \prod_{i=1}^{n-2}X_i = 1 \right)P(X_{n-1} = -1) \\
      & = \frac{1}{2}.
    \end{align*}
    By symmetry, $P(X_n = 1) = \frac{1}{2}$ as well.
  \end{claimproof}

  \begin{claim}
    Any subfamily $\{ X_{i_1}, \dots, X_{i_{n-1}} \}$ of size $n - 1$ is jointly independent.
  \end{claim}
  \begin{claimproof}
    Let $\{ X_{i_1}, \dots, X_{i_{n-1}} \}$ be any subfamily of size $n - 1$. Note that if $X_n$ is not a member of the subfamily, then the claim is trivial. So without loss of generality assume the subfamily is $\{ X_2, \dots, X_{n-1}, X_n \}$. Then it suffices to show that for any $\{ a_2, \dots, a_{n} \} \in \{0,1\}^{n-1}$,
    \begin{equation}
      P(X_2 = a_2, \dots, X_{n-1} = a_{n-1}, X_n = a_n) = P(X_2 = a_2) \cdots P(X_{n-1} = a_{n-1}) P(X_n = a_n).
      \label{2.1}
    \end{equation}
    Define $a_1 = 1$ if $\text{sign}(\prod_{i=2}^{n-1}a_i) = \text{sign}(a_n)$ and $a_1 = -1$ otherwise. Then 
    \[
      P(X_n = a_n | X_2 = a_2, \dots , X_{n-1} = a_{n-1}) = P(X_1 = a_1),
    \]
    so
    \begin{align*}
      P(X_2 = a_2, \dots, X_{n-1} = a_{n-1}, X_n = a_n) & = P(X_n = a_n | X_2 = a_2, \dots , X_{n-1} = a_{n-1})\cdot \\
      & \ \ \ \ \ \ \ \ \ P(X_2 = a_2, \dots , X_{n-1} = a_{n-1}) \\
      & = P(X_1 = a_1)P(X_2 = a_2, \dots, X_{n-1} = a_{n-1}) \\
      & = P(X_1 = a_1)P(X_2 = a_2) \dots P(X_{n-1} = a_{n-1}) \\
      & = \left(\frac{1}{2}\right)^{n-1} \\
      & = P(X_2 = a_2) \cdots P(X_{n-1} = a_{n-1}) P(X_n = a_n),
    \end{align*}
    by the joint independence of $X_2, \dots, X_{n-1}$ and Claim 1. Thus \eqref{2.1} holds.
  \end{claimproof}

  \begin{claim}
    The family $\{ X_1, \dots, X_n \}$ is not jointly independent.
  \end{claim}
  \begin{claimproof}
    Note that
    \[
      P(X_1 = 1, \dots, X_{n-1} = 1, X_n = -1) = 0 \neq \left(\frac{1}{2}\right)^{n} = P(X_1 = 1) \dots P(X_{n-1}=1)P(X_n = -1).
    \]
  \end{claimproof}

\end{Solution}

%------------------------------------------------------------------------------------------------------------------%
% Question 3
%------------------------------------------------------------------------------------------------------------------%


%------------------------------------------------------------------------------------------------------------------%
% Question 4
%------------------------------------------------------------------------------------------------------------------%

\subsection*{4}
\begin{tcolorbox}
  Suppose $(\Omega, \mathcal{F}, P)$ is a probability space where $\mathcal{F} = \sigma(\mathcal{A})$, for some algebra $\mathcal{A}$. Prove that for all $A, B \in \mathcal{F}$, there exists $\{ A_n \}_{n=0}^{\infty}$, $\{ B_n \}_{n=0}^{\infty}$, where $A_n, B_n \in \mathcal{A}$ for all $n \in \mathbb{N}$, such that $P(A \triangle B) \rightarrow 0$, $P(B\triangle B_n) \rightarrow 0$, and $P((A\cap B) \triangle (A_n \cap B_n)) \rightarrow 0$ as $n \rightarrow \infty$
\end{tcolorbox}
\begin{Solution}
  Let
  \[
    \mathcal{L} = \{ A \in \mathcal{F} : \exists A_0, A_1, \dots \in \mathcal{A}, P(A \triangle A_n) \rightarrow 0 \}.
  \]
  The proof will proceed as follows: first we will show that $\mathcal{L}$ is a $\lambda$-system that of course contains $\mathcal{A}$ (Claim 1). Then, since $\mathcal{A}$ qualifies as a $\pi$-system, it follows by the $\pi-\lambda$ theorem that $\mathcal{F} \subset \mathcal{L}$ (Claim 2). Finally, in Claim 3, we will show that for any $A, B \in \mathcal{L}$ with corresponding sequences $\{ A_n \}_{n=0}^{\infty}$ and $\{ B_n \}_{n=0}^{\infty}$ in $\mathcal{A}$ satisfying $P(A \triangle A_n) \rightarrow 0$ and $P(B\triangle B_n) \rightarrow 0$ as $n \rightarrow \infty$, it follows that $P( (A \cap B) \triangle (A_n \cap B_n) ) \rightarrow 0$ as $n \rightarrow \infty$.

  \begin{claim}
    $\mathcal{L}$ is a $\lambda$-system containing $\mathcal{A}$.
  \end{claim}
  \begin{claimproof}
    The fact that $\mathcal{L}$ contains $\mathcal{A}$ is trivial. To show that $\mathcal{L}$ is a $\lambda$-system, there are 3 properties we need to verify: that $\Omega \in \mathcal{L}$, $\mathcal{L}$ is closed under the set difference operation, and that $\mathcal{L}$ is closed under countable unions from below.
    \begin{enumerate}
      \item Clearly $\Omega \in \mathcal{L}$ since $\Omega \in \mathcal{A}$.
      \item Now suppose $A, B \in \mathcal{L}$ and $A \subset B$. We need to show that $B - A \in \mathcal{L}$. Let $\{ A_n \}_{n=0}^{\infty}$ and $\{ B_n \}_{n=0}^{\infty}$ be sequences of sets in $\mathcal{A}$ such that $P(A \triangle A_n) \rightarrow 0$ and $P(B \triangle B_n) \rightarrow 0$. Let $C := B - A$ and $C_n := B_n - A_n$. Then $C_n \in \mathcal{A}$ for each $n \in \mathbb{N}$ since $\mathcal{A}$ is an algebra. Further,
        \begin{align*}
          P(C\triangle C_n) & = P((C - C_n) \cup (C_n - C)) \\
          & = P(((B - A) \cap (B_n - A_n)^{c}) \cup ((B_n - A_n) - (B - A)^{c} )) \\
          & = P( (B \cap A^{c} \cap (B_n^{c} \cup A_n)) \cup (B_n \cap A_n^c \cap (B^c \cup A)) ) \\
          & = P( (B \cap B_n^c \cup A^c) \cup (B \cap A^c \cap A_n) \cup (B_n \cap B^c \cup A_n^c) \cup (B_n \cap A_n^c \cap A) ) \\
          & \leq P( (B - B_n) \cup (A_n - A) \cup (B_n - B) \cup (A - A_n) ) \\
          & \leq P( (B - B_n) \cup (B_n - B) ) + P( (A - A_n) \cup (A_n - A) ) \\
          & = P(A \triangle A_n) + P(B \triangle B_n) \rightarrow 0.
        \end{align*}
        Thus $C \in \mathcal{L}$.
      \item Now suppose $A_n \in \mathcal{L}$ for all $n \in \mathbb{N}$ an $A_n \uparrow A$. We need to show $A \in \mathcal{L}$. For each $n \in \mathbb{N}$, let $\{ A_{n,k} \}_{k=0}^{\infty}$ be a sequence of sets in $\mathcal{A}$ such that $P(A_n \triangle A_{n,k}) \rightarrow 0$ as $k \rightarrow \infty$. Then let $B_n := A_{n,k'}$, for some $k'$ such that 
        \begin{equation}
          P(A_n \triangle B_n) < 2^{-n}.
          \label{4.1}
        \end{equation}
        
        Now let $\epsilon > 0$. Then choose $n > 0$ large enough such that $2^{-n + 1} < \epsilon$ and 

        \begin{equation}
          P(A \triangle A_n) = P(A - A_n) < 2^{-n}
          \label{4.2}
        \end{equation}
        
        (we can do this, of course, by the continuity of measure from below). Thus,
        \begin{align*}
          P( A \triangle B_n ) & = P( (A - B_n) \cup (B_n - A) ) \\
          & = P( (A \cap B_n^c) \cup (B_n \cap A^c) ) \\
          & = P( (A \cap B_n^c \cap A_n) \cup (A \cap B_n^c \cap A_n^c) \cup (B_n \cap A^c \cap A_n) \cup (B_n \cap A^c \cap A_n^c) ) \\
          & \leq P( (A - A_n) \cup (A_n - A) \cup (A_n - B_n) \cup (B_n - A_n) ) \\
          & \leq P( (A - A_n) \cup (A_n - A) ) + P( (A_n - B_n) \cup (B_n - A_n) ) \\
          & = P(A \triangle A_n) + P(A_n \triangle B_n) \\
          \text{(by \eqref{4.1} and \eqref{4.2})} \ \ \ \ & \leq 2^{-n} + 2^{-n} \\
          & = 2^{-n + 1} < \epsilon.
        \end{align*}
        Since $\epsilon > 0$ was arbitrary, $P(A \triangle B_n) \rightarrow 0$ as $n \rightarrow \infty$. Hence $A \in \mathcal{L}$.
    \end{enumerate}
  \end{claimproof}

  \begin{claim}
    $\mathcal{F} \subset \mathcal{L}$.
  \end{claim}
  \begin{claimproof}
    Since $\mathcal{A}$ is $\pi$-system that generates $\mathcal{F}$ and $\mathcal{L}$ is a $\lambda$-system containing $\mathcal{A}$, the result follows from the $\pi-\lambda$ theorem.
  \end{claimproof}

  \begin{claim}
    For any $A, B \in \mathcal{L}$ with corresponding sequences $\{ A_n \}_{n=0}^{\infty}$ and $\{ B_n \}_{n=0}^{\infty}$ in $\mathcal{A}$ satisfying $P(A \triangle A_n) \rightarrow 0$ and $P(B\triangle B_n) \rightarrow 0$ as $n \rightarrow \infty$, it follows that $P( (A \cap B) \triangle (A_n \cap B_n) ) \rightarrow 0$ as $n \rightarrow \infty$.
  \end{claim}
  \begin{claimproof}
    We have
    \begin{align*}
      P((A \cap B) \triangle (A_n \cap B_n)) & = P( ((A \cap B) - (A_n \cap B_n)) \cup ((A_n \cap B_n) - (A \cap B))) \\
      & = P( (A \cap B \cap (A_n^c \cup B_n^c)) \cup (A_n \cap B_n \cap (A^c \cup B^c)) ) \\
      & = P( (A \cap B \cap A_n^c) \cup (A \cap B \cap B_n^c) \cup (A_n \cap B_n \cap A^c) \cup (A_n \cap B_n \cap B^c) ) \\
      & \leq P( (A - A_n) \cup (A_n - A) \cup (B - B_n) \cup (B_n - B) ) \\
      & \leq P(A \triangle A_n) + P(B \triangle B_n) \rightarrow 0 \text{ as $n\rightarrow \infty$}.
    \end{align*}
  \end{claimproof}

  So by Claims 2 and 3, we are done.
\end{Solution}

%------------------------------------------------------------------------------------------------------------------%
% Question 5.
%------------------------------------------------------------------------------------------------------------------%

\subsection*{5}
\begin{tcolorbox}
  Suppose $X_1, X_2, \dots$ and $Y_1, Y_2, \dots$ are random variables defined on a common probability space $(\Omega, \mathcal{F}, P)$. Let $\mathcal{G} = \sigma(X_1, X_2, \dots)$ and $\mathcal{H} = \sigma(Y_1, Y_2, \dots)$. Assume that every finite family $(X_1, X_2, \dots, X_n)$ is independent of every finite family $(Y_1, Y_2, \dots, Y_m)$. Prove that $\mathcal{G}$ and $\mathcal{H}$ are independent.
\end{tcolorbox}
\begin{Solution}
  Let 
  \begin{align*}
    \mathcal{A} & := \{ A \subset \Omega : A \in \sigma(X_1, \dots, X_n), n \geq 1 \}, \text{ and } \\
    \mathcal{B} & := \{ B \subset \Omega : B \in \sigma(Y_1, \dots, Y_n), n \geq 1 \}.
  \end{align*}

  \begin{claim}
    $\mathcal{A}$ and $\mathcal{B}$ are algebras.
  \end{claim}
  \begin{claimproof}
    We will only give an argument for $\mathcal{A}$, as the argument for $\mathcal{B}$ follows by symmetry. Well, clearly $\Omega \in \mathcal{A}$ since $\Omega \in \sigma(X_1, \dots, X_n)$ for all $n \geq 1$. Further, if $A \in \mathcal{A}$, then there exists $n \geq 1$ such that $A \in \sigma(X_1, \dots, X_n)$. Thus $A^c \in \sigma(X_1, \dots, X_n)$, and so $A^c \in \mathcal{A}$. Hence $\mathcal{A}$ is closed under complementation. Lastly, suppose $A_1, A_2 \in \mathcal{A}$. Then there exists $k, m \geq 1$ such that $A_1 \in \sigma(X_1, \dots, X_k)$ and $A_2 \in \sigma(X_1, \dots, X_m)$. Now take $n := \max \{ k, m\}$. Then $A_1, A_2 \in \sigma(X_1, \dots, X_n)$ since $\sigma(X_1, \dots, X_k), \sigma(X_1, \dots, X_m) \subset \sigma(X_1, \dots, X_n)$. So then $A_1 \cup A_2 \in \sigma(X_1, \dots, X_n)$. Thus $A_1 \cup A_2 \in \mathcal{A}$, i.e. $\mathcal{A}$ is closed under finite unions. Hence $\mathcal{A}$ is an algebra.
  \end{claimproof}

  \begin{claim}
    $\mathcal{G} \subset \sigma(\mathcal{A})$ and $\mathcal{H} \subset \sigma(\mathcal{B})$.
  \end{claim}
  \begin{claimproof}
    By definition, $\mathcal{G}$ is the smallest $\sigma$-algebra containing all sets of the form $X_n^{-1}[A]$, where $A$ is a Borel set. But of course $X_n^{-1}[A] \in \sigma(X_1, \dots, X_n) \subset \mathcal{A}$. Hence $\mathcal{G} \subset \sigma(\mathcal{A})$. By symmetry, $\mathcal{H} \subset \sigma(\mathcal{B})$
  \end{claimproof}

  \begin{claim}
    If $A, A_1, A_2, \dots \in \mathcal{F}$ and $P(A \triangle A_n) \rightarrow 0$, then $P(A_n) \rightarrow P(A)$.
  \end{claim}
  \begin{claimproof}
    Note that since $P(A \triangle A_n) = P(A \cap A_n^c) + P(A_n \cap A^c) \rightarrow 0$ as $n \rightarrow \infty$,
    \begin{equation}
      P(A \cap A_n^c) \rightarrow 0 \ \ \ \text{ and  } \ \ \ P(A_n \cap A^c) \rightarrow 0 \ \ \ \text{ as $n \rightarrow \infty$ }.
      \label{5.1}
    \end{equation}
    Now, for each $n \geq 1$, $A = (A \cap A_n) \cup (A \cap A_n^c) \subset A_n \cup (A \cap A_n^c)$, and therefore
    \[
      P(A) \leq P(A_n) + P(A \cap A_n^c).
    \]
    Thus, $P(A) \leq \liminf_{n\rightarrow \infty} [ P(A) + P(A \cap A_n^c) ]$, and so by \eqref{5.1},
    \begin{equation}
      \liminf_{n\rightarrow \infty} P(A_n) \geq P(A).
      \label{5.2}
    \end{equation}
    Similarly, $A_n \subset A \cup (A_n \cap A^c)$, for any $n \geq 1$, and so
    \[
      \limsup_{n\rightarrow \infty} [P(A_n) - P(A_n \cap A^c)] \leq P(A).
    \]
    So by \eqref{5.1} again,
    \begin{equation}
      \limsup_{n\rightarrow\infty} P(A_n) \leq P(A).
      \label{5.3}
    \end{equation}
    Hence, by \eqref{5.2} and \eqref{5.3}, $\lim_{n\rightarrow \infty} P(A_n) = P(A)$.
  \end{claimproof}

  Now suppose $A \in \mathcal{G}$ and $B \in \mathcal{H}$. By Claim 2, $A \in \sigma(\mathcal{A})$ and $B \in \sigma(\mathcal{H})$. Thus, by Claim 1 and the previous exercise, $A$ and $B$ can be approximated by sequences $A_1, A_2, \dots \in \mathcal{A}$ and $B_1, B_2, \dots \in \mathcal{B}$, respectively. That is, $P(A \triangle A_n) \rightarrow 0$, $P(B \triangle B_n) \rightarrow 0$, and
  \[
    P((A \cap B) \triangle (A_n \cap B_n)) \rightarrow 0
  \]
  as $n \rightarrow \infty$. So by Claim 3,
  \[
    P(A \cap B) = \linf{n} P(A_n \cap B_n) = \linf{n} P(A_n) P(B_n) = P(A) P(B).
  \]
  Therefore $\mathcal{G}$ and $\mathcal{H}$ are independent.
\end{Solution}

\end{document}
