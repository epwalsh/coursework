\documentclass[12pt]{article}
\usepackage{amsmath}
\usepackage{amsfonts}
\usepackage{parskip}
\usepackage{amsthm}
\usepackage{thmtools}
\usepackage[headheight=15pt]{geometry}
\geometry{a4paper, left=20mm, right=20mm, top=30mm, bottom=30mm}
\usepackage{graphicx}
\usepackage{bm} % for bold font in math mode - command is \bm{text}
\usepackage{enumitem}
\usepackage{fancyhdr}
\usepackage{amssymb} % for stacked arrows and other shit
\pagestyle{fancy}
\usepackage{changepage}
\usepackage{mathcomp}
\usepackage{tcolorbox}

\declaretheoremstyle[headfont=\normalfont]{normal}
\declaretheorem[style=plain]{Theorem}
\declaretheorem[style=plain]{Proposition}
\declaretheorem[style=plain]{Lemma}
\newcounter{ProofCounter}
\newcounter{ClaimCounter}[ProofCounter]
\newcounter{SubClaimCounter}[ClaimCounter]
\newenvironment{Proof}{\stepcounter{ProofCounter}\textsc{Proof.}}{\hfill$\square$}
\newenvironment{Solution}{\stepcounter{ProofCounter}\textbf{Solution:}}{\hfill$\square$}
\newenvironment{claim}[1]{\vspace{1mm}\stepcounter{ClaimCounter}\par\noindent\underline{\bf Claim \theClaimCounter:}\space#1}{}
\newenvironment{claimproof}[1]{\par\noindent\underline{Proof of claim \theClaimCounter:}\space#1}{\hfill $\blacksquare$ Claim \theClaimCounter}
\newenvironment{subclaim}[1]{\stepcounter{SubClaimCounter}\par\noindent\emph{Subclaim \theClaimCounter.\theSubClaimCounter:}\space#1}{}
% \newenvironment{subclaimproof}[1]{\begin{adjustwidth}{2em}{0pt}\par\noindent\emph{Proof of subclaim \theClaimCounter.\theSubClaimCounter:}\space#1}{\hfill
% $\blacksquare$ \emph{Subclaim \theClaimCounter.\theSubClaimCounter}\vspace{5mm}\end{adjustwidth}}
\newenvironment{subclaimproof}[1]{\par\noindent\emph{Proof of subclaim \theClaimCounter.\theSubClaimCounter:}\space#1}{\hfill
$\Diamond$ \emph{Subclaim \theClaimCounter.\theSubClaimCounter}}

\allowdisplaybreaks{}

% chktex-file 3

\lhead{Evan P. Walsh}
\chead{MATH 521}
\rhead{\thepage}
\cfoot{}

% Custom commands.
\newcommand\toinfty{\rightarrow\infty}
\newcommand\toinf{\rightarrow\infty}
\newcommand{\sinf}[1]{\sum_{#1=0}^{\infty}}
\newcommand{\linf}[1]{\lim_{#1\rightarrow\infty}}

\begin{document}\thispagestyle{empty}
\begin{center}
  \Large \textsc{math 521 -- Midterm 1 -- fall 2018} \\ 
  \vspace{5mm}
  \large Evan Pete Walsh
\end{center}

%------------------------------------------------------------------------------------------------------------------%
% Question 1
%------------------------------------------------------------------------------------------------------------------%

\subsection*{1}
\begin{Solution}
  We will make use of the following results from Royden and Fitzpatrick's book on real analysis \cite{RF}. \\

  \begin{Theorem}[Section 2.7, Proposition 20 \cite{RF}]
    The Cantor-Lebesgue function is an increasing continuous function that maps $[0, 1]$ onto $[0, 1]$. Further, its derivative exists and is zero almost everywhere on $[0, 1]$ with respect to Lebesgue measure.
  \end{Theorem}

  \vspace{5mm}

  \begin{Theorem}[Section 6.5, Theorem 14 \cite{RF}]
    If $f$ is integrable over the closed, bounded interval $[a, b]$, then
    \[
      \frac{d}{dx} \int_{a}^{x} f = f(x) \ \ \text{for almost all $x \in (a, b)$}.
    \]
  \end{Theorem}

  \vspace{5mm}

  Now let $h$ be the Cantor-Lebesgue function and let $F(t) := \frac{1}{2}(h(t) + t)$, for $t \in [0, 1]$. By Theorem 1, $F$ is continuous, strictly increasing, and maps $[0, 1]$ onto $[0, 1]$. Therefore $F$ is a valid distribution function.
  
  Further, $f := F^{-1}$ is well-defined, continuous, and strictly increasing as well. Hence $f$ is also measurable. Therefore we can define the random variable $Y := f(X)$. By the Probability Integral Transformation (PIT), $Y$ has distribution function $F$.

  We claim that $Y$ does not have a density. By way of contradiction suppose $Y$ has density $g$. Then for $y \in [0, 1]$,
  \[
    P(Y \leq y) = \int_{0}^{y} g(t) dt = F(y).
  \]
  Thus, by Theorem 2,
  \[
    \frac{d}{dy} F(y) = g(y) \ \text{for almost all $y \in [0, 1]$}.
  \]
  However, by Theorem 1, $\frac{d}{dy} F(y) \equiv \frac{1}{2}$ since the Cantor-Lebesgue function $h$ is singular. Hence $g \equiv \frac{1}{2}$ almost everywhere.
  So,
  \[
    P(Y \leq 1) = \int_{0}^{1} \frac{1}{2} dt = \frac{1}{2}.
  \]
  But this is insane since $P(Y \leq 1) = 1$.
\end{Solution}

%------------------------------------------------------------------------------------------------------------------%
% Question 2
%------------------------------------------------------------------------------------------------------------------%

\newpage
\subsection*{2}
\begin{Solution}

  \textbf{Part i.} Let $m_1 < \infty$. Then for $n$ such that $2^{n} > m_1$,
  \begin{align*}
    P(N_n \geq m_1) = 1 - P(N_n < m_1) & = 1 - \sum_{j=0}^{m_1 - 1}(2^{-n})^j(1 - 2^{-n})^{2^n - j} \\
    & \geq 1 - m_1(1 - 2^{-n})^{2^n - m_1 + 1}.
  \end{align*}
  But $m_1(1 - 2^{-n})^{2^n - m_1 + 1} \rightarrow 0$ as $n \rightarrow \infty$. Thus
  \[
    P(N_n \geq m_1) \rightarrow 1 \ \ \text{as $n \rightarrow \infty$}.
  \]
  Hence we can actually give a slightly stronger statement than we are trying to prove. That is, for any $m_1 < \infty$ and \emph{any} $p_1 \in (0, 1)$, there exists $n_1 < \infty$ such that for all $n \geq n_1$,
  \[
    P(N_n \geq m_1) \geq p_1.
  \]

  \textbf{Part ii.}
\end{Solution}

\bibliographystyle{apalike}
\begin{thebibliography}{9}
  \bibitem[RF, 2010]{RF}
    H.L. Royden and P.M. Fitzpatrick.
    \textit{Real Analysis}. Fourth edition.
    Pearson Prentice Hall, 2010.
\end{thebibliography}

\end{document}
