\documentclass[12pt]{article}
\usepackage{amsmath}
\usepackage{amsfonts}
\usepackage{parskip}
\usepackage{amsthm}
\usepackage{thmtools}
\usepackage[headheight=15pt]{geometry}
\geometry{a4paper, left=20mm, right=20mm, top=30mm, bottom=30mm}
\usepackage{graphicx}
\usepackage{bm} % for bold font in math mode - command is \bm{text}
\usepackage{enumitem}
\usepackage{fancyhdr}
\usepackage{amssymb} % for stacked arrows and other shit
\pagestyle{fancy}
\usepackage{changepage}
\usepackage{mathcomp}
\usepackage{tcolorbox}

\declaretheoremstyle[headfont=\normalfont]{normal}
\declaretheorem[style=plain]{Theorem}
\declaretheorem[style=plain]{Proposition}
\declaretheorem[style=plain]{Lemma}
\newcounter{ProofCounter}
\newcounter{ClaimCounter}[ProofCounter]
\newcounter{SubClaimCounter}[ClaimCounter]
\newenvironment{Proof}{\stepcounter{ProofCounter}\textsc{Proof.}}{\hfill$\square$}
\newenvironment{Solution}{\stepcounter{ProofCounter}\textbf{Solution:}}{\hfill$\square$}
\newenvironment{claim}[1]{\vspace{1mm}\stepcounter{ClaimCounter}\par\noindent\underline{\bf Claim \theClaimCounter:}\space#1}{}
\newenvironment{claimproof}[1]{\par\noindent\underline{Proof of claim \theClaimCounter:}\space#1}{\hfill $\blacksquare$ Claim \theClaimCounter}
\newenvironment{subclaim}[1]{\stepcounter{SubClaimCounter}\par\noindent\emph{Subclaim \theClaimCounter.\theSubClaimCounter:}\space#1}{}
% \newenvironment{subclaimproof}[1]{\begin{adjustwidth}{2em}{0pt}\par\noindent\emph{Proof of subclaim \theClaimCounter.\theSubClaimCounter:}\space#1}{\hfill
% $\blacksquare$ \emph{Subclaim \theClaimCounter.\theSubClaimCounter}\vspace{5mm}\end{adjustwidth}}
\newenvironment{subclaimproof}[1]{\par\noindent\emph{Proof of subclaim \theClaimCounter.\theSubClaimCounter:}\space#1}{\hfill
$\Diamond$ \emph{Subclaim \theClaimCounter.\theSubClaimCounter}}

\allowdisplaybreaks{}

% chktex-file 3

\lhead{Evan P. Walsh}
\chead{MATH 521}
\rhead{\thepage}
\cfoot{}

% Custom commands.
\newcommand\toinfty{\rightarrow\infty}
\newcommand\toinf{\rightarrow\infty}
\newcommand{\sinf}[1]{\sum_{#1=0}^{\infty}}
\newcommand{\linf}[1]{\lim_{#1\rightarrow\infty}}

\begin{document}\thispagestyle{empty}
\begin{center}
  \Large \textsc{math 521 -- Midterm 1 -- fall 2018} \\ 
  \vspace{5mm}
  \large Evan Pete Walsh
\end{center}

%------------------------------------------------------------------------------------------------------------------%
% Question 1
%------------------------------------------------------------------------------------------------------------------%

\subsection*{1}
\begin{Solution}
  We will make use of the following results from Royden and Fitzpatrick's book on real analysis \cite{RF}. \\

  \begin{Theorem}[Section 2.7, Proposition 20 \cite{RF}]
    The Cantor-Lebesgue function is an increasing continuous function that maps $[0, 1]$ onto $[0, 1]$. Further, its derivative exists and is zero almost everywhere on $[0, 1]$ with respect to Lebesgue measure.
  \end{Theorem}

  \vspace{5mm}

  \begin{Theorem}[Section 6.5, Theorem 14 \cite{RF}]
    If $f$ is integrable over the closed, bounded interval $[a, b]$, then
    \[
      \frac{d}{dx} \int_{a}^{x} f = f(x) \ \ \text{for almost all $x \in (a, b)$}.
    \]
  \end{Theorem}

  \vspace{5mm}

  Now let $h$ be the Cantor-Lebesgue function and let $F(t) := \frac{1}{2}(h(t) + t)$, for $t \in [0, 1]$. By Theorem 1, $F$ is continuous, strictly increasing, and maps $[0, 1]$ onto $[0, 1]$. Therefore $F$ is a valid distribution function.
  
  Further, $f := F^{-1}$ is well-defined, continuous, and strictly increasing as well. Hence $f$ is also measurable. Therefore we can define the random variable $Y := f(X)$. By the Probability Integral Transformation (PIT), $Y$ has distribution function $F$.

  We claim that $Y$ does not have a density. By way of contradiction suppose $Y$ has density $g$. Then for almost all $y \in [0, 1]$,
  \[
    P(Y \leq y) = \int_{0}^{y} g(t) dt = F(y).
  \]
  Thus, by Theorem 2,
  \[
    \frac{d}{dy} F(y) = g(y) \ \text{for almost all $y \in [0, 1]$}.
  \]
  However, by Theorem 1, $\frac{d}{dy} F(y) \equiv \frac{1}{2}$ almost everywhere since the Cantor-Lebesgue function $h$ is singular. Hence $g \equiv \frac{1}{2}$ almost everywhere.
  So,
  \[
    P(Y \leq 1) = \int_{0}^{1}g(t)dt = \int_{0}^{1} \frac{1}{2} dt = \frac{1}{2}.
  \]
  But this is insane since $P(Y \leq 1) = 1$.
\end{Solution}

%------------------------------------------------------------------------------------------------------------------%
% Question 2
%------------------------------------------------------------------------------------------------------------------%

\newpage
\subsection*{2}
\begin{Solution}

  \textbf{Part i.}

  \textbf{Part ii.} Note that
  \begin{align*}
    P(M_n = 0) = P(X_k \leq 2^{2n}, \forall 1 \leq k \leq 2^{n}) = P(X_1 \leq 2^{2n})^{2^n} & = \left( \sum_{j=1}^{n}2^{-j} \right)^{2^n} \\
    & = \left( 1 - \sum_{j=n+1}^{\infty}2^{-j} \right)^{2^n} \\
    & = \left( 1 - 2^{-n}\sum_{j=1}^{\infty} 2^{-j} \right)^{2^{n}} \\
    & = (1 - 2^{-n})^{2^n}.
  \end{align*}
  But $(1 - 2^{-n})^{2^n} \rightarrow e^{-1}$ as $n \rightarrow \infty$. Thus for all $\epsilon > 0$, there exists $n_2 < \infty$ such that for each $n \geq n_2$,
  \[
    P(M_n = 0) \geq p_2,
  \]
  where $p_2 := e^{-1} - \epsilon$.

  \textbf{Part iii.} By Markov's inequality,
  \begin{align*}
    P(S_n^* \geq a 2^{2n}) \leq \frac{1}{a2^{2n}} ES_n^* & = \frac{1}{a2^{2n}} \sum_{k=1}^{2^{n}} EX_k\bm{1}_{X_k < 2^{2n}} \\
    & = \frac{1}{a2^{2n}} \sum_{k=1}^{2^n}\sum_{j=1}^{n-1} 2^{-j} 2^{2j} \\
    & = \frac{1}{a2^{2n}} 2^{n}\sum_{j=1}^{n-1} 2^{j} \\
    & = \frac{1}{a2^{n}} \sum_{j=1}^{n-1}2^{j} \\
    & = \frac{1}{a} \sum_{j=1}^{n-1} 2^{j-n} \\
    \text{($\ell = n-j$)} \ \ \ & = \frac{1}{a} \sum_{\ell=1}^{n-1}2^{-l} \\
    & = \frac{1}{a} \left( 1 - \sum_{\ell=n}^{\infty} 2^{-\ell} \right).
  \end{align*}
  But 
  \[
    \left( 1 - \sum_{\ell=n}^{\infty} 2^{-\ell} \right)  \leq 1
  \]
  provided $n > 1$. So for all $n > 1$,
  \[
    P(S_n^* \geq a 2^{2n})  \leq \frac{1}{a}.
  \]
  Thus we can choose any $a > 1$ and set $p_3 := \frac{1}{a}$, $n_3 := 2$. So for all $n \geq n_3$,
  \[
    P(S_n^* \geq a 2^{2n})  \leq p_3.
  \]

  \textbf{Part iv.}
\end{Solution}


%------------------------------------------------------------------------------------------------------------------%
% Question 3
%------------------------------------------------------------------------------------------------------------------%

\newpage
\subsection*{3}
\begin{Solution}
  \textbf{Part i.} Suppose $X$ and $Y$ are independent and $L_X(t)$, $L_Y(t)$ exist. Then by independence,
  \[
    L_{X + Y}(t) = Ee^{-t(X + Y)} = Ee^{-tX}e^{-tY} = Ee^{-tX}Ee^{-tY} = L_X(t) L_Y(t).
  \]

  \textbf{Part ii.} Suppose $S_n$ is Binomial$(n, 1/2)$. Then
  \begin{align*}
    L_{X}(t) = Ee^{-tX} = \sum_{k=0}^{n} e^{-tk} \binom{n}{k} \left( \frac{1}{2} \right)^n & = \sum_{k=0}^{n} \binom{n}{k} e^{-tk} \left( \frac{1}{2} \right)^k \left( \frac{1}{2} \right)^{n-k} \\
    & = \sum_{k=0}^{n} \binom{n}{k} \left( \frac{e^{-t}}{2} \right)^k \left( \frac{1}{2} \right)^{n-k} \\
    \text{(by the binomial theorem)} \ \ \ & = \left( \frac{e^{-t}}{2} + \frac{1}{2} \right)^{n}.
  \end{align*}

  \textbf{Part iii.} Suppose $Z$ standard normal. Then
  \begin{align*}
    L_Z(t) = \int_{-\infty}^{\infty} e^{-tz} \frac{1}{\sqrt{2\pi}} e^{-z^2 / 2} dz & = \int_{\infty}^{\infty} \frac{1}{\sqrt{2\pi}} e^{-\frac{1}{2}(z^2 + 2tz + t^2 - t^2)} dz \\
    & = \int_{-\infty}^{\infty} \frac{1}{\sqrt{2\pi}} e^{-\frac{1}{2}(z + t)^2 + t^2 / 2} dz \\
    & = e^{t^2 / 2} \int_{-\infty}^{\infty}\underbrace{ \frac{1}{\sqrt{2\pi}} e^{-\frac{1}{2}(z + t)^2} }_{\text{density of $N(-t, 1)$}} dz \\
    & = e^{t^2 / 2}.
  \end{align*}

  \textbf{Part iv.} We will show that the result holds with $c = 2$. In doing so, we will make use of the following lemma. \\

  \begin{Lemma}[A generalization of the limit definition of the exponential function]
    If a sequence $b_n \rightarrow 0$ as $n \rightarrow \infty$, then for all $a \in \mathbb{R}$,
    \[
      \left( 1 + \frac{a + b_n}{n} \right)^{n} \rightarrow e^{a} \ \ \text{ as $n\toinf$. }
    \]
  \end{Lemma}
  \begin{Proof}
    We will use without proof the limit definition of the exponential function, that is $e^x = \lim_{n\rightarrow \infty}(1 + x/n)^n$.
    Let $\epsilon > 0$. Then for large enough $n$, $a - \epsilon \leq a + b_n \leq a + \epsilon$. Therefore,
    \[
      \limsup_{n\rightarrow \infty} \left( 1 + \frac{a + b_n}{n} \right)^{n} \leq \limsup_{n\rightarrow \infty}\left( 1 + \frac{a + \epsilon}{n} \right)^{n} = e^{a + \epsilon},
    \]
    and similarly,
    \[
      \liminf_{n\rightarrow \infty} \left( 1 + \frac{a + b_n}{n} \right)^{n} \geq \liminf_{n\rightarrow \infty}\left( 1 + \frac{a - \epsilon}{n} \right)^{n} = e^{a - \epsilon}.
    \]
    Putting these together,
    \[
      e^{a - \epsilon} \leq \liminf_{n\rightarrow\infty}\left( 1 + \frac{a + b_n}{n} \right)^{n} \leq \limsup_{n\rightarrow \infty} \left( 1 + \frac{a + b_n}{n} \right)^{n} \leq e^{a + \epsilon}.
    \]
    But since $\epsilon > 0$ was arbitrary, $\left( 1 + \frac{a + b_n}{n} \right)^{n} \rightarrow e^{a}$ as $n\toinf$.
  \end{Proof}

  \vspace{5mm}

  Now, note that for all $c \in \mathbb{R}$, we have
  \begin{align*}
    L_{cR_n}(t) = Ee^{-tcR_n} = Ee^{-tc(S_n - n/2) / \sqrt{n}} & = e^{tcn / 2\sqrt{n}} Ee^{-tcS_n / \sqrt{n}} \\
    & = e^{tcn / 2\sqrt{n}} L_{S_n}(tc / \sqrt{n}).
  \end{align*}
  Thus, by Part ii. with $c = 2$,
  \begin{equation}
    L_{2R_n}(t) = e^{tn/\sqrt{n}} \left( \frac{e^{-t/\sqrt{n}}}{2} + \frac{1}{2} \right)^{n} = \left( \frac{e^{t/\sqrt{n}} + e^{-t/\sqrt{n}}}{2} \right)^{n}.
    \label{3.4.1}
  \end{equation}
  Using the series definition of the exponential, we can write
  \begin{align}
    e^{t/\sqrt{n}} + e^{-t/\sqrt{n}}  = \sum_{k=0}^{\infty} \frac{(t/\sqrt{n})^k}{k!} + \sum_{k=0}^{\infty} \frac{(-t/\sqrt{n})^k}{k!} & = \sum_{k=0}^{\infty} \frac{(t/\sqrt{n})^k + (-t/\sqrt{n})^k}{k!} \nonumber \\
    & = 2 + \frac{t^2}{n} + \sum_{k=3}^{\infty} \frac{(t/\sqrt{n})^k + (-t/\sqrt{n})^k}{k!} \nonumber \\
    & = 2 + \frac{t^2}{n} + \sum_{j=2}^{\infty} \frac{(t/\sqrt{n})^{2j}}{(2j)!}.
    \label{3.4.2}
  \end{align}
  Then if $b_n := n\sum_{j=2}^{\infty} \frac{(t/\sqrt{n})^{2j}}{(2j)!}$, we have
  \begin{equation}
    L_{2R_n}(t) = \left( 1 + \frac{t^2 / 2 + b_n}{n} \right)^{n},
    \label{3.4.3}
  \end{equation}
  by \eqref{3.4.1} and \eqref{3.4.2}. Thus, provided we can show that $b_n \rightarrow 0$ as $n \rightarrow \infty$, Lemma 1 gives us
  \[
    \lim_{n\rightarrow\infty} L_{2R_n}(t) = e^{t^2/2} = L_{Z}(t).
  \]
  Well,
  \begin{equation}
    b_n = n\sum_{j=2}^{\infty} \frac{(t/\sqrt{n})^{2j}}{(2j)!} = n\sum_{j=2}^{\infty} \frac{(t^2 / n)^j}{(2j)!} \leq n\sum_{j=2}^{\infty} (t^2 / n)^j = n\sum_{j=1}^{\infty}(t^2 / n)^j - t^2.
    \label{3.4.4}
  \end{equation}
  But for large enough $n$, $t^2 < n$, and so $\sum_{j=1}^{\infty}(t^2 / n)^j$ becomes a covergent geometric series. Hence
  \begin{equation}
    n\sum_{j=1}^{\infty}(t^2 / n)^j - t^2 = \frac{n(t^2/n)}{1 - t^2/n} - t^2 = \frac{nt^2}{n - t^2} - t^2 = \frac{nt^2 - (n-t^2)t^2}{n-t^2} = \frac{t^4}{n-t^2} \rightarrow 0
    \label{3.4.5}
  \end{equation}
  as $n \rightarrow \infty$. So by \eqref{3.4.4} and \eqref{3.4.5}, we are done.

\end{Solution}

\bibliographystyle{apalike}
\begin{thebibliography}{9}
  \bibitem[RF, 2010]{RF}
    H.L. Royden and P.M. Fitzpatrick.
    \textit{Real Analysis}. Fourth edition.
    Pearson Prentice Hall, 2010.
\end{thebibliography}

\end{document}
