\documentclass[12pt]{article}
\usepackage{amsmath}
\usepackage{amsfonts}
\usepackage{parskip}
\usepackage{amsthm}
\usepackage{thmtools}
\usepackage[headheight=15pt]{geometry}
\geometry{a4paper, left=20mm, right=20mm, top=30mm, bottom=30mm}
\usepackage{graphicx}
\usepackage{bm} % for bold font in math mode - command is \bm{text}
\usepackage{enumitem}
\usepackage{fancyhdr}
\usepackage{amssymb} % for stacked arrows and other shit
\pagestyle{fancy}
\usepackage{changepage}
\usepackage{mathcomp}
\usepackage{tcolorbox}

\declaretheoremstyle[headfont=\normalfont]{normal}
\declaretheorem[style=normal]{Theorem}
\declaretheorem[style=normal]{Proposition}
\declaretheorem[style=normal]{Lemma}
\newcounter{ProofCounter}
\newcounter{ClaimCounter}[ProofCounter]
\newcounter{SubClaimCounter}[ClaimCounter]
\newenvironment{Proof}{\stepcounter{ProofCounter}\textsc{Proof.}}{\hfill$\square$}
\newenvironment{Solution}{\stepcounter{ProofCounter}\textbf{Solution:}}{\hfill$\square$}
\newenvironment{claim}[1]{\vspace{1mm}\stepcounter{ClaimCounter}\par\noindent\underline{\bf Claim \theClaimCounter:}\space#1}{}
\newenvironment{claimproof}[1]{\par\noindent\underline{Proof of claim \theClaimCounter:}\space#1}{\hfill $\blacksquare$ Claim \theClaimCounter}
\newenvironment{subclaim}[1]{\stepcounter{SubClaimCounter}\par\noindent\emph{Subclaim \theClaimCounter.\theSubClaimCounter:}\space#1}{}
% \newenvironment{subclaimproof}[1]{\begin{adjustwidth}{2em}{0pt}\par\noindent\emph{Proof of subclaim \theClaimCounter.\theSubClaimCounter:}\space#1}{\hfill
% $\blacksquare$ \emph{Subclaim \theClaimCounter.\theSubClaimCounter}\vspace{5mm}\end{adjustwidth}}
\newenvironment{subclaimproof}[1]{\par\noindent\emph{Proof of subclaim \theClaimCounter.\theSubClaimCounter:}\space#1}{\hfill
$\Diamond$ \emph{Subclaim \theClaimCounter.\theSubClaimCounter}}

\allowdisplaybreaks{}

% chktex-file 3

\lhead{Evan P. Walsh}
\chead{MATH 521}
\rhead{\thepage}
\cfoot{}

% Custom commands.
\newcommand\toinfty{\rightarrow\infty}
\newcommand\toinf{\rightarrow\infty}
\newcommand{\sinf}[1]{\sum_{#1=0}^{\infty}}
\newcommand{\linf}[1]{\lim_{#1\rightarrow\infty}}

\begin{document}\thispagestyle{empty}
\begin{center}
  \Large \textsc{math 521 -- ASSIGNMENT II -- fall 2018} \\ 
  \vspace{5mm}
  \large Evan Pete Walsh
\end{center}


%------------------------------------------------------------------------------------------------------------------%
% Question 1
%------------------------------------------------------------------------------------------------------------------%

\subsection*{1.2.4}
\begin{tcolorbox}
  Show that if $F(x) = P(X \leq x)$ is continuous, then $Y := F(X)$ has a uniform distribution on $[0, 1]$. That is, if $y \in [0, 1]$, then $P(Y \leq y) = y$.
\end{tcolorbox}
\begin{Solution}
  Define $Y := F(X)$ on the probability space $([0, 1], \mathcal{B}[0, 1], P_Y)$, where $\mathcal{B}[0, 1]$ is the Borel sigma algebra on $[0, 1]$ and $P_Y$ is the probability measure induced by $F(X)$, that is $P_Y(A) = P(F(X) \in A)$ for all $A \in \mathcal{B}$.

  We will prove the result by way of the following small claims. First, define 
  \[
    F^{-1}(y) = \inf\{x \in [0, 1] : F(x) = y\}, \ \text{for all $y \in [0, 1]$.} 
  \]
  Of course the ``true" inverse of $F$ may not exist, but our $F^{-1}$ is still well-defined by the intermediate value theorem.

  \begin{claim}
    For all $y \in [0, 1]$, $F(F^{-1}(y)) = y$.
  \end{claim}
  \begin{claimproof}
    By way of contradiction suppose there exists $y_0 \in [0, 1]$ such that $F(F^{-1}(y_0)) \neq y_0$. Let $\{ x_n \}_{n=0}^{\infty}$ be a sequence in $\mathbb{R}$ converging to $F^{-1}(y)$, such that $F(x_n) = y$ for all $n \in \mathbb{N}$. So then $\linf{n} F(x_n) = y \neq F^{-1}(y)$. But this contradicts the continuity of $F$.
  \end{claimproof}

  \begin{claim}
    If $F(x) < y$, then $x < F^{-1}(y)$.
  \end{claim}
  \begin{claimproof}
    Suppose $F(x) < y$. Then, similar to in Claim 1, it follows by continuity of $F$ that
    \[
      x < \inf\{ x' : F(x') = y\} = F^{-1}(y).
    \]
  \end{claimproof}

  \begin{claim}
    If $y < F(x)$, then $F^{-1}(y) < x$.
  \end{claim}
  \begin{claimproof}
    Suppose $y < F(x)$. Then $F^{-1}(y) = \inf\{ x' : F(x') = y \} < x$, since $y < F(x)$ and $F$ is non-decreasing.
  \end{claimproof}

  Now, for any $\epsilon > 0$,
  \begin{align*}
    P_Y(Y \leq y) & = P(F(X) \leq y) \\
    & \leq P(F(X) < y + \epsilon) \\
    & \leq P(X < F^{-1}(y + \epsilon)) \text{ by Claim 2 } \\
    & \leq P(X \leq F^{-1}(y + \epsilon)) \\
    & = F(F^{-1}(y + \epsilon)) \\
    & = y + \epsilon \text{, by Claim 1. }
  \end{align*}
  Since this holds for arbitrary $\epsilon > 0$,
  \begin{equation}
    P_Y(Y \leq y) \leq y.
    \label{1.1}
  \end{equation}

  Further,
  \begin{align*}
    1 - P_Y(Y \leq y) & = P(F(X) > y) \\
    & \leq P(X > F^{-1}(y)) \text{ by Claim 3 } \\
    & = 1 - F(F^{-1}(y)) \\
    & = 1 - y.
  \end{align*}
  Hence,
  \begin{equation}
    P_Y(Y \leq y) \geq y.
    \label{1.2}
  \end{equation}
  So by \eqref{1.1} and \eqref{1.2}, $P_Y(Y \leq y) = y$.

\end{Solution}


\end{document}
