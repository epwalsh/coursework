\documentclass[12pt]{article}
\usepackage{amsmath}
\usepackage{amsfonts}
\usepackage{parskip}
\usepackage{amsthm}
\usepackage{thmtools}
\usepackage[headheight=15pt]{geometry}
\geometry{a4paper, left=20mm, right=20mm, top=30mm, bottom=30mm}
\usepackage{graphicx}
\usepackage{bm} % for bold font in math mode - command is \bm{text}
\usepackage{enumitem}
\usepackage{fancyhdr}
\usepackage{amssymb} % for stacked arrows and other shit
\pagestyle{fancy}
\usepackage{changepage}
\usepackage{mathcomp}
\usepackage{tcolorbox}

\declaretheoremstyle[headfont=\normalfont]{normal}
\declaretheorem[style=normal]{Theorem}
\declaretheorem[style=normal]{Proposition}
\declaretheorem[style=normal]{Lemma}
\newcounter{ProofCounter}
\newcounter{ClaimCounter}[ProofCounter]
\newcounter{SubClaimCounter}[ClaimCounter]
\newenvironment{Proof}{\stepcounter{ProofCounter}\textsc{Proof.}}{\hfill$\square$}
\newenvironment{Solution}{\stepcounter{ProofCounter}\textbf{Solution:}}{\hfill$\square$}
\newenvironment{claim}[1]{\vspace{1mm}\stepcounter{ClaimCounter}\par\noindent\underline{\bf Claim \theClaimCounter:}\space#1}{}
\newenvironment{claimproof}[1]{\par\noindent\underline{Proof of claim \theClaimCounter:}\space#1}{\hfill $\blacksquare$ Claim \theClaimCounter}
\newenvironment{subclaim}[1]{\stepcounter{SubClaimCounter}\par\noindent\emph{Subclaim \theClaimCounter.\theSubClaimCounter:}\space#1}{}
% \newenvironment{subclaimproof}[1]{\begin{adjustwidth}{2em}{0pt}\par\noindent\emph{Proof of subclaim \theClaimCounter.\theSubClaimCounter:}\space#1}{\hfill
% $\blacksquare$ \emph{Subclaim \theClaimCounter.\theSubClaimCounter}\vspace{5mm}\end{adjustwidth}}
\newenvironment{subclaimproof}[1]{\par\noindent\emph{Proof of subclaim \theClaimCounter.\theSubClaimCounter:}\space#1}{\hfill
$\Diamond$ \emph{Subclaim \theClaimCounter.\theSubClaimCounter}}

\allowdisplaybreaks{}

% chktex-file 3

\title{MATH 502: Assignment V}
\author{Evan P. Walsh}
\makeatletter
\makeatother
\lhead{Evan P. Walsh}
\chead{MATH 502: Assignment V}
\rhead{\thepage}
\cfoot{}

\begin{document}
\maketitle


\subsection*{1}
\begin{Solution}
  \begin{enumerate}
    \item[3.] Let $(X,d)$ be a metric space and suppose $F \subseteq X$ is closed and $K \subseteq X$ is compact.
      
      \begin{claim}
        $F\cap K \neq \emptyset$ if and only if $m := \inf\left\{ d(x,y) : x \in F, y \in K \right\} = 0$.
      \end{claim}
      \begin{claimproof}
        $(\Rightarrow)$ Trivial.

        $(\Leftarrow)$ Suppose $m = 0$. Then there exists a sequence $\{(x_n, y_n)\}_{n=0}^{\infty}$ such that $x_n \in F, y_n \in K$ for all $n \in
        \mathbb{N}$ and $d(x_n, y_n) \rightarrow 0$. Since $K$ is compact, there exists a subsequence $\{y_{n_k}\}_{k=0}^{\infty}$ that converges to
        some $y \in K$. Thus $d(x_n, y) \leq d(x_n, y_n) + d(y_n, y) \rightarrow 0$ as $n \rightarrow \infty$, so $x_n \rightarrow y$. But since $F$
        is closed, $y \in F$. Hence $F \cap K \supseteq \{y\}$.
      \end{claimproof}

      If we replace the assumption that $K$ is compact with the looser assumption that $K$ is closed, then the above claim fails. To see this,
      consider the metric space $\mathbb{R}$ with the usual metrix induced by $|\cdot |$. Let $F := \mathbb{N}$ and let $K := \{n + 2^{-n}: n \in
      \mathbb{N}\}$. So both $F$ and $K$ are closed, $F \cap K = \emptyset$, but $m \leq \lim_{n\rightarrow\infty}|n - (n+2^{-n})| = 0$.

    \item[4.] Let $K := K_1 \times \dots \times K_{n}$ and let $D := D_1$ be the metric on $K$.

      \begin{claim}
        $K$ equiped with the metric $D$ is compact.
      \end{claim}
      \begin{claimproof}
        Since $K$ is a metric space, it suffices to show that $K$ is sequentially compact.
        Let $\left\{ x_k \right\}_{k=0}^{\infty}$ be a sequence in $K$, where $x_k := (x_{k,1}, \dots, x_{k,n})$ for each $k \in \mathbb{N}$.
        Since $K_1$ is compact, there exists a subsequence $\{x_{k_j, 1}\}_{j=0}^{\infty}$ of $\{x_{k,1}\}_{k=0}^{\infty}$ that
        converges in $K_1$. But since $K_2$ is compact, there exists a convergent subsequence
        $\{x_{k_{j_i},2}\}_{i=0}^{\infty}$ of $\{x_{k_j,2}\}_{j=0}^{\infty}$ that converges in $K_2$. Continuing in this manner, we can construct a
        subsequence $\{x_{k}'\}_{k=0}^{\infty}$ of $\{x_{k}\}_{k=0}^{\infty}$, where $x_{k}' = (x_{k,1}', \dots, x_{k,n}')$, 
        such that $\{x_{k,m}'\}_{k=0}^{\infty}$ converges in $K_m$ for each $m = 1, \dots, n$. Let $y_{m}$ be the limit of
        $\{x_{k,m}'\}_{k=0}^{\infty}$ be each $m = 1, \dots, n$. Let $y := (y_1, \dots, y_n) \in K$. Then 
        \[
          \lim_{k\rightarrow\infty} d(x_{k}', y) = \lim_{k\rightarrow\infty} \sum_{m=1}^{n}d_{m}(x_{k,m}', y_{m}) =
          \sum_{m=1}^{n}\lim_{k\rightarrow\infty} d_{m}(x_{k,m}', y_{m}) = 0.
        \]
        Hence $\left\{ x_{k}' \right\}_{k=0}^{\infty}$ converges in $K$.
      \end{claimproof}


    \item[6.] Let $E$ be a normed space and let $K, L \subseteq E$ be compact.

      \begin{claim}
        $K + L$ is compact.
      \end{claim}
      \begin{claimproof}
        It suffices to show that $K + L$ is sequentially compact. Thus, let $\left\{ z_n \right\}_{n=0}^{\infty}$ be a sequence in $K + L$.  
        By the previous question, $K\times L$ equipped with the $D$ metric, which is induced by the norm $\|\cdot \|$ restricted to $K$ and $L$, is
        compact. Hence there exists a subsequence $\left\{ (x_{n_k}, y_{n_k}) \right\}_{k=0}^{\infty}$ of $\left\{ (x_n, y_n) \right\}_{n=0}^{\infty}$
        that converges to $(x, y) \in K\times L$ with respect to $D$. But then $\left\{ x_{n_k} \right\}_{k=0}^{\infty}$ converges to $x$ and 
        $\left\{ y_{n_k} \right\}_{k=0}^{\infty}$ converges to $y$ with respect to the metric induced by $\|\cdot \|$. Let $z := x + y \in K + L$.  
        Then
        \[
          \limsup_{k\rightarrow\infty}\|z_k - z\| \leq \limsup_{k\rightarrow\infty}\|x_k - x\| + \|y_k - y\| = 0.
        \]
        Thus $\left\{ z_{n_k} \right\}_{k=0}^{\infty}$ converges to $z \in K + L$. Hence $K + L$ is compact.
      \end{claimproof}

  \end{enumerate}
\end{Solution}

\subsection*{2}
\begin{Solution}
  Let $\left\{ r_n \right\}_{n=0}^{\infty}$ be an enumeration of the rationals in $[0,1]$ and define
  \[
    f(x) := \sum \left\{ 2^{-n} : r_n < x \right\}
  \]
  for $x \in [0,1]$.
  \begin{claim}
    $f$ is continuous at every irrational in $[0,1]$.
  \end{claim}
  \begin{claimproof}
    Let $x \in [0,1] - \mathbb{Q}$. Let $\epsilon > 0$. Then there exists $n_{\epsilon} \in \mathbb{N}$ such that $2^{-n_{\epsilon}} < \epsilon$. Now 
    choose $\delta > 0$ such that $B_{\delta}(x) \cap \left\{ r_0, r_1, \dots, r_{n_{\epsilon}} \right\} = \emptyset$. Then for $t \in
    B_{\delta}(x)$,
    \begin{align*}
      |f(t) - f(x)| & = \left\{ \begin{array}{cl}
          \sum \left\{ 2^{-n} : x \leq r_{n} < t \right\} & \text{ if } x \leq t \\ \\
          \sum\left\{ 2^{-n} : t \leq r_n < x  \right\} & \text{ if } x > t 
      \end{array} \right. \\
      & < \sum_{n=n_{\epsilon}+1}^{\infty} 2^{-n} = 2^{-n_{\epsilon}} < \epsilon.
    \end{align*}
    Hence $f$ continuous at $x$.
  \end{claimproof}

  \begin{claim}
    $f$ left-continuous at every rational in $(0,1]$.
  \end{claim}
  \begin{claimproof}
    Let $x \in (0,1] \cap \mathbb{Q}$. Let $\epsilon > 0$ and choose $n_{\epsilon} \in \mathbb{N}$ such that $2^{-n_{\epsilon}} < \epsilon$. Now
    choose $\delta > 0$ such that $(x-\delta, x) \cap \left\{ r_0, r_1, \dots, r_{n_{\epsilon}} \right\} = \emptyset$. Then for $t \in
    (x-\delta, x)$,
    \[
      |f(t) - f(x)| = \sum \left\{ 2^{-n} : t \leq r_{n} < x \right\} < \sum_{n=n_{\epsilon}+1}^{\infty} 2^{-n} = 2^{-n_{\epsilon}} < \epsilon.
    \]
    Hence $f$ left-continuous at $x$.
  \end{claimproof}

  \begin{claim}
    $f$ discontinuous from the right at every rational in $[0, 1)$.
  \end{claim}
  \begin{claimproof}
    Let $x \in [0, 1) \cap \mathbb{Q}$. Then there exists $n_x \in \mathbb{N}$ such that $x = r_{n_x}$. Then for any $\delta > 0$,
    $f(x + \delta) - f(x) \geq 2^{-n_x}$, and therefore 
    \[
      \lim_{t\rightarrow x^{+}}f(t) \geq 2^{-n_x} > f(x).
    \]
    Hence $f$ discontinuous from the right at $x$.
  \end{claimproof}

  Now let $F(x) := \int_{0}^{x}f(t)dt$.

  \begin{claim}
    $F$ is not differentiable at the rationals in $(0,1)$.
  \end{claim}
  \begin{claimproof}
    Let $x \in (0,1) \cap \mathbb{Q}$. Then there exists $n_{x} \in \mathbb{N}$ such that $x = r_{n_x}$. Thus, for any $h > 0$,
    \begin{equation}
      \frac{F(x+h) - F(x)}{h} = \frac{1}{h}\int_{x}^{x+h}f(t)dt \geq \frac{1}{h} \int_{x}^{x+h}2^{-n_x}dt \equiv 2^{-n_x},
        \label{2.1}
    \end{equation}
    since $f(t) \geq 2^{-n_x}$ for all $t \in (x, x+h)$. However,
    \begin{equation}
      \frac{F(x) - F(x-h)}{h} = \frac{1}{h}\int_{x-h}^{x}f(t)dt \leq \frac{1}{h}\int_{x-h}^{x}f(x)dt < 2^{-n_x}.
      \label{2.2}
    \end{equation}
    So by \eqref{2.1} and \eqref{2.2}, $f$ is not differentiable at $x$.
  \end{claimproof}

\end{Solution}



\subsection*{3}
\begin{Solution}
  Since Lipschitz continuity is a stronger condition than uniform continuity, and thus continuity, $\text{Lip}[a,b] \subset C[a,b]$. 

  \begin{claim}
    $\text{Lip}[a,b]$ is a subspace of $C[a,b]$.
  \end{claim}
  \begin{claimproof}
    Clearly the zero-function $f(x) \equiv 0$ is in $\text{Lip}[a,b]$ since any $M > 0$ is a Lipschitz constant for $f$. Now suppose $f$ and $g$ and
    any function in $\text{Lip}[a,b]$ and $c \in \mathbb{R}$. Let $M_f$ and $M_g$ be Lipschitz constants for $f$ and $g$, respectively. Let $h := f +
    g$. Then for all
    $x, y \in [a,b]$,
    \[
      |h(y) - h(x)| \leq |f(y) - f(x)| + |g(y) - g(x)| \leq M_f |y - x| + M_g |y - x| = (M_f + M_g) |y - x|,
    \]
    and thus $M_f + M_g$ is a Lipschitz constant for $h$, so $h \in \text{Lip}[a,b]$. Further,
    \[
      |cf(y) - cf(x)| = |c| \cdot |f(y) - f(x)| \leq |c| M_f |y - x|,
    \]
    and so $|c|\cdot M_f$ is a Lipschitz constant for $cf$, and thus $cf \in \text{Lip}[a,b]$.
  \end{claimproof}

  Now let $M_0 > 0$ and set $A := \left\{ f \in \text{Lip}[a,b] : M_0\text{ is a Lipschitz constant for $f$ and }|f(x)| \leq M_0 \right\}$.

  \begin{claim}
    $A$ is compact in $C[a,b]$.
  \end{claim}
  \begin{claimproof}
    We will proceed by way of the Arzel{\`a}-Ascoli Theorem. To that end, we need to show that $A$ is uniformly bounded and equicontinuous. Note that the
    condition of $A$ being uniformly bounded is equivalent to the condition that $\left\{ f(x) : x \in [a,b], f \in A \right\}$ is relatively compact
    in $\mathbb{R}$.

    \begin{subclaim}
      $A$ is uniformly bounded by $M_0(b - a + 1)$.
    \end{subclaim}
    \begin{subclaimproof}
      Let $f \in A$. Note that $f(a) \leq M_0$ and 
      \[
        |f(x) - f(a)| \leq M_0 |x - a| \leq M_0 |b - a|. 
      \]
      Hence $|f(x)| \leq |f(x)| + M_0(b - a) \leq M_0 + M_0(b-a)$.
    \end{subclaimproof}

    \begin{subclaim}
      $A$ is equicontinuous.
    \end{subclaim}
    \begin{subclaimproof}
      Let $\epsilon > 0$. Take $\delta := \frac{\epsilon}{M_0}$. Then if $x,y \in [a,b]$ such that $|x - y| < \delta$, then
      \[
        |f(x) - f(y)| \leq M_0|x - y| < \epsilon
      \]
      for all $f \in A$.
    \end{subclaimproof}

    By subclaims 2.1 and 2.2 and the Arzel{\`a}-Ascoli Theorem, $A$ is relatively compact in $C[a,b]$. Thus it remains to show that $A$ is closed.
    Well, if $\left\{ f_n \right\}_{n=0}^{\infty}$ is a sequence in $A$ that converges uniformly to some $f \in C[a,b]$, then clearly 
    \begin{equation}
      |f(a)| = \lim_{n\rightarrow\infty}|f_{n}(a)| \leq M_0,
      \label{3.1}
    \end{equation}
    and for any $x, y \in [a,b]$,
    \begin{equation}
      |f(x) - f(y)| = \lim_{n\rightarrow\infty}|f_n(x) - f_n(y)| \leq M_0 |x - y|.
      \label{3.2}
    \end{equation}
    So by \eqref{3.1} and \eqref{3.2}, $f \in A$ and hence $A$ is closed.
  \end{claimproof}

\end{Solution}

\subsection*{4}
\begin{Solution}
  For $f \in C^{1}[0,1]$, let $\|f\| := \sup_{0\leq x \leq 1}|f(x)| + \sup_{0\leq x\leq 1}|f'(x)|$.

  \begin{claim}
    $C^{1}[0,1]$ is a complete metric space.
  \end{claim}
  \begin{claimproof}
    Let $\left\{ f_{n} \right\}_{n=0}^{\infty}$ be a Cauchy sequence in $C^{1}[0,1]$. Then, by definition of $\|\cdot\|$, $\left\{ f_{n}
    \right\}_{n=0}^{\infty}$ and $\left\{ f_{n}' \right\}_{n=0}^{\infty}$ are Cauchy in $C[0,1]$. Hence there exists $f, g \in C[0,1]$ such that $f$
    is the limit of $\left\{ f_n \right\}$ is $C[0,1]$ and $g$ is the limit of $\left\{ f_n' \right\}$ in $C[0,1]$. Hence, if we can show that $f'
    \equiv g$, then 
    \[
      \|f_n - f\| = \sup_{0\leq x\leq 1}|f_n(x) - f(x)| + \sup_{0\leq x \leq 1}|f_n'(x) - g'(x)| \longrightarrow 0
    \]
    as $n\rightarrow \infty$, and thus $f_n \rightarrow f \in C^{1}[0,1]$, which proves that $C^{1}[0,1]$ is complete. To that end, note that by the
    Fundemantal Theorem of Calculus,
    \begin{equation}
      f_{n}(x) = \int_{0}^{x}f_{n}'(t)dt
      \label{4.1}
    \end{equation}
    for each $n \in \mathbb{N}$ and $x \in [0,1]$. Also, since $f_{n}' \rightarrow g$ uniformly, the sequence $\left\{ f_n' \right\}$ is bounded, i.e.
    there exists $M > 0$ such that $f_{n} \leq M$ for all $n \in \mathbb{N}$. Hence,
    \[
      f(x) = \lim_{n\rightarrow\infty} f_{n}(x) = \lim_{n\rightarrow\infty}\int_{0}^{x}f_{n}'(t)dt = \int_{0}^{x} \lim_{n\rightarrow\infty}f_{n}'(t)dt
      = \int_{0}^{x}g(t)dt,
    \]
    for all $x \in [0,1]$ by \eqref{4.1} and the Dominated Convergence Theorem. Hence, since $g \in C[0,1]$, $f' \equiv g$ by the Fundemantal Theorem
    of Calculus once again.
  \end{claimproof}

  Now let $B_1 := \left\{ f \in C^{1} : \|f\| \leq 1 \right\}$.

  \begin{claim}
    $B_1$ is NOT compact in $C^{1}[0,1]$.
  \end{claim}
  \begin{claimproof}
    $B_1$ is just the closed unit ball, and since $C^{1}[0,1]$ is infinite dimensional, Riesz's Lemma says that $B_1$ is not compact.
  \end{claimproof}

\end{Solution}



\end{document}
