\documentclass[12pt]{article}
\usepackage{amsmath}
\usepackage{amsfonts}
\usepackage{parskip}
\usepackage{amsthm}
\usepackage{thmtools}
\usepackage[headheight=15pt]{geometry}
\geometry{a4paper, left=20mm, right=20mm, top=30mm, bottom=30mm}
\usepackage{graphicx}
\usepackage{bm} % for bold font in math mode - command is \bm{text}
\usepackage{enumitem}
\usepackage{fancyhdr}
\usepackage{amssymb} % for stacked arrows and other shit
\pagestyle{fancy}
\usepackage{changepage}
\usepackage{mathcomp}
\usepackage{tcolorbox}

\declaretheoremstyle[headfont=\normalfont]{normal}
\declaretheorem[style=normal]{Theorem}
\declaretheorem[style=normal]{Proposition}
\declaretheorem[style=normal]{Lemma}
\newcounter{ProofCounter}
\newcounter{ClaimCounter}[ProofCounter]
\newcounter{SubClaimCounter}[ClaimCounter]
\newenvironment{Proof}{\stepcounter{ProofCounter}\textsc{Proof.}}{\hfill$\square$}
\newenvironment{Solution}{\stepcounter{ProofCounter}\textbf{Solution:}}{\hfill$\square$}
\newenvironment{claim}[1]{\vspace{1mm}\stepcounter{ClaimCounter}\par\noindent\underline{\bf Claim \theClaimCounter:}\space#1}{}
\newenvironment{claimproof}[1]{\par\noindent\underline{Proof of claim \theClaimCounter:}\space#1}{\hfill $\blacksquare$ Claim \theClaimCounter}
\newenvironment{subclaim}[1]{\stepcounter{SubClaimCounter}\par\noindent\emph{Subclaim \theClaimCounter.\theSubClaimCounter:}\space#1}{}
% \newenvironment{subclaimproof}[1]{\begin{adjustwidth}{2em}{0pt}\par\noindent\emph{Proof of subclaim \theClaimCounter.\theSubClaimCounter:}\space#1}{\hfill
% $\blacksquare$ \emph{Subclaim \theClaimCounter.\theSubClaimCounter}\vspace{5mm}\end{adjustwidth}}
\newenvironment{subclaimproof}[1]{\par\noindent\emph{Proof of subclaim \theClaimCounter.\theSubClaimCounter:}\space#1}{\hfill
$\Diamond$ \emph{Subclaim \theClaimCounter.\theSubClaimCounter}}

\allowdisplaybreaks{}

% chktex-file 3

\title{MATH 502: Assignment V}
\author{Evan P. Walsh}
\makeatletter
\makeatother
\lhead{Evan P. Walsh}
\chead{MATH 502: Assignment V}
\rhead{\thepage}
\cfoot{}

\begin{document}
\maketitle


\subsection*{1}
\begin{Solution}
  \begin{enumerate}
    \item[5.]

  \end{enumerate}
\end{Solution}

\subsection*{2}
\begin{Solution}
  Let $\left\{ r_n \right\}_{n=0}^{\infty}$ be an enumeration of the rationals in $[0,1]$ and define
  \[
    f(x) := \sum \left\{ 2^{-n} : r_n < x \right\}
  \]
  for $x \in [0,1]$.
  \begin{claim}
    $f$ is continuous at every irrational in $[0,1]$.
  \end{claim}
  \begin{claimproof}
    Let $x \in [0,1] - \mathbb{Q}$. Let $\epsilon > 0$. Then there exists $n_{\epsilon} \in \mathbb{N}$ such that $2^{-n_{\epsilon}} < \epsilon$. Now 
    choose $\delta > 0$ such that $B_{\delta}(x) \cap \left\{ r_0, r_1, \dots, r_{n_{\epsilon}} \right\} = \emptyset$. Then for $t \in
    B_{\delta}(x)$,
    \begin{align*}
      |f(t) - f(x)| & = \left\{ \begin{array}{cl}
          \sum \left\{ 2^{-n} : x \leq r_{n} < t \right\} & \text{ if } x \leq t \\ \\
          \sum\left\{ 2^{-n} : t \leq r_n < x  \right\} & \text{ if } x > t 
      \end{array} \right. \\
      & < \sum_{n=n_{\epsilon}+1}^{\infty} 2^{-n} = 2^{-n_{\epsilon}} < \epsilon.
    \end{align*}
    Hence $f$ continuous at $x$.
  \end{claimproof}

  \begin{claim}
    $f$ left-continuous at every rational in $(0,1]$.
  \end{claim}
  \begin{claimproof}
    Let $x \in (0,1] \cap \mathbb{Q}$. Let $\epsilon > 0$ and choose $n_{\epsilon} \in \mathbb{N}$ such that $2^{-n_{\epsilon}} < \epsilon$. Now
    choose $\delta > 0$ such that $(x-\delta, x) \cap \left\{ r_0, r_1, \dots, r_{n_{\epsilon}} \right\} = \emptyset$. Then for $t \in
    (x-\delta, x)$,
    \[
      |f(t) - f(x)| = \sum \left\{ 2^{-n} : t \leq r_{n} < x \right\} < \sum_{n=n_{\epsilon}+1}^{\infty} 2^{-n} = 2^{-n_{\epsilon}} < \epsilon.
    \]
    Hence $f$ left-continuous at $x$.
  \end{claimproof}

  \begin{claim}
    $f$ discontinuous from the right at every rational in $[0, 1)$.
  \end{claim}
  \begin{claimproof}
    Let $x \in [0, 1) \cap \mathbb{Q}$. Then there exists $n_x \in \mathbb{N}$ such that $x = r_{n_x}$. Then for any $\delta > 0$,
    $f(x + \delta) - f(x) \geq 2^{-n_x}$, and therefore 
    \[
      \lim_{t\rightarrow x^{+}}f(t) \geq 2^{-n_x} > f(x).
    \]
    Hence $f$ discontinuous from the right at $x$.
  \end{claimproof}

  Now let $F(x) := \int_{0}^{x}f(t)dt$.

  \begin{claim}
    $F$ is not differentiable at the rationals in $(0,1)$.
  \end{claim}
  \begin{claimproof}
    Let $x \in (0,1) \cap \mathbb{Q}$. Then there exists $n_{x} \in \mathbb{N}$ such that $x = r_{n_x}$. Thus, for any $h > 0$,
    \begin{equation}
      \frac{F(x+h) - F(x)}{h} = \frac{1}{h}\int_{x}^{x+h}f(t)dt \geq \frac{1}{h} \int_{x}^{x+h}2^{-n_x}dt \equiv 2^{-n_x},
        \label{2.1}
    \end{equation}
    since $f(t) \geq 2^{-n_x}$ for all $t \in (x, x+h)$. However,
    \begin{equation}
      \frac{F(x) - F(x-h)}{h} = \frac{1}{h}\int_{x-h}^{x}f(t)dt \leq \frac{1}{h}\int_{x-h}^{x}f(x)dt < 2^{-n_x}.
      \label{2.2}
    \end{equation}
    So by \eqref{2.1} and \eqref{2.2}, $f$ is not differentiable at $x$.
  \end{claimproof}

\end{Solution}


\end{document}
