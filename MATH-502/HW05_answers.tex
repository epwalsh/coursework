\documentclass[12pt]{article}
\usepackage{amsmath}
\usepackage{amsfonts}
\usepackage{parskip}
\usepackage{amsthm}
\usepackage{thmtools}
\usepackage[headheight=15pt]{geometry}
\geometry{a4paper, left=20mm, right=20mm, top=30mm, bottom=30mm}
\usepackage{graphicx}
\usepackage{bm} % for bold font in math mode - command is \bm{text}
\usepackage{enumitem}
\usepackage{fancyhdr}
\usepackage{amssymb} % for stacked arrows and other shit
\pagestyle{fancy}
\usepackage{changepage}
\usepackage{mathcomp}
\usepackage{tcolorbox}

\declaretheoremstyle[headfont=\normalfont]{normal}
\declaretheorem[style=normal]{Theorem}
\declaretheorem[style=normal]{Proposition}
\declaretheorem[style=normal]{Lemma}
\newcounter{ProofCounter}
\newcounter{ClaimCounter}[ProofCounter]
\newcounter{SubClaimCounter}[ClaimCounter]
\newenvironment{Proof}{\stepcounter{ProofCounter}\textsc{Proof.}}{\hfill$\square$}
\newenvironment{Solution}{\stepcounter{ProofCounter}\textbf{Solution:}}{\hfill$\square$}
\newenvironment{claim}[1]{\vspace{1mm}\stepcounter{ClaimCounter}\par\noindent\underline{\bf Claim \theClaimCounter:}\space#1}{}
\newenvironment{claimproof}[1]{\par\noindent\underline{Proof of claim \theClaimCounter:}\space#1}{\hfill $\blacksquare$ Claim \theClaimCounter}
\newenvironment{subclaim}[1]{\stepcounter{SubClaimCounter}\par\noindent\emph{Subclaim \theClaimCounter.\theSubClaimCounter:}\space#1}{}
% \newenvironment{subclaimproof}[1]{\begin{adjustwidth}{2em}{0pt}\par\noindent\emph{Proof of subclaim \theClaimCounter.\theSubClaimCounter:}\space#1}{\hfill
% $\blacksquare$ \emph{Subclaim \theClaimCounter.\theSubClaimCounter}\vspace{5mm}\end{adjustwidth}}
\newenvironment{subclaimproof}[1]{\par\noindent\emph{Proof of subclaim \theClaimCounter.\theSubClaimCounter:}\space#1}{\hfill
$\Diamond$ \emph{Subclaim \theClaimCounter.\theSubClaimCounter}}

\allowdisplaybreaks{}

% chktex-file 3

\title{MATH 502: Assignment V}
\author{Evan P. Walsh}
\makeatletter
\makeatother
\lhead{Evan P. Walsh}
\chead{MATH 502: Assignment V}
\rhead{\thepage}
\cfoot{}

\begin{document}
\maketitle


\subsection*{1}
\begin{Solution}
  \begin{enumerate}
    \item[5.]

  \end{enumerate}
\end{Solution}

\subsection*{2}
\begin{Solution}
  Let $\left\{ r_n \right\}_{n=0}^{\infty}$ be an enumeration of the rationals in $[0,1]$ and define
  \[
    f(x) := \sum \left\{ 2^{-n} : r_n < x \right\}
  \]
  for $x \in [0,1]$.
  \begin{claim}
    $f$ is continuous at every irrational in $[0,1]$.
  \end{claim}
  \begin{claimproof}
    Let $x \in [0,1] - \mathbb{Q}$. Let $\epsilon > 0$. Then there exists $n_{\epsilon} \in \mathbb{N}$ such that $2^{-n_{\epsilon}} < \epsilon$. Now 
    choose $\delta > 0$ such that $B_{\delta}(x) \cap \left\{ r_0, r_1, \dots, r_{n_{\epsilon}} \right\} = \emptyset$. Then for $t \in
    B_{\delta}(x)$,
    \begin{align*}
      |f(t) - f(x)| & = \left\{ \begin{array}{cl}
          \sum \left\{ 2^{-n} : x \leq r_{n} < t \right\} & \text{ if } x \leq t \\ \\
          \sum\left\{ 2^{-n} : t \leq r_n < x  \right\} & \text{ if } x > t 
      \end{array} \right. \\
      & < \sum_{n=n_{\epsilon}+1}^{\infty} 2^{-n} = 2^{-n_{\epsilon}} < \epsilon.
    \end{align*}
    Hence $f$ continuous at $x$.
  \end{claimproof}

  \begin{claim}
    $f$ left-continuous at every rational in $(0,1]$.
  \end{claim}
  \begin{claimproof}
    Let $x \in (0,1] \cap \mathbb{Q}$. Let $\epsilon > 0$ and choose $n_{\epsilon} \in \mathbb{N}$ such that $2^{-n_{\epsilon}} < \epsilon$. Now
    choose $\delta > 0$ such that $(x-\delta, x) \cap \left\{ r_0, r_1, \dots, r_{n_{\epsilon}} \right\} = \emptyset$. Then for $t \in
    (x-\delta, x)$,
    \[
      |f(t) - f(x)| = \sum \left\{ 2^{-n} : t \leq r_{n} < x \right\} < \sum_{n=n_{\epsilon}+1}^{\infty} 2^{-n} = 2^{-n_{\epsilon}} < \epsilon.
    \]
    Hence $f$ left-continuous at $x$.
  \end{claimproof}

  \begin{claim}
    $f$ discontinuous from the right at every rational in $[0, 1)$.
  \end{claim}
  \begin{claimproof}
    Let $x \in [0, 1) \cap \mathbb{Q}$. Then there exists $n_x \in \mathbb{N}$ such that $x = r_{n_x}$. Then for any $\delta > 0$,
    $f(x + \delta) - f(x) \geq 2^{-n_x}$, and therefore 
    \[
      \lim_{t\rightarrow x^{+}}f(t) \geq 2^{-n_x} > f(x).
    \]
    Hence $f$ discontinuous from the right at $x$.
  \end{claimproof}

  Now let $F(x) := \int_{0}^{x}f(t)dt$.

  \begin{claim}
    $F$ is not differentiable at the rationals in $(0,1)$.
  \end{claim}
  \begin{claimproof}
    Let $x \in (0,1) \cap \mathbb{Q}$. Then there exists $n_{x} \in \mathbb{N}$ such that $x = r_{n_x}$. Thus, for any $h > 0$,
    \begin{equation}
      \frac{F(x+h) - F(x)}{h} = \frac{1}{h}\int_{x}^{x+h}f(t)dt \geq \frac{1}{h} \int_{x}^{x+h}2^{-n_x}dt \equiv 2^{-n_x},
        \label{2.1}
    \end{equation}
    since $f(t) \geq 2^{-n_x}$ for all $t \in (x, x+h)$. However,
    \begin{equation}
      \frac{F(x) - F(x-h)}{h} = \frac{1}{h}\int_{x-h}^{x}f(t)dt \leq \frac{1}{h}\int_{x-h}^{x}f(x)dt < 2^{-n_x}.
      \label{2.2}
    \end{equation}
    So by \eqref{2.1} and \eqref{2.2}, $f$ is not differentiable at $x$.
  \end{claimproof}

\end{Solution}



\subsection*{3}
\begin{Solution}
  Since Lipschitz continuity is a stronger condition than uniform continuity, and thus continuity, $\text{Lip}[a,b] \subset C[a,b]$. 

  \begin{claim}
    $\text{Lip}[a,b]$ is a subspace of $C[a,b]$.
  \end{claim}
  \begin{claimproof}
    Clearly the zero-function $f(x) \equiv 0$ is in $\text{Lip}[a,b]$ since any $M > 0$ is a Lipschitz constant for $f$. Now suppose $f$ and $g$ and
    any function in $\text{Lip}[a,b]$ and $c \in \mathbb{R}$. Let $M_f$ and $M_g$ be Lipschitz constants for $f$ and $g$, respectively. Let $h := f +
    g$. Then for all
    $x, y \in [a,b]$,
    \[
      |h(y) - h(x)| \leq |f(y) - f(x)| + |g(y) - g(x)| \leq M_f |y - x| + M_g |y - x| = (M_f + M_g) |y - x|,
    \]
    and thus $M_f + M_g$ is a Lipschitz constant for $h$, so $h \in \text{Lip}[a,b]$. Further,
    \[
      |cf(y) - cf(x)| = |c| \cdot |f(y) - f(x)| \leq |c| M_f |y - x|,
    \]
    and so $|c|\cdot M_f$ is a Lipschitz constant for $cf$, and thus $cf \in \text{Lip}[a,b]$.
  \end{claimproof}

  Now let $M_0 > 0$ and set $A := \left\{ f \in \text{Lip}[a,b] : M_0\text{ is a Lipschitz constant for $f$ and }|f(x)| \leq M_0 \right\}$.

  \begin{claim}
    $A$ is compact in $C[a,b]$.
  \end{claim}

\end{Solution}

\subsection*{4}
\begin{Solution}
  For $f \in C^{1}[0,1]$, let $\|f\| := \sup_{0\leq x \leq 1}|f(x)| + \sup_{0\leq x\leq 1}|f'(x)|$.

  \begin{claim}
    $C^{1}[0,1]$ is a complete metric space.
  \end{claim}
  \begin{claimproof}
    Let $\left\{ f_{n} \right\}_{n=0}^{\infty}$ be a Cauchy sequence in $C^{1}[0,1]$. Then, by definition of $\|\cdot\|$, $\left\{ f_{n}
    \right\}_{n=0}^{\infty}$ and $\left\{ f_{n}' \right\}_{n=0}^{\infty}$ are Cauchy in $C[0,1]$. Hence there exists $f, g \in C[0,1]$ such that $f$
    is the limit of $\left\{ f_n \right\}$ is $C[0,1]$ and $g$ is the limit of $\left\{ f_n' \right\}$ in $C[0,1]$. Hence, if we can show that $f'
    \equiv g$, then 
    \[
      \|f_n - f\| = \sup_{0\leq x\leq 1}|f_n(x) - f(x)| + \sup_{0\leq x \leq 1}|f_n'(x) - g'(x)| \longrightarrow 0
    \]
    as $n\rightarrow \infty$, and thus $f_n \rightarrow f \in C^{1}[0,1]$, which proves that $C^{1}[0,1]$ is complete. To that end, note that by the
    Fundemantal Theorem of Calculus,
    \begin{equation}
      f_{n}(x) = \int_{0}^{x}f_{n}'(t)dt
      \label{4.1}
    \end{equation}
    for each $n \in \mathbb{N}$ and $x \in [0,1]$. Also, since $f_{n}' \rightarrow g$ uniformly, the sequence $\left\{ f_n' \right\}$ is bounded, i.e.
    there exists $M > 0$ such that $f_{n} \leq M$ for all $n \in \mathbb{N}$. Hence,
    \[
      f(x) = \lim_{n\rightarrow\infty} f_{n}(x) = \lim_{n\rightarrow\infty}\int_{0}^{x}f_{n}'(t)dt = \int_{0}^{x} \lim_{n\rightarrow\infty}f_{n}'(t)dt
      = \int_{0}^{x}g(t)dt,
    \]
    for all $x \in [0,1]$ by \eqref{4.1} and the Dominated Convergence Theorem. Hence, since $g \in C[0,1]$, $f' \equiv g$ by the Fundemantal Theorem
    of Calculus once again.
  \end{claimproof}

  Now let $B_1 := \left\{ f \in C^{1} : \|f\| \leq 1 \right\}$.

  \begin{claim}
    $B_1$ is compact in $C^{1}[0,1]$.
  \end{claim}
  \begin{claimproof}
    
  \end{claimproof}

\end{Solution}



\end{document}
