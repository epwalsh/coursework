\documentclass[12pt]{article}
\usepackage{amsmath}
\usepackage{amsfonts}
\usepackage{parskip}
\usepackage{amsthm}
\usepackage{thmtools}
\usepackage[headheight=15pt]{geometry}
\geometry{a4paper, left=20mm, right=20mm, top=30mm, bottom=30mm}
\usepackage{graphicx}
\usepackage{bm} % for bold font in math mode - command is \bm{text}
\usepackage{enumitem}
\usepackage{fancyhdr}
\usepackage{amssymb} % for stacked arrows and other shit
\pagestyle{fancy}
\usepackage{changepage}
\usepackage{mathcomp}
\usepackage{tcolorbox}
\usepackage{eufrak}

\declaretheoremstyle[headfont=\normalfont]{normal}
\declaretheorem[style=normal]{Theorem}
\declaretheorem[style=normal]{Proposition}
\declaretheorem[style=normal]{Lemma}
\newcounter{ProofCounter}
\newcounter{ClaimCounter}[ProofCounter]
\newcounter{SubClaimCounter}[ClaimCounter]
\newenvironment{Proof}{\stepcounter{ProofCounter}\textsc{Proof.}}{\hfill$\square$}
\newenvironment{Solution}{\stepcounter{ProofCounter}\textbf{Solution:}}{\hfill$\square$}
\newenvironment{claim}[1]{\vspace{1mm}\stepcounter{ClaimCounter}\par\noindent\underline{\bf Claim \theClaimCounter:}\space#1}{}
\newenvironment{claimproof}[1]{\par\noindent\underline{Proof of claim \theClaimCounter:}\space#1}{\hfill $\blacksquare$ Claim \theClaimCounter}
\newenvironment{subclaim}[1]{\stepcounter{SubClaimCounter}\par\noindent\emph{Subclaim \theClaimCounter.\theSubClaimCounter:}\space#1}{}
% \newenvironment{subclaimproof}[1]{\begin{adjustwidth}{2em}{0pt}\par\noindent\emph{Proof of subclaim \theClaimCounter.\theSubClaimCounter:}\space#1}{\hfill
% $\blacksquare$ \emph{Subclaim \theClaimCounter.\theSubClaimCounter}\vspace{5mm}\end{adjustwidth}}
\newenvironment{subclaimproof}[1]{\par\noindent\emph{Proof of subclaim \theClaimCounter.\theSubClaimCounter:}\space#1}{\hfill
$\Diamond$ \emph{Subclaim \theClaimCounter.\theSubClaimCounter}}

\allowdisplaybreaks{}

% chktex-file 3

\lhead{Evan P. Walsh}
\chead{\textsc{Math 502 Assignment XI}}
\rhead{\thepage}
\cfoot{}

\begin{document}\thispagestyle{empty}
\begin{center}
  \Large \textsc{math 502 -- ASSIGNMENT XI -- spring 2017} \\ 
  \vspace{5mm}
  \large Evan Pete Walsh
\end{center}


\subsection*{1 [Sec. 4.2 \#2, 3, 4]}
\begin{Solution}
  \begin{enumerate}
    \item[\#2.] Let $D$ be a countable dense subset of $K$ and $\{d_0, d_1, d_2, \dots\}$ an enumeration of $D$. Define $\kappa : \mathbb{N}
      \rightarrow K$ by $\kappa(n) := d_{n}$.

      By the Stone-\v{C}ech compactification there exists a continuous function $\hat{\kappa} : \beta \mathbb{N} \rightarrow K$ such that $\kappa(n) =
      \hat{\kappa}\circ \iota(n)$ for each $n \in \mathbb{N}$. Thus it remains to show that $\hat{\kappa}$ is a surjection.

      Let $y \in K$. Since $D$ is dense in $K$, there exists a net $\left\{ y_\alpha \right\}_{\alpha\in\mathbb{A}}$ in $D$ such that $y_\alpha \rightarrow y$.

      For each $\alpha \in \mathbb{N}$, let $x_\alpha:= \iota \circ \kappa^{-1}(y_\alpha) \in \beta X$. Since $\beta X$ is compact, there exists a
      subnet $\left\{ x_{\alpha_\beta} \right\}_{\beta\in\mathbb{B}}$ such that $x_{\alpha_\beta}$ converges. So let $x := \lim x_{\alpha_\beta}$.

      Then by the continuity of $\hat{\kappa}$,
      \[
        \hat{\kappa}(x) = \lim \hat{\kappa}(x_{\alpha_\beta}) = \lim \hat{\kappa} \circ \iota \circ \kappa^{-1}(y_{\alpha_\beta}) = \lim
        y_{\alpha_\beta} = y.
      \]
      Hence $\hat{\kappa}$ is surjective.

    \item[\#3.]
      \begin{claim}
        $(X, \tau)$ is connected if and only if $C_b(X, \mathbb{F})$ has no non-trivial idempotents (i.e. no idempotents other than the constant
        functions 0 and 1).
      \end{claim}
      \begin{claimproof}
        First note that $f \in C_b(X, \mathbb{F})$ is idempotent if and only if $f(x) \in \{0,1\}$ for all $x \in X$.

        $(\Rightarrow)$ We will prove the contrapositive. Suppose there exists $f \in C_b(X, \mathbb{F})$ such that $f^2 \equiv f$ but $f \not\equiv
        0$ and $f \not\equiv 1$. Then $U := f^{-1}[\{1\}]$ and $V := f^{-1}[\{0\}]$ are non-empty, open, and cover $X$. Hence $(U, V)$ is a disconnection of $X$.

        $(\Leftarrow)$ We prove the contrapositive again. Suppose $U, V$ is a disconnection of $X$. Then let 
        \[
          f(x) := \left\{ \begin{array}{cl}
              1 & \text{ if } x \in U \\
              0 & \text{ if } x \in V
          \end{array} \right. .
        \]
        Then clearly $f$ is continuous, idempotent, and non-constant.
      \end{claimproof}

      \begin{claim}
        A completely regular space is connected if and only if its Stone-\v{C}ech compactification is.
      \end{claim}
      \begin{claimproof}
        Let $(X, \tau)$ be a completely regular topological space and $X \beta$ the Stone-\v{C}ech compactification of $X$. Identify $X$ with $\iota(X)$.
        By the previous claim it suffices to show that $C_b(X, \mathbb{F})$ contains a non-constant idempotent function if and only if $C(\beta X,
        \mathbb{F})$ contains a non-constant idempotent function.

        Note that for any $f,g \in C_b(X, \mathbb{F})$ with $\tilde{f}, \tilde{g} \in C(\beta X, \mathbb{F})$ their respective extensions to $\beta X$,
        \begin{equation}
          \sup_{x \in X}|f(x) - g(x)| = \sup_{x \in \beta X}|\tilde{f}(x) - \tilde{g}(x)|,
          \label{1}
        \end{equation}
        since $X$ is dense in $X\beta$. Hence if $f \in C_b(X, \mathbb{F})$ is idempotent and non-constant, its extension $\tilde{f} \in C(\beta X, \mathbb{F})$
        is idempotent and non-constant. 
        
        On the other hand, if $\tilde{f} \in C(\beta X, \mathbb{F})$ is idempotent and non-constant, then clearly its restriction to
        $X$ is idempotent. Thus it remains to show that $f := \tilde{f}\big|_{X}$ is non-constant. But by \eqref{1} 
        \[
          1 = \sup_{x \in \beta X}|\tilde{f}(x) - 0| = \sup_{x \in X}|f(x) - 0|\  \text{ and } \
          1 = \sup_{x \in \beta X}|\tilde{f}(x) - 1| = \sup_{x \in X}|f(x) - 1|.
        \]
        Hence $f$ is non-constant.
      \end{claimproof}

    \item[\#4.] Let $(X, \tau)$ be a discrete topological space.

      (a) Let $f \in C_b(X, \mathcal{F})$ and $\epsilon > 0$. Since $f$ bounded, there exists $M > 0$ such that $|f| \leq M$. Now, for each $n \geq 1$
      with $n \epsilon / 2 \leq M + \epsilon / 2$, let 
      \[
        E_n := \{ x \in X : (n - 1)\epsilon / 2 < f(x) \leq n \epsilon / 2\} \text{ and } E_{-n} := \{x \in X : -n \epsilon / 2 \leq f(x) <
        (n-1)\epsilon / 2 \}.
      \]
      Then all of the $E_n$'s together with the $E_{-n}$'s are pairwise disjoint and cover $X$.

      Further, note that for any characteristic function $\chi_A$, $A \subseteq X$, $\chi_A[X] \subseteq
      \{0,1\}$. Hence every characteristic function is idempotent. Now let 
      \[
        g(x) := \sum_{\{n : n\epsilon / 2 \leq M + \epsilon / 2\}} (n\epsilon / 2)[ \chi_{E_n}(x) - \chi_{E_{-n}}(x)].
      \]
      Then $g$ is in the linear span of idempotents, and 
      \[
        |f(x) - g(x)| \leq \epsilon / 2 \ \ \text{ for all } x \in X.
      \]
      Hence $\sup_{x \in X}|f(x) - g(x)| < \epsilon$, and so the linear span of idempotents is dense in $C_b(X, \mathcal{F})$.

      (b) To show that $\beta X$ is zero-dimensional, we will make use the following claim.
      \begin{claim}
        The linear span of idempotents is dense in $C(\beta X, \mathbb{F})$.
      \end{claim}
      \begin{claimproof}
        Identify $X$ with $\iota(X)$.
        For $\tilde{f} \in C(\beta X, \mathcal{F})$, $f := \tilde{f}\big|_{X} \in C_b(X, \mathcal{F})$. Hence by part (a), for any $\epsilon > 0$, 
        there exists $g : X \rightarrow \mathbb{F}$ in the linear span of the idempotents on $X$ such that
        \[
          \sup_{x \in X}|g(x) - f(x)| < \epsilon.
        \]
        Let $\tilde{g} : \beta X \rightarrow \mathbb{F}$ be the unique extension of $g$ to $\beta X$. 
        \begin{subclaim}
          $\tilde{g}$ is in the linear span of idempotents on $\beta X$.
        \end{subclaim}
        \begin{subclaimproof}
          Note that by \eqref{1}, if $h$ is idempotent on $X$, then the unique extension of $h$ to $\beta X$ is idempotent on $\beta X$.
          Hence if $g = \sum_{i=1}^{n}a_i h_i$, where $a_i \in \mathbb{F}$ and $h_i$ is idempotent on $X$ for each $i = 1, \dots, n$, then by \eqref{1} again,
          \[
            \tilde{g} \equiv \sum_{i=1}^{n}a_i \tilde{h}_i,
          \]
          where $\tilde{h}_i$ is the idempotent extension of $h_i$ for each $i = 1,\dots, n$.
        \end{subclaimproof}

        Now, using \eqref{1} for a third time,
        \[
          \sup_{x \in \beta X}|\tilde{f}(x) - \tilde{g}(x)| = \sup_{x \in X}|f(x) - g(x)| < \epsilon.
        \]
        Hence the linear span of idempotents on $\beta X$ is dense in $C(\beta X, \mathbb{F})$.
      \end{claimproof}

      Now let $x, y \in \beta X$ with $x \neq y$. Since $\beta X$ is $T_1$, $\{x\}$ and $\{y\}$ are closed. Thus, by Urysohn's lemma, there exists
      $\tilde{f}
      \in C(\beta X, \mathbb{F})$ such that $\tilde{f}(x) = 0$ and $\tilde{f}(y) = 1$. 
      
      By the above claim there exists $\tilde{g}$ in the linear span of idempotents on
      $\beta X$ such that $\sup_{t \in \beta X}|\tilde{f}(t) - \tilde{g}(t)| < 1/2$. Thus, since $g(\beta X) \subseteq \{0,1\}$,
      $\tilde{g}(x) = 0$ and $\tilde{g}(y) = 1$. So by continuity
      \[
        U := g^{-1}[\{0\}] \text{ and } V := g^{-1}[\{1\}]
      \]
      are closed (and open), non-empty, disjoint, and cover $\beta X$ with $x \in U$ and $y \in V$. Hence $\beta X$ is
      zero-dimensional.
  \end{enumerate}
\end{Solution}

\subsection*{2}
\begin{tcolorbox}
  (a) Let $\tau$ be the topology on $\mathbb{R}$ defined by the base $\mathcal{B} := \{[a,b) : a,b \in \mathbb{R}\}$. Is $(\mathbb{R}, \tau)$
  metrizable?

  (b) Let $\tau'$ be the topology on $\mathbb{R}$ defined by the base $\mathcal{B}' := \{[a,b) : a, b \in \mathbb{Q}\}$. Is $(\mathbb{R}, \tau')$
  metrizable?
\end{tcolorbox}
\begin{Solution}
  (a) \textbf{No, $(\mathbb{R}, \tau)$ is not metrizable.} If it was metrizable, then the Sorgenfrey plane would also be metrizable since the Sorgenfrey plane
  is the topological product of $(\mathbb{R}, \tau)$ with itself. But Example 3.5.14 shows
  that the Sorgenfrey plane is not metrizable.

  (b) \textbf{Yes, $(\mathbb{R}, \tau')$ is metrizable.} To prove this, we will construct a metric $d$ that induces $\tau'$.

  Consider the function $g$ defined in Assignment VIII Question 3 for some enumeration $\{q_0, q_1, \dots \}$ of the rationals. 
  Recall that $g$ was shown to be continuous as a function from $(\mathbb{R}, \tau')$ to
  $(\mathbb{R}, |\cdot|)$. Now let $\tilde{g}(x) := x + g(x)$ and define $d$ as 
  \[
    d(x,y) := |\tilde{g}(x) - \tilde{g}(y)|.
  \]
  Clearly $d(x,y) = d(y,x)$ and $d(x,z) \leq d(x,y) + d(y,z)$ for all $x, y, z \in \mathbb{R}$. Also $d(x, y) = 0$ if and only if $x = y$ since
  $\tilde{g}$ is strictly increasing on $\mathbb{R}$.
  Hence $d$ is a valid metric.

  Since $x \mapsto x$ is continuous as a function from  $(\mathbb{R}, |\cdot|)
  \rightarrow (\mathbb{R}, |\cdot|)$ and $\tau'$ is finer than
  $|\cdot|$, $x \mapsto x$ is also continuous as a function from $(\mathbb{R}, \tau') \rightarrow (\mathbb{R}, |\cdot|)$.
  Hence $\tilde{g}$ is continuous as a function from $(\mathbb{R}, \tau') \rightarrow (\mathbb{R}, |\cdot|)$. Thus the open balls
  \[
    B_r(x) := \{ y \in \mathbb{R} : |\tilde{g}(y) - \tilde{g}(x)| < r \}, \ x \in \mathbb{R}, r > 0
  \]
  are open in $\tau'$. So the topology generated by $d$ is coarser than $\tau'$. Conversely, let $[a,b) \in \mathcal{B}'$. Then there exists $n_0, n_1 \in
  \mathbb{N}$ such that $q_{n_0} = a$ and $q_{n_1} = b$. Then since $\tilde{g}$ has a ``jump'' of $2^{-{n_0}}$ at $a$ and $2^{-n_1}$ as $b$,
  \[
    [a,b) = \bigcup \left\{ B_r(x) : x \in [a,b) \wedge r \leq \min\{ 2^{-n_0}, 2^{-n_1}\} \right\},
  \]
  which of course is open in the topology induced by $d$. So $\tau'$ is coarser than the topology induced by $d$.
  Hence the topology induced by $d$ is $\tau'$.
\end{Solution}

\subsection*{3}
\begin{tcolorbox}
  Let $(X, \tau)$ be a locally compact Hausdorff space which is not compact. 
  Let $\kappa : X \rightarrow X_{\infty}$ be the map $\kappa(x) = x$, where $X$ is viewed as a subspace of $X_{\infty}$. If $\iota: X \rightarrow \beta
  X$ is the embedding of Theorem 4.2.4, then the theorem states that there is a unique continuous map $\hat{\kappa} : \beta X \rightarrow X_{\infty}$
  such that $\hat{\kappa} \circ \iota \equiv \kappa$. Let $\omega$ be a point of $\beta X - \iota(X)$. What is $\hat{\kappa}(\omega)$?
\end{tcolorbox}
\begin{Solution}
  We claim that $\hat{\kappa}(\omega)$ is the unique point in $X_{\infty} - X$.

  By way of contradiction suppose $\hat{\kappa}(\omega) \in X$.
  Since $\iota(X)$ is dense in $\beta X$, we can find a net $\left\{ \omega_{\alpha} \right\}_{\alpha\in\mathbb{A}}$ in $\iota(X)$ converging to $\omega$.
  Then
  \[
    \hat{\kappa}(\omega) \stackrel{(a)}{=} \lim \hat{\kappa}(\omega_\alpha) \stackrel{(b)}{=} \lim\kappa\circ
    \iota^{-1}(\omega_\alpha) \stackrel{(c)}{=}
    \lim\iota^{-1}(\omega_\alpha) \in X,
  \]
  where (a) follows from the continuity of $\hat{\kappa}$, (b) follows from the definition of $\hat{\kappa}$ and since $\omega_\alpha \in \iota(X)$
  for all $\alpha \in \mathbb{A}$, and
  (c) follows by the definition of $\kappa$.
  So $\omega = \lim \omega_\alpha \in \iota(X)$ by the continuity of $\iota$. This is a contradiction.
\end{Solution}

\end{document}
