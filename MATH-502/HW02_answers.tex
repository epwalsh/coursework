\documentclass[12pt]{article}
\usepackage{amsmath}
\usepackage{amsfonts}
\usepackage{parskip}
\usepackage{amsthm}
\usepackage{thmtools}
\usepackage[headheight=15pt]{geometry}
\geometry{a4paper, left=20mm, right=20mm, top=30mm, bottom=30mm}
\usepackage{graphicx}
\usepackage{bm} % for bold font in math mode - command is \bm{text}
\usepackage{enumitem}
\usepackage{fancyhdr}
\usepackage{amssymb} % for stacked arrows and other shit
\pagestyle{fancy}
\usepackage{changepage}
\usepackage{mathcomp}
\usepackage{tcolorbox}

\declaretheoremstyle[headfont=\normalfont]{normal}
\declaretheorem[style=normal]{Theorem}
\declaretheorem[style=normal]{Proposition}
\declaretheorem[style=normal]{Lemma}
\newcounter{ProofCounter}
\newcounter{ClaimCounter}[ProofCounter]
\newcounter{SubClaimCounter}[ClaimCounter]
\newenvironment{Proof}{\stepcounter{ProofCounter}\textsc{Proof.}}{\hfill$\square$}
\newenvironment{Solution}{\stepcounter{ProofCounter}\textbf{Solution:}}{\hfill$\square$}
\newenvironment{claim}[1]{\vspace{1mm}\stepcounter{ClaimCounter}\par\noindent\underline{\bf Claim \theClaimCounter:}\space#1}{}
\newenvironment{claimproof}[1]{\par\noindent\underline{Proof of claim \theClaimCounter:}\space#1}{\hfill $\blacksquare$ Claim \theClaimCounter}
\newenvironment{subclaim}[1]{\stepcounter{SubClaimCounter}\par\noindent\emph{Subclaim \theClaimCounter.\theSubClaimCounter:}\space#1}{}
% \newenvironment{subclaimproof}[1]{\begin{adjustwidth}{2em}{0pt}\par\noindent\emph{Proof of subclaim \theClaimCounter.\theSubClaimCounter:}\space#1}{\hfill
% $\blacksquare$ \emph{Subclaim \theClaimCounter.\theSubClaimCounter}\vspace{5mm}\end{adjustwidth}}
\newenvironment{subclaimproof}[1]{\par\noindent\emph{Proof of subclaim \theClaimCounter.\theSubClaimCounter:}\space#1}{\hfill
$\Diamond$ \emph{Subclaim \theClaimCounter.\theSubClaimCounter}}

\allowdisplaybreaks{}

% chktex-file 3

\title{MATH 502: HW 2}
\author{Evan ``Pete'' Walsh}
\makeatletter
\makeatother
\lhead{Evan P. Walsh}
\chead{MATH 502: HW 2}
\rhead{\thepage}
\cfoot{}

\begin{document}
\maketitle

\subsection*{1}
\begin{Solution}
  Let $a\in \mathbb{Q}$ and $r > 0$. By definition,
  \[
    \partial B_r(a) = \{x \in \mathbb{Q} : \ \forall \ \epsilon > 0, B_{\epsilon}(x) \cap B_r(a) \neq \emptyset, B_{\epsilon}(x) \cap (\mathbb{Q} -
    B_r(a)) \neq \emptyset\}.
  \]
  Now, if $r \in \mathbb{Q}$, then $a - r, a + r \in \mathbb{Q}$. So clearly $\partial B_r(a) = \{a - r, a + r\}$. However, if $r \in \mathbb{R} -
  \mathbb{Q}$, then $a - r, a + r \in \mathbb{R} - \mathbb{Q}$. In that case we claim that $\partial B_r(a) = \emptyset$. By way of contradiction,
  suppose there exists $x_0 \in \partial B_r(a)$. Thus $x_0 \notin B_r(a)$, otherwise there would exist an open ball around $x_0$ contained in
  $B_r(a)$, a contradiction. But if $x_0 \notin B_r(a)$, then either $x < a - r$ and $x > a + r$. Without loss of generality assume the former. Then
  $B_{|x-a+r|}(x_0) \cap B_r(a) = \emptyset$, a contradiction. Hence
  \[
    \partial B_r(a) = \left\{ \begin{array}{cl}
        \{a - r, a + r\} & \text{ if } r \in \mathbb{Q} \\
        \emptyset & \text{ if } r \in \mathbb{R} - \mathbb{Q}.
    \end{array} \right.
  \]
\end{Solution}


\subsection*{2}
\begin{Solution}
  \begin{description}
    \item[Section 2.2 \#1:] Let $(X, d)$ be a metric space. Without loss of generality assume $X \neq \emptyset$.

      \begin{claim}
        Every singleton subset of $X$ is closed.
      \end{claim}
      \begin{claimproof}
        Let $x_0 \in X$. Let $x_1 \in X - \{x_0\}$. Then $B_{|x_1 - x_0|}(x_1) \cap \{x_0\} = \emptyset$, so $B_{|x_1 - x_0|}(x_1) \subseteq X -
        \{x_0\}$. Hence $X - \{x_0\}$ open, so $\{x_0\}$ closed.
      \end{claimproof}

      By Claim 1, any finite $E := \{x_0, x_1, \dots, x_n\} \subseteq X$ is closed since $E = \cup_{j=1}^{n}\{x_j\}$, a finite union of closed sets.

    \item[Section 2.2 \#2:] Let $a_0 \in S + U$. Then there exists $x_0 \in S$, $y_0 \in U$ such that $a_0 = x_0 + y_0$. Hence there exists $r
      > 0$ such that $B_r(y_0) \subset U$. Then 
      \[
        B_r(y_0) + x_0 := \{y + x_0 : y\in B_r(y_0)\} \subset S + U.
      \]
      Hence it remains to show that $B_r(y_0) + x_0$ is open. But 
      \[
        \{y + x_0 : y \in B_r(y_0)\} = \{y + x_0: \|y - y_0\| < r \} = \{y : \|y - y_0 - x_0\| < r\} = B_r(y_0 + x_0),
      \]
      which is an open ball.

    \item[Section 2.2 \#4:] Let $x \in \bar{S}$. Then for each $r > 0$, $B_r(x) \cap S \neq \emptyset$. Hence, for each $r > 0$, there exists $y \in
      S$ such that $d(x,y) < r$. So $0 = \inf\{d(x,y): y \in S\} = \text{dist}(x,S)$. Hence $\bar{S} \subseteq \{x \in X : \text{dist}(x,S) = 0\}$.

      Similarly, let $x \in X$ such that $\text{dist}(x,S) = 0$. Then for each $r > 0$, there exists $y \in S$ such that $d(x,y) < r$, i.e. $B_r(x)
      \cap S \neq \emptyset$. Hence $x \in \bar{S}$ by definition. So $\{x \in X: \text{dist}(x,S) = 0\} \subseteq \bar{S}$.

      Hence we can conclude that $\bar{S} = \{x \in X: \text{dist}(x,S) = 0\}$.

    \item[Section 2.2 \#5:] Let $S$ be the set of all functions $f : \mathbb{N} \rightarrow \mathbb{Q}$ such that all but a finite number of natural
      numbers are mapped to 0. We will show that $S$ is countable and dense in $Y$.

      \begin{claim}
        $S$ is countable.
      \end{claim}
      \begin{claimproof}
        Let for each $n \in \mathbb{N}$, let $S_n$ be the subset of $S$ consisting of no functions for which there exists $k > n$ such that $f(k) \neq
        0$. Then there is clearly a bijection between $S_n$ and $\mathbb{Q}^{n}$, which is a countable set. Hence each $S_n$ is countable.
        But $S = \cup_{n\in\mathbb{N}}S_n$, the countable union of countable sets. Hence $S$ is countable.
      \end{claimproof}

      \begin{claim}
        $S$ is dense in $Y$.
      \end{claim}
      \begin{claimproof}
        Let $f \in Y$ and $r > 0$. We need to show that there exists $g \in S$ such that $d(f,g) < r$. Now, by definition of $Y$, there exists $N \in
        \mathbb{N}$ such that $|f(n)| < r/2$ for all $n > N$. We can then choose $g \in S_N$ such that $|g(n) - f(n)| < r/2$ for all $1 \leq n \leq
        N$, since $\mathbb{Q}$ is dense in $\mathbb{R}$. Thus,
        \[
          d(f,g) = \sup_{n\in\mathbb{N}}|f(n) - g(n)| \leq r / 2 < r.
        \]
        So $S$ is dense in $Y$.
      \end{claimproof}


      By claims 2 and 3, $Y$ is separable.
  \end{description}
\end{Solution}

\subsection*{3}
\begin{Solution}
  \begin{description}
    \item[Section 2.3 \#1:] Let $\left\{ X^{(n)} \right\}_{n=1}^{\infty}$ be a sequence in $X$.

      $(\Rightarrow)$ Suppose $\left\{ X^{(n)} \right\}_{n=1}^{\infty}$ converges to $x^{(0)} := (x_1^{(0)}, x_2^{(0)}, \dots) \in X$. By way of
      contradiction suppose there exists $k_0 \in \mathbb{N}$ such that $\left\{ x_{k_0}^{(n)} \right\}_{n=1}^{\infty}$ does not converge to
      $x_{k_0}^{(0)}$. Then there exists $r > 0$ such that for all $N \in \mathbb{N}$, there exists $n > N$ with $d_{k_0}(x_{k_0}^{(n)},
      x_{k_0}^{(0)}) \geq r$, and thus 
      \[
        d(x^{(n)}, x^{(0)}) \geq \left( \frac{1}{2^{k_0}} \right)\left( \frac{r}{1 + r} \right) \ \ \text{infinitely often.}
      \]
      Hence $\left\{ X^{(n)} \right\}_{n=1}^{\infty}$ does not converge to $x^{(0)}$, a contradiction.

      $(\Leftarrow)$ Now suppose $x_k^{(n)} \rightarrow x_k^{(0)}$ as $n \rightarrow \infty$, for each $k \in \mathbb{N}$. Let $r > 0$. Choose $K \in
      \mathbb{N}$ such that $\sum_{k > K}2^{-k} < r / 2$. Now choose $N \in \mathbb{N}$ such that $d_k(x_k^{(n)}, x_k^{(0)}) < r/2$ for all $n \geq
      N$, $0 \leq k \leq K$. Then for $n \geq N$,
      \[
        d(x^{(n)}, x^{(0)}) < \sum_{k=1}^{K}\left( \frac{1}{2^k} \right)\left( \frac{r/2}{1 + r/2} \right) + \frac{r}{2} <
        \frac{r}{2}\sum_{k=1}^{\infty}\frac{1}{2^k} + \frac{r}{2} = r.
      \]
      Hence $\left\{ X^{(n)} \right\}_{n=1}^{\infty}$ converges to $x^{(0)}$.

    \item[Section 2.3 \#3:] Let $f(x) := \text{dist}(x, S)$. We want to show that $f$ is continuous on $X$. To that end, let $x_0 \in X$ and $r > 0$.
      Then for each $x \in X$ such that $d(x,x_0) < r/2$, and $y \in S$,
      \[
        d(x,y) \leq d(x,x_0) + d(x_0, y).
      \]
      So, taking the infimum over $y$ on both sides of the inequality above, we have 
      \[
        \text{dist}(x, S) \leq r/2 + \text{dist}(x_0, S), \ \text{ i.e. } f(x) \leq f(x_0) + r/2.
      \]
      By symmetry,
      \[
        \text{dist}(x_0, S) \leq r/2 + \text{dist}(x, S), \ \text{ i.e. } f(x) \geq f(x_0) - r/2.
      \]
      Hence $|f(x) - f(x_0)| \leq r / 2 < r$ for all $x \in X$ with $d(x,x_0) < r/2$. Thus $f$ continuous.
        
  \end{description}
\end{Solution}



\end{document}
