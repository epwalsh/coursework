\documentclass[12pt]{article}
\usepackage{amsmath}
\usepackage{amsfonts}
\usepackage{parskip}
\usepackage{amsthm}
\usepackage{thmtools}
\usepackage[headheight=15pt]{geometry}
\geometry{a4paper, left=20mm, right=20mm, top=30mm, bottom=30mm}
\usepackage{graphicx}
\usepackage{bm} % for bold font in math mode - command is \bm{text}
\usepackage{enumitem}
\usepackage{fancyhdr}
\usepackage{amssymb} % for stacked arrows and other shit
\pagestyle{fancy}
\usepackage{changepage}
\usepackage{mathcomp}
\usepackage{tcolorbox}

\declaretheoremstyle[headfont=\normalfont]{normal}
\declaretheorem[style=normal]{Theorem}
\declaretheorem[style=normal]{Proposition}
\declaretheorem[style=normal]{Lemma}
\newcounter{ProofCounter}
\newcounter{ClaimCounter}[ProofCounter]
\newcounter{SubClaimCounter}[ClaimCounter]
\newenvironment{Proof}{\stepcounter{ProofCounter}\textsc{Proof.}}{\hfill$\square$}
\newenvironment{Solution}{\stepcounter{ProofCounter}\textbf{Solution:}}{\hfill$\square$}
\newenvironment{claim}[1]{\vspace{1mm}\stepcounter{ClaimCounter}\par\noindent\underline{\bf Claim \theClaimCounter:}\space#1}{}
\newenvironment{claimproof}[1]{\par\noindent\underline{Proof of claim \theClaimCounter:}\space#1}{\hfill $\blacksquare$ Claim \theClaimCounter}
\newenvironment{subclaim}[1]{\stepcounter{SubClaimCounter}\par\noindent\emph{Subclaim \theClaimCounter.\theSubClaimCounter:}\space#1}{}
% \newenvironment{subclaimproof}[1]{\begin{adjustwidth}{2em}{0pt}\par\noindent\emph{Proof of subclaim \theClaimCounter.\theSubClaimCounter:}\space#1}{\hfill
% $\blacksquare$ \emph{Subclaim \theClaimCounter.\theSubClaimCounter}\vspace{5mm}\end{adjustwidth}}
\newenvironment{subclaimproof}[1]{\par\noindent\emph{Proof of subclaim \theClaimCounter.\theSubClaimCounter:}\space#1}{\hfill
$\Diamond$ \emph{Subclaim \theClaimCounter.\theSubClaimCounter}}

\allowdisplaybreaks{}

% chktex-file 3

\title{MATH 502: HW 2}
\author{Evan P. Walsh}
\makeatletter
\makeatother
\lhead{Evan P. Walsh}
\chead{MATH 502: HW 2}
\rhead{\thepage}
\cfoot{}

\begin{document}
\maketitle

\subsection*{1}
\begin{Solution}
  Let $a\in \mathbb{Q}$ and $r > 0$. By definition,
  \[
    \partial B_r(a) = \{x \in \mathbb{Q} : \ \forall \ \epsilon > 0, B_{\epsilon}(x) \cap B_r(a) \neq \emptyset, B_{\epsilon}(x) \cap (\mathbb{Q} -
    B_r(a)) \neq \emptyset\}.
  \]
  Now, if $r \in \mathbb{Q}$, then $a - r, a + r \in \mathbb{Q}$. So clearly $\partial B_r(a) = \{a - r, a + r\}$. However, if $r \in \mathbb{R} -
  \mathbb{Q}$, then $a - r, a + r \in \mathbb{R} - \mathbb{Q}$. In that case we claim that $\partial B_r(a) = \emptyset$. By way of contradiction,
  suppose there exists $x_0 \in \partial B_r(a)$. Thus $x_0 \notin B_r(a)$, otherwise there would exist an open ball around $x_0$ contained in
  $B_r(a)$, a contradiction. But if $x_0 \notin B_r(a)$, then either $x < a - r$ and $x > a + r$. Without loss of generality assume the former. Then
  $B_{|x-a+r|}(x_0) \cap B_r(a) = \emptyset$, a contradiction. Hence
  \[
    \partial B_r(a) = \left\{ \begin{array}{cl}
        \{a - r, a + r\} & \text{ if } r \in \mathbb{Q} \\
        \emptyset & \text{ if } r \in \mathbb{R} - \mathbb{Q}.
    \end{array} \right.
  \]
\end{Solution}


\subsection*{2}
\begin{Solution}
  \begin{description}
    \item[Section 2.2 \#1:] Let $(X, d)$ be a metric space. Without loss of generality assume $X \neq \emptyset$.

      \begin{claim}
        Every singleton subset of $X$ is closed.
      \end{claim}
      \begin{claimproof}
        Let $x_0 \in X$. Let $x_1 \in X - \{x_0\}$. Then $B_{|x_1 - x_0|}(x_1) \cap \{x_0\} = \emptyset$, so $B_{|x_1 - x_0|}(x_1) \subseteq X -
        \{x_0\}$. Hence $X - \{x_0\}$ open, so $\{x_0\}$ closed.
      \end{claimproof}

      By Claim 1, any finite $E := \{x_0, x_1, \dots, x_n\} \subseteq X$ is closed since $E = \cup_{j=1}^{n}\{x_j\}$, a finite union of closed sets.

    \item[Section 2.2 \#2:]
    \item[Section 2.2 \#4:]
    \item[Section 2.2 \#5:]
  \end{description}
\end{Solution}

\subsection*{3}
\begin{Solution}
  \begin{description}
    \item[Section 2.3 \#1:]
    \item[Section 2.3 \#3:]
  \end{description}
\end{Solution}



\end{document}
