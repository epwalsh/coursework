\documentclass[12pt]{article}
\usepackage{amsmath}
\usepackage{amsfonts}
\usepackage{parskip}
\usepackage{amsthm}
\usepackage{thmtools}
\usepackage[headheight=15pt]{geometry}
\geometry{a4paper, left=20mm, right=20mm, top=30mm, bottom=20mm}
\usepackage{graphicx}
\usepackage{bm} % for bold font in math mode - command is \bm{text}
\usepackage{enumitem}
\usepackage{fancyhdr}
\usepackage{amssymb} % for stacked arrows and other shit
\pagestyle{fancy}
\usepackage{changepage}
\usepackage{mathcomp}
\usepackage{tcolorbox}

\declaretheoremstyle[headfont=\normalfont]{normal}
\declaretheorem[style=normal]{Theorem}
\declaretheorem[style=normal]{Proposition}
\declaretheorem[style=normal]{Lemma}
\newcounter{ProofCounter}
\newcounter{ClaimCounter}[ProofCounter]
\newcounter{SubClaimCounter}[ClaimCounter]
\newenvironment{Proof}{\stepcounter{ProofCounter}\textsc{Proof.}}{\hfill$\square$}
\newenvironment{Solution}{\stepcounter{ProofCounter}\textbf{Solution:}}{\hfill$\square$}
\newenvironment{claim}[1]{\vspace{1mm}\stepcounter{ClaimCounter}\par\noindent\underline{\bf Claim \theClaimCounter:}\space#1}{}
\newenvironment{claimproof}[1]{\par\noindent\underline{Proof of claim \theClaimCounter:}\space#1}{\hfill $\blacksquare$ Claim \theClaimCounter}
\newenvironment{subclaim}[1]{\stepcounter{SubClaimCounter}\par\noindent\emph{Subclaim \theClaimCounter.\theSubClaimCounter:}\space#1}{}
% \newenvironment{subclaimproof}[1]{\begin{adjustwidth}{2em}{0pt}\par\noindent\emph{Proof of subclaim \theClaimCounter.\theSubClaimCounter:}\space#1}{\hfill
% $\blacksquare$ \emph{Subclaim \theClaimCounter.\theSubClaimCounter}\vspace{5mm}\end{adjustwidth}}
\newenvironment{subclaimproof}[1]{\par\noindent\emph{Proof of subclaim \theClaimCounter.\theSubClaimCounter:}\space#1}{\hfill
$\Diamond$ \emph{Subclaim \theClaimCounter.\theSubClaimCounter}}

\allowdisplaybreaks{}

% chktex-file 3

\title{MATH 502: Assignment IV}
\author{Evan P. Walsh}
\makeatletter
\makeatother
\lhead{Evan P. Walsh}
\chead{MATH 502: Assignment IV}
\rhead{\thepage}
\cfoot{}

\begin{document}
\maketitle

\subsection*{1}
\begin{Solution}
  \begin{enumerate}
    \item[(i)] Let $Y := \left\{ f \in X : \lim_{x\rightarrow \infty}f(x) = 0 \right\}$.

      Yes, this subspace is complete. To see this, let $\left\{ f_{n} \right\}$ be a Cauchy sequence in $Y$. Since $X$ is complete, there exists a limit $f
      \in X$ such that $f(x) = \lim_{n\rightarrow \infty}f_{n}(x)$ for all $x \in \mathbb{R}$. We aim to show that $f \in Y$. By way of contradiction,
      suppose $f \notin Y$, i.e. there exists $\epsilon > 0$ such that for each $M > 0$, there exists $x_{M} > M$ with 
      \begin{equation}
        |f(x_{M})| \geq \epsilon.
        \label{1.1}
      \end{equation}
      Now, since $f_{n} \rightarrow f$, there exists $N_{\epsilon} \in \mathbb{N}$ such that for all $n \geq N_{\epsilon}$, 
      \begin{equation}
        D(f_n, f) < \epsilon /2.
        \label{1.2}
      \end{equation}
      Now fix $n \geq N_{\epsilon}$. Then for each $M > 0$, $|f_{n}(x_{M})|\geq \epsilon / 2$ by \eqref{1.1} and \eqref{1.2}. Hence, taking
      $M \rightarrow \infty$, 
      \[
        \liminf_{M\rightarrow\infty} |f_{n}(x_{M})| \geq \epsilon / 2.
      \]
      This is a contradiction.

    \item[(ii)] Let $Y := \left\{ f\in X : f \text{ differentiable} \right\}$.

      No, this subspace is not complete. We will show this by constructing a sequence of differentiable functions that converge uniformly (i.e. w/
      respect to the metric $D$) to a
      function that is not differentiable. To that end, for each $n \geq 1$, let 
      \[
        f_{n}(x) := (x^{2} + n^{-1})^{1/2} - \left( \frac{x^2}{x^2 + 1} \right)\left( x^{2} + n^{-1} \right)^{1/2} \ \ x \in \mathbb{R}.
      \]
      Clearly this function is well-defined, continuous, and differentiable for each $n \geq 1$. To see that each $f_{n} \in X$, note that 
      \[
        \lim_{x\rightarrow+\infty}f_{n}(x) = \lim_{x\rightarrow -\infty}f_{n}(x) = \lim_{x\rightarrow\infty} \frac{(x^2 + n^{-1})^{1/2}}{x^2+1} = 0,
      \]
      and hence each $f_n$ is bounded. To see that $f_{n} \rightarrow f$ uniformly, let $n > m \geq 1$. Then 
      \begin{align*}
        |f_{n}(x) - f_{m}(x)| & = \bigg| \frac{1}{x^2 + 1}\left[ (x^2 + n^{-1})^{1/2} - (x^{2} + m^{-1})^{1/2} \right]\bigg| \\
        & \leq \bigg| \frac{1}{x^2 + 1}\left[ (x^2 + n^{-1})^{1/2} - |x| \right]\bigg| \\
        & \leq n^{-1} \ \text{ for all } x \in \mathbb{R}.
      \end{align*}
      Hence the convergence is uniform.
      However,
      \[
        \lim_{n\rightarrow \infty}f_{n}(x) = |x| - \frac{|x|^3}{x^2 + 1} \ \ \forall \ x\in \mathbb{R},
      \]
      which is not differentiable.

    \item[(iii)] Let $Y := \left\{ f \in X : \int_{0}^{1}f(x)dx = 0 \right\}$.

      Yes, $Y$ is a complete subspace. To see why, let $\left\{ f_{n} \right\}_{n=1}^{\infty}$ be a Cauchy sequence in $Y$. Let $f$ be the limit in
      $X$ of $\left\{ f_{n} \right\}$. We need to show that $f \in Y$. Well, since $f_n \rightarrow f$ uniformly on $\mathbb{R}$, and hence uniformly
      on $[0,1]$, there exists $N \in \mathbb{N}$ such that $D(f_n, f) \leq 1$ for all $n \geq N$. Hence 
      \begin{equation}
        |f_{n}(x)| \leq \sup_{x\in[0,1]}|f(x)| + 1
        \label{1.3}
      \end{equation}
      for all $n \geq N$. Thus, since \eqref{1.3} is satisfied for all but finitely many $n$, it must hold that the entire sequence of functions is
      bounded on $[0,1]$ (and also on $\mathbb{R}$) by some number $0 \leq M < \infty$. Hence, by the bounded convergence theorem (or dominated
      convergence theorem),
      \[
        0 = \lim_{n\rightarrow\infty}\int_{0}^{\infty}f_{n}(x)dx = \int_{0}^{1}f(x)dx.
      \]
      So $f \in Y$.
  \end{enumerate}
\end{Solution}


\subsection*{2}
\begin{Solution}
  \begin{claim}
    $(X,d')$ complete and Cauchy sequences in $(X,d')$ are also Cauchy in $(X,d)$.
  \end{claim}
  \begin{claimproof}
    Suppose $\left\{ x_n \right\}_{n=0}^{\infty}$ is Cauchy with respect to $d'$. Then $\left\{ y_n := \log(x_n) \right\}_{n=0}^{\infty}$ is Cauchy in
    $(\mathbb{R}, |\cdot|)$, and thus there exists $y \in \mathbb{R}$ such that $y = \lim y_n$. Now take $x := \exp(y) \in \mathbb{R}_{+}$. Thus,
    \[
      0 = \lim_{n\rightarrow\infty}|y_n - y| = \lim_{n\rightarrow \infty}|\log(x_n) - \log(x)| = \lim_{n\rightarrow \infty}d'(x_n, x).
    \]
    Hence $\left\{ x_n \right\}$ converges in $(X, d')$, so $(X, d')$ Cauchy. It remains to show that $\left\{ x_n \right\}$ Cauchy in $(X,d)$. Well,
    by continuity of $\exp(\cdot)$ with respect to the metric induced by $|\cdot|$,
    \begin{equation}
      0 = \lim_{n\rightarrow\infty}|\exp(y_n) - \exp(y)| = \lim_{n\rightarrow\infty}|x_n - x|.
      \label{3.1}
    \end{equation}
    Also, by continuity of $g(t) := t^{-1}$ on $\mathbb{R}_{+}$,
    \begin{equation}
      0 = \lim_{n\rightarrow\infty}|1/x_n - 1/x|.
      \label{3.2}
    \end{equation}
    So by \eqref{3.1} and \eqref{3.2},
    \[
      0 \lim_{n\rightarrow\infty}|x_n - x| + |1/x_n - 1/x| = \lim_{n\rightarrow\infty}d(x_n, x).
    \]
    So $\left\{ x_n \right\}$ is convergent in $(X,d)$ and therefore Cauchy.
  \end{claimproof}

  \begin{claim}
    $(X,d)$ is complete and Cauchy sequences in $(X,d)$ are also Cauchy in $(X,d')$.
  \end{claim}
  \begin{claimproof}
    The fact that $(X,d)$ is complete immediately follows from Proposition 2.4.7. Now suppose $\left\{ x_n \right\}_{n=0}^{\infty}$ is Cauchy in
    $(X,d)$. Let $x := \lim x_n$. Thus, 
    \[
      \lim_{n\rightarrow\infty}|x_n - x| \leq \lim_{n\rightarrow\infty}d(x_n, x) = 0. 
    \]
    Then by continuity of
    $\log(\cdot)$ with repsect to $|\cdot|$, 
    \[
      \lim_{n\rightarrow\infty}|\log(x_n) - \log(x)| = \lim_{n\rightarrow\infty}d'(x_n,x) = 0.
    \]
    Hence $\left\{ x_n \right\}$ is convergent in $(X, d')$ and hence Cauchy.
  \end{claimproof}

  By claims 1 and 2, $d, d'$ are Cauchy equivalent.

  \begin{claim}
    $d,d'$ are not strongly equivalent.
  \end{claim}
  \begin{claimproof}
    By way of contradiction suppose $d, d'$ are strongly equivalent. Then there exists $m, M > 0$ such that for each $x,y \in X$,
    \[
      md(x,y) \leq d'(x,y) \leq Md(x,y),
    \]
    which implies 
    \[
      m \leq \frac{d'(x,y)}{d(x,y)} = \frac{|\log(x) - \log(y)|}{|x - y| + |1/x - 1/y|} \leq M
    \]
    whenever $x \neq y$. However,
    \[
      \frac{d'(x,1)}{d(x,1)} = \frac{|\log(x)|}{|x| + |1/x|} \rightarrow 0 \ \ \text{as } x \rightarrow\infty.
    \]
    This is a contradiction.
  \end{claimproof}

\end{Solution}


\subsection*{3}
\begin{Solution}
  No, $(Y,d_{Y})$ is not necessarily complete. Consider $(X, d_{X}) := (\mathbb{R}, |\cdot|)$ and $(Y,d_{Y}) := \big( (0,1), |\cdot|\big)$.
  Then take $f : X \rightarrow Y$ as $f(x) := e^{x} / (e^{x} + 1)$. So $f$ is a continuous bijection but clearly $(Y, d_{Y})$ is not complete.
\end{Solution}


\subsection*{4}
\begin{Solution}
  It is clear from the definition of each $f_{n}$ that 
  \[
    \lim_{n\rightarrow\infty} f_{n}(x) = f(x) := \left\{ \begin{array}{cl}
        0 & \text{ if } x = 0 \\
        0 & \text{ if } x \in [0,1] - \mathbb{Q} \\
        \frac{1}{q} & \text{ if } x = \frac{p}{q}
    \end{array} \right.,\ \  x \in [0,1].
  \]
  \begin{claim}
    The convergence is uniform.
  \end{claim}
  \begin{claimproof}
    Note that for any $n \in \mathbb{N}$, $|f_n(x) - f(x)| > 0$ if and only if $x = \frac{p}{q}$ where $q > n$, in which case $|f_n(x) - f(x)| =
    \frac{1}{q} < \frac{1}{n}$.
    Hence
    \[
      \sup_{x\in[0,1]}|f_n(x) - f(x)| \leq \frac{1}{n} \rightarrow 0
    \]
    as $n\rightarrow\infty$, i.e. $f_{n}$ converges to $f$ uniformly.
  \end{claimproof}

  \begin{claim}
    $f$ continuous at $0$ and every irrational in $[0,1]$.
  \end{claim}
  \begin{claimproof}
    First we will show that $f$ is continuous at 0. Let $x_0, x_1, x_2, \dots \in (0,1]$ such that $x_n \rightarrow 0$. 
    Without loss of generality we can assume $x_n \in \mathbb{Q}$ for all $n \in \mathbb{N}$.
    It suffices to
    show that $f(x_n) \rightarrow 0$. For each $n \in \mathbb{N}$, let $q_n, p_{n} \geq 1$ such that $x_n = \frac{p_n}{q_n}$. Then clearly $q_n \rightarrow
    \infty$ as $n \rightarrow \infty$. But this means precisely that $f(x_n) \rightarrow 0$.

    Now suppose $x \in [0,1] - \mathbb{Q}$. Let $x_0, x_1, \dots \in (0,1]$ such that $x_n \rightarrow x$. Again, assume $x_n \in \mathbb{Q}$ for each
    $n \in \mathbb{N}$ and write $x_n = \frac{p_n}{q_n}$ where $q_n, p_n \geq 1$. Then it suffices to show that $q_n \rightarrow \infty$ to show that
    $f(x_n) \rightarrow 0$. So, by way
    of contradiction suppose not, i.e. suppose there exists $N \in \mathbb{N}$ such that $q_n \leq N$ infinitely often. But since $p_n \leq q_n$,
    there are only finitely many different values that $x_n$ can take on if $p_n \leq q_n \leq N$. Hence the sequence repeats a certain value infinitely
    often. But this repeated value cannot possibly be $x$ itself, since $x$ is irrational. Hence $\left\{ x_n \right\}$ does not converge to $x$. This
    is a contradiction.
  \end{claimproof}

\end{Solution}

\subsection*{5}
\begin{Solution}
  \begin{enumerate}
    \item[(5)] Consider $\mathbb{R}$ with the usual metric induced by $|\cdot|$. For $n\geq 1$, let $F_n := [n, \infty)$. Then clearly each $F_n$ is
      closed and $F_1 \supset F_2 \supset \dots$. However, $\text{diam}(F_n) = \infty$ for all $n \in \mathbb{N}$, and 
      \[
        \bigcap_{n=1}^{\infty}F_{n} = \emptyset.
      \]

    \item[(7)] Let $\theta > 0$ and define $F_n := \left\{ t \in [0,1]: |f_n(t) - f_k(t)| \leq \theta \text{ for all } k \geq n \right\}$.
      \begin{enumerate}
        \item Note that 
          \begin{align*}
            F_n & = \bigcap_{k\geq n}\left\{ t : |f_n(t) - f_k(t)| \leq \theta \right\}\\
            & = \bigcap_{k\geq n}\left\{ t : f_n(t) - f_k(t)\leq \theta \right\} \cap \left\{ t : f_k(t) - f_n(t) \leq \theta \right\}\\
            & = \bigcap_{k\geq n}(f_n - f_k)^{-1}\big[(-\infty,\theta]\big] \cap  (f_k - f_n)^{-1}\big[(-\infty,\theta]\big] \\
          \end{align*}
          which is closed as the countable intersection of closed sets, since $f_n - f_k$ and $f_k - f_n$ are continuous. Therefore it remains to show
          that $[0,1] = \cup_{n=1}^{\infty}F_n$. By way of contradiction suppose there exists $x' \in [0,1]$ such that $x \notin
          \cup_{n=1}^{\infty}F_n$. Then $x \notin F_n$ for all $n \in \mathbb{N}$, which implies that for all $n \in \mathbb{N}$, there exists $k \geq
          n$ such that $|f_n(x') - f_k(x')| > \theta$. Therefore $\left\{ f_n(x') \right\}_{n=1}^{\infty}$ is not Cauchy, and thus not convergent.
          This is a contradiction.

        \item Let $\epsilon > 0$ and define $F_n' := \left\{ t \in I: |f_n(t) - f_k(t)| \leq \epsilon / 3 \text{ for all } k \geq n \right\}$, $n\geq
          1$.
          By the work done in part (a), each $F_n'$ is closed and $\cup_{n=1}^{\infty}F_n' = I$. Thus, since $I$ is non-trivial and therefore has a
          nonempty interior $\mathring{I}$, there exists an $n_0 \geq 1$ such that $F_{n_0}'$ has a nonempty interior. Hence there exists a
          non-trivial open subinterval $O \subseteq \mathring{F_{n_0}'}$ by Corollary 2.4.17, since $\mathring{F_{n_0}'}$ is open and open sets in
          $\mathbb{R}$ are composed of
          the union of countable many open intervals. Now let $x_0 \in O$. 
          Since $f_{n_0}$ is continuous, we can choose $\delta > 0$ small enough such that $f_{n_0}[B_{\delta}(x_0)] \subset B_{\epsilon / 6}(f_{n_0}(x_0))$.
          Now let $J := [x_0 - \delta / 2, x_0 + \delta / 2] \subset \mathring{F_{n_0}} \subset \mathring{I}$. Then for each $s, t \in J$,
          \begin{equation}
            |f_{n_0}(s) - f_{n_0}(t)| < 2\times \frac{\epsilon}{6} = \frac{\epsilon}{3}.
            \label{6.1}
          \end{equation}

          \begin{claim}
            $|f(s) - f(t)| < \epsilon$ for all $s,t \in J$.
          \end{claim}
          \begin{claimproof}
            Note that since $f_{n}(s) \rightarrow f(s)$ for each $s \in J$, 
            \begin{equation}
              |f_{n_0}(s) - f(s)| = \lim_{k\rightarrow\infty}|f_{n_0}(s) - f_{k}(s)| \leq \epsilon / 6
              \label{6.2}
            \end{equation}
            since $s \in F_{n_{0}}$. Thus, by \eqref{6.1} and \eqref{6.2},
            \[
              |f(s) - f(t)| \leq |f(s) - f_{n_0}(s)| + |f_{n_0}(s) - f_{n_0}(t)| + |f_{n_0}(t) - f(t)| \leq 3\times \frac{\epsilon}{3} =
              \epsilon.
            \]
          \end{claimproof}

        \item This follows directly from applying the result in part (b) inductively. By part (b), we can choose a closed subinterval $I_1 \subset
          \mathring{I}$ such that
          $|f(t) - f(s)| \leq 1$ for all $s, t \in I_1$ and so that $\text{len}(I_1) \leq 1$. Note that since $I_1 \subseteq [0,1]$, it is trivial
          that $\text{len}(I_1) \leq 1$. However, given the construction in part (b), it is clear that we can make these sets arbitrarily small, which
          is important for the next step. Now, by way of induction, suppose $I_n \subset \mathring{I}_{n-1}$, for some $n\geq 1$, has been chosen such that $|f(t) - f(s)| \leq
          \frac{1}{n}$ for all $s,t \in I_n$ and $\text{len}(I_n) \leq \frac{1}{n}$. Then apply part (b) with $I_n$ as the ``reference set $I$'' and $\epsilon :=
          \frac{1}{n+1}$, and ensure that the resulting set $J \equiv I_{n+1}$ has length less than $\frac{1}{n+1}$.

          In this way we now have a sequence $I_1, I_2, \dots$ satisfying the conditions that we wished to achieve. But by Cantor's intersection
          theorem, $\cap_{n=1}^{\infty}F_{n}$ contains precisily one point $x_0 \in [0,1]$, and at $x_0$, we claim that $f$ is continuous.

          \begin{claim}
            $f$ continuous at $x_0$.
          \end{claim}
          \begin{claimproof}
            Let $\epsilon > 0$. Then choose $n' \geq 1$ such that $\frac{1}{n'} < \epsilon$. Since $x_0 \in F_n$ for all $n \geq 1$,
            $x_0 \in I_{n'+1} \subset \mathring{I}_{n'}$. Hence there exists $\delta > 0$ such that $B_{\delta}(x_0) \subset \mathring{I}_{n'}$.
            Thus, if $x \in B_{\delta}(x_0)$, then $x \in I_{n'}$. So by definition of $I_{n'}$,
            \[
              |f(x) - f(x_0)| \leq \frac{1}{n'} < \epsilon.
            \]
            So $f$ continuous at $x_0$.
          \end{claimproof}

        \item By way of contradiction suppose that the set of points at which $f$ is continuous is not dense in $[0,1]$. Then there exists $x_0 \in
          [0,1]$ and $r > 0$ such that $f$ is discontinuous for all points in $I := B_{r}(x_0)$. Let $\epsilon > 0$. Now, by part (c), there exists a
          sequence $I\supset I_1 \supset \mathring{I_1} \supset I_2 \supset \mathring{I_2} \dots$ of closed subintervals of $I$ such that $|f(t) -
          f(s)| \leq \frac{1}{n}$ for all $s, t, \in I_n$. But then by Claim 2, if $x_0 \in I$ such that $\{x_0\} = \cap_{n=1}^{\infty}I_n$, then $f$
          is continuous at $x_0$. This is a contradiction.

      \end{enumerate}
  \end{enumerate}
\end{Solution}


\end{document}
