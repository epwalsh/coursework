\documentclass[12pt]{article}
\usepackage{amsmath}
\usepackage{amsfonts}
\usepackage{parskip}
\usepackage{amsthm}
\usepackage{thmtools}
\usepackage[headheight=15pt]{geometry}
\geometry{a4paper, left=20mm, right=20mm, top=30mm, bottom=30mm}
\usepackage{graphicx}
\usepackage{bm} % for bold font in math mode - command is \bm{text}
\usepackage{enumitem}
\usepackage{fancyhdr}
\usepackage{amssymb} % for stacked arrows and other shit
\pagestyle{fancy}
\usepackage{changepage}
\usepackage{mathcomp}
\usepackage{tcolorbox}
\usepackage{eufrak}

\declaretheoremstyle[headfont=\normalfont]{normal}
\declaretheorem[style=normal]{Theorem}
\declaretheorem[style=normal]{Proposition}
\declaretheorem[style=normal]{Lemma}
\newcounter{ProofCounter}
\newcounter{ClaimCounter}[ProofCounter]
\newcounter{SubClaimCounter}[ClaimCounter]
\newenvironment{Proof}{\stepcounter{ProofCounter}\textsc{Proof.}}{\hfill$\square$}
\newenvironment{Solution}{\stepcounter{ProofCounter}\textbf{Solution:}}{\hfill$\square$}
\newenvironment{claim}[1]{\vspace{1mm}\stepcounter{ClaimCounter}\par\noindent\underline{\bf Claim \theClaimCounter:}\space#1}{}
\newenvironment{claimproof}[1]{\par\noindent\underline{Proof of claim \theClaimCounter:}\space#1}{\hfill $\blacksquare$ Claim \theClaimCounter}
\newenvironment{subclaim}[1]{\stepcounter{SubClaimCounter}\par\noindent\emph{Subclaim \theClaimCounter.\theSubClaimCounter:}\space#1}{}
% \newenvironment{subclaimproof}[1]{\begin{adjustwidth}{2em}{0pt}\par\noindent\emph{Proof of subclaim \theClaimCounter.\theSubClaimCounter:}\space#1}{\hfill
% $\blacksquare$ \emph{Subclaim \theClaimCounter.\theSubClaimCounter}\vspace{5mm}\end{adjustwidth}}
\newenvironment{subclaimproof}[1]{\par\noindent\emph{Proof of subclaim \theClaimCounter.\theSubClaimCounter:}\space#1}{\hfill
$\Diamond$ \emph{Subclaim \theClaimCounter.\theSubClaimCounter}}

\allowdisplaybreaks{}

% chktex-file 3

\title{MATH 502: Assignment VII}
\author{Evan P. Walsh}
\makeatletter
\makeatother
\lhead{Evan P. Walsh}
\chead{MATH 502: Assignment VII}
\rhead{\thepage}
\cfoot{}

\begin{document}
\maketitle


\subsection*{1}
\begin{Solution}
  Let $\tau := \{\emptyset\} \cup \left\{ U \subseteq \mathbb{N} : \liminf_{n\rightarrow\infty}\frac{ |U\cap\{1,\dots,n\}|}{n} = 1  \right\}$.

  To verify that $\tau$ is indeed a topology on $\mathbb{N}$, note that clearly $\mathbb{N} \in \tau$, and if $U, V \in \tau$, then either $U \cap V =
  \emptyset \in \tau$, or 
  \begin{align*}
    1 \geq \liminf_{n\rightarrow\infty} \frac{|U\cap V\cap\{1,\dots, n\}|}{n} & = \liminf_{n\rightarrow\infty}\frac{|U\cap\{1,\dots,n\} -
    (\{1,\dots,n\}-V)|}{n} \\
    \text{(since $|\{1,\dots,n\} - V| < \infty$) } & = \liminf_{n\rightarrow\infty} \frac{|U\cap\left\{ 1,\dots, n \right\}|}{n} - \frac{|\left\{ 1,
    \dots, n\right\} - V|}{n} \\
    & \geq \liminf_{n\rightarrow\infty} \frac{|U\cap\left\{ 1,\dots,n \right\}|}{n} - \limsup_{n\rightarrow\infty}\frac{|\left\{ 1,\dots,n \right\} -
    V|}{n} \\
    & = \liminf_{n\rightarrow\infty} \frac{|U\cap\left\{ 1,\dots,n \right\}|}{n} - \limsup_{n\rightarrow\infty}\frac{n - |\left\{ 1,\dots,n
      \right\}\cap V|}{n} \\
      & \geq \liminf_{n\rightarrow\infty} \frac{|U\cap\left\{ 1,\dots,n \right\}|}{n} + 1 - \liminf_{n\rightarrow\infty}\frac{|\left\{ 1,\dots,
      n\right\}\cap V|}{n} \\
      & = 1,
  \end{align*}
  and hence $U\cap V \in \tau$. Lastly, suppose $\{U_{\lambda} : \lambda \in \Lambda\} \subseteq \tau$. Without loss of generality assume 
  there exists at least one $\lambda' \in \Lambda$ such that 
  $U_{\lambda'} \neq \emptyset$. Then
  \[
    1 \geq \liminf_{n\rightarrow\infty}\frac{\big|\bigcup_{\lambda\in\Lambda}U_{\lambda}\cap\left\{ 1,\dots,n \right\}\big|}{n} \geq
    \liminf_{n\rightarrow\infty}\frac{|U_{\lambda'}\cap\left\{ 1,\dots,n \right\}|}{n} = 1.
  \]
  Hence $\tau$ is a topology.

  \begin{claim}
    $\tau$ is not Hausdorff.
  \end{claim}
  \begin{claimproof}
    By way of contradiction suppose $\tau$ is Hausdorff. Then if $x\neq y \in \mathbb{N}$, there exists $U, V \in \tau$ such that $x \in U$, $y \in V$
    and $U\cap V = \emptyset$. But then since $U\cup V \in \tau$ and $U\cap V \neq \emptyset$,
    \begin{align*}
      1 = \liminf_{n\rightarrow\infty}\frac{|(U\cup V)\cap\left\{ 1,\dots,n \right\}|}{n} & = \liminf_{n\rightarrow\infty} \frac{|U \cap \left\{ 
      1,\dots,n\right\}|}{n} + \frac{|V\cap\left\{ 1,\dots,n \right\}|}{n} \\
      & \geq  \liminf_{n\rightarrow\infty} \frac{|U \cap \left\{1,\dots,n\right\}|}{n} + \liminf_{n\rightarrow\infty}\frac{|V\cap\left\{ 1,\dots,n
      \right\}|}{n} = 2.
    \end{align*}
    This is a contradiction. Hence $\tau$ is not Hausdorff.
  \end{claimproof}

\end{Solution}


\subsection*{2}
\begin{Solution}
  Let $\tau$ be defined by $U \in \tau$ if $U = \emptyset$ or $\mathbb{R} - U$ countable.
  \begin{enumerate}
    \item[(a)] Yes, $f$ is continuous. Note that if $F$ closed in $\tau$, then $F$ countable. Therefore $f^{-1}[F]$ is clearly also countable, and
      thus is closed in $\tau$ as well.

    \item[(b)] No. Note that $[0,1]$ is closed in $|\cdot|$, but $f^{-1}([0,1]) = [-1,1]$ is not closed in $\tau$ since it is uncountable.

    \item[(c)] No. Let $F := f[\mathbb{Q}]$, which is countable and therefore closed in $\tau$. But $f^{-1}[F] = \mathbb{Q}$ is not closed in
      $|\cdot|$.
  \end{enumerate}
\end{Solution}


\subsection*{3}
\begin{Solution}
  Let $I$ be a proper ideal in $\mathcal{R}$. Let $\mathcal{S}$ be the set of all proper ideals in $\mathcal{R}$ containing $I$. Then $\mathcal{S}$ is an ordered
  set with the ordering of set inclusion, i.e. we say $I_1 \preceq I_2$ iff $I_1 \subseteq I_2$. Now let $\mathcal{A}$ be a totally ordered subset of
  $\mathcal{S}$.

  \begin{claim}
    $\mathcal{A}$ has an upper bound.
  \end{claim}
  \begin{claimproof}
    Let $\tilde{I} := \cup_{I\in \mathcal{A}}I$. 

    \begin{subclaim}
      $\tilde{I}$ is an ideal. 
    \end{subclaim}
    \begin{subclaimproof}
      Let $a \in \tilde{I}$. Then $a \in I_a$ for some $I_a \in \mathcal{A}$. Hence $ra \in I_a \subseteq \tilde{I}$ for all $r \in \mathcal{R}$. Now
      take any $a, b \in \tilde{I}$. Then there exists $I_a, I_b \in \mathcal{A}$ such that $a \in I_a$ and $b \in I_b$. Since $\mathcal{A}$ is
      totally ordered, either $I_a \subseteq I_b$ or $I_b \subseteq I_a$. Without loss of generality assume $I_a \subseteq I_b$. Then $a, b \in I_b$,
      and thus $a - b \in I_b \subseteq \tilde{I}$.
    \end{subclaimproof}
    
    To see that $\tilde{I}$ is a proper ideal, assume not. Then $1 \in
    \tilde{I}$ and hence
    $1 \in I_0$ for some $I_0 \in \mathcal{A}$. But then $I_0$ is not a proper ideal. This is a contradiction. Thus $\tilde{I}$ is proper.
    Also clearly $I \subseteq \tilde{I}$.
    Hence $\tilde{I} \in \mathcal{A}$, and since
    $I' \preceq \tilde{I}$ for
    all $I' \in \mathcal{A}$, we have that $\tilde{I}$ is an upper bound for $\mathcal{A}$.

  \end{claimproof}

  By Claim 1, the conditions for Zorn's Lemma apply to $\mathcal{S}$. Hence $\mathcal{S}$ has a maximal element, i.e. $I$ is contained in a maximal
  ideal.
\end{Solution}



\newpage
\subsection*{4}
\begin{Solution}
  \begin{enumerate}
    \item[4.] Let $S' := \left\{ x \in X : x \text{ is the limit of a sequence in }S \right\}$.

      First we will show that $\overline{S} \subseteq S'$. 
      Let $x \in \overline{S}$ and let $\mathcal{B}_x = \left\{ B_k \right\}_{k=0}^{\infty}$ be a countable base for $\mathcal{N}_x$. Then for
      each $n \in \mathbb{N}$, there exists 
      \[
        x_n \in \left( \bigcap_{k=0}^{n}B_k \right) \cap S
      \]
      by Proposition 3.1.18. We claim then that $x_n \rightarrow x$. Thus, we need to show that for any $N \in \mathcal{N}_x$, there exists $n_{N} \in
      \mathbb{N}$ such that if $n \geq n_{N}$, then $x_n \in N$. Well, since $\mathcal{B}_x$ is a base, there exists $n_{N} \in \mathbb{N}$ such that
      $B_{n_N} \subseteq N$, and therefore $x_n \in \cap_{k=0}^{n}B_{k} \subseteq \cap_{k=0}^{n_{N}}B_k \subseteq N$ for all $n \geq n_{N}$.
      So $x = \lim x_n$. Hence $\overline{S} \subseteq S'$.

      Now let $x \in S'$ so that there exists a sequence $\left\{ x_n \right\}_{n=0}^{\infty}$ in $S$ converging to $x$. By way of contradiction
      suppose $x \in U:= X - \overline{S}$. Then there exists $n_{U} \in \mathbb{N}$ such that if $n \geq n_{U}$ then $x_n \in U$. But this is a
      contradiction since $x_n \in S \subseteq \overline{S}$ for all $n \in \mathbb{N}$. Thus $x \in \overline{S}$, and so $S' \subseteq
      \overline{S}$.

    \item[5.] (a) Let $I$ be an ideal of $S$. Then for $a,b \in \phi^{-1}[I]$ and $r \in R$, $\phi(a) - \phi(b) \in I$ and $\phi(r)\phi(a) \in I$, so
      $a - b, ra \in \phi^{-1}[I]$ since $\phi$ is a homomorphism. Further, if $I$ is proper, then $1 \notin I$, and therefore $1 \notin \phi^{-1}[I]$
      since $\phi$ is unital. Hence $\phi^{-1}[I]$ proper when $I$ is proper. Lastly, if $I$ is prime, then for $a, b \in R$, if $ab \in
      \phi^{-1}[I]$, then $\phi(ab) = \phi(a)\phi(b) \in I$, and so either $\phi(a) \in I$ or $\phi(b) \in I$. Without loss of generality assume
      $\phi(a) \in I$. Then clearly $a \in \phi^{-1}[I]$. Hence $\phi^{-1}[I]$ is prime when $I$ is prime.

      (b) Let $F$ be closed in in Spec$(\mathcal{R})$. Then there exists an ideal $I$ of $\mathcal{R}$ such that $F = V(I)$. Note that since
      $\phi^{-1}$ is also a unital homomorphism, $\phi(I)$ is an ideal of $\mathcal{S}$ by part (a). Thus,
      \begin{align*}
        (\phi^*)^{-1}[F] & = \left\{ p \in \text{Spec}(\mathcal{S}) : \phi^*(p) \in F \right\} \\
        & = \left\{ p \in \text{Spec}(\mathcal{S}) : \phi^{-1}[p] \in F \right\} \\
        & = \left\{ p \in \text{Spec}(\mathcal{S}) : \phi^{-1}[p] \in \text{Spec}(\mathcal{R}) \wedge \phi^{-1}[p] \supseteq I \right\} \\
        \text{(by part (a), since $\phi^{-1}[p]$ is a prime ideal)}\  & = \left\{ p \in \text{Spec}(\mathcal{S}) : \phi^{-1}[p] \supseteq I \right\} \\
        & = V(\phi(I)),
      \end{align*}
      which by definition is closed in Spec$(\mathcal{S})$. Hence $\phi^*$ is continuous.

    \item[11.] Let $C[0,1]:=C([0,1], \mathbb{F})$.
      Let $f : [0,1] \rightarrow \mathcal{F}$ and $N \in N_f$. Thus there exists $C = \{x_1, \dots, x_n\} \in
      \mathcal{F}$ and $\epsilon > 0$ with $N_{f,C,\epsilon} \subseteq N$. But since $\{x_1, \dots, x_n\}$ is a finite set of points in $[0,1]$,
      there is a polynomial $p$ such that $\sup_{1\leq k \leq n}|f(x_k) - p(x_k)| < \epsilon$. Thus there exists a continuous function $p \in
      N_{f,C,\epsilon}$. So $N \cap C[0,1] \supseteq N_{f,C,\epsilon}\cap C[0,1] \neq \emptyset$, i.e. $C[0,1]$ is dense in
      $(F([0,1],\mathbb{F}),\mathcal{T}_{\mathcal{F}})$.

      Now define $f : [0,1] \rightarrow \mathcal{F}$ by
      \[
        f(x) := \left\{ \begin{array}{cl}
            1 & \text{ if } t \notin [0,1] \cap \mathbb{Q}, \\
            0 & \text{ if } t \in [0,1] \cap \mathbb{Q}.
        \end{array} \right.
      \]
      We claim that there is no sequence of continuous functions from $[0,1]$ to $\mathbb{F}$ that converges in $\mathcal{T}_{\mathcal{F}}$ to $f$.
      Well, from Example 3.2.13(b), we know that convergence with respect to $\mathcal{T}_{\mathcal{F}}$ is pointwise convergence. However, by
      Exercise 2.4.7, if $f$ is the limit of continuous functions in $C[0,1]$, then $f$ is continuous on a dense set in $[0,1]$.
      However, this is clearly not the case since $f$ is nowhere continuous.
  \end{enumerate}
\end{Solution}

\subsection*{5}
\begin{Solution}
  Let $\mathcal{B} := \left\{ U_{a,b} := [a,b) : a \leq b \wedge a,b\in \mathbb{R} \right\}$ and let $\tau$ be the topology associated with the base $\mathcal{B}$.
  Also let $|\cdot|$ denote the usual topology on $\mathbb{R}$.

  (a) Let $U \in |\cdot|$. Then $U$ is the union of a countable number of open intervals. Hence, to show that $U \in \tau$, it suffices to show that
  any open interval $(a,b)$, $a,b \in \mathbb{R}$, can be expressed as the union of basic sets from $\mathcal{B}$, which clearly holds since 
  \[
    (a,b) = \bigcup_{n=0}^{\infty}[a + 2^{-n}, b) \in \tau.
  \]

  (b) By part (c) and Exercise 3.1.4(a), $\tau$ is first countable.

  (c) Yes, $\tau$ is second countable. Consider the collection of open sets $\mathcal{B}' := \left\{ U_{a,b} : a \leq b \wedge a,b \in \mathbb{Q} \right\}$. Clearly
  $\mathcal{B}'$ is countable and whenever $a < b$, $a,b \in \mathbb{R}$, there exists sequences $\left\{ r_{a,n} \right\}_{n=0}^{\infty}$ and 
  $\left\{ r_{b,n} \right\}_{n=0}^{\infty}$ in $\mathbb{Q}$ such that $r_{a,n} \leq a < r_{b,n} < b$ for all $n \in \mathbb{N}$ since $\mathbb{Q}$ is
  dense in $\mathbb{R}$ with respect to $|\cdot|$. Thus $U_{r_{a,n},r_{b,n}} \in \mathcal{B}'$ for all $n \in \mathbb{N}$ and 
  \[
    U_{a,b} = \bigcup_{n=0}^{\infty}U_{r_{a,n},r_{b,n}}.
  \]
  Hence, since any basic set $U_{a,b} \in \mathcal{B}$ can be expressed as the union of open sets in $\mathcal{B}'$, $\mathcal{B}'$ serves as a base
  as well. So $\tau$ is second countable since $\mathcal{B}'$ is countable.

  (d) Yes, $\tau$ is separable. We claim that $\mathbb{Q}$ is dense in $\tau$. By Exercise 3.2.4 and part (b), it suffices to show that for any $x \in
  \mathbb{R}$, $x$ is the limit of a sequence of rationals with respect to $\tau$. Let $N \in \mathcal{N}_x$. Then there exists $a < b$ such that $x \in U_{a,b} \subset N$.
  Since $\mathbb{Q}$ is dense with respect to $|\cdot|$, there exists a sequence of rationals $\left\{ r_n \right\}_{n=0}^{\infty}$ such that $r_n
  \rightarrow x$ and $r_n > x$ for all $n \in \mathbb{N}$. Hence there exists $n_{a,b} \in\mathbb{N}$ such that for $|r_n - x| < |b - x|$,
  and therefore $x < r_n < b$, for all $n
  \geq n_{a,b}$. Hence $r_n \in U_{a,b}$ for all $n \geq n_{a,b}$. So $x = \lim r_n$ with respect to $\tau$.
\end{Solution}

\subsection*{6}
\begin{Solution}
  By the work done in 5(c), $\tau' \supseteq \tau$. But since $\mathcal{B}' \subseteq \mathcal{B}$, we also clearly have $\tau' \subseteq \tau$.
  Hence the answers to this problem are the same as to problem 5.
\end{Solution}

\end{document}
