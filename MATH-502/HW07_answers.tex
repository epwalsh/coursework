\documentclass[12pt]{article}
\usepackage{amsmath}
\usepackage{amsfonts}
\usepackage{parskip}
\usepackage{amsthm}
\usepackage{thmtools}
\usepackage[headheight=15pt]{geometry}
\geometry{a4paper, left=20mm, right=20mm, top=30mm, bottom=30mm}
\usepackage{graphicx}
\usepackage{bm} % for bold font in math mode - command is \bm{text}
\usepackage{enumitem}
\usepackage{fancyhdr}
\usepackage{amssymb} % for stacked arrows and other shit
\pagestyle{fancy}
\usepackage{changepage}
\usepackage{mathcomp}
\usepackage{tcolorbox}
\usepackage{eufrak}

\declaretheoremstyle[headfont=\normalfont]{normal}
\declaretheorem[style=normal]{Theorem}
\declaretheorem[style=normal]{Proposition}
\declaretheorem[style=normal]{Lemma}
\newcounter{ProofCounter}
\newcounter{ClaimCounter}[ProofCounter]
\newcounter{SubClaimCounter}[ClaimCounter]
\newenvironment{Proof}{\stepcounter{ProofCounter}\textsc{Proof.}}{\hfill$\square$}
\newenvironment{Solution}{\stepcounter{ProofCounter}\textbf{Solution:}}{\hfill$\square$}
\newenvironment{claim}[1]{\vspace{1mm}\stepcounter{ClaimCounter}\par\noindent\underline{\bf Claim \theClaimCounter:}\space#1}{}
\newenvironment{claimproof}[1]{\par\noindent\underline{Proof of claim \theClaimCounter:}\space#1}{\hfill $\blacksquare$ Claim \theClaimCounter}
\newenvironment{subclaim}[1]{\stepcounter{SubClaimCounter}\par\noindent\emph{Subclaim \theClaimCounter.\theSubClaimCounter:}\space#1}{}
% \newenvironment{subclaimproof}[1]{\begin{adjustwidth}{2em}{0pt}\par\noindent\emph{Proof of subclaim \theClaimCounter.\theSubClaimCounter:}\space#1}{\hfill
% $\blacksquare$ \emph{Subclaim \theClaimCounter.\theSubClaimCounter}\vspace{5mm}\end{adjustwidth}}
\newenvironment{subclaimproof}[1]{\par\noindent\emph{Proof of subclaim \theClaimCounter.\theSubClaimCounter:}\space#1}{\hfill
$\Diamond$ \emph{Subclaim \theClaimCounter.\theSubClaimCounter}}

\allowdisplaybreaks{}

% chktex-file 3

\title{MATH 502: Assignment VII}
\author{Evan P. Walsh}
\makeatletter
\makeatother
\lhead{Evan P. Walsh}
\chead{MATH 502: Assignment VII}
\rhead{\thepage}
\cfoot{}

\begin{document}
\maketitle


\subsection*{1}
\begin{Solution}
  Let $\tau := \{\emptyset\} \cup \left\{ U \subseteq \mathbb{N} : \liminf_{n\rightarrow\infty}\frac{ |U\cap\{1,\dots,n\}|}{n} = 1  \right\}$.

  To verify that $\tau$ is indeed a topology on $\mathbb{N}$, note that clearly $\mathbb{N} \in \tau$, and if $U, V \in \tau$, then either $U \cap V =
  \emptyset \in \tau$, or 
  \begin{align*}
    1 \geq \liminf_{n\rightarrow\infty} \frac{|U\cap V\cap\{1,\dots, n\}|}{n} & = \liminf_{n\rightarrow\infty}\frac{|U\cap\{1,\dots,n\} -
    (\{1,\dots,n\}-V)|}{n} \\
    \text{(since $|\{1,\dots,n\} - V| < \infty$) } & = \liminf_{n\rightarrow\infty} \frac{|U\cap\left\{ 1,\dots, n \right\}|}{n} - \frac{|\left\{ 1,
    \dots, n\right\} - V|}{n} \\
    & \geq \liminf_{n\rightarrow\infty} \frac{|U\cap\left\{ 1,\dots,n \right\}|}{n} - \limsup_{n\rightarrow\infty}\frac{|\left\{ 1,\dots,n \right\} -
    V|}{n} \\
    & = \liminf_{n\rightarrow\infty} \frac{|U\cap\left\{ 1,\dots,n \right\}|}{n} - \limsup_{n\rightarrow\infty}\frac{n - |\left\{ 1,\dots,n
      \right\}\cap V|}{n} \\
      & \geq \liminf_{n\rightarrow\infty} \frac{|U\cap\left\{ 1,\dots,n \right\}|}{n} + 1 - \liminf_{n\rightarrow\infty}\frac{|\left\{ 1,\dots,
      n\right\}\cap V|}{n} \\
      & = 1,
  \end{align*}
  and hence $U\cap V \in \tau$. Lastly, suppose $\{U_{\lambda} : \lambda \in \Lambda\} \subseteq \tau$. Without loss of generality assume 
  there exists at least one $\lambda' \in \Lambda$ such that 
  $U_{\lambda'} \neq \emptyset$. Then
  \[
    1 \geq \liminf_{n\rightarrow\infty}\frac{\big|\bigcup_{\lambda\in\Lambda}U_{\lambda}\cap\left\{ 1,\dots,n \right\}\big|}{n} \geq
    \liminf_{n\rightarrow\infty}\frac{|U_{\lambda'}\cap\left\{ 1,\dots,n \right\}|}{n} = 1.
  \]
  Hence $\tau$ is a topology.

  \begin{claim}
    $\tau$ is not Hausdorff.
  \end{claim}
  \begin{claimproof}
    By way of contradiction suppose $\tau$ is Hausdorff. Then if $x\neq y \in \mathbb{N}$, there exists $U, V \in \tau$ such that $x \in U$, $y \in V$
    and $U\cap V = \emptyset$. But then since $U\cup V \in \tau$ and $U\cap V \neq \emptyset$,
    \begin{align*}
      1 = \liminf_{n\rightarrow\infty}\frac{|(U\cup V)\cap\left\{ 1,\dots,n \right\}|}{n} & = \liminf_{n\rightarrow\infty} \frac{|U \cap \left\{ 
      1,\dots,n\right\}|}{n} + \frac{|V\cap\left\{ 1,\dots,n \right\}|}{n} \\
      & \geq  \liminf_{n\rightarrow\infty} \frac{|U \cap \left\{1,\dots,n\right\}|}{n} + \liminf_{n\rightarrow\infty}\frac{|V\cap\left\{ 1,\dots,n
      \right\}|}{n} = 2.
    \end{align*}
    This is a contradiction. Hence $\tau$ is not Hausdorff.
  \end{claimproof}

\end{Solution}


\subsection*{2}
\begin{Solution}
  Let $\tau$ be defined by $U \in \tau$ if $U = \emptyset$ or $\mathbb{R} - U$ countable.
  \begin{enumerate}
    \item[(a)] Yes, $f$ is continuous. Note that if $F$ closed in $\tau$, then $F$ countable. Therefore $f^{-1}[F]$ is clearly also countable, and
      thus is closed in $\tau$ as well.

    \item[(b)] No. Note that $[0,1]$ is closed in $|\cdot|$, but $f^{-1}([0,1]) = [-1,1]$ is not closed in $\tau$ since it is uncountable.

    \item[(c)] No. Let $F := f[\mathbb{Q}]$, which is countable and therefore closed in $\tau$. But $f^{-1}[F] = \mathbb{Q}$ is not closed in
      $|\cdot|$.
  \end{enumerate}
\end{Solution}


\subsection*{3}
\begin{Solution}
  Let $I$ be a proper ideal in $\mathcal{R}$. Let $\mathcal{S}$ be the set of all proper ideals in $\mathcal{R}$ containing $I$. Then $\mathcal{S}$ is an ordered
  set with the ordering of set inclusion, i.e. we say $I_1 \preceq I_2$ iff $I_1 \subseteq I_2$. Now let $\mathcal{A}$ be a totally ordered subset of
  $\mathcal{S}$.

  \begin{claim}
    $\mathcal{A}$ has an upper bound.
  \end{claim}
  \begin{claimproof}
    Let $\tilde{I} := \cup_{I\in \mathcal{A}}I$. 

    \begin{subclaim}
      $\tilde{I}$ is an ideal. 
    \end{subclaim}
    \begin{subclaimproof}
      Let $a \in \tilde{I}$. Then $a \in I_a$ for some $I_a \in \mathcal{A}$. Hence $ra \in I_a \subseteq \tilde{I}$ for all $r \in \mathcal{R}$. Now
      take any $a, b \in \tilde{I}$. Then there exists $I_a, I_b \in \mathcal{A}$ such that $a \in I_a$ and $b \in I_b$. Since $\mathcal{A}$ is
      totally ordered, either $I_a \subseteq I_b$ or $I_b \subseteq I_a$. Without loss of generality assume $I_a \subseteq I_b$. Then $a, b \in I_b$,
      and thus $a - b \in I_b \subseteq \tilde{I}$.
    \end{subclaimproof}
    
    To see that $\tilde{I}$ is a proper ideal, assume not. Then $1 \in
    \tilde{I}$ and hence
    $1 \in I_0$ for some $I_0 \in \mathcal{A}$. But then $I_0$ is not a proper ideal. This is a contradiction. Thus $\tilde{I}$ is proper.
    Also clearly $I \subseteq \tilde{I}$.
    Hence $\tilde{I} \in \mathcal{A}$, and since
    $I' \preceq \tilde{I}$ for
    all $I' \in \mathcal{A}$, we have that $\tilde{I}$ is an upper bound for $\mathcal{A}$.

  \end{claimproof}

  By Claim 1, the conditions for Zorn's Lemma apply to $\mathcal{S}$. Hence $\mathcal{S}$ has a maximal element, i.e. $I$ is contained in a maximal
  ideal.
\end{Solution}


\end{document}
