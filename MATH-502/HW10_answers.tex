\documentclass[12pt]{article}
\usepackage{amsmath}
\usepackage{amsfonts}
\usepackage{parskip}
\usepackage{amsthm}
\usepackage{thmtools}
\usepackage[headheight=15pt]{geometry}
\geometry{a4paper, left=20mm, right=20mm, top=30mm, bottom=30mm}
\usepackage{graphicx}
\usepackage{bm} % for bold font in math mode - command is \bm{text}
\usepackage{enumitem}
\usepackage{fancyhdr}
\usepackage{amssymb} % for stacked arrows and other shit
\pagestyle{fancy}
\usepackage{changepage}
\usepackage{mathcomp}
\usepackage{tcolorbox}
\usepackage{eufrak}

\declaretheoremstyle[headfont=\normalfont]{normal}
\declaretheorem[style=normal]{Theorem}
\declaretheorem[style=normal]{Proposition}
\declaretheorem[style=normal]{Lemma}
\newcounter{ProofCounter}
\newcounter{ClaimCounter}[ProofCounter]
\newcounter{SubClaimCounter}[ClaimCounter]
\newenvironment{Proof}{\stepcounter{ProofCounter}\textsc{Proof.}}{\hfill$\square$}
\newenvironment{Solution}{\stepcounter{ProofCounter}\textbf{Solution:}}{\hfill$\square$}
\newenvironment{claim}[1]{\vspace{1mm}\stepcounter{ClaimCounter}\par\noindent\underline{\bf Claim \theClaimCounter:}\space#1}{}
\newenvironment{claimproof}[1]{\par\noindent\underline{Proof of claim \theClaimCounter:}\space#1}{\hfill $\blacksquare$ Claim \theClaimCounter}
\newenvironment{subclaim}[1]{\stepcounter{SubClaimCounter}\par\noindent\emph{Subclaim \theClaimCounter.\theSubClaimCounter:}\space#1}{}
% \newenvironment{subclaimproof}[1]{\begin{adjustwidth}{2em}{0pt}\par\noindent\emph{Proof of subclaim \theClaimCounter.\theSubClaimCounter:}\space#1}{\hfill
% $\blacksquare$ \emph{Subclaim \theClaimCounter.\theSubClaimCounter}\vspace{5mm}\end{adjustwidth}}
\newenvironment{subclaimproof}[1]{\par\noindent\emph{Proof of subclaim \theClaimCounter.\theSubClaimCounter:}\space#1}{\hfill
$\Diamond$ \emph{Subclaim \theClaimCounter.\theSubClaimCounter}}

\allowdisplaybreaks{}

% chktex-file 3

\lhead{Evan P. Walsh}
\chead{\textsc{Math 502 Assignment X}}
\rhead{\thepage}
\cfoot{}

\begin{document}\thispagestyle{empty}
\begin{center}
  \Large \textsc{math 502 -- ASSIGNMENT X -- spring 2017} \\ 
  \vspace{5mm}
  \large Evan Pete Walsh
\end{center}


\subsection*{1}
\begin{tcolorbox}
  Prove that if $(X, \tau)$ is regular and $Y \subseteq X$ is a subspace, then $Y$ is regular.
\end{tcolorbox}
\begin{Solution}
  Let $(X, \tau)$ be a regular topological space and $Y \subseteq X$ a subspace. Let $y \in Y$ and $F \subseteq Y$ closed in the subspace topology.

  By definition of the subspace topology, there exists $G \subseteq X$ such that $G$ is closed in the $\tau$ topology and $F = G \cap Y$.
  Also, since $y \in Y - F$, $y \in Y - G$. Hence by the regularity of $X$, there exists open $\tilde{U}, \tilde{V} \in \tau$ such that $y \in
  \tilde{U}$, $G \subseteq \tilde{V}$, and $\tilde{U} \cap \tilde{V} = \emptyset$.

  Now let $U := \tilde{U} \cap Y$ and $V := \tilde{V} \cap Y$, which are both open in the subspace topology. Then $x \in U$, $F \subseteq V$, and 
  $U \cap V = (\tilde{U} \cap Y)  \cap ( \tilde{V} \cap Y) = \tilde{U} \cap \tilde{V} \subseteq \emptyset$. Hence $Y$ is regular.
\end{Solution}

\subsection*{2}
\begin{tcolorbox}
  Suppose $g, h : [0,1] \rightarrow \mathbb{R}$ are continuous functions such that $g(0) = h(0)$. Prove by direct construction and then using the
  Tietze Extension Theorem that there exists a continuous $f : [0,1] \times [0,1] \rightarrow \mathbb{R}$ such that $f(x,0) \equiv g(x)$ for all $x \in
  [0,1]$ and $f(0,y) \equiv h(y)$ for all $y \in [0,1]$.
\end{tcolorbox}
\begin{Solution}
  
\end{Solution}

\subsection*{3 [Sec. 4.1 \#2, 3, 4, and 7]}
\begin{Solution}
  \begin{enumerate}
    \item[\#1.] 
  \end{enumerate}
\end{Solution}

\subsection*{4} 
\begin{tcolorbox}
  Show that the product of two regular topological spaces is regular.
\end{tcolorbox}
\begin{Solution}
  Suppose $(X_1, \tau_1)$ and $(X_2, \tau_2)$ are regular topological spaces. Let $(X, \tau)$ be the topological product corresponding to $X_1 \times
  X_2$. Let $x := (x_1, x_2) \in X$ and $F \subseteq X$ closed such that $x \notin F$. We need to show that there exists open $U, V \subseteq X$ such
  that $x \in U$, $F \subseteq V$, and $U \cap V = \emptyset$. 
  
  To that end, first note that $F$ is the intersection of sets of the form 
  \[
    \pi_1^{-1}[F_1] \cup \pi_2^{-1}[F_2]
  \]
  where $F_1 \subseteq X_1$ and $F_2 \subseteq X_2$ are closed in $\tau_1, \tau_2$, respectively. Thus there must exist some closed $F_1' \subseteq X_1$ and
  closed $F_2' \subseteq X_2$ such that $F \subseteq \pi_1^{-1}[F_1'] \cup \pi_2{-1}[F_2']$ and $x \notin \pi_1^{-1}[F_1'] \cup \pi_2^{-1}[F_2']$.
  Then $x_1 \notin F_1'$ and $x_2 \notin F_2'$. 
  
  Now, since $X_1$ and $X_2$ are regular, there exist disjoint, open $U_1, V_1 \subseteq X_1$ 
  such that $x_1 \in U_1$ and $F_1' \subseteq V_1$, and disjoint, open $U_2, V_2 \subseteq X_2$ such that 
  $x_2 \in U_2$ and $F_2' \subseteq V_2$. 

  Define the $\tau$-open sets $U$ and $V$ by $U := \pi_1^{-1}[U_1] \cap \pi_2^{-1}[U_2]$ and 
  $V := \pi_1^{-1}[V_1] \cup \pi_2^{-1}[V_2]$. We claim that $U$ and $V$ satisfy the conditions needed for regularity of $X$. 
  
  Indeed, clearly $x \in
  U$ and $F \subseteq \pi_{1}^{-1}[F_1'] \cup \pi_2^{-1}[F_2'] \subseteq \pi_1^{-1}[V_1] \cup \pi_2^{-1}[V_2] = V$. Hence it remains to show that $U
  \cap V = \emptyset$.

  Well, if there exists $y := (y_1, y_2) \in U \cap V$, then $y_1 \in U_1$, $y_2 \in U_2$, and either $y_1 \in V_1$ or $y_2 \in V_2$. In either case,
  we have a contradiction since $U_1 \cap V_1 = \emptyset$ and $U_2 \cap V_2 = \emptyset$.
\end{Solution}

\end{document}
