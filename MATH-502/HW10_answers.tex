\documentclass[12pt]{article}
\usepackage{amsmath}
\usepackage{amsfonts}
\usepackage{parskip}
\usepackage{amsthm}
\usepackage{thmtools}
\usepackage[headheight=15pt]{geometry}
\geometry{a4paper, left=20mm, right=20mm, top=30mm, bottom=30mm}
\usepackage{graphicx}
\usepackage{bm} % for bold font in math mode - command is \bm{text}
\usepackage{enumitem}
\usepackage{fancyhdr}
\usepackage{amssymb} % for stacked arrows and other shit
\pagestyle{fancy}
\usepackage{changepage}
\usepackage{mathcomp}
\usepackage{tcolorbox}
\usepackage{eufrak}

\declaretheoremstyle[headfont=\normalfont]{normal}
\declaretheorem[style=normal]{Theorem}
\declaretheorem[style=normal]{Proposition}
\declaretheorem[style=normal]{Lemma}
\newcounter{ProofCounter}
\newcounter{ClaimCounter}[ProofCounter]
\newcounter{SubClaimCounter}[ClaimCounter]
\newenvironment{Proof}{\stepcounter{ProofCounter}\textsc{Proof.}}{\hfill$\square$}
\newenvironment{Solution}{\stepcounter{ProofCounter}\textbf{Solution:}}{\hfill$\square$}
\newenvironment{claim}[1]{\vspace{1mm}\stepcounter{ClaimCounter}\par\noindent\underline{\bf Claim \theClaimCounter:}\space#1}{}
\newenvironment{claimproof}[1]{\par\noindent\underline{Proof of claim \theClaimCounter:}\space#1}{\hfill $\blacksquare$ Claim \theClaimCounter}
\newenvironment{subclaim}[1]{\stepcounter{SubClaimCounter}\par\noindent\emph{Subclaim \theClaimCounter.\theSubClaimCounter:}\space#1}{}
% \newenvironment{subclaimproof}[1]{\begin{adjustwidth}{2em}{0pt}\par\noindent\emph{Proof of subclaim \theClaimCounter.\theSubClaimCounter:}\space#1}{\hfill
% $\blacksquare$ \emph{Subclaim \theClaimCounter.\theSubClaimCounter}\vspace{5mm}\end{adjustwidth}}
\newenvironment{subclaimproof}[1]{\par\noindent\emph{Proof of subclaim \theClaimCounter.\theSubClaimCounter:}\space#1}{\hfill
$\Diamond$ \emph{Subclaim \theClaimCounter.\theSubClaimCounter}}

\allowdisplaybreaks{}

% chktex-file 3

\lhead{Evan P. Walsh}
\chead{\textsc{Math 502 Assignment X}}
\rhead{\thepage}
\cfoot{}

\begin{document}\thispagestyle{empty}
\begin{center}
  \Large \textsc{math 502 -- ASSIGNMENT X -- spring 2017} \\ 
  \vspace{5mm}
  \large Evan Pete Walsh
\end{center}


\subsection*{1}
\begin{tcolorbox}
  Prove that if $(X, \tau)$ is regular and $Y \subseteq X$ is a subspace, then $Y$ is regular.
\end{tcolorbox}
\begin{Solution}
  Let $(X, \tau)$ be a regular topological space and $Y \subseteq X$ a subspace. Let $y \in Y$ and $F \subseteq Y$ closed in the subspace topology.

  By definition of the subspace topology, there exists $G \subseteq X$ such that $G$ is closed in the $\tau$ topology and $F = G \cap Y$.
  Also, since $y \in Y - F$, $y \in Y - G$. Hence by the regularity of $X$, there exists open $\tilde{U}, \tilde{V} \in \tau$ such that $y \in
  \tilde{U}$, $G \subseteq \tilde{V}$, and $\tilde{U} \cap \tilde{V} = \emptyset$.

  Now let $U := \tilde{U} \cap Y$ and $V := \tilde{V} \cap Y$, which are both open in the subspace topology. Then $x \in U$, $F \subseteq V$, and 
  $U \cap V = (\tilde{U} \cap Y)  \cap ( \tilde{V} \cap Y) = \tilde{U} \cap \tilde{V} \subseteq \emptyset$. Hence $Y$ is regular.
\end{Solution}

\subsection*{2}
\begin{tcolorbox}
  Suppose $g, h : [0,1] \rightarrow \mathbb{R}$ are continuous functions such that $g(0) = h(0)$. Prove by direct construction and then using the
  Tietze Extension Theorem that there exists a continuous $f : [0,1] \times [0,1] \rightarrow \mathbb{R}$ such that $f(x,0) \equiv g(x)$ for all $x \in
  [0,1]$ and $f(0,y) \equiv h(y)$ for all $y \in [0,1]$.
\end{tcolorbox}
\begin{Solution}
  First we will proceed by direct construction. Note that $h(y) \equiv h(x,y)$ and $g(x) \equiv g(x,y)$ are continuous on $[0,1] \times [0,1]$.
  Thus, as the sum and product of continuous functions, 
  \[
    f(x,y) := g(x) + (1-x)[h(y) - h(0)] + (1-y)[g(x) - g(0)]
  \]
  is continuous and satisfies the desired conditions.

  On the other hand, the set $Y := \{(x,y) \subseteq [0,1]\times [0,1] : x = 0 \text{ or } y = 0\}$ is closed in $[0,1] \times [0,1]$ and 
  \[
    \tilde{f}(x,y) := \left\{ \begin{array}{cl}
        g(x) & \text{ if } y = 0 \\
        h(y) & \text{ if } x = 0 \\
    \end{array} \right. \ (x,y) \in Y
  \]
  is continuous on $Y$ since $g(0) = h(0)$. Further, the image of $Y$ under $\tilde{f}$ is compact in $\mathbb{R}$ since 
  \[
    \tilde{f}[Y] = g([0,1]) \cup h([0,1])
  \]
  and $g([0,1]), h([0,1])$ are both compact in $\mathbb{R}$. Hence $\tilde{f} \in C_b(Y, \mathbb{R})$. So, since $[0,1] \times [0,1]$ is a metric space, and
  thus is normal, there exists $f \in C_b([0,1]\times[0,1], \mathbb{R})$ such that $f\big|_{Y} \equiv \tilde{f}$ by the Tietze Extension
  Theorem.
\end{Solution}

\subsection*{3 [Sec. 4.1 \#2, 3, 4, and 7]}
\begin{Solution}
  \begin{enumerate}
    \item[\#2.] 
      \begin{claim}
        An open subset of a compact Hausdorff space is locally compact.
      \end{claim}
      \begin{claimproof}
        Let $X$ be a compact Hausdorff space and $Y$ an open subspace. Now let $y \in Y$ be arbitrary. We need to show that $\mathcal{N}_y$ contains a
        compact subset of $Y$.

        Since $X$ is compact and Hausdorff, $X$ is normal, and since $X$ is $T_1$, $\{y\}$ is closed.
        Thus, by Lemma 4.1.1, there exists $V$ open in $X$ such that $\{y\} \subseteq V \subseteq \bar{V} \subseteq Y$.
        So $F := \bar{V}$ is compact since it is a closed subset of the compact space $X$, and $F \in \mathcal{N}_y$ since $y \in F \subseteq Y$.

      \end{claimproof} 

      \begin{claim}
        For a locally compact Hausdorff space $(X, \tau)$, the neighborhood system $\mathcal{N}_x$ has a base consisting of compact sets for each $x
        \in X$.
      \end{claim}
      \begin{claimproof}
        Let $x \in X$. We need to show that for each $N \in \mathcal{N}_x$ there exists a compact set $K \in \mathcal{N}_x$ such that $K \subseteq N$. 

        Well, by the local compactness of $X$, there exists some compact $E \in \mathcal{N}_x$. Then if 
        $N \in \mathcal{N}_x$, $E \cap N \in \mathcal{N}_x$, and so there exists open $U \subseteq E \cap N$ with $x \in U$. 
        
        Thus by Claim 1, since $U \subseteq E$, $U$
        is a locally compact subspace of $E$. Hence there exists a compact $K$ which is in the neighborhood system of $x$ relative to $U$. 
        So $K$ contains a set $U'$ which contains $x$ and is open in the relative topology of $U$. 
        
        Of course $U'$ must also be open in $\tau$, and so $K \in \mathcal{N}_x$. Thus, since $K \subseteq U \subseteq N$, we are done.
      \end{claimproof}

    \item[\#3.] We could define $f : X \rightarrow \mathbb{R}$ by 
      \[
        f(x) := \frac{d(x, F)}{d(x, F) + d(x, G)}, \ x \in X.
      \]
      Since $F$ and $G$ are closed, $d(x, F) > 0$ if $x \notin F$ and $d(x, G) > 0$ if $x \notin G$. Thus, since $F$ and $G$ are disjoint, $d(x, F) +
      d(x, G) > 0$ for all $x \in X$. Hence $f$ is well-defined and also continuous since $d(x, F)$ and $d(x, G)$ are continuous. 

      Further, clearly $f[X] \subseteq [0,1]$, and
      since $d(x, F) \equiv 0$ for all $x \in F$ and $d(x, G) \equiv 0$ for all $x \in G$,
      \[
        f\big|_{F}(x) \equiv 0 \ \ \text{ and } \ \ f\big|_{G}(x) = \frac{d(x, F)}{d(x, F) + 0} \equiv 1.
      \]

    \item[\#4.] Let $(X,\tau)$ be completely regular, let $K \subseteq X$ be compact and $F \subseteq X$ closed such that $K \cap F = \emptyset$.

      \begin{claim}
        For each open $U \subseteq X$ such that $K \subset U$, there exists an open subset $V$ of $X$ such that 
        \[
          K \subseteq V \subseteq \bar{V} \subseteq U.
        \]
      \end{claim}
      \begin{claimproof}
        Let $G := X - U$.
        Since $X$ is completely regular, for each $x \in K$ there exists disjoint open sets $\tilde{U}$ and $\tilde{V}$ such that $x \in \tilde{V}$
        and $G \subseteq \tilde{U}$. Now let 
        \[
          \mathcal{V} := \left\{ \tilde{V} : \text{$\tilde{V}$ open and $\exists \ x \in K$ with $x \in \tilde{V}$ and open $\tilde{U} \supseteq G$
          with $\tilde{V} \cap \tilde{U} = \emptyset$} \right\}.
        \]
        Then $\mathcal{V}$ is an open cover of $K$. Hence, by the compactness of $K$ there exists a finite subcover 
        $V_1, \dots, V_n \in \mathcal{V}$ with corresponding sets $U_1, \dots, U_n$ such that $U_i \supseteq G$ and $V_i \cap U_i = \emptyset$ for
        each $i = 1, \dots, n$.

        Now let $V := V_1 \cap \dots \cap V_n$. Thus $V$ is open and $V \subseteq X - (U_1 \cup \dots \cup U_n)$, so
        \[
          \bar{V} \subseteq X - (U_1 \cap \dots \cap U_n) \subseteq X - (X - G) = U.
        \]
        Therefore we have $K \subseteq V \subseteq \bar{V} \subseteq U$, so we are done.
      \end{claimproof}

      By Claim 3 we can now apply the exact same argument as in the proof of Urysohn's lemma to derive a continuous function $\tilde{f} : X \rightarrow [0,1]$
      such that $\tilde{f}\big|_{K} \equiv 0$ and $\tilde{f}\big|_{F} \equiv 1$. Then by taking $f := 1 - \tilde{f}$, we are done.

    \item[\#7.] Let $(X, \tau)$ be a Hausdorff space.

      $(\Rightarrow)$ First suppose $(X, \tau)$ has a base $\mathcal{B}$ consisting of clopen sets. Let $\mathbb{I} := \mathcal{B}$ be the index set,
      and define $\iota : X \rightarrow \{0,1\}^{\mathbb{I}}$ by $\iota(x) := \left( \chi_{B}(x) \right)_{B\in\mathbb{I}}$, where $\chi_{A}$ is the
      characteristic function for the set $A$.

      \begin{claim}
        $\iota$ is injective.
      \end{claim}
      \begin{claimproof}
        Let $x \neq y \in X$. Then by the Hausdorff property, there exists $B_x, B_y \in \mathcal{B}$ such that $x \in B_x, y \in B_y$ and $B_x \cap
        B_y = \emptyset$. Thus $\pi_{B_x}\circ\iota(x) = 1 \neq 0 = \pi_{B_x}\circ \iota(y)$. So $\iota(x) \neq \iota(y)$.
      \end{claimproof}

      \begin{claim}
        $\iota$ continuous.
      \end{claim}
      \begin{claimproof}
        Let $B \in \mathbb{I}$. Then $\iota^{-1} \circ \pi_{B}^{-1}[\{1\}] = B$ and $\iota^{-1} \circ \pi_{B}^{-1}[\{0\}] = X - B$ are both open since $B$ is clopen.
        Thus $\iota^{-1}\circ\pi_{B}^{-1}[V]$ is open in $X$ for any $V \subseteq \{0,1\}$. Now, if $U \subseteq \{0,1\}^{\mathbb{I}}$ is open in the product
        topology, then $U$ is the union of sets of the form 
        \[
          \pi_{B_1}^{-1}[U_1] \cap \dots \pi_{B_n}^{-1}[U_n],
        \]
        where for each $j = 1, \dots, n$, $B_j \in \mathbb{I}$ and $U_j \subseteq \{0,1\}$.  Hence $\iota^{-1}[U]$ is open in $X$ for any open $U \subseteq
        \{0,1\}^{\mathbb{I}}$.
      \end{claimproof}

      Let $Y := \iota[X]$.

      \begin{claim}
        $\iota$ is an open mapping onto the subspace $Y$.
      \end{claim}
      \begin{claimproof}
        It suffices to show that $\iota[B]$ is open in the subspace $Y$ for each $B \in \mathcal{B}$.
        But 
        \[
          \iota[B] = \pi_{B}^{-1}[\{1\}] \cap Y,
        \]
        which is open in $Y$ since $\pi_B^{-1}[\{1\}]$ is open in $\{0,1\}^{\mathbb{I}}$ by the definition of the product topology.
      \end{claimproof}

      So by Claims 4, 5, and 6, $\iota$ is a homeomorphism from $X$ into $Y$.

      $(\Leftarrow)$ Now suppose that $(X, \tau)$ is homeomorphic to a subspace $Y$ of $\{0, 1\}^{\mathbb{I}}$ for some index set $\mathbb{I}$.
      If we can show that a base for $\{0,1\}^{\mathbb{I}}$ consists of clopen sets, then clearly the corresponding base for the subspace $Y$ consists 
      of clopen sets. It then follows that since $Y$ is homeomorphic to $X$, $X$ has a base consisting of clopen sets. Well\dots

      \begin{claim}
        $\{0,1\}^{\mathbb{I}}$ has a base consisting of clopen sets.
      \end{claim}
      \begin{claimproof}
        Consider the usual base $\mathcal{B}_{\mathbb{I}}$ of the product space $\{0,1\}^{\mathbb{I}}$, consisting of sets of the form
        \[
          \pi_{i_1}^{-1}[U_1] \cap \dots \cap \pi_{i_n}^{-1}[U_n],
        \]
        where $i_j \in \mathbb{I}$ and $U_j \subseteq \{0,1\}$ for each $j = 1, \dots, n$. Since $\pi_{i}^{-1}[U]$ is open in $\{0,1\}^{\mathbb{I}}$ 
        for any $i \in \mathbb{I}$ and $U \subseteq \{0,1\}$, it follows that $\pi_{i}^{-1}[\{0,1\}-U]$ open for any $U \subseteq \{0,1\}$.
        Thus, with $i_j$ and $U_j$ as above,
        \[
          \{0,1\}^{\mathbb{I}} - \left(\pi_{i_1}^{-1}[U_1] \cap \dots \cap \pi_{i_n}^{-1}[U_n]\right) = \pi_{i_1}^{-1}[\{0,1\}-U_1] \cup \dots \cup
          \pi_{i_n}^{-1}[\{0,1\} - U_n]
        \]
        is open. Hence $\mathcal{B}_{\mathbb{I}}$ consists of clopen sets.
      \end{claimproof}
  \end{enumerate}
\end{Solution}

\subsection*{4} 
\begin{tcolorbox}
  Show that the product of two regular topological spaces is regular.
\end{tcolorbox}
\begin{Solution}
  Suppose $(X_1, \tau_1)$ and $(X_2, \tau_2)$ are regular topological spaces. Let $(X, \tau)$ be the topological product corresponding to $X_1 \times
  X_2$. Let $x := (x_1, x_2) \in X$ and $F \subseteq X$ closed such that $x \notin F$. We need to show that there exists open $U, V \subseteq X$ such
  that $x \in U$, $F \subseteq V$, and $U \cap V = \emptyset$. 
  
  To that end, first note that $F$ is the intersection of sets of the form 
  \[
    \pi_1^{-1}[F_1] \cup \pi_2^{-1}[F_2]
  \]
  where $F_1 \subseteq X_1$ and $F_2 \subseteq X_2$ are closed in $\tau_1, \tau_2$, respectively. Thus there must exist some closed $F_1' \subseteq X_1$ and
  closed $F_2' \subseteq X_2$ such that $F \subseteq \pi_1^{-1}[F_1'] \cup \pi_2{-1}[F_2']$ and $x \notin \pi_1^{-1}[F_1'] \cup \pi_2^{-1}[F_2']$.
  Then $x_1 \notin F_1'$ and $x_2 \notin F_2'$. 
  
  Now, since $X_1$ and $X_2$ are regular, there exist disjoint, open $U_1, V_1 \subseteq X_1$ 
  such that $x_1 \in U_1$ and $F_1' \subseteq V_1$, and disjoint, open $U_2, V_2 \subseteq X_2$ such that 
  $x_2 \in U_2$ and $F_2' \subseteq V_2$. 

  Define the $\tau$-open sets $U$ and $V$ by $U := \pi_1^{-1}[U_1] \cap \pi_2^{-1}[U_2]$ and 
  $V := \pi_1^{-1}[V_1] \cup \pi_2^{-1}[V_2]$. We claim that $U$ and $V$ satisfy the conditions needed for regularity of $X$. 
  
  Indeed, clearly $x \in
  U$ and $F \subseteq \pi_{1}^{-1}[F_1'] \cup \pi_2^{-1}[F_2'] \subseteq \pi_1^{-1}[V_1] \cup \pi_2^{-1}[V_2] = V$. Hence it remains to show that $U
  \cap V = \emptyset$.

  Well, if there exists $y := (y_1, y_2) \in U \cap V$, then $y_1 \in U_1$, $y_2 \in U_2$, and either $y_1 \in V_1$ or $y_2 \in V_2$. In either case,
  we have a contradiction since $U_1 \cap V_1 = \emptyset$ and $U_2 \cap V_2 = \emptyset$.
\end{Solution}

\end{document}
