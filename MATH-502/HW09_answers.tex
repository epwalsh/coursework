\documentclass[12pt]{article}
\usepackage{amsmath}
\usepackage{amsfonts}
\usepackage{parskip}
\usepackage{amsthm}
\usepackage{thmtools}
\usepackage[headheight=15pt]{geometry}
\geometry{a4paper, left=20mm, right=20mm, top=30mm, bottom=30mm}
\usepackage{graphicx}
\usepackage{bm} % for bold font in math mode - command is \bm{text}
\usepackage{enumitem}
\usepackage{fancyhdr}
\usepackage{amssymb} % for stacked arrows and other shit
\pagestyle{fancy}
\usepackage{changepage}
\usepackage{mathcomp}
\usepackage{tcolorbox}
\usepackage{eufrak}

\declaretheoremstyle[headfont=\normalfont]{normal}
\declaretheorem[style=normal]{Theorem}
\declaretheorem[style=normal]{Proposition}
\declaretheorem[style=normal]{Lemma}
\newcounter{ProofCounter}
\newcounter{ClaimCounter}[ProofCounter]
\newcounter{SubClaimCounter}[ClaimCounter]
\newenvironment{Proof}{\stepcounter{ProofCounter}\textsc{Proof.}}{\hfill$\square$}
\newenvironment{Solution}{\stepcounter{ProofCounter}\textbf{Solution:}}{\hfill$\square$}
\newenvironment{claim}[1]{\vspace{1mm}\stepcounter{ClaimCounter}\par\noindent\underline{\bf Claim \theClaimCounter:}\space#1}{}
\newenvironment{claimproof}[1]{\par\noindent\underline{Proof of claim \theClaimCounter:}\space#1}{\hfill $\blacksquare$ Claim \theClaimCounter}
\newenvironment{subclaim}[1]{\stepcounter{SubClaimCounter}\par\noindent\emph{Subclaim \theClaimCounter.\theSubClaimCounter:}\space#1}{}
% \newenvironment{subclaimproof}[1]{\begin{adjustwidth}{2em}{0pt}\par\noindent\emph{Proof of subclaim \theClaimCounter.\theSubClaimCounter:}\space#1}{\hfill
% $\blacksquare$ \emph{Subclaim \theClaimCounter.\theSubClaimCounter}\vspace{5mm}\end{adjustwidth}}
\newenvironment{subclaimproof}[1]{\par\noindent\emph{Proof of subclaim \theClaimCounter.\theSubClaimCounter:}\space#1}{\hfill
$\Diamond$ \emph{Subclaim \theClaimCounter.\theSubClaimCounter}}

\allowdisplaybreaks{}

% chktex-file 3

\lhead{Evan P. Walsh}
\chead{MATH 502: Assignment IX}
\rhead{\thepage}
\cfoot{}

\begin{document}\thispagestyle{empty}
\begin{center}
  \Large \textsc{math 502 -- ASSIGNMENT IX -- spring 2017} \\ 
  \vspace{5mm}
  \large Evan Pete Walsh
\end{center}


\subsection*{1 [Sec. 3.4 \#2, 4, 6, and 8]}
\begin{Solution}
  \begin{enumerate}
    \item[\# 2.] $(\Rightarrow)$ For the forward direction, first assume the $(X,\tau)$ is path connected. Let $i_0 \in \mathbb{I}$ and $x_{i_0},
      y_{i_0} \in X_{i_0}$. Then choose $x_i, y_i \in X_i$ arbitrarily for $i \neq i_0$. By assumption, there exists continuous $\gamma : [0,1]
      \rightarrow X$ such that $\gamma(0) = x := (x_i)_{i\in\mathbb{I}}$ and $\gamma(1) = y := (y_i)_{i\in\mathbb{I}}$.
      But by definition of $(X, \tau)$, the coordinate projection $\pi_{i_0} : X \rightarrow X_{i_0}$ is continuous. Thus, $\pi_{i_0} \circ \gamma :
      [0,1] \rightarrow (X_{i_0}, \tau_{i_0})$ is continuous and $\pi_{i_0} \circ \gamma(0) = x_{i_0}$, $\pi_{i_0} \circ \gamma(1) = y_{i_0}$. Hence
      $(X_{i_0}, \tau_{i_0})$ is path connected.

      $(\Leftarrow)$ Now assume that $(X_i, \tau_i)$ is path connected for each $i \in \mathbb{I}$. Let $x := \left( x_i \right)_{i\in\mathbb{I}}$, 
      $\left( y_i \right)_{i\in\mathbb{I}} \in X$. Then by assumption, for each $i \in \mathbb{I}$, there exists continuous $\gamma_i : [0,1]
      \rightarrow X_i$ such that $\gamma_i(0) = x_i$ and $\gamma_i(1) = y_i$. So let $\gamma : [0,1] \rightarrow X$ be defined by $\gamma(t) := \left(
      \gamma_i(t) \right)_{i\in\mathbb{I}}$. Since $\gamma(0) = x$ and $\gamma(1) = y$ by definition, it remains to show that $\gamma$ is continuous.
      To that end, let $U \in \tau$. We want to show that $\gamma^{-1}[U]$ is open in $[0,1]$. However, any open set in $\tau$ can be written as
      the union of sets of the form 
      \begin{equation}
        \pi_{i_1}^{-1}[U_1] \cap \dots \cap \pi_{i_n}^{-1}[U_n]
        \label{eq1}
      \end{equation}
      where $i_1, \dots, i_n \in \mathbb{I}$ and $U_j \in \tau_{i_j}$ for each $j = 1, \dots, n$. Thus we may assume without loss of generality
      that $U$ is of the form in \eqref{eq1}. Hence
      \begin{align*}
        \gamma^{-1}[U] = \gamma^{-1}\bigg[\pi_{i_1}^{-1}[U_1] \cap \dots \cap \pi_{i_n}^{-1}[U_n]\bigg] & =
        \bigcap_{j=1}^{n}\gamma^{-1}\bigg[\pi_{i_j}^{-1}[U_j]\bigg] \\
        & = \bigcap_{j=1}^{n} \gamma_{i_j}^{-1}\bigg[ \pi_{i_j}^{-1}[U_j] \bigg],
      \end{align*}
      which is open in $[0,1]$ by the continuity of $\pi_{i_j} \circ \gamma_{i_j}$. Hence $\gamma$ is continuous, and so $(X,\tau)$ is path connected.

    \item[\# 4.] Fix $y \in Y$ and consider the family $\mathcal{G} := \left\{ (X \times \{x\}) \cup (\{x\} \times Y) : x \in X \right\}$.

      \begin{claim}
        For each $x \in X$, $(X \times \{y\}) \cup (\{x\} \times Y)$ is a connected subspace.
      \end{claim}
      \begin{claimproof}
        Let $f(x) : X \rightarrow X \times Y$ be defined by $f(x) := (x,y)$. To show that $f$ is continuous, let $V$ be open in
        the product space $X \times Y$. Without loss of generality we can assume $V$ can be expressed as
        \begin{equation}
          V = \pi_{X}^{-1}[V_x] \cap \pi_{Y}^{-1}[V_y], 
          \label{eq1.2}
        \end{equation}
        where $\pi_{X}, \pi_{Y}$ are the coordinate projections from $X\times Y \rightarrow X$ and $X\times
        Y \rightarrow Y$, respectively, and $V_x \in \tau_{X}, V_y \in \tau_{Y}$ (otherwise $V$ would be a union of sets in the form of \eqref{eq1.2}). Thus,
        \[
          f^{-1}[U] = f^{-1}[V \cap X \times \{y\}] = f^{-1}\big[\pi_{X}^{-1}[V_x] \cap \pi_{Y}^{-1}[V_y] \cap (X \times \{y\})\big] = f^{-1}[V_x] = V_x
          \in \tau_X.
        \]
        Hence $f$ is continuous. Thus, by Proposition 3.4.11, $f[X] = X\times \{y\}$ is a connected subspace of $X \times Y$. The same argument shows
        that $\{x\} \times Y$ is connected for each $x \in X$. Thus, since $(X \times \{y\}) \cap (\{x\} \times Y) = \{(x,y)\} \neq \emptyset$, 
        \[
          (X \times \{y\}) \cap (\{x\} \times Y)
        \]
        is connected for each $x \in X$ by Proposition 3.4.16.
      \end{claimproof}

      By Claim 1 and since for each $x_0 \neq x_1 \in X$, 
      \[
        \big[(X \times \{y\}) \cup (\{x_0\} \times Y)\big] \cap \big[(X \times \{y\}) \cup (\{x_1\} \times Y)\big] = X \times \{y\} \neq \emptyset,
      \]
      we can apply Proposition 3.4.16 to the family $\mathcal{G}$. Hence 
      \[
        \bigcup_{x \in X} (X \times \{y\}) \cup (\{x\} \times Y) = X \times Y
      \]
      is connected.

    \item[\# 6.] Since $G_0$ is a component, it is closed. Thus it remains to show that $G_0$ is a normal subgroup.

      \begin{claim}
        $ab \in G_0$ for all $a,b \in G_0$.
      \end{claim}
      \begin{claimproof}
        Let $a', b' \in G_0$. By definition of the topology on $G$, $(a,b) \mapsto ab$ is continuous. Thus, since the image of a connected set under a continuous function
        is connected, and since $(e,e) = ee = e \in G_0$, it must be that $a'b' \in G_0$.
      \end{claimproof}

      \begin{claim}
        For each $a \in G_0$, $a^{-1} \in G_0$.
      \end{claim}
      \begin{claimproof}
        Similary to Claim 2, this follows from the fact that $a \mapsto a^{-1}$ is continuous and $e = e^{-1} \in G_0$.
      \end{claimproof}

      So by Claim 2 and Claim 3, $G_0$ is a subgroup. The following claim establishes that $G_0$ is normal.

      \begin{claim}
        For each $g \in G$, $gG_0 g^{-1} \subseteq G_0$.
      \end{claim}
      \begin{claimproof}
        Let $g \in G$. Then by Exercise \# 4 above, $\{g\} \times G_0$ is connected in the product topology since $\{g\}$ is connected in $G$.
        Therefore $gG_0$ is connected in $G$ since $(a,b) \mapsto ab$ is continuous. Reusing this logic, $gG_0 \times \{g^{-1}\}$ is
        connected in the product topology, so $gG_0g^{-1}$ is connected in $G$. Therefore, since $e \in G_0$ and $geg^{-1} = e$, it must be that
        $gG_0g^{-1} \subseteq G_0$.
      \end{claimproof}

    \item[\# 8.] $(\Leftarrow)$ Suppose $(X_i, \tau_i)$ totally disconnected for each $i \in \mathbb{I}$.
      Let $C \subseteq X$ be any connected subset. 
      By way of contradiction suppose there exists distinct $x = (x_i)_{i\in\mathbb{I}}, y = (y_i)_{i\in\mathbb{I}} \in C$.
      Then there exists $i_0 \in \mathbb{I}$ such that $x_{i_0} \neq y_{i_0}$.
      Since $\pi_{i_0}$ continuous, $\pi_{i_0}[C]$ is connected in $(X_{i_0}, \tau_{i_0})$ and contains $x_{i_0}, y_{i_0}$. This is a contradiction.

      $(\Rightarrow)$ Now suppose $(X, \tau)$ is totally disconnected. By way of contradiction suppose there exists $i_0 \in \mathbb{I}$ such that
      $(X_{i_0}, \tau_{i_0})$ is not totally
      disconnected. Then there exists connected $C_0 \subseteq X_{i_0}$ which contains two distinct points $x_{i_0}, y_{i_0}$.
      For $i \neq i_{0}$, choose $x_i \equiv y_i$ arbitrarily. Let $x := (x_i)_{i\in\mathbb{I}}$, $y := (y_i)_{i\in\mathbb{I}}$. Then by Exercise \#
      5, the product 
      \[ F := \{ (w_i)_{i\in\mathbb{I}} : w_i = x_i, i \neq i_0 \wedge  w_{i_0} \in C\} \subseteq X \]
      is connected. But this is a contradiction since $x, y \in F$ are distinct.
  \end{enumerate}
\end{Solution}

\subsection*{2}
\begin{Solution}
  \begin{enumerate}
    \item[(a)] We will show that $X$ is not compact. Since $X$ is a metric space, it suffices to show that $X$ is not sequentially compact. Thus we will construct a
      sequence of functions for which there is no convergent subsequence. In particular, for $n \geq 1$, let 
      \[
        f_n(x) := x^n, \ \ x \in [0,1].
      \]
      Clearly $f_1, f_2, \dots \in X$ and yet for each $x \in [0,1)$, $f_n(x) \rightarrow 0$. Hence $\left\{ f_n \right\}_{n=1}^{\infty}$ converges
      pointwise to the function $g : [0,1] \rightarrow [0,1]$ defined by
      \[
        g(x) := \left\{ \begin{array}{cl}
            0 & \text{ if } 0 \leq x < 1, \\
            1 & \text{ if } x = 1,
          \end{array}
        \right. \ \ \ x \in [0,1].
      \]
      Hence every subsequence of $\left\{ f_n \right\}_{n=1}^{\infty}$ also converges pointwise to $g$. But $g \notin X$ since $g$ is not strictly
      increasing. Thus $X$ is not
      sequentially compact.

    \item[(b)] We will show that $X$ is not locally compact. By way of contradiction suppose $X$ is locally compact. Let $f \in X$ be defined by
      $f(x) := x$. Then since we are assuming that $X$ is locally compact, there exists a compact neighborhood $N_f$ of $f$. So since
      $N_f$ is a neighborhood of $f$, there exists an $r > 0$ such that the open ball $B_r[f] := \{ g\in X : d(f,g) < r\}$ is contained in
      $N_f$. Similar to part (a), we will derive a contradiction by constructing a sequence in $B_r[f]$, and thus in $N_f$, that converges pointwise to a
      function that is not in $X$. In particular, for $n \geq 1$, let $f_n:[0,1]\rightarrow[0,1]$ be defined by 
      \[
        f_n(x) := \left\{ \begin{array}{cl} 
            x^n & \text{ if } 0 \leq x \leq \frac{r}{2}, \\
            x & \text{ if } \frac{r}{2} < x \leq 1,
          \end{array}
        \right. \ \ \ x \in [0,1].
      \]
      Then for each $n \geq 1$, $|f_n(x) - f(x)| = 0$ for $x > r / 2$, and $|f_n(x) - f(x)| \leq |f(x)| \leq r / 2$ for $0 \leq x \leq r/2$. Hence
      $f_n \in B_r[f]$ for each $n \geq 1$. Further, $\left\{ f_n \right\}_{n=1}^{\infty}$ converges pointwise to the $g : [0,1] \rightarrow [0,1]$ defined
      by 
      \[
        g(x) := \left\{ \begin{array}{cl}
            0 & \text{ if } 0 \leq x \leq \frac{r}{2}, \\
            x & \text{ if } \frac{r}{2} < x \leq 1, 
          \end{array}
        \right. \ \ \ x \in [0,1].
      \]
      But $g \notin X$ since $g$ is not strictly increasing. Thus we have found a sequence in $N_f$ that has no convergent subsequence.


    \item[(c)] We will show that $X$ is not separable. So, by way of contradiction suppose $X$ is separable. Then there exists a countable, dense
      subset of $\mathcal{S} \subset X$. 

      \begin{claim}
        There exists $x_0 \in (0,1)$ such that each function in $\mathcal{S}$ is continuous at $x_0$.
      \end{claim}
      \begin{claimproof}
        It is well known that monotone functions can have at most countably many discontinuities (and any such
        discontinuity is a ``jump'' discontinuity). Therefore, since $\mathcal{S}$ countable and each function in $\mathcal{S}$ has countably many
        discontinuities, the set of all discontinuities of functions in $\mathcal{S}$ is countable. Hence, since $(0,1)$ is uncountable, there
        exists $x_0 \in (0,1)$ such that no function in $\mathcal{S}$ has a discontinuity at $x_0$.
      \end{claimproof}

      With $x_0 \in (0,1)$ as in the claim above, let $f \in X$ be defined by
      \[
        f(x) := \left\{ \begin{array}{cl}
            \frac{x}{4x_0} & \text{ if } 0 \leq x \leq x_0, \\ \\
            \left( \frac{1-\frac{3}{4}}{1-x_0} \right)(x - x_0) + \frac{3}{4} & \text{ if } x_0 < x \leq 1,
          \end{array}
        \right. \ \ \  x \in [0,1].
      \]
      So $f$ has a jump of $1/2$ at $x_0$ and is continuous otherwise, and therefore 
      \begin{equation}
        |f(x) - f(x_0)| > \frac{1}{2}
        \label{2.3}
      \end{equation}
      whenever $x > x_0$.
      Now, since $\mathcal{S}$ is dense in $X$, there exists $g \in \mathcal{S}$ such
      that $d(f,g) < 1/6$, i.e. 
      \begin{equation}
        \sup_{0\leq x\leq 1}|f(x) - g(x)| < \frac{1}{6}.
        \label{2.1}
      \end{equation}
      Further, by the above claim and since $g \in \mathcal{S}$, $g$ is continuous at $x_0$. Hence there exists $x_1 > x_0$ such that 
      \begin{equation}
        |g(x_1) - g(x_0)| < \frac{1}{6}.
        \label{2.2}
      \end{equation}
      So by \eqref{2.1} and \eqref{2.2}, 
      \[
        |f(x_1) - f(x_0)| \leq |f(x_1) - g(x_1)| + |g(x_1) - g(x_0)| + |g(x_0) - f(x_0)| < \frac{1}{6} + \frac{1}{6} + \frac{1}{6} = \frac{1}{2}.
      \]
      But this contradicts \eqref{2.3}.
      

    \item[(d)] We will show that $X$ is path connected, and therefore connected as well. To that end, let $f, g \in X$. Define $\gamma : [0,1]
      \rightarrow X$ by $\gamma(t) := t f + (1 - t) g$. We will show that $\gamma$ is a path connecting $f$ and $g$.

      \begin{claim}
        $\gamma$ is well-defined.
      \end{claim}
      \begin{claimproof}
        To see that $\gamma$ is well-defined, note that for each $t \in [0,1]$, $t f$ and $(1 - t)g$
        are strictly increasing since $f$ and $g$ are strictly increasing. Therefore $t f + (1-t) g$ is strictly increasing since the sum of strictly increasing
        functions is strictly increasing. Further, 
        \[
          t f(0) + (1- t)g(0) = 0 \ \text{  and  }\ t f(1) + (1-t) g(1) = 1. 
        \]
        Hence $\gamma(t) \in X$ for all $t \in [0,1]$.
      \end{claimproof}

      \begin{claim}
        $\gamma$ is continuous.
      \end{claim}
      \begin{claimproof}
        Let $t_0 \in [0,1]$ and $\epsilon > 0$. We need to show that there exists $\delta > 0$ such that for all $t \in [0,1]$ with $|t - t_0| <
        \delta$, it holds that $d(\gamma(t), \gamma(t_0)) < \epsilon$. To that end, let $\delta := \epsilon / 2$. Then if $|t - t_0| < \delta$,
        \begin{align*}
          d(\gamma(t), \gamma(t_0)) & = \sup_{0\leq x \leq 1} \big| t f(x) + (1-t)g(x) - t_0 f(x) - (1-t_0)g(x) \big| \\
          & = \sup_{0\leq x \leq 1}\big| (t - t_0) f(x) + (t_0 - t)g(x) \big| \\
          & \leq \sup_{0\leq x \leq 1} \big| t - t_0 \big| \big| f(x)\big| + \big| t - t_0\big| \big|g(x)\big| \\
          & \leq 2 \big| t - t_0\big| < \epsilon.
        \end{align*}
        Hence $\gamma$ is continuous.
      \end{claimproof}

      By the above two claims, we are done.
  \end{enumerate}
\end{Solution}

\end{document}
