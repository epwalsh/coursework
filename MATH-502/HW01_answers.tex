\documentclass[12pt]{article}
\usepackage{amsmath}
\usepackage{amsfonts}
\usepackage{parskip}
\usepackage{amsthm}
\usepackage{thmtools}
\usepackage[headheight=15pt]{geometry}
\geometry{a4paper, left=20mm, right=20mm, top=30mm, bottom=30mm}
\usepackage{graphicx}
\usepackage{bm} % for bold font in math mode - command is \bm{text}
\usepackage{enumitem}
\usepackage{fancyhdr}
\usepackage{amssymb} % for stacked arrows and other shit
\pagestyle{fancy}
\usepackage{changepage}
\usepackage{mathcomp}
\usepackage{tcolorbox}

\declaretheoremstyle[headfont=\normalfont]{normal}
\declaretheorem[style=normal]{Theorem}
\declaretheorem[style=normal]{Proposition}
\declaretheorem[style=normal]{Lemma}
\newcounter{ProofCounter}
\newcounter{ClaimCounter}[ProofCounter]
\newcounter{SubClaimCounter}[ClaimCounter]
\newenvironment{Proof}{\stepcounter{ProofCounter}\textsc{Proof.}}{\hfill$\square$}
\newenvironment{Solution}{\stepcounter{ProofCounter}\textbf{Solution:}}{\hfill$\square$}
\newenvironment{claim}[1]{\vspace{1mm}\stepcounter{ClaimCounter}\par\noindent\underline{\bf Claim \theClaimCounter:}\space#1}{}
\newenvironment{claimproof}[1]{\par\noindent\underline{Proof of claim \theClaimCounter:}\space#1}{\hfill $\blacksquare$ Claim \theClaimCounter}
\newenvironment{subclaim}[1]{\stepcounter{SubClaimCounter}\par\noindent\emph{Subclaim \theClaimCounter.\theSubClaimCounter:}\space#1}{}
% \newenvironment{subclaimproof}[1]{\begin{adjustwidth}{2em}{0pt}\par\noindent\emph{Proof of subclaim \theClaimCounter.\theSubClaimCounter:}\space#1}{\hfill
% $\blacksquare$ \emph{Subclaim \theClaimCounter.\theSubClaimCounter}\vspace{5mm}\end{adjustwidth}}
\newenvironment{subclaimproof}[1]{\par\noindent\emph{Proof of subclaim \theClaimCounter.\theSubClaimCounter:}\space#1}{\hfill
$\Diamond$ \emph{Subclaim \theClaimCounter.\theSubClaimCounter}}

\allowdisplaybreaks{}

% chktex-file 3

\title{MATH 502: HW 1}
\author{Evan ``Pete'' Walsh}
\makeatletter
\makeatother
\lhead{Evan P. Walsh}
\chead{MATH 502: HW 1}
\rhead{\thepage}
\cfoot{}

\begin{document}
\maketitle

\subsection*{1}
\begin{Solution}
  We will show that $\phi \geq 0$ is necessary and sufficient for $\rho$ to be a semi-norm on $C[0,1]$, while $\phi \geq 0$ and vanishing on no
  subinterval of $[0,1]$ is necessary and sufficient for $\rho$ to be a norm.

  \begin{claim}
    If $\phi \geq 0$, then for all $f,g \in C[0,1]$ and $\alpha \in \mathbb{F}$, $\rho(\alpha f) = |\alpha|\rho(f)$ and $\rho(f + g) \leq \rho(f) +
    \rho(g)$.
  \end{claim}
  \begin{claimproof}
    Follows from the linearity of integration and the triangle inequality on $\mathbb{F}$.

  \end{claimproof}

  \begin{claim}
    $\phi \geq 0$ if and only if $\rho$ is a semi-norm.
  \end{claim}
  \begin{claimproof}
    $(\Rightarrow)$ First suppose $\phi \geq 0$. Then clearly $\rho(f) \geq 0$. The rest follows from Claim 1.

    $(\Leftarrow)$ Now suppose $\rho$ is a semi-norm. By way of contradiction suppose there exists $x_0 \in [0,1]$ such that $\phi(x_0) < 0$.
    Then take $f(x) := \phi(x) \cdot \chi_{\{x : \phi(x) < 0\}}(x)$. Then $f \in C[0,1]$ but 
    \[
      \rho(f) = \int_{[0,1]}|f(x)|\phi(x)dx = \int_{\{x: \phi(x) < 0\}}-[\phi(x)]^2dx < 0,
    \]
    since $\{x : \phi(x) < 0\}$ is open and non-empty.
    This is a contradiction.
  \end{claimproof}

  \begin{claim}
    $\phi \geq 0$ and vanishes on no non-trivial subinterval interval if and only if $\rho$ is a norm.
  \end{claim}
  \begin{claimproof}
    $(\Leftarrow)$ First suppose $\rho$ is a norm. Then $\rho$ is a semi-norm, so $\phi \geq 0$ by Claim 2. Now, by way of contradiction, suppose that
    $\phi$ vanishes on some non-trivial subinterval $I \subseteq [0,1]$. Without loss of generality assume $I = (a,b)$. Define
    \[
      f(x) := \left\{ \begin{array}{cl}
          0 & \text{ if } x \leq a \\
          x - a & \text{ if } a < x < \frac{a+b}{2} \\
          b - x & \text{ if } \frac{a + b}{2} \leq x < b \\
          0 & \text{ if } x \geq b \\
      \end{array} \right. .
    \]
    Then $f \in C[0,1]$ and $\rho(f) = 0$. But $f \not\equiv 0$, which contradicts the assumption that $\rho$ is a norm.

    $(\Rightarrow)$ Now suppose $\phi \geq 0$ and vanishes and no non-trivial subinterval of $[0,1]$. By Claim 1, it suffices to show that $\rho(f) =
    0$ implies $f \equiv 0$ for all $f \in C[0,1]$. By way contradiction, suppose there exists $f \in C[0,1]$ such that $\rho(f) = 0$ but $f \not\equiv 0$.
    Then by continuity, 
    \[ E := \{x \in [0,1]: |f(x)| > 0\} \] 
    is open (as a subset of $[0,1]$)and non-empty, and thus there exists a non-empty open subinterval interval $I$ such that 
    $|f(x)| > 0$ on $I$. So 
    \begin{equation}
      0 = \int_{[0,1]}|f(x)|\phi(x)dx \geq \int_{I}|f(x)|\phi(x)dx \geq 0.
      \label{1.1}
    \end{equation}
    But since $|f(x)|\phi(x)$ is continuous, \eqref{1.1} implies that $|f(x)|\phi(x) \equiv 0$ on $I$.\footnote{See Rudin, \emph{Principles of
    Mathematical Analysis}, Page 138 Problem 2.} So $\phi \equiv 0$ on $I$, a contradiction.
  \end{claimproof}

\end{Solution}


\subsection*{2}
\begin{Solution}
  We will show that $\rho$ is a semi-norm but not a norm.

  \begin{claim}
    $\rho$ is semi-norm.
  \end{claim}
  \begin{claimproof}
    Clearly $\rho \geq 0$. Now let $f, g \in C^1[0,1]$ and $\alpha \in \alpha \in \mathbb{R}$. By the linearity of differentiation,
    \[
      \rho(\alpha f) = \max_{0\leq x \leq 1}|\alpha f'(x)| = |\alpha|\left[ \max_{0\leq x \leq 1}|f'(x)| \right] = |\alpha| \rho(f),
    \]
    and 
    \[
      \rho(f + g) = \max_{0\leq x\leq 1}|f'(x) + g'(x)| \leq \max_{0\leq x\leq 1}|f'(x)| + |g'(x)| \leq \max_{0\leq x\leq 1}|f'(x)| + \max_{0\leq x
      \leq 1}|g'(x)| = \rho(f) + \rho(g).
    \]
    Hence $\rho$ is a semi-norm.
  \end{claimproof}

  \begin{claim}
    $\rho$ is not a norm.
  \end{claim}
  \begin{claimproof}
    Consider the function $f \in C^1[0,1]$ where $f \equiv 1$. Then clearly $\rho(f) = 0$, but $f \not\equiv 0$.
    Hence $\rho$ is not a norm.
  \end{claimproof}

\end{Solution}


\subsection*{3}
\begin{Solution}
  Let $d_1(x,y) := \sqrt{|x - y|}$, $d_2(x,y) := (x - y)^2$ for $x, y \in \mathbb{R}$, and $d_3(x,y) := |\log(y/x)|$ for $x, y \in \mathbb{R}^+$.
  We will show that $d_1$ and $d_3$ are metrics but $d_2$ is not.
  Clearly $d_i$ satisfies $d_i(x,y) \geq 0$, $d_i(x,y) = d_i(y,x)$, and $d_i(x,y) = 0$ if and only if $x = y$, for each $i = 1,3$.
  Therefore to show that each $d_1$ and $d_3$ are metrics, it remains to show that the triangle inequality holds.
  First let $x, y, z \in \mathbb{R}$. Then
  \[
    d_1(x,z) = \sqrt{|x - z|} \leq \sqrt{|x-y| + |y-z|} \leq \sqrt{|z-y|} + \sqrt{|y-z|} = d_1(x,y) + d_1(y,z).
  \]
  Hence $d_1$ is a metric. Similarly, let $x,y,z \in \mathbb{R}^+$. Then
  \[
    d_3(x,z) = |\log(x) - \log(z)| \leq |\log(x) - \log(y)| + |\log(y) - \log(z)| = d_3(x,y) + d_3(y,z).
  \]
  So $d_3$ is also a metric. To show that $d_2$ is not a metric, let $x, y, z \in \mathbb{R}$ such that $z < y < x$. Then
  \begin{align*}
    d_2(x,z) = (z - z)^2 = (x - y + y - z)^2 & = (x - y)^2 + 2(x-y)(y-z) + (y-z)^2 \\
    & > (x-y)^2 + (y-z)^2 \\
    & = d_2(x,y) + d_2(y,z),
  \end{align*}
  since $2(x-y)(y-z) > 0$. Hence $d_2$ does not satisfy the triangle inequality, so is not a norm.
\end{Solution}

\subsection*{4}
\begin{Solution}
  \begin{description}
    \item[Section 2.1 \#1:] Clearly $d(A,B) = d(B, A)$ for all $A, B \in X$. Now suppose $d(A,B) = 0$. Then
      \[
        |(A-B)\cup (B-A)| = 0
      \]
      and so $|(A-B)| = |(B-A)| = 0$, which implies $A-B = B-A = \emptyset$ and thus $A = B$. It remains to verify the triangle inequality. Let $A, B,
      C \in X$. Then
      \begin{align*}
        d(A,C) & = |(A-B)\cup (C-D)| \\
        & = \big| [(A-C)-B]\cup[(A-C)\cap B] \cup [(C-A)-B] \cup [(C-A)\cap B] \bigg| \\
        & \leq \bigg| [(A-C)-B] \cup [(C-A)\cap B] \bigg| + \bigg| [(A-C)\cap B] \cup [(C-A) - B] \bigg| \\
        & \leq \bigg| (A-B) \cup (B-A) \bigg| + \bigg| (B-C) \cup (C-B) \bigg| \\
        & = d(A,B) + d(B,C).
      \end{align*}
      Hence $d$ is a metric.

    \item[Section 2.1 \#2:] Let $f,g,h \in B(S,Y)$. Clearly $D(f,g) = D(g,f)$. Now let $y \in S$. Then 
      \[
        \sup_x d\left( f(x), g(x) \right) \leq \sup_x \left[ d\left( f(x),f(y) \right) + d\left( f(y),g(y) \right) + d\left( g(y), g(x) \right)\right] < \infty.
      \]
      Further, suppose $D(f,g) = 0$. Then $d\left( f(x),g(x) \right) = 0$ for all $x \in S$. Hence $f(x) = g(x)$.
      Lastly, it remains to verify the triangle inequality. Well,
      \begin{align*}
        D(f,h) = \sup_x d\left( f(x),h(x) \right) & \leq \sup_x \left[ d\left( f(x),g(x) \right) + d\left( g(x),h(x) \right) \right] \\
        & \leq \sup_x d\left( f(x),g(x) \right) + \sup_x d\left( g(x), h(x) \right) \\
        & = D(f,g) + D(g,h).
      \end{align*}

    \item[Section 2.1 \#4:] No, $|||\cdot |||$ is not a norm. To see why, let $\alpha \in \mathbb{F}, \alpha\neq 0$, $x,y \in E$. Then
      \[
        |||\alpha x||| = \frac{ |\alpha| \cdot ||x||}{1 + |\alpha| \cdot ||x||} \neq \frac{ |\alpha| \cdot ||x||}{1 + ||x||} = |\alpha|\cdot |||x|||.
      \]
      Hence $|||\cdot |||$ is not a norm.
  \end{description}
\end{Solution}




\end{document}
