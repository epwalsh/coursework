\documentclass[12pt]{article}
\usepackage{amsmath}
\usepackage{amsfonts}
\usepackage{parskip}
\usepackage{amsthm}
\usepackage{thmtools}
\usepackage[headheight=15pt]{geometry}
\geometry{a4paper, left=20mm, right=20mm, top=30mm, bottom=30mm}
\usepackage{graphicx}
\usepackage{bm} % for bold font in math mode - command is \bm{text}
\usepackage{enumitem}
\usepackage{fancyhdr}
\usepackage{amssymb} % for stacked arrows and other shit
\pagestyle{fancy}
\usepackage{changepage}
\usepackage{mathcomp}
\usepackage{tcolorbox}

\declaretheoremstyle[headfont=\normalfont]{normal}
\declaretheorem[style=normal]{Theorem}
\declaretheorem[style=normal]{Proposition}
\declaretheorem[style=normal]{Lemma}
\newcounter{ProofCounter}
\newcounter{ClaimCounter}[ProofCounter]
\newcounter{SubClaimCounter}[ClaimCounter]
\newenvironment{Proof}{\stepcounter{ProofCounter}\textsc{Proof.}}{\hfill$\square$}
\newenvironment{Solution}{\stepcounter{ProofCounter}\textbf{Solution:}}{\hfill$\square$}
\newenvironment{claim}[1]{\vspace{1mm}\stepcounter{ClaimCounter}\par\noindent\underline{\bf Claim \theClaimCounter:}\space#1}{}
\newenvironment{claimproof}[1]{\par\noindent\underline{Proof of claim \theClaimCounter:}\space#1}{\hfill $\blacksquare$ Claim \theClaimCounter}
\newenvironment{subclaim}[1]{\stepcounter{SubClaimCounter}\par\noindent\emph{Subclaim \theClaimCounter.\theSubClaimCounter:}\space#1}{}
% \newenvironment{subclaimproof}[1]{\begin{adjustwidth}{2em}{0pt}\par\noindent\emph{Proof of subclaim \theClaimCounter.\theSubClaimCounter:}\space#1}{\hfill
% $\blacksquare$ \emph{Subclaim \theClaimCounter.\theSubClaimCounter}\vspace{5mm}\end{adjustwidth}}
\newenvironment{subclaimproof}[1]{\par\noindent\emph{Proof of subclaim \theClaimCounter.\theSubClaimCounter:}\space#1}{\hfill
$\Diamond$ \emph{Subclaim \theClaimCounter.\theSubClaimCounter}}

\allowdisplaybreaks{}

% chktex-file 3

\title{MATH 502: HW 1}
\author{Evan P. Walsh}
\makeatletter
\makeatother
\lhead{Evan P. Walsh}
\chead{MATH 502: HW 1}
\rhead{\thepage}
\cfoot{}

\begin{document}
\maketitle

\subsection*{1}
\begin{Proof}
  We will show that $\phi \geq 0$ is necessary and sufficient for $\rho$ to be a semi-norm on $C[0,1]$, while $\phi \geq 0$ and vanishing on no
  subinterval of $[0,1]$ is necessary and sufficient for $\rho$ to be a norm.

  \begin{claim}
    If $\phi \geq 0$, then for all $f,g \in C[0,1]$ and $\alpha \in \mathbb{F}$, $\rho(\alpha f) = |\alpha|\rho(f)$ and $\rho(f + g) \leq \rho(f) +
    \rho(g)$.
  \end{claim}
  \begin{claimproof}
    Follows from the linearity of integration.
  \end{claimproof}

  \begin{claim}
    $\phi \geq 0$ if and only if $\rho$ is a semi-norm.
  \end{claim}
  \begin{claimproof}
    $(\Rightarrow)$ First suppose $\phi \geq 0$. Then clearly $\rho(f) \geq 0$. The rest follows from Claim 1.

    $(\Leftarrow)$ Now suppose $\rho$ is a semi-norm. By way of contradiction suppose there exists $x_0 \in [0,1]$ such that $\phi(x_0) < 0$.
    Then take $f(x) := \phi(x) \cdot \chi_{\{x : \phi(x) < 0\}}(x)$. Then $f \in C[0,1]$ but 
    \[
      \rho(f) = \int_{[0,1]}|f(x)|\phi(x)dx = \int_{\{x: \phi(x) < 0\}}-[\phi(x)]^2dx < 0.
    \]
    This is a contradiction.
  \end{claimproof}

  \begin{claim}
    $\phi \geq 0$ and vanishes on no non-trivial subinterval interval if and only if $\rho$ is a norm.
  \end{claim}
  \begin{claimproof}
    $(\Leftarrow)$ First suppose $\rho$ is a norm. Then $\rho$ is a semi-norm, so $\phi \geq 0$ by Claim 2. Now, by way of contradiction, suppose that
    $\phi$ vanishes on some non-trivial subinterval $I \subseteq [0,1]$. Without loss of generality assume $I = (a,b)$. Define
    \[
      f(x) := \left\{ \begin{array}{cl}
          0 & \text{ if } x \leq a \\
          x - a & \text{ if } a < x < \frac{a+b}{2} \\
          b - x & \text{ if } \frac{a + b}{2} \leq x < b \\
          0 & \text{ if } x \geq b \\
      \end{array} \right. .
    \]
    Then $f \in C[0,1]$ and $\rho(f) = 0$. But $f \not\equiv 0$. This contradicts the assumption that $\rho$ is a norm.

    $(\Rightarrow)$ Now suppose $\phi \geq 0$ and vanishes and no non-trivial subinterval of $[0,1]$. By Claim 1, it suffices to show that $\rho(f) =
    0$ implies $f \equiv 0$ for all $f \in C[0,1]$. By way contradiction, suppose there exists $f \in C[0,1]$ such that $\rho(f) = 0$ but $f \not\equiv 0$.
    Then by continuity, 
    \[ E := \{x \in [0,1]: |f(x)| > 0\} \] 
    is open and non-empty, and thus there exists a non-empty open interval $I$ such that 
    $|f(x)| > 0$ on $I$. So 
    \begin{equation}
      0 = \int_{[0,1]}|f(x)|\phi(x)dx \geq \int_{I}|f(x)|\phi(x)dx \geq 0.
      \label{1.1}
    \end{equation}
    But since $|f(x)|\phi(x)$ is continuous, \eqref{1.1} implies that $|f(x)|\phi(x) \equiv 0$ on $I$. So $\phi \equiv 0$ on $I$, a contradiction.
  \end{claimproof}

\end{Proof}


\end{document}

