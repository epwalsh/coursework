\documentclass[12pt]{article}
\usepackage{amsmath}
\usepackage{amsfonts}
\usepackage{parskip}
\usepackage{amsthm}
\usepackage{thmtools}
\usepackage[headheight=15pt]{geometry}
\geometry{a4paper, left=20mm, right=20mm, top=30mm, bottom=30mm}
\usepackage{graphicx}
\usepackage{bm} % for bold font in math mode - command is \bm{text}
\usepackage{enumitem}
\usepackage{fancyhdr}
\usepackage{amssymb} % for stacked arrows and other shit
\pagestyle{fancy}
\usepackage{changepage}
\usepackage{mathcomp}
\usepackage{tcolorbox}
\usepackage{eufrak}

\declaretheoremstyle[headfont=\normalfont]{normal}
\declaretheorem[style=normal]{Theorem}
\declaretheorem[style=normal]{Proposition}
\declaretheorem[style=normal]{Lemma}
\newcounter{ProofCounter}
\newcounter{ClaimCounter}[ProofCounter]
\newcounter{SubClaimCounter}[ClaimCounter]
\newenvironment{Proof}{\stepcounter{ProofCounter}\textsc{Proof.}}{\hfill$\square$}
\newenvironment{Solution}{\stepcounter{ProofCounter}\textbf{Solution:}}{\hfill$\square$}
\newenvironment{claim}[1]{\vspace{1mm}\stepcounter{ClaimCounter}\par\noindent\underline{\bf Claim \theClaimCounter:}\space#1}{}
\newenvironment{claimproof}[1]{\par\noindent\underline{Proof of claim \theClaimCounter:}\space#1}{\hfill $\blacksquare$ Claim \theClaimCounter}
\newenvironment{subclaim}[1]{\stepcounter{SubClaimCounter}\par\noindent\emph{Subclaim \theClaimCounter.\theSubClaimCounter:}\space#1}{}
% \newenvironment{subclaimproof}[1]{\begin{adjustwidth}{2em}{0pt}\par\noindent\emph{Proof of subclaim \theClaimCounter.\theSubClaimCounter:}\space#1}{\hfill
% $\blacksquare$ \emph{Subclaim \theClaimCounter.\theSubClaimCounter}\vspace{5mm}\end{adjustwidth}}
\newenvironment{subclaimproof}[1]{\par\noindent\emph{Proof of subclaim \theClaimCounter.\theSubClaimCounter:}\space#1}{\hfill
$\Diamond$ \emph{Subclaim \theClaimCounter.\theSubClaimCounter}}

\allowdisplaybreaks{}

% chktex-file 3

\lhead{Evan P. Walsh}
\chead{MATH 502: Assignment VIII}
\rhead{\thepage}
\cfoot{}

\begin{document}\thispagestyle{empty}
\begin{center}
  \Large \textsc{math 502 -- ASSIGNMENT VIII -- spring 2017} \\ 
  \vspace{5mm}
  \large Evan Pete Walsh
\end{center}


\subsection*{1 [Sec. 3.3 \#2, 4, 7, and 8]}
\begin{Solution}
  \begin{enumerate}
    \item[\# 2.] The forward direction of the proof is trivial by the definition of compactness. For the reverse direction, suppose that each open
      cover of $X$ consisting of basic sets has a finite subcover. Let $\mathcal{O}$ be an open cover of $X$. Let 
      \[
        \mathcal{U} := \left\{ U \in \mathcal{B} : \exists\ O \in \mathcal{O} \text{ s.t. } U \subseteq O \right\}.
      \]
      Then since $\mathcal{B}$ is a base, 
      \[
        \bigcup_{O \in \mathcal{O}}O = \bigcup_{O \in \mathcal{O}}\left(\bigcup_{U\in \mathcal{U}, U \subseteq O}U \right)= \bigcup_{U\in \mathcal{U}}U.
      \]
      Thus $\mathcal{U}$ is a cover of $X$, so by assumption there exists $U_1, \dots, U_n \in \mathcal{U}$ such that 
      \[ U_1 \cup \dots \cup U_n = X. \]
      Now, for each $j = 1,\dots, n$, there exists $O_j \in \mathcal{O}$ such that $O_j \supseteq U_j$ Therefore 
      \[ O_1 \cup \dots \cup O_n = X. \]
      Hence $\mathcal{O}$ has a finite subcover. Since $\mathcal{O}$ was arbitrary, $X$ is compact.

    \item[\# 4.] $(\Rightarrow)$ First suppose $f$ is continuous. Let $i \in \mathbb{I}$. We will show that $\pi_i \circ f$ is continuous by
      showing that the pre-image of any open set in $Y_i$ is open in $X$.
      To that end, let $U \in \tau_{i}$. Then by definition of $\tau_{Y}$,
      $\pi_i^{-1}[U] \in \tau_Y$. Thus, since $f$ is continuous $f^{-1}\circ \pi_{i}^{-1}[U] \in \tau_X$. Hence $\pi_i \circ f$ is continuous.

      $(\Leftarrow)$ Now suppose $\pi_i \circ f : X \rightarrow Y_i$ is continuous for all $i \in \mathbb{I}$. We will show that $f$ is continuous by
      showing that the pre-image of any open set in $Y$ is open in $X$. To that end, let $U \in \tau_Y$. Then by Proposition 3.2.7, we may express $U$
      as 
      \[
        U = \bigcup\left\{ \pi_{i_1}^{-1}[U_1] \cap \dots \cap \pi_{i_n}^{-1}[U_n] : U_{j} \in \tau_{i_j} \forall j = 1,\dots, n \wedge \pi_{i_1}^{-1}[U_1] \cap \dots \cap \pi_{i_n}^{-1}[U_n]
        \subseteq U \right\},
      \]
      i.e. $U$ is the union of sets of the form $\pi_{i_1}^{-1}[U_1] \cap \dots \cap \pi_{i_n}^{-1}[U_n]$, where $U_j \in \tau_{i_j}$ for each $j =
      1,\dots, n$. Now, by assumption $f^{-1}\circ \pi_{i_j}^{-1}[U_{j}] \in \tau_X$ whenever $U_j \in \tau_{i_j}$, and therefore 
      \[
        f^{-1}\left[ \cap_{j=1}^{n}\pi_{i_j}^{-1}[U_j] \right] = \bigcap_{j=1}^{n}f^{-1}\circ \pi_{i_j}^{-1}[U_j] \in \tau_X
      \]
      since the finite intersection of open sets is open. Hence
      \begin{align*}
        f^{-1}[U] & = f^{-1}\left[ \bigcup\left\{ \cap_{j=1}^{n} \pi_{i_j}^{-1}[U_j] : U_{j} \in \tau_{i_j} \forall j = 1,\dots,
            n \wedge \cap_{j=1}^{n}\pi_{i_j}^{-1}[U_j]\subseteq U \right\} \right] \\
            & = \bigcup \left\{ f^{-1}\left[ \cap_{j=1}^{n}\pi_{i_j}^{-1}[U_j] \right] : U_{j} \in \tau_{i_j} \forall j = 1,\dots, n
          \wedge \cap_{j=1}^{n}\pi_{i_j}^{-1}[U_j]\subseteq U \right\}  \\
          & = \bigcup \left\{ \bigcap_{j=1}^{n}f^{-1}\circ \pi_{i_j}^{-1}[U_{j}] : U_{j} \in \tau_{i_j} \forall j = 1,\dots, n \wedge \cap_{j=1}^n \pi_{i_j}^{-1}[U_j]\subseteq U \right\} 
          \in \tau_X
      \end{align*}
      since the arbitrary union of open sets is open. Thus, since $U$ was arbitrary $f$ is continuous.

    \item[\# 7.] $(\Rightarrow)$ First suppose $(X, \tau)$ is Hausdorff. Let $i_0 \in \mathbb{I}$. Without loss of generality assume $X_{i_0}$ has at
      least two elements, otherwise we are done.
      Let $x_{i_0}, y_{i_0} \in X_{i_0}$ such that $x_{i_0} \neq y_{i_0}$. Then for each $i \neq i_{0}$, choose $x_i \equiv y_i \in X_i$ so that 
      $x := \left( x_i \right)_{i\in\mathbb{I}}, y := \left( y_i \right)_{i\in\mathbb{I}} \in X$ agree on every coordinate except $i_0$.
      So, since $(X, \tau)$ is Hausdorff, there exists open $U,V \in \tau$ such that $x \in U$, $y \in V$ and $U \cap V = \emptyset$. Then by Proposition
      3.2.7, $U$ and $V$ can be expressed as the union of sets of the form
      \begin{equation}
        \pi_{i_0}^{-1}[U_{0}] \cap \bigcap_{j=1}^{n} \pi_{i_j}^{-1}[U_j]   \ \text{ and }\ 
        \pi_{i_0}^{-1}[V_{0}] \cap \bigcap_{k=1}^{m} \pi_{i_k}^{-1}[V_k]
        \label{1.0}
      \end{equation}
      respectively, where $x_{i_j} \in U_j \in \tau_{i_j}, y_{i_k} \in V_{k} \in \tau_{i_k}$ for each $j = 0,\dots, n$, $k = 1,\dots,m$.
      But since $x_{i_j} \equiv y_{i_j}$, $x_{i_k} \equiv y_{i_k}$ for all $j = 1, \dots, n$ and $k = 1,\dots, m$, 
      \[
        \left( \bigcap_{j=1}^{n}\pi_{i_j}^{-1}[U_j] \right) \cap \left( \bigcap_{k=1}^{m}\pi_{i_k}^{-1}[V_k] \right) \neq \emptyset.
      \]
      Thus it must be that the $U_0$ and $V_0$ in \eqref{1.0} are disjoint in $X_{i_0}$.
      Hence $U_0$ is an open neighborhood of $x_{i_0}$ and $V_0$ is an open neighborhood of $y_{i_0}$ in $X_{i_0}$ with $U_0$, $V_0$ disjoint.
      So since $i_0 \in \mathbb{I}$ and $x_{i_0}, y_{i_0}$ in $X_{i_0}$ were arbitrary, $(X_i, \tau_i)$ is Hausdorff for all $i \in \mathbb{I}$.

      $(\Leftarrow)$ Now suppose $(X_i, \tau_i)$ is Hausdorff for each $i \in \mathbb{I}$. Let $x := (x_i)_{i\in\mathbb{I}}, y := \left( y_i
      \right)_{i\in\mathbb{I}} \in X$ such that $x \neq y$. Then there exists $i_0 \in \mathbb{I}$ such that $x_{i_0} \neq y_{i_0}$. Now, since $(X_{i_0}, \tau_{i_0})$ is
      Hausdorff, there exists $U_{0}, V_{0} \in \tau_{i_0}$ such that $x_{i_0} \in U_{0}, y_{i_0} \in V_0$ and $U_0 \cap V_0 = \emptyset$.
      Then set $U := \pi_{i_0}^{-1}[U_0]$ and $V := \pi_{i_0}^{-1}[V_0]$. Then $U,V \in \tau$ by Proposition 3.2.7 and clearly $x \in U$, $y \in V$,
      $U\cap V = \emptyset$. Hence $(X,\tau)$ is Hausdorff.

    \item[\# 8.] $(\Rightarrow)$ Suppose $\left\{ (x,x) : x \in X \right\}$ is closed. Let $x_0, y_0 \in X$ such that $x_0 \neq y_0$.
      Then 
      \[
        (x_0, y_0) \in X \times X - \left\{ (x,x) : x \in X \right\}, 
      \]
      which is open. So there exists an open $O \subseteq X \times X$ such that 
      \[(x_0, y_0) \in O \subseteq X \times X - \left\{ (x,x) : x \in X \right\}.\] So by Proposition 3.2.7, $O$ can be expressed as the union of sets
      of the form $U \times V$ where $U, V \in \tau$. Let $U_0, V_0$ be any such sets. So $x_0 \in U_0$ and $y_0 \in V_0$, and since 
      \[
        U_0 \times V_0 \subseteq X \times X - \left\{ (x,x) : x \in X \right\},
      \]
      it must be that $x \neq y$ for all $x \in U_0$ and $y \in V_0$. Hence $U_0 \cap V_0 = \emptyset$. So since $x_0, y_0 \in X$ were arbitrary, 
      $(X, \tau)$ is Hausdorff.

      $(\Leftarrow)$ Suppose $X$ is Hausdorff. Let
      \[
        (x_0,y_0) \in X\times X - \left\{ (x,x) : x \in X\right\}.
      \]
      Then $x_0 \neq y_0$, so
      there exists open $U, V \subseteq X$ such that $x_0 \in U$, $y_0 \in V$ and $U \cap V = \emptyset$.
      Then let $O := U \times V$, which is open in the topological product by Proposition 3.2.7.
      Further, $(x_0, y_0) \in O$, and since $x \neq y$ for all $x \in U, y \in V$, $O \subseteq X\times X - \left\{ (x,x) : x \in X\right\}$.
      Hence $X\times X - \left\{ (x,x) : x \in X\right\}$ is open, and so $\left\{ (x,x) : x \in X\right\}$ is closed.
  \end{enumerate}
\end{Solution}

\subsection*{2}
\begin{Solution}
  In problem 6 of the last homework we showed that $\tau$ second countable and finer than the usual topology on $\mathbb{R}$.

  \begin{claim}
    $[0,1]$ is compact.
  \end{claim}
  \begin{claimproof}
    Let $\mathcal{U}$ be an open cover of $[0,1]$ with respect to $\tau$.
    By the result from Sec. 3.2 Exercise 2, we may assume without loss of generality that $\mathcal{U} \subseteq \mathcal{B}$.
    Since $[0,1] \subseteq \bigcup_{U \in \mathcal{U}}U$, there exists $U_0 \equiv [a_0,b_0) \in \mathcal{U}$ such that $0 \in U_0$. Hence $a_0 \leq 0 \leq
    b_0$. Now let $0 < \epsilon < b$, and set $I := [\epsilon, 1]$. Let $\mathcal{B}' := \{(a,b) : [a,b) \in \mathcal{B}\}$ so that $\mathcal{B}'$ is
    an open cover of $I$ under the usual topology of $\mathbb{R}$. Hence there exists $(a_1, b_1), \dots, (a_n, b_n) \in \mathcal{B}'$ such that 
    \[
      I \subseteq \cup_{i=1}^{n}(a_i, b_i)
    \]
    since $I$ is compact under the usual topology. But then 
    \[
      [0,1] = [0,\epsilon) \cup [\epsilon, 1] \subseteq [a_0, b_0) \cup \bigcup_{i=1}^{n}(a_i, b_i) \subseteq [a_0, b_0) \cup \bigcup_{i=1}^{n}[a_i,
      b_i),
    \]
    so $[a_0, b_0), [a_1, b_1), \dots, [a_n, b_n) \in \mathcal{B}$ is a finite subcover of $[0,1]$. Hence $[0,1]$ is compact in $\tau$.

  \end{claimproof}

  \begin{claim}
    $\mathbb{R}$ is locally compact with respect to $\tau$.
  \end{claim}
  \begin{claimproof}
    Let $x \in \mathbb{R}$. The work done in Claim 1 shows that any interval of the form $[a,b]$ is compact in $\tau$. Thus $[x,x+1]$ is compact and
    is a neighborhood of $x$ since $[x,x+1]\supset [x,x+1) \in \tau$.
    Since $x$ was arbitrary, $\mathbb{R}$ is locally compact with respect to $\tau$.
  \end{claimproof}

\end{Solution}

\subsection*{3}
\begin{Solution}
  Let $\left\{ r_n \right\}_{n=0}^{\infty}$ be an enumeration of the rationals and define $f, g: (\mathbb{R}, \tau) \rightarrow (\mathbb{R}, |\cdot|)$
  by
  \[
    f(x) := \sum_{r_n < x}2^{-n} \ \text{ and } \ g(x) := \sum_{r_n \leq x}2^{-n},
  \]
  where $\tau$ is as in Problem 2. We will show that $g$ is continuous and $f$ is not continuous. 

  \begin{claim}
    $g$ is continuous
  \end{claim}
  \begin{claimproof}
    Since $g$ viewed as a function from $(\mathbb{R}, |\cdot|) \rightarrow (\mathbb{R}, |\cdot|)$ is continuous from the right at every point $x \in
    \mathbb{R}$ and continuous at each irrational, it follows that $g^{-1}[I]$ is either an open interval $(a,b)$ or a interval of the form $[a,b)$,
    with $a \in \mathbb{Q}$, for any open interval $I \in |\cdot|$. In both cases, the pre-image of $I$ under $g$ is open in $\tau$.
    Hence $g$ is continuous.
  \end{claimproof}

  \begin{claim}
    $f$ is not continuous.
  \end{claim}
  \begin{claimproof}
    Note that since $0 \in \mathbb{Q}$, $f$ has a jump discontinuity at $0$. Let $y := f(0)$ and let $n_0 \in \mathbb{N}$ such that $r_{n_0} = 0$.
    Then 
    \[
      f^{-1}[(-1, y + 2^{-n_0-1})] = (-\infty, 0],
    \]
    which is not open in $\tau$. Hence $f$ is not continuous.
  \end{claimproof}

\end{Solution}


\end{document}
