\documentclass[12pt]{article}
\usepackage{amsmath}
\usepackage{amsfonts}
\usepackage{parskip}
\usepackage{amsthm}
\usepackage{thmtools}
\usepackage[headheight=15pt]{geometry}
\geometry{a4paper, left=20mm, right=20mm, top=30mm, bottom=30mm}
\usepackage{graphicx}
\usepackage{bm} % for bold font in math mode - command is \bm{text}
\usepackage{enumitem}
\usepackage{fancyhdr}
\usepackage{amssymb} % for stacked arrows and other shit
\pagestyle{fancy}
\usepackage{changepage}
\usepackage{mathcomp}
\usepackage{tcolorbox}
% \usepackage{eufrak}

\declaretheoremstyle[headfont=\normalfont]{normal}
\declaretheorem[style=normal]{Theorem}
\declaretheorem[style=normal]{Proposition}
\declaretheorem[style=normal]{Lemma}
\newcounter{ProofCounter}
\newcounter{ClaimCounter}[ProofCounter]
\newcounter{SubClaimCounter}[ClaimCounter]
\newenvironment{Proof}{\stepcounter{ProofCounter}\textsc{Proof.}}{\hfill$\square$}
\newenvironment{Solution}{\stepcounter{ProofCounter}\textbf{Solution:}}{\hfill$\square$}
\newenvironment{claim}[1]{\vspace{1mm}\stepcounter{ClaimCounter}\par\noindent\underline{\bf Claim \theClaimCounter:}\space#1}{}
\newenvironment{claimproof}[1]{\par\noindent\underline{Proof of claim \theClaimCounter:}\space#1}{\hfill $\blacksquare$ Claim \theClaimCounter}
\newenvironment{subclaim}[1]{\stepcounter{SubClaimCounter}\par\noindent\emph{Subclaim \theClaimCounter.\theSubClaimCounter:}\space#1}{}
% \newenvironment{subclaimproof}[1]{\begin{adjustwidth}{2em}{0pt}\par\noindent\emph{Proof of subclaim \theClaimCounter.\theSubClaimCounter:}\space#1}{\hfill
% $\blacksquare$ \emph{Subclaim \theClaimCounter.\theSubClaimCounter}\vspace{5mm}\end{adjustwidth}}
\newenvironment{subclaimproof}[1]{\par\noindent\emph{Proof of subclaim \theClaimCounter.\theSubClaimCounter:}\space#1}{\hfill
$\Diamond$ \emph{Subclaim \theClaimCounter.\theSubClaimCounter}}

\allowdisplaybreaks{}

% chktex-file 3

\lhead{Evan P. Walsh}
\chead{\textsc{Math 502 Assignment XII}}
\rhead{\thepage}
\cfoot{}

\begin{document}\thispagestyle{empty}
\begin{center}
  \Large \textsc{math 502 -- ASSIGNMENT XII -- spring 2017} \\ 
  \vspace{5mm}
  \large Evan Pete Walsh
\end{center}

\subsection*{1}
\begin{tcolorbox}
  Let $f : [0,1] \rightarrow \mathbb{R}$ be defined by $f(x) := 2x + 1$. Let $\mathcal{A}$ be the algebra over $\mathbb{R}$ generated by $f$. Thus, a
  typical element of $\mathcal{A}$ is a function $g := a_1 f + a_2 f^2 + \dots + a_n f^n$ for $a_j \in \mathbb{R}$, $1 \leq j \leq n$, and $n \in
  \mathbb{N}$. Is $\mathcal{A}$ dense in $C([0,1], \mathbb{R})$?
\end{tcolorbox}
\begin{Solution}
  
\end{Solution}


\newpage
\subsection*{2}
\begin{tcolorbox}
  Let $f : [0, \infty) \rightarrow \mathbb{R}$ be defined by $f(x) := e^{-x}$. Is the algebra generated by $f$ dense in $C_0([0,\infty), \mathbb{R})$?
\end{tcolorbox}
\begin{Solution}
  
\end{Solution}


\newpage
\subsection*{3}
\begin{tcolorbox}
  Let $S \subset \mathbb{R}^2$, with the topology induced from the metric topology of $\mathbb{R}^2$,
  \[
    S := \{(x,y) : 1 \leq x \leq 2, 1 \leq y \leq 2\}.
  \]
  Let $f, g : S \rightarrow \mathbb{R}$ be defined by $f(x,y) := 1 / x$ and $g(x,y) := 1 / y$. Let $\mathcal{A}$ be the algebra of functions generated
  by $f$ and $g$. Is $\mathcal{A}$ dense in $C(S, \mathbb{R})$?
\end{tcolorbox}
\begin{Solution}
  
\end{Solution}


\newpage
\subsection*{4}
\begin{tcolorbox}
  The unit circle $\mathbb{T}$ can be expressed as the set of complex numbers $z$ such that $|z| = 1$. Let $\mathcal{A}$ be the algebra of functions
  $f : \mathbb{T} \rightarrow \mathbb{C}$ such that $f(z) = \sum_{-n}^{n}a_j z^{j}$, $a_j \in \mathbb{C}$, $1 \leq j \leq n$, $n \in \mathbb{N}$. Is
  $\mathcal{A}$ dense in $C(\mathbb{T}, \mathbb{C})$?
\end{tcolorbox}
\begin{Solution}
  
\end{Solution}


\newpage
\subsection*{5 [Sec. 4.3 \#2, 5, 6, 8]}
\begin{Solution}
  \begin{enumerate}
    \item[\#2.] 
  \end{enumerate}
\end{Solution}

\end{document}
