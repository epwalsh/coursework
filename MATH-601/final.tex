\documentclass[12pt]{article}
\usepackage{amsmath}
\usepackage{amsfonts}
\usepackage{parskip}
\usepackage{amsthm}
\usepackage{thmtools}
\usepackage[headheight=15pt]{geometry}
\geometry{a4paper, left=20mm, right=20mm, top=30mm, bottom=30mm}
\usepackage{graphicx}
\usepackage{bm} % for bold font in math mode - command is \bm{text}
\usepackage{enumitem}
\usepackage{fancyhdr}
\usepackage{amssymb} % for stacked arrows and other shit
\pagestyle{fancy}
\usepackage{changepage}
\usepackage{mathcomp}
\usepackage{tcolorbox}
% \usepackage{eufrak}

\declaretheoremstyle[headfont=\normalfont]{normal}
\declaretheorem[style=normal]{Theorem}
\declaretheorem[style=normal]{Proposition}
\declaretheorem[style=normal]{Lemma}
\newcounter{ProofCounter}
\newcounter{ClaimCounter}[ProofCounter]
\newcounter{SubClaimCounter}[ClaimCounter]
\newenvironment{Proof}{\stepcounter{ProofCounter}\textsc{Proof.}}{\hfill$\square$}
\newenvironment{Solution}{\stepcounter{ProofCounter}\textbf{Solution:}}{\hfill$\square$}
\newenvironment{claim}[1]{\vspace{1mm}\stepcounter{ClaimCounter}\par\noindent\underline{\bf Claim \theClaimCounter:}\space#1}{}
\newenvironment{claimproof}[1]{\par\noindent\underline{Proof of claim \theClaimCounter:}\space#1}{\hfill $\blacksquare$ Claim \theClaimCounter}
\newenvironment{subclaim}[1]{\stepcounter{SubClaimCounter}\par\noindent\emph{Subclaim \theClaimCounter.\theSubClaimCounter:}\space#1}{}
% \newenvironment{subclaimproof}[1]{\begin{adjustwidth}{2em}{0pt}\par\noindent\emph{Proof of subclaim \theClaimCounter.\theSubClaimCounter:}\space#1}{\hfill
% $\blacksquare$ \emph{Subclaim \theClaimCounter.\theSubClaimCounter}\vspace{5mm}\end{adjustwidth}}
\newenvironment{subclaimproof}[1]{\par\noindent\emph{Proof of subclaim \theClaimCounter.\theSubClaimCounter:}\space#1}{\hfill
$\Diamond$ \emph{Subclaim \theClaimCounter.\theSubClaimCounter}}

\allowdisplaybreaks{}

% chktex-file 3

\lhead{Evan P. Walsh}
\chead{MATH 601: Final Exam}
\rhead{\thepage}
\cfoot{}

\begin{document}\thispagestyle{empty}
\begin{center}
  \Large \textsc{math 601 -- Final Exam -- fall 2017} \\ 
  \vspace{5mm}
  \large Evan Pete Walsh
\end{center}

\subsection*{1}
\begin{tcolorbox}
  Suppose $A \subseteq \mathbb{N}$. Prove that $A$ is recursive if and only if the increasing enumeration of $A$ is recursive.
\end{tcolorbox}

\begin{Solution}
  Let $w : \mathbb{N} \rightarrow \mathbb{N}$ be the increasing enumeration of $A$.

  $(\Rightarrow)$ Suppose $A$ is recursive. Then note that 
  \begin{align*}
    w(0) & = \mu y [ y \in A ] \\
    w(n + 1) & = \mu y [ y > w(n) \wedge y \in A ] \text{ for all } n \geq 0.
  \end{align*}
  Hence, by the theorem of sequential recursion, $w$ is recursive.

  $(\Leftarrow)$ Now suppose $w$ is recursive. We need to show that $A$ is recursive. But note that 
  \[ x \in A \Leftrightarrow (\exists n \leq x) w(n) = x. \]
  Hence $A$ is recursive by closure under bounded quantification.
\end{Solution}


\newpage
\subsection*{2}
\begin{tcolorbox}
  A \emph{completion} of Peano Arithmetic is a complete theory $T$ in $\mathcal{L}_{PA}$ that includes the axioms of Peano Arithmetic. Prove there are
  uncountable many completions of Peano Arithmetic.
\end{tcolorbox}

\begin{Solution}
  Let $\{\Phi_0, \Phi_1, \dots\}$ be an enumeration of the sentences of $\mathcal{L}_{PA}$.
  We will show that given any $\nu \in \{0,1\}^{\omega}$, we can construct a unique completion $T_\nu$ of $PA$. From there it
  follows that there are an uncountable number of completions of $PA$ since $\{0,1\}^{\omega}$ is of course uncountable.
  The following claim will be helpful.

  \begin{claim}
    Suppose $T_0 \subseteq \text{Wff}$ is finite and $PA \cup T_0$ is consistent. Then $PA \cup T_0$ is incomplete.
  \end{claim}
  \begin{claimproof}
    Let $T_1 := PA \cup T_0$.
    It suffices to show that the conditions for Theorem 8.2 hold for $T_1$. Well, since $T_1$ is finite, $T_1$ is recursive. Therefore $T_1$ is
    recursively enumerable.

    Also, since (1) - (6) in the statement of Theorem 8.2 are all consequences of $PA$, they are also consequences of $T_1$ by monotonicity.
  \end{claimproof}

  Now let $\nu \in \{0, 1\}^{\omega}$. We will now recursively construct $T_\nu$ as follows. Set $T_{\nu,0} := PA = PA \cup \emptyset$.
  So $T_{\nu,0}$ is consistent and finite. Hence, by Claim 1, there exists a minimal $k_0 \geq 0$ such that $T_{\nu,0} \not \vdash \ulcorner \Phi_{k_0} \urcorner$ and
  $T_{\nu,0} \not \vdash \ulcorner \neg \Phi_{k_0} \urcorner$. Then let $T_{\nu,1}'$ be such that for all $0 \leq k < k_0$, 
  $\ulcorner \Phi_k \urcorner \in T_{\nu,1}'$
  if $T_{\nu,0} \vdash \ulcorner \Phi_k \urcorner$ or $\ulcorner \neg \Phi_k \urcorner \in T_{\nu,1}'$ if $T_{\nu,0} \vdash \ulcorner \neg \Phi_k \urcorner$.
  Let 
  \[
    T_{\nu,1}'' := \left\{ \begin{array}{cl}
        PA \cup \{ \ulcorner \Phi_{k_0} \urcorner \} & \text{ if } \nu(0) = 0, \\
        PA \cup \{ \ulcorner \neg \Phi_{k_0} \urcorner \} & \text{ if } \nu(0) = 1.
    \end{array} \right.
  \]
  Let $T_{\nu,1} := T_{\nu,1}' \cup T_{\nu,1}''$.
  In either case, $T_{\nu,1}$ is consistent and finite. 
  
  Now, by way of induction, suppose that for $n > 0$, $T_{\nu,n}$ has been constructed. Then by Claim 1 we can find a minimal $k_n \geq 0$ such that 
  $T_{\nu,n} \not \vdash \ulcorner \Phi_{k_n} \urcorner$ and $T_{\nu,n} \not \vdash \ulcorner \neg \Phi_{k_n} \urcorner$. Then let $T_{\nu,n+1}'$ be such that for 
  all $0 \leq k < k_n$, $\ulcorner \Phi_k \urcorner \in T_{\nu,n+1}'$
  if $T_{\nu,n} \vdash \ulcorner \Phi_k \urcorner$ or $\ulcorner \neg \Phi_k \urcorner \in T_{\nu,n+1}'$ if $T_{\nu,n} \vdash \ulcorner \neg \Phi_k \urcorner$.
  Let 
  \[
    T_{\nu,n+1}'' := \left\{ \begin{array}{cl}
        PA \cup \{ \ulcorner \Phi_{k_n} \urcorner \} & \text{ if } \nu(n) = 0, \\
        PA \cup \{ \ulcorner \neg \Phi_{k_n} \urcorner \} & \text{ if } \nu(n) = 1.
    \end{array} \right.
  \]
  Then set $T_{\nu,n+1} := T_{\nu,n+1}' + T_{\nu,n+1}''$. Clearly $T_{\nu,n+1}$ is consistent by its construction.

  Now let $T_{\nu} := \cup_{n\geq 0} T_{\nu,n}$.

  \begin{claim}
    For all $\nu \in \{0, 1\}^{\omega}$, $T_{\nu}$ is consistent.
  \end{claim}
  \begin{claimproof}
    By way of contradiction suppose not. Then by compactness, there exists a finite set $T \subset T_{\nu}$ that is inconsistent. But since $T$ is
    finite, there exists $n_0 \in \mathbb{N}$ such that $T \subseteq T_{\nu,n_0}$. So by monotonicity, $T_{\nu,n_0}$ is inconsistent. This is a
    contradiction.
  \end{claimproof}

  \begin{claim}
    For all $\nu \in \{0, 1\}^{\omega}$, $T_{\nu}$ is complete.
  \end{claim}
  \begin{claimproof}
    By construction, for every sentence $\Phi$, either $\ulcorner\Phi\urcorner \in T_{\nu}$ or $\ulcorner \neg \Phi \urcorner \in T_{\nu}$.
    Hence $T_{\nu}$ is complete by Proposition 5.18 in Chapter 3.
  \end{claimproof}

  \begin{claim}
    If $\nu \neq \nu'$, then $T_{\nu} \neq T_{\nu'}$.
  \end{claim}
  \begin{claimproof}
    Let $n \in \mathbb{N}$ be the minimal natural number such that $\nu(n) \neq \nu'(n)$. Without loss of generality assume $\nu(n) = 0$ and $\nu'(n)
    = 1$.
    Let $k_n$ be the minimal natural number such that both
    $T_{\nu,n}$ and $T_{\nu',n}$ do not prove $\Phi_{k_n}$ nor its negation. Then, by definition, $\ulcorner \Phi_{k_n} \urcorner \in T_{\nu,n}$ and 
    $\ulcorner \neg \Phi_{k_n} \urcorner \in T_{\nu',n}$. So $\ulcorner \Phi_{k_n} \urcorner \in T_{\nu}$ and
    $\ulcorner \neg \Phi_{k_n} \urcorner \in T_{\nu'}$. Hence $T_{\nu} \neq T_{\nu'}$.
  \end{claimproof}

  By the above claims, $\{ T_\nu : \nu \in \{0,1\}^{\omega} \}$ is an uncountable collections of completions of PA.
\end{Solution}


\newpage
\subsection*{3}
\begin{tcolorbox}
  Prove the G\"odel-Henkin completeness theorem for uncountable theories.
\end{tcolorbox}
We first need the following Lemma. \vspace{.5cm}

\begin{Lemma}
  \emph{
    Suppose $T$ is a consistent theory in a language $\mathcal{L}$. For each existential sentence $\Psi = \ulcorner (\exists \alpha) \Phi \urcorner$
    of $\mathcal{L}$, let $\kappa_{\Psi}$ be a unique new constant symbol that is not in $\mathcal{L}$. Then 
    \[
      T' := T \cup \{ \ulcorner ((\exists \alpha) \Phi \Rightarrow \Phi') \urcorner : \Psi = \ulcorner (\exists \alpha) \Phi \urcorner \text{ is an
      existential sentence of $\mathcal{L}$ and $\Phi' = \Phi[\alpha / \kappa_{\Psi}]$} \}
    \]
    is a consistent theory in the language $\mathcal{L}'$ consisting of the new constants $\{ \kappa_{\beta} \}$ added on to $\mathcal{L}$.
  }
\end{Lemma}
\begin{Proof}
  Let $\ulcorner (\exists \alpha_0) \Phi_0 \urcorner, \ulcorner (\exists \alpha_1)
  \Phi_1 \urcorner, \dots$ be an enumeration of the existential sentences of $\mathcal{L}$ (possibly invoking the Well Ordering Theorem if the set
  of existential sentences is uncountable), and $\kappa_0, \kappa_1, \dots$ the corresponds new constants. 
  Now set $T_0 := T \cup \{ \ulcorner((\exists \alpha_0) \Phi_0 \Rightarrow \Phi_0')\urcorner \}$, where $\Phi_0' := \Phi_0[\alpha_0 / \kappa_{0}]$.
  Let $\mathcal{L}_0$ be the result of adding the constant $\kappa_0$ to $\mathcal{L}$.
  \begin{claim}
    $T_0$ is consistent is the language $\mathcal{L}_0$.
  \end{claim}
  \begin{claimproof}
    By way of contradiction suppose not. Then it must be that 
    \[ T \vdash \ulcorner \neg ((\exists \alpha_0) \Phi_0 \Rightarrow \Phi_0') \urcorner. \]
    So $T \vdash \ulcorner ((\exists \alpha_0) \Phi_0 \wedge \neg \Phi_0') \urcorner$. Hence $T \vdash \ulcorner (\exists \alpha_0) \Phi_0 \urcorner$ and 
    $T \vdash \ulcorner \neg \Phi_0'\urcorner$. But then, by Lemma 4.12 in Chapter 4 and generalization, $T \vdash \ulcorner (\forall \alpha_0) \neg
    \Phi_0 \urcorner$, i.e. $T \vdash \ulcorner \neg (\exists \alpha_0) \Phi_0 \urcorner$. This is a contradiction. Hence $T_0$ is consistent in the
    lanuage $\mathcal{L}_0$.
  \end{claimproof}

  Now, for $\beta > 0$, we will define $T_{\beta}$ and $\mathcal{L}_\beta$ inductively as follows. If $\beta$ is a successor ordinal, let 
  \[ T_\beta := T_{\beta - 1} \cup \{ \ulcorner((\exists \alpha_\beta) \Phi_\beta \Rightarrow \Phi_\beta')\urcorner \}, \]
  and $\beta$ is a limit ordinal, let 
  \[ T_\beta := (\cup_{\gamma < \beta} T_{\gamma} )\cup \{ \ulcorner((\exists \alpha_\beta) \Phi_\beta \Rightarrow \Phi_\beta')\urcorner \}, \]
  where $\Phi_\beta' := \Phi[\alpha_\beta / \kappa_{\beta}]$. Let $\mathcal{L}_\beta$ be the language consisting of adding $\{ \kappa_{\gamma} : \gamma \leq
  \beta \}$ to $\mathcal{L}$.

  \begin{claim}
    $T_\beta$ is consistent in $\mathcal{L}_\beta$.
  \end{claim}
  \begin{claimproof}
    In the case that $\beta$ is a successor ordinal, the proof is exactly like in Claim 1. So without loss of generality suppose $\beta$ is a limit
    ordinal. By way of contradiction assume $T_\beta$ is inconsistent. Then it must be that 
    \[ \cup_{\gamma < \beta}T_\gamma \vdash \ulcorner \neg ((\exists \alpha_\beta) \Phi_\beta \Rightarrow \Phi_\beta') \urcorner. \]
    So by monotonicity, there exists $\gamma' < \beta$ such that 
    \[ T_{\gamma'} \vdash \ulcorner \neg ((\exists \alpha_\beta) \Phi_\beta \Rightarrow \Phi_\beta') \urcorner. \]
    But by the inductive assumption, $T_{\gamma'}$ is consistent. So we can now proceed as in Claim 1 to derive a contradiction.
  \end{claimproof}

  It remains to show that $T'$ is consistent. 

  \begin{claim}
    $T'$ consistent.
  \end{claim}
  \begin{claimproof}
    By way of contradiction suppose not. Let $\Phi$ be a well-formed forumla in the language of $\mathcal{L}'$. So $T' \vdash \Phi$ and $T' \vdash
    \ulcorner \neg \Phi \urcorner$.
    But note that $T' = T_0 \cup T_1 \cup \dots$.
    Hence, by monotonicity, there exists $\beta_0, \beta_1$ such that $T_{\beta_0} \vdash \Phi$ and $T_{\beta_1} \vdash \ulcorner \neg \Phi
    \urcorner$. Let $\gamma$ be the greater of $\beta_0$ and $\beta_1$. Then $T_{\gamma} \vdash \Phi$ and $T_{\gamma} \vdash \ulcorner \neg \Phi
    \urcorner$. Hence $T_{\gamma}$ is inconsistent. This is a contradiction.

  \end{claimproof}

\end{Proof}

\textbf{Now for the proof of the completeness theorem.}

\begin{Proof}
  $(\Leftarrow)$ Suppose $T$ is a theory and $\mathcal{A}$ is a model for $T$. By way of contradiction suppose $T$ is inconsistent. Let $\Phi$ be a
  wff. Then $T \vdash \Phi$ and $T \vdash \ulcorner \neg \Phi \urcorner$. So by the Soundness Theorem, $\mathcal{A} \models \Phi$ and $\mathcal{A}
  \models \ulcorner \neg \Phi \urcorner$, i.e. $\mathcal{A} \not \models \Phi$. This is a contradiction.

  $(\Rightarrow)$ Let $T$ be a consistent theory in a language $\mathcal{L}$. We will inductively construct a "maximal" theory $T'$ which is
  consistent and has witnesses.
  Let $T_0 = T$ and $\mathcal{L}_0 = \mathcal{L}$. By way of induction, for $n > 0$ let $\kappa_{n,\Psi}$ be a new unique constant not in $\mathcal{L}_n$
  for each existential sentence $\Psi$ of $\mathcal{L}_n$. Let $\mathcal{L}_{n+1}$ be the language formed by adding $\{ \kappa_{n,\Psi} \}$ to
  $\mathcal{L}_n$, and set
  \[
    T_{n+1} := T_n \cup \{ \ulcorner ((\exists \alpha) \Phi \Rightarrow \Phi') \urcorner : \Psi = \ulcorner (\exists \alpha) \Phi \urcorner \text{ is an
    existential sentence of $\mathcal{L}_n$ and $\Phi' = \Phi[\alpha / \kappa_{n,\Psi}]$} \}.
  \]
  Then by Lemma 1, $T_{n+1}$ is
  a consistent theory in the language $\mathcal{L}_{n+1}$. Now let $T' := \cup_{n=0}^{\infty} T_{n}$ and $\mathcal{L}'$ be the ``sum" of all the languages
  $\mathcal{L}_n$.

  \begin{claim}
    $T'$ is consistent.
  \end{claim}
  \begin{claimproof}
    This follows since each $T_{n}$ is consistent and $T_0 \subset T_1 \subset T_2 \subset \dots$.
  \end{claimproof}

  \begin{claim}
    $T'$ has witnesses.
  \end{claim}
  \begin{claimproof}
    Suppose $\Psi = \ulcorner (\exists \alpha) \Phi \urcorner$ is a theorem of $T'$. Then $\Psi$ is a theorem of $T_n$ for some $n \in \mathbb{N}$.
    But then $\ulcorner ((\exists \alpha) \Phi \Rightarrow \Phi') \urcorner \in T_{n+1}$, where $\Phi' = \Phi[\alpha / \kappa_{n,\Psi}]$. Hence
    $T_{n+1} \vdash \Phi'$, so $T' \vdash \Phi'$.

  \end{claimproof}

  By claims 1 and 2 and Theorem 4.7 of Chapter 4, $T'$ has a model $\mathcal{A}$. Let $\mathcal{B}$ be the reduct of $\mathcal{A}$ to $\mathcal{L}$.
  Then $\mathcal{B} \models T$.
  
\end{Proof}

\newpage
\subsection*{4}
\begin{tcolorbox}
  Give an example of two countable models of the same language that are not elementarily equivalent.
\end{tcolorbox}
\begin{Solution}
  Consider $\mathcal{A} := (\mathbb{Q}, \leq)$ and $\mathcal{B} := (\mathbb{Q} \cap [0,1], \leq)$ as interpretations of the language $\mathcal{L} :=
  \{`\leq'\}$. Let $\Phi := `(\forall x_0)(\exists x_1)(\leq (x_0, x_1) \wedge \neg (x_0 = x_1))'$.

  \begin{claim}
    $\mathcal{A} \models \Phi$.
  \end{claim}
  \begin{claimproof}
    $\mathcal{A} \models \Phi$ if and only if for all $a \in \mathbb{Q}$, 
    \[ \mathcal{A} \models `(\exists x_1)(\leq(x_0, x_1) \wedge \neg (x_0 =x_1))'[\{(x_0, a)\}], \] 
    if and only if for all $a \in \mathbb{Q}$, there is some $b \in \mathbb{Q}$ such that 
    \[ \mathcal{A} \models `(\leq(x_0, x_1) \wedge \neg (x_0 =x_1))'[\{(x_0, a), (x_1, b)\}], \] 
    if and only if for all $a \in \mathbb{Q}$, there is some $b \in \mathbb{Q}$ such that $a \leq b$ and $a \neq b$, which of course holds since we
    can always take $b := a + 1$.
  \end{claimproof}

  \begin{claim}
    $\mathcal{B} \not \models \Phi$.
  \end{claim}
  \begin{claimproof}
    Similarly to Claim 1, we have $\mathcal{B} \models \Phi$ 
    if and only if for all $a \in \mathbb{Q} \cap [0,1]$, there is some $b \in \mathbb{Q} \cap [0,1]$ such that $a \leq b$ and $a \neq b$.
    But clearly this does not hold for $a := 1$.

  \end{claimproof}

  By claims 1 and 2, $\mathcal{A}$ and $\mathcal{B}$ are not elementarily equivalent.

\end{Solution}

\end{document}
