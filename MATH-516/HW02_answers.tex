\documentclass[12pt]{article}
\usepackage{amsmath}
\usepackage{amsfonts}
\usepackage{parskip}
\usepackage{amsthm}
\usepackage{thmtools}
\usepackage[headheight=15pt]{geometry}
\geometry{a4paper, left=20mm, right=20mm, top=30mm, bottom=30mm}
\usepackage{graphicx}
\usepackage{bm} % for bold font in math mode - command is \bm{text}
\usepackage{enumitem}
\usepackage{fancyhdr}
\usepackage{amssymb} % for stacked arrows and other shit
\pagestyle{fancy}
\usepackage{changepage}
\usepackage{mathcomp}
\usepackage{tcolorbox}

\declaretheoremstyle[headfont=\normalfont]{normal}
\declaretheorem[style=normal]{Theorem}
\declaretheorem[style=normal]{Proposition}
\declaretheorem[style=normal]{Lemma}
\newcounter{ProofCounter}
\newcounter{ClaimCounter}[ProofCounter]
\newcounter{SubClaimCounter}[ClaimCounter]
\newenvironment{Proof}{\stepcounter{ProofCounter}\textit{Proof.}}{\hfill$\square$}
\newenvironment{claim}[1]{\vspace{1mm}\stepcounter{ClaimCounter}\par\noindent\underline{\bf Claim \theClaimCounter:}\space#1}{}
\newenvironment{claimproof}[1]{\par\noindent\underline{Proof of claim \theClaimCounter:}\space#1}{\hfill $\blacksquare$ Claim \theClaimCounter}
\newenvironment{subclaim}[1]{\stepcounter{SubClaimCounter}\par\noindent\emph{Subclaim \theClaimCounter.\theSubClaimCounter:}\space#1}{}
% \newenvironment{subclaimproof}[1]{\begin{adjustwidth}{2em}{0pt}\par\noindent\emph{Proof of subclaim \theClaimCounter.\theSubClaimCounter:}\space#1}{\hfill
% $\blacksquare$ \emph{Subclaim \theClaimCounter.\theSubClaimCounter}\vspace{5mm}\end{adjustwidth}}
\newenvironment{subclaimproof}[1]{\par\noindent\emph{Proof of subclaim \theClaimCounter.\theSubClaimCounter:}\space#1}{\hfill
$\Diamond$ \emph{Subclaim \theClaimCounter.\theSubClaimCounter}}

\title{MATH 516: HW 2}
\author{Evan P. Walsh}
\makeatletter
\let\runauthor\@author
\let\runtitle\@title
\makeatother
\lhead{\runauthor}
\chead{\runtitle}
\rhead{\thepage}
\cfoot{}

\begin{document}
\maketitle


\subsection*{1 [RF 17.19]}
\begin{tcolorbox}
Show that any measure induced by an outer measure is complete.
\end{tcolorbox}

\begin{Proof}
Suppose $A$ is $\mu^{*}$-measurable and $\mu(A) = 0$. Let $E \subseteq A$ and $B \subseteq X$. 
By subadditivity,
\begin{equation}
\mu^*(B) = \mu^*\left( (B\cap E) \cup (B\cap (X-E)) \right) \leq \mu^{*}(B\cap E) + \mu^{*}(B\cap(X-E)),
\label{1.1}
\end{equation}
Now, by monotonicity, 
\begin{equation*}
\mu^{*}(B\cap E) \leq \mu^*(A) = 0 \qquad \text{and} \qquad \mu^{*}(B\cap(X-E)) \leq \mu^*(B),
\end{equation*}
since $B\cap E \subseteq E \subseteq A$ and $B\cap (X-E) \subseteq B$. Thus
\begin{equation}
\mu^*(B\cap E) + \mu^*(B\cap (X-E)) = \mu^*(B\cap E) \leq \mu^*(B).
\label{1.3}
\end{equation}
So by \eqref{1.1} and \eqref{1.3}, $E$ is measurable.
\end{Proof}


\newpage 
\subsection*{2 [RF 17.7 i]}
\begin{tcolorbox}
Let $(X,\mathcal{M})$ be a measurable space. Verify that if $\mu$ and $\nu$ are measures defined on $\mathcal{M}$, then the set function $\lambda$ defined on
$\mathcal{M}$ by $\lambda(E) := \mu(E) + \nu(E)$ also is a measure. We denote $\lambda$ by $\mu + \nu$.
\end{tcolorbox}
\begin{Proof}
Clearly
\begin{equation*}
\lambda(\emptyset) = \mu(\emptyset) + \nu(\emptyset) = 0 + 0 = 0.
\end{equation*}
Now suppose $\left\{ A_{n} \right\}_{n=0}^{\infty} \subseteq \mathcal{M}$ is pairwise disjoint. Then 
\begin{align*}
\lambda\left( \cup_{n=0}^{\infty}A_{n} \right) = \mu\left( \cup_{n=0}^{\infty}A_{n} \right) + \nu\left( \cup_{n=0}^{\infty}A_{n} \right) = 
\sum_{n=0}^{\infty}\mu(A_{n}) + \sum_{n=0}^{\infty}\nu(A_{n}) & = \sum_{n=0}^{\infty}\left[ \mu(A_{n}) + \nu(A_{n}) \right] \\
& = \sum_{n=0}^{\infty}\lambda(A_{n}).
\end{align*}
Therefore $\lambda$ is a measure.
\end{Proof}

\newpage 
\subsection*{3 [RF 17.7 ii]}
\begin{tcolorbox}
Let $(X,\mathcal{M})$ be a measurable space. Verify that if $\mu$ and $\nu$ are measures on $\mathcal{M}$ and $\mu \geq \nu$, then there is a measure
$\lambda$ on $\mathcal{M}$ for which $\mu = \nu + \lambda$.
\end{tcolorbox}
\begin{Proof}
Set $\lambda(A) := \mu(A) - \nu(A)$ for every $A \in \mathcal{M}$. By assumption, $\lambda \geq 0$. Further,
\begin{equation*}
\lambda(\emptyset) = \mu(\emptyset) - \nu(\emptyset) = 0 - 0 = 0.
\end{equation*}
Now suppose $\left\{ A_{n} \right\}_{n=0}^{\infty} \subseteq \mathcal{M}$ is pairwise disjoint. We need to show 
\begin{equation}
\lambda\left( \cup_{n=0}^{\infty}A_{n} \right) = \sum_{n=0}^{\infty}\lambda(A_{n}).
\label{3.2}
\end{equation}

\end{Proof}

\newpage 
\subsection*{4 [RF 17.7 iii]}
\begin{tcolorbox}
Let $(X,\mathcal{M})$ be a measurable space. Verify that if $\nu$ is $\sigma$-finite, the measure $\lambda$ as in 3 is unique.
\end{tcolorbox}

\newpage 
\subsection*{5 [RF 17.7 iv]}
\begin{tcolorbox}
Let $(X,\mathcal{M})$ be a measurable space. Show that in general the measure $\lambda$ as in 3 need not be unique but that there is always a smallest
such $\lambda$.
\end{tcolorbox}





\end{document}

