\documentclass[12pt]{article}
\usepackage{amsmath}
\usepackage{amsfonts}
\usepackage{parskip}
\usepackage{amsthm}
\usepackage{thmtools}
\usepackage[headheight=15pt]{geometry}
\geometry{a4paper, left=20mm, right=20mm, top=30mm, bottom=30mm}
\usepackage{graphicx}
\usepackage{bm} % for bold font in math mode - command is \bm{text}
\usepackage{enumitem}
\usepackage{fancyhdr}
\usepackage{amssymb} % for stacked arrows and other shit
\pagestyle{fancy}
\usepackage{changepage}
\usepackage{mathcomp}
\usepackage{tcolorbox}

\declaretheoremstyle[headfont=\normalfont]{normal}
\declaretheorem[style=normal]{Theorem}
\declaretheorem[style=normal]{Proposition}
\declaretheorem[style=normal]{Lemma}
\newcounter{ProofCounter}
\newcounter{ClaimCounter}[ProofCounter]
\newcounter{SubClaimCounter}[ClaimCounter]
\newenvironment{Proof}{\stepcounter{ProofCounter}\textsc{Proof.}}{\hfill$\square$}
\newenvironment{claim}[1]{\vspace{1mm}\stepcounter{ClaimCounter}\par\noindent\underline{\bf Claim \theClaimCounter:}\space#1}{}
\newenvironment{claimproof}[1]{\par\noindent\underline{Proof of claim \theClaimCounter:}\space#1}{\hfill $\blacksquare$ Claim \theClaimCounter}
\newenvironment{subclaim}[1]{\stepcounter{SubClaimCounter}\par\noindent\emph{Subclaim \theClaimCounter.\theSubClaimCounter:}\space#1}{}
% \newenvironment{subclaimproof}[1]{\begin{adjustwidth}{2em}{0pt}\par\noindent\emph{Proof of subclaim \theClaimCounter.\theSubClaimCounter:}\space#1}{\hfill
% $\blacksquare$ \emph{Subclaim \theClaimCounter.\theSubClaimCounter}\vspace{5mm}\end{adjustwidth}}
\newenvironment{subclaimproof}[1]{\par\noindent\emph{Proof of subclaim \theClaimCounter.\theSubClaimCounter:}\space#1}{\hfill
$\Diamond$ \emph{Subclaim \theClaimCounter.\theSubClaimCounter}}

\title{MATH 516: HW 2}
\author{Evan P. Walsh}
\makeatletter
\let\runauthor\@author
\let\runtitle\@title
\makeatother
\lhead{\runauthor}
\chead{\runtitle}
\rhead{\thepage}
\cfoot{}

\begin{document}
\maketitle


\subsection*{1 [RF 17.19]}
\begin{tcolorbox}
Show that any measure induced by an outer measure is complete.
\end{tcolorbox}

\begin{Proof}
Suppose $E \subseteq A$ and $A$ is $\mu^{*}$-measurable with $\mu(A) = 0$. Let $B \subseteq X$. We need to show that $E$ is $\mu^{*}$-measurable, i.e.
$\mu^*(B) = \mu^*(B\cap E) + \mu^*(B\cap(X-E))$. By subadditivity,
\begin{equation}
\mu^*(B) = \mu^*\left( (B\cap E) \cup (B\cap (X-E)) \right) \leq \mu^{*}(B\cap E) + \mu^{*}(B\cap(X-E)),
\label{1.1}
\end{equation}
Now, by monotonicity, 
\begin{equation*}
\mu^{*}(B\cap E) \leq \mu^*(A) = 0 \qquad \text{and} \qquad \mu^{*}(B\cap(X-E)) \leq \mu^*(B),
\end{equation*}
since $B\cap E \subseteq E \subseteq A$ and $B\cap (X-E) \subseteq B$. Thus
\begin{equation}
\mu^*(B\cap E) + \mu^*(B\cap (X-E)) = \mu^*(B\cap (X-E)) \leq \mu^*(B).
\label{1.3}
\end{equation}
So by \eqref{1.1} and \eqref{1.3}, $E$ is measurable.
\end{Proof}


\subsection*{2 [RF 17.7 i]}
\begin{tcolorbox}
Let $(X,\mathcal{M})$ be a measurable space. Verify that if $\mu$ and $\nu$ are measures defined on $\mathcal{M}$, then the set function $\lambda$ defined on
$\mathcal{M}$ by $\lambda(E) := \mu(E) + \nu(E)$ also is a measure. We denote $\lambda$ by $\mu + \nu$.
\end{tcolorbox}
\begin{Proof}
Clearly
\begin{equation*}
\lambda(\emptyset) = \mu(\emptyset) + \nu(\emptyset) = 0 + 0 = 0.
\end{equation*}
Now suppose $A_{0}, A_{1}, \hdots \in \mathcal{M}$ such that $\left\{ A_{n} \right\}_{n=0}^{\infty}$ is pairwise disjoint. Then 
\begin{align*}
\lambda\left( \cup_{n=0}^{\infty}A_{n} \right) = \mu\left( \cup_{n=0}^{\infty}A_{n} \right) + \nu\left( \cup_{n=0}^{\infty}A_{n} \right) = 
\sum_{n=0}^{\infty}\mu(A_{n}) + \sum_{n=0}^{\infty}\nu(A_{n}) & = \sum_{n=0}^{\infty}\left[ \mu(A_{n}) + \nu(A_{n}) \right] \\
& = \sum_{n=0}^{\infty}\lambda(A_{n}).
\end{align*}
Therefore $\lambda$ is a measure.
\end{Proof}

\newpage 
\subsection*{3 [RF 17.7 ii]}
\begin{tcolorbox}
Let $(X,\mathcal{M})$ be a measurable space. Verify that if $\mu$ and $\nu$ are measures on $\mathcal{M}$ and $\mu \geq \nu$, then there is a measure
$\lambda$ on $\mathcal{M}$ for which $\mu = \nu + \lambda$.
\end{tcolorbox}

\begin{Proof}
Let $\lambda : \mathcal{M} \rightarrow [0,\infty]$ be defined by 
\[ \lambda(A) := \sup \left\{ \mu(B) - \nu(B) : B \in \mathcal{M} \text{ such that } \nu(B) < \infty \text{ and } B \subseteq A \right\}, \ A \in
\mathcal{M}. \]
\begin{claim}
If $\nu(A) < \infty$ then $\lambda(A) = \mu(A) - \nu(A)$.
\end{claim}
\begin{claimproof}
Since $A \subseteq A$ and $\nu(A) < \infty$, 
\begin{equation}
\lambda(A) \geq \mu(A) - \nu(A),
\label{3.1} 
\end{equation}
by definition of $\lambda$. Now let $\epsilon > 0$. Using the
definition of $\lambda$ again, there exists some $B \subseteq A$ such that $\lambda(A) - \epsilon \leq \mu(B) - \nu(B)$. But then 
\[ \lambda(A) \leq \mu(B) - \nu(B) + \mu(A-B) - \nu(A-B) + \epsilon = \mu(A) - \nu(A) + \epsilon. \]
But then 
\begin{equation}
\lambda(A) \leq \mu(A) - \nu(A). 
\label{3.2}
\end{equation}
So by \eqref{3.1} and \eqref{3.2}, $\lambda(A) = \mu(A) - \nu(A)$.
\end{claimproof}

\begin{claim}
$\lambda(\emptyset) = 0$
\end{claim}
\begin{claimproof}
By claim 1, $\lambda(\emptyset) = \mu(\emptyset) - \nu(\emptyset) = 0$.
\end{claimproof}

\begin{claim}
$\lambda\left( \cup_{n=0}^{\infty}A_{n} \right) = \sum_{n=0}^{\infty}\lambda(A_{n})$ whenever $A_{0}, A_{1}, \hdots \in \mathcal{M}$ such that
$\left\{ A_{n} \right\}_{n=0}^{\infty}$ is pairwise disjoint.
\end{claim}
\begin{claimproof}
Suppose $A_{0}, A_{1}, \hdots \in \mathcal{M}$ such that $\left\{ A_{n} \right\}_{n=0}^{\infty}$ is pairwise disjoint. Let $B \subseteq
\cup_{n=0}^{\infty} A_{n}$ such that $\nu(B) < \infty$. Note that $B = \cup_{n=0}^{\infty}B\cap A_{n}$, where $\left\{ B\cap A_{n}
\right\}_{n=0}^{\infty}$ is pairwise disjoint. Thus,
\begin{align*}
\mu(B) - \nu(B) & = \sum_{n=0}^{\infty}\left[ \mu(B\cap A_{n}) - \nu(B\cap A_{n}) \right] \\
& \leq \sum_{n=0}^{\infty}\sup\left\{ 
\mu(E) - \nu(E) : E \in \mathcal{M} \text{ such that } \nu(E) < \infty \text{ and } E \subseteq A_{n} \right\} \\
& = \sum_{n=0}^{\infty}\lambda(A_{n}).
\end{align*}
Hence by taking the supremum over all such $B$ on the left,
\begin{equation}
\lambda\left( \cup_{n=0}^{\infty}A_{n} \right) \leq \sum_{n=0}^{\infty}\lambda(A_{n}).
\label{3.3}
\end{equation}
It remains to show that 
\begin{equation}
\sum_{n=0}^{\infty}\lambda(A_{n}) \leq \lambda\left( \cup_{n=0}^{\infty} A_{n} \right).
\label{3.4}
\end{equation}
Without loss of generality we can assume $\lambda\left( \cup_{n=0}^{\infty}A_{n} \right) < \infty$. 
Therefore by \eqref{3.3} $\lambda(A_{n}) < \infty$ for all $n \in \mathbb{N}$. Thus, for each $n \in \mathbb{N}$, we can choose $B_{n}
\subseteq A_{n}$ such that $B_{n} \in \mathcal{M}$, $\nu(B_{n}) < \infty$ and $\lambda(A_{n}) \leq \mu(B_{n}) - \nu(B_{n}) + \epsilon\cdot
2^{-(n+1)}$. Note that for each $k \in \mathbb{N}$, $\cup_{n=0}^{k}B_{n} \subseteq \cup_{n=0}^{\infty}A_{n}$ and $\nu\left( \cup_{n=0}^{k}B_{n}
\right) < \infty$. Hence,
\begin{align*}
\sum_{n=0}^{k}\lambda(A_{n}) \leq \sum_{n=0}^{k}\left[ \mu(B_{n}) - \nu(B_{n}) + \epsilon 2^{-(n+1)}\right] & = \mu\left( \cup_{n=0}^{k}B_{n} \right)
- \nu\left( \cup_{n=0}^{k}B_{n} \right) + \epsilon\sum_{n=0}^{k}2^{-(n+1)} \\
& \leq \lambda\left( \cup_{n=0}^{\infty}A_{n} \right) + \epsilon.
\end{align*}
Since the right-hand side is independent of $k$, we can take $k$ to infinity on the left. Thus,
\[ \lim_{k\rightarrow\infty}\sum_{n=0}^{\infty}\lambda(A_{n}) = \sum_{n=0}^{\infty}\lambda(A_{n}) \leq \lambda\left( \cup_{n=0}^{\infty}A_{n}
\right) + \epsilon. \]
Since $\epsilon > 0$ was arbitrary, we now get \eqref{3.4}. So by \eqref{3.3} and \eqref{3.4}, we are done.
\end{claimproof}

By claims 2 and 3, $\lambda$ is a measure, and by claim 1 $\lambda$ satisfies the condition that $\mu = \nu + \lambda$.
\end{Proof}


\subsection*{4 [RF 17.7 iii]}
\begin{tcolorbox}
Let $(X,\mathcal{M})$ be a measurable space. Verify that if $\nu$ is $\sigma$-finite, the measure $\lambda$ as in 3 is unique.
\end{tcolorbox}

\begin{Proof}
Suppose $\lambda_{1}, \lambda_{2}$ are measures that satisfy $\mu = \nu + \lambda_{i}$, $i = 1,2$. Since $\nu$ is $\sigma$-finite, $X =
\cup_{n=0}^{\infty}E_{n}$, where $E_{n} \in \mathcal{M}$ and $\nu(E_{n}) < \infty$ for all $n \in \mathbb{N}$. Without loss of generality we can assume $\left\{ E_{n}
\right\}_{n=0}^{\infty}$ is pairwise disjoint. Let $A \in \mathcal{M}$. If $\nu(A) < \infty$, then clearly 
\[ \lambda_{1}(A) = \lambda_{2}(A) = \mu(A) - \nu(A) \] 
in order for the condition on the $\lambda_{i}$'s to be satisfied. Now suppose $\nu(A) = \infty$. Note that $\nu(A\cap E_{n}) < \infty$ 
for each $n \in \mathbb{N}$. Therefore 
\[ \lambda_{i}(A\cap E_{n}) = \mu(A\cap E_{n}) - \nu(A\cap E_{n}), \ i = 1,2. \]
But since $\left\{ E_{n} \right\}_{n=0}^{\infty}$ is pairwise disjoint and $A \subseteq X = \cup_{n=0}^{\infty}E_{n}$,
\[ \lambda_{1}(A) = \sum_{n=0}^{\infty}\lambda_{1}(A\cap E_{n}) = \sum_{n=0}^{\infty}\lambda_{2}(A\cap E_{n}) = \lambda_{2}(A). \]
\end{Proof}



\newpage 
\subsection*{5 [RF 17.7 iv]}
\begin{tcolorbox}
Let $(X,\mathcal{M})$ be a measurable space. Show that in general the measure $\lambda$ as in 3 need not be unique but that there is always a smallest
such $\lambda$.
\end{tcolorbox}

To see that the measure $\lambda$ does not need to be unique, in general, consider
the following example. Let $(X,\mathcal{M}) := (\mathbb{N}, \mathcal{P}(\mathbb{N}))$.
Let $\mu$ and $\nu$ both be defined by 
\[ \mu(A) = \nu(A) := \left\{ \begin{array}{cl}
0 & \text{ if } A = \emptyset \\
\infty & \text{ otherwise.} \end{array} \right. \]
Let $\lambda_{1} \equiv 0$ and $\lambda_{2} :=$ counting measure. Then 
$\infty = \mu(A) = \nu(A) + \lambda_{i}(A)$, for $i = 1,2$ and $A \neq \emptyset$.

We will now show that the measure constructed in 3 is always a smallest such $\lambda$.

\begin{Proof}
Let $\lambda$ denote the measure constructed in 3. Let $\lambda'$ be another measure on $(X,\mathcal{M})$ such that $\mu = \nu + \lambda'$. Clearly
$\lambda(E) = \lambda'(E)$ whenever $\nu(E) < \infty$. With that in mind, let $A \in \mathcal{M}$ and let $B \subseteq A$ such that $B \in
\mathcal{M}$ and $\nu(B) < \infty$. Then 
\[ \lambda(B) = \lambda'(B) \leq \lambda'(A), \]
by monotonicity of $\lambda'$. Thus, taking the supremum over all such $B$ on the left-hand side, we get 
\[ \lambda(A) \leq \lambda'(A). \]
\end{Proof}




\end{document}

