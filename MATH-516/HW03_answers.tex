\documentclass[12pt]{article}
\usepackage{amsmath}
\usepackage{amsfonts}
\usepackage{parskip}
\usepackage{amsthm}
\usepackage{thmtools}
\usepackage[headheight=15pt]{geometry}
\geometry{a4paper, left=20mm, right=20mm, top=30mm, bottom=30mm}
\usepackage{graphicx}
\usepackage{bm} % for bold font in math mode - command is \bm{text}
\usepackage{enumitem}
\usepackage{fancyhdr}
\usepackage{amssymb} % for stacked arrows and other shit
\pagestyle{fancy}
\usepackage{changepage}
\usepackage{mathcomp}
\usepackage{tcolorbox}
\usepackage{yfonts}

\declaretheoremstyle[headfont=\normalfont]{normal}
\declaretheorem[style=normal]{Theorem}
\declaretheorem[style=normal]{Proposition}
\declaretheorem[style=normal]{Lemma}
\newcounter{ProofCounter}
\newcounter{ClaimCounter}[ProofCounter]
\newcounter{SubClaimCounter}[ClaimCounter]
\newenvironment{Proof}{\stepcounter{ProofCounter}\textit{Proof.}}{\hfill$\square$}
\newenvironment{claim}[1]{\vspace{1mm}\stepcounter{ClaimCounter}\par\noindent\underline{\bf Claim \theClaimCounter:}\space#1}{}
\newenvironment{claimproof}[1]{\par\noindent\underline{Proof of claim \theClaimCounter:}\space#1}{\hfill $\blacksquare$ Claim \theClaimCounter}
\newenvironment{subclaim}[1]{\stepcounter{SubClaimCounter}\par\noindent\emph{Subclaim \theClaimCounter.\theSubClaimCounter:}\space#1}{}
% \newenvironment{subclaimproof}[1]{\begin{adjustwidth}{2em}{0pt}\par\noindent\emph{Proof of subclaim \theClaimCounter.\theSubClaimCounter:}\space#1}{\hfill
% $\blacksquare$ \emph{Subclaim \theClaimCounter.\theSubClaimCounter}\vspace{5mm}\end{adjustwidth}}
\newenvironment{subclaimproof}[1]{\par\noindent\emph{Proof of subclaim \theClaimCounter.\theSubClaimCounter:}\space#1}{\hfill
$\Diamond$ \emph{Subclaim \theClaimCounter.\theSubClaimCounter}}

\title{MATH 516: HW 3}
\author{Evan P. Walsh}
\makeatletter
\let\runauthor\@author
\let\runtitle\@title
\makeatother
\lhead{\runauthor}
\chead{\runtitle}
\rhead{\thepage}
\cfoot{}

\begin{document}
\maketitle

\subsection*{1}
\begin{tcolorbox}
Define a measure on $\left\{ 0,1 \right\}^{\omega}$ to model a biased coin.
\end{tcolorbox}

We will use the same algebra defined in the course notes. Let $0 \leq q \leq 1$. Let $A \in \mathcal{A}$. Then we can write $A = \cup_{j <
n}B(\sigma_{j})$, where $\left\{ B(\sigma_{j}) \right\}_{j < n}$ is pairwise disjoint. For each $j < n$, let $n_{j} := |\sigma_{j}|$ and let
$\sigma_{j}(i)$ denote the $i$'th entry in $\sigma_{j}$. Then let 
\[ p(A) := \sum_{j < n}\prod_{i=1}^{n_{j}}\left[ q^{\sigma_{j}(i)}(1-q)^{1-\sigma_{j}(i)} \right]. \]
If a 1 represents heads and 0 represents tails, then we can think of each $\sigma_{j}$ as a realization of $n_{j}$ coin flips. Then the product term inside
the summation above represents the probability, according to $q$, of the event $\sigma_{j}$, where $q$ represents the probability of flipping heads on
a single flip. Hence $p(A)$ is the probability of getting any of the $\sigma_{j}$ sequences. Using the same steps as in the course notes, we can show
that $p$ is a premeasure on $A$. Then we can extend $p$ to the measure $\mu_{p}$ using Theorem (I)(2)(xvii).



\subsection*{2}
\begin{tcolorbox}
Suppose $X := \mathbb{N}$ and $\mathcal{M} := \{\emptyset\}$ and $p(\emptyset) = 0$. Describe $\mu_{p}^{*}$.
\end{tcolorbox}

Then 
\[ \mu_{p}^{*}(E) = \left\{ \begin{array}{cl}
0 & \text{ if } E = \emptyset, \\
\infty & \text{ if } E \neq \emptyset.
\end{array} \right. \]
The is because for any $\mathcal{P}(X) \supseteq E \neq \emptyset$, 
\[ \left\{ \sum_{n=0}^{\infty}p(A_{n}) : E \subseteq \cup_{n=0}^{\infty}A_{n} \text{ such that } A_{n} \in \mathcal{M}\ \forall \ n \in \mathbb{N}
\right\} = \emptyset. \]
Further, $\mu_{p}^{*}$ is actually a measure over $\mathcal{P}(X)$.



\subsection*{3}



\subsection*{4}


\subsection*{5}
\begin{tcolorbox}
Suppose $(X, \mathcal{S}, \mu)$ is a $\sigma$-finite measure space. Suppose $\mathfrak{D} \subseteq \mathcal{S}$ is a disjoint class; i.e. $A\cap B = \emptyset$
whenever $A, B \in \mathfrak{D}$ and $A \neq B$. Prove that for each measurable $E \subseteq X$, $\{D \in \mathfrak{D} : \mu(E\cap D) > 0\}$ is
countable.
\end{tcolorbox}

\begin{Proof}
Since $(X, \mathcal{S}, \mu)$ is $\sigma$-finite, there exists $A_{0}, A_{1}, \hdots \in \mathcal{S}$ such that $X = \cup_{n=0}^{\infty}A_{n}$ and
$\mu(A_{n}) < \infty$ for each $n \in \mathbb{N}$. Without loss of generality we can assume $\left\{ A_{n} \right\}_{n=0}^{\infty}$ is pairwise
disjoint. Let $E \in \mathcal{S}$. Denote $\mathfrak{D}_{n,k} := \left\{ D \in \mathfrak{D} : \mu(E\cap D\cap A_{n}) > 2^{-k} \right\}$ for each $n,k
\in \mathbb{N}$. Note that 
\begin{align*}
\left\{ D \in \mathfrak{D} : \mu(E \cap D) > 0 \right\} & = \cup_{n=0}^{\infty}\left\{ D \in \mathfrak{D} : \mu(E\cap D\cap A_{n}) > 0 \right\} \\
& = \cup_{n=0}^{\infty}\cup_{k=0}^{\infty}\left\{ D \in \mathfrak{D} : \mu(E\cap D\cap A_{n}) > 2^{-k} \right\} \\
& = \cup_{n=0}^{\infty}\cup_{k=0}^{\infty}\mathfrak{D}_{n,k}.
\end{align*}
Therefore it suffices to show that $\mathfrak{D}_{n,k}$ is countable for each $n,k \in \mathbb{N}$. In fact, each $\mathfrak{D}_{n,k}$ must
necessarily be finite. To see this, assume by way of contradiction that there exists some $n', k' \in \mathbb{N}$ such that $\mathfrak{D}_{n',k'}$ is
infinite (without loss of generality, assume $\mathfrak{D}_{n',k'}$ is countably infinite, otherwise we would just consider a countably infinite
subset of $\mathfrak{D}_{n',k'}$). But since $\mathfrak{D}_{n',k'}$ is a disjoint class, 
\[ \mu(A_{n'}) \geq \sum_{D\in \mathfrak{D}_{n',k'}}\mu(E\cap D\cap A_{n'}) \geq \sum_{D\in\mathfrak{D}_{n',k'}}2^{-k} = \infty. \]
This is a contradiction. Hence each $\mathfrak{D}_{n,k}$ is finite. Since the countable sum of finite sets is a countable set, we are done.
\end{Proof}


\end{document}
