\documentclass[12pt]{article}
\usepackage{amsmath}
\usepackage{amsfonts}
\usepackage{parskip}
\usepackage{amsthm}
\usepackage{thmtools}
\usepackage[headheight=15pt]{geometry}
\geometry{a4paper, left=20mm, right=20mm, top=30mm, bottom=30mm}
\usepackage{graphicx}
\usepackage{bm} % for bold font in math mode - command is \bm{text}
\usepackage{enumitem}
\usepackage{fancyhdr}
\usepackage{amssymb} % for stacked arrows and other shit
\pagestyle{fancy}
\usepackage{changepage}
\usepackage{mathcomp}
\usepackage{tcolorbox}
\usepackage{yfonts}

\declaretheoremstyle[headfont=\normalfont]{normal}
\declaretheorem[style=normal]{Theorem}
\declaretheorem[style=normal]{Proposition}
\declaretheorem[style=normal]{Lemma}
\newcounter{ProofCounter}
\newcounter{ClaimCounter}[ProofCounter]
\newcounter{SubClaimCounter}[ClaimCounter]
\newenvironment{Proof}{\stepcounter{ProofCounter}\textsc{Proof.}}{\hfill$\square$}
\newenvironment{claim}[1]{\vspace{1mm}\stepcounter{ClaimCounter}\par\noindent\underline{\bf Claim \theClaimCounter:}\space#1}{}
\newenvironment{claimproof}[1]{\par\noindent\underline{Proof of claim \theClaimCounter:}\space#1}{\hfill $\blacksquare$ Claim \theClaimCounter}
\newenvironment{subclaim}[1]{\stepcounter{SubClaimCounter}\par\noindent\emph{Subclaim \theClaimCounter.\theSubClaimCounter:}\space#1}{}
% \newenvironment{subclaimproof}[1]{\begin{adjustwidth}{2em}{0pt}\par\noindent\emph{Proof of subclaim \theClaimCounter.\theSubClaimCounter:}\space#1}{\hfill
% $\blacksquare$ \emph{Subclaim \theClaimCounter.\theSubClaimCounter}\vspace{5mm}\end{adjustwidth}}
\newenvironment{subclaimproof}[1]{\par\noindent\emph{Proof of subclaim \theClaimCounter.\theSubClaimCounter:}\space#1}{\hfill
$\Diamond$ \emph{Subclaim \theClaimCounter.\theSubClaimCounter}}

\title{MATH 516: HW 3}
\author{Evan P. Walsh}
\makeatletter
\let\runauthor\@author
\let\runtitle\@title
\makeatother
\lhead{\runauthor}
\chead{\runtitle}
\rhead{\thepage}
\cfoot{}

\begin{document}
\maketitle

\subsection*{1}
\begin{tcolorbox}
Define a measure on $\left\{ 0,1 \right\}^{\omega}$ to model a biased coin.
\end{tcolorbox}

We will use the same algebra defined in the course notes. Let $0 \leq q \leq 1$. Let $A \in \mathcal{A}$. Then we can write $A = \cup_{j <
n}B(\sigma_{j})$, where $\left\{ B(\sigma_{j}) \right\}_{j < n}$ is pairwise disjoint. For each $j < n$, let $n_{j} := |\sigma_{j}|$ and let
$\sigma_{j}(i)$ denote the $i$'th entry in $\sigma_{j}$. Then let 
\[ p(A) := \sum_{j < n}\prod_{i=1}^{n_{j}}\left[ q^{\sigma_{j}(i)}(1-q)^{1-\sigma_{j}(i)} \right]. \]
If a 1 represents heads and 0 represents tails, then we can think of each $\sigma_{j}$ as a realization of $n_{j}$ coin flips. Then the product term inside
the summation above represents the probability, according to $q$, of the event $\sigma_{j}$, where $q$ represents the probability of flipping heads on
a single flip. Hence $p(A)$ is the probability of getting any of the $\sigma_{j}$ sequences. Using the same steps as in the course notes, we can show
that $p$ is a premeasure on $A$. Then we can extend $p$ to the measure $\mu_{p}$ using Theorem (I)(2)(xvii).



\subsection*{2}
\begin{tcolorbox}
Suppose $X := \mathbb{N}$ and $\mathcal{M} := \{\emptyset\}$ and $p(\emptyset) = 0$. Describe $\mu_{p}^{*}$.
\end{tcolorbox}

Then 
\[ \mu_{p}^{*}(E) = \left\{ \begin{array}{cl}
0 & \text{ if } E = \emptyset, \\
\infty & \text{ if } E \neq \emptyset.
\end{array} \right. \]
The is because for any $\mathcal{P}(X) \supseteq E \neq \emptyset$, 
\[ \left\{ \sum_{n=0}^{\infty}p(A_{n}) : E \subseteq \cup_{n=0}^{\infty}A_{n} \text{ such that } A_{n} \in \mathcal{M}\ \forall \ n \in \mathbb{N}
\right\} = \emptyset. \]
Further, $\mu_{p}^{*}$ is actually a measure over $\mathcal{P}(X)$.



\subsection*{3 [RF 17.36]}
\begin{tcolorbox}
Let $\mu$ be a finite premeasure on an algebra $\mathcal{S}$, and $\mu^{*}$ the induced outer measure. Show that a subset $E$ of $X$ is
$\mu^*$-measurable if and only if for each $\epsilon > 0$ there exists a set $A \in \mathcal{S}_{\delta}$ with $A\subseteq E$ such that $\mu^*(E-A) <
\epsilon$.
\end{tcolorbox}
\begin{Proof} Let $E \subseteq X$.

$(\Rightarrow)$ First assume that $E$ is $\mu^*$-measurable. Let $\epsilon > 0$. Since the premeasure $\mu$ is finite, $\mu^*(E)$ and $\mu^*(E^{c})$ are finite. Therefore
there exists $A_{0}, A_1, \hdots \in \mathcal{S}$ such that $\cup_{n=0}^{\infty}A_{n} \supseteq E^{c}$ and $\sum_{n=0}^{\infty}\mu^*(A_{n}) <
\mu^{*}(E^c) + \epsilon$. By subadditivity,
\begin{equation}
\mu^{*}\left( \cup_{n=0}^{\infty}A_{n} \right) \leq \sum_{n=0}^{\infty}\mu^*(A_{n}) < \mu^*(E^c) + \epsilon.
\label{3.1}
\end{equation}
Now set $A := \left( \cup_{n=0}^{\infty}A_{n} \right)^{c}$. Then $A \in \mathcal{S}_{\delta}$ and $A\subseteq E$. By the measurability of $E$,
\begin{align*}
& \mu^{*}(E\cap A^{c}) + \mu^{*}(E^{c}\cap A^{c}) = \mu^{*}(A^{c}) \\
\Rightarrow \qquad & \mu^{*}(E\cap A^{c}) = \mu^{*}(A^{c}) - \mu^{*}(E^{c}\cap A^{c}) \\
\text{by } \eqref{3.1} \Rightarrow \qquad & \mu^*(E - A) = \mu^*\left( \cup_{n=0}^{\infty}A_{n} \right) - \mu^{*}(E^{c}) < \epsilon.
\end{align*}

$(\Leftarrow)$ Now suppose for all $\epsilon > 0$, there exists some $A \in \mathcal{S}_{\delta}$ such that $A \subseteq E$ and $\mu^{*}(E - A) <
\epsilon$. Then for all $n \in \mathbb{N}$, there exists $A_{n} \in \mathcal{S}_{\delta}$ such that $A_{n} \subseteq E$ and $\mu^{*}(E - A_{n}) < 2^{-n}$.
Let $A := \cup_{n=0}^{\infty}A_{n}$. Since the $\mu^*$-measurable sets form a $\sigma$-algebra, each $A_{n}$ is $\mu^*$-measurable and thus $A$ is
also $\mu^*$-measurable.
\begin{claim}
$\mu^{*}(A) = \mu^{*}(E)$.
\end{claim}
\begin{claimproof}
By monotonicity, $\mu^*(A) \leq \mu^*(E)$.
Now note that since each $A_{n}$ is measurable, $\mu^{*}(E - A_{n}) = \mu^*(E) - \mu^*(A_{n})$ (to see this, just rearrange using the definition of measurable).
Therefore, for each $n \in \mathbb{N}$, $\mu^{*}(E - A_{n}) = \mu^{*}(E) - \mu^{*}(A_{n}) < 2^{-n}$. Thus,
\[ mu^{*}(E) < \mu^{*}(A_{n}) + 2^{-n} \leq \mu^*(A) + 2^{-n}. \]
Hence $\mu^*(E) \leq \mu^*(A)$. So $\mu^{*}(A) = \mu^{*}(E)$.
\end{claimproof}

By claim 1, $\mu^*(E - A) = \mu^*(E) - \mu^*(A) = 0$. Therefore since outer measures are complete, $\mu^*(E - A)$ is $\mu^{*}$-measurable. Therefore 
$E = A \cup (E - A)$ is measurable.
\end{Proof}


\newpage
\subsection*{4 [RF 17.34]}
\begin{tcolorbox}
If we start with an outer measure $\mu^{*}$ on $2^{X}$ and form the induced measure $\bar{\mu}$ on the $\mu^{*}$-measurable sets, we can view
$\bar{\mu}$ as a set function and denote by $\mu^{+}$ the outer measure induced by $\bar{\mu}$.
\begin{enumerate}[label=(\alph*)]
\item Show that for each set $E \subseteq X$ we have $\mu^{+}(E) \geq \mu^{*}(E)$.
\item For a given set $E$, show that $\mu^{+}(E) = \mu^*(E)$ if and only if there is a $\mu^*$-measurable set $A \supseteq E$ with $\mu^*(A) =
\mu^*(E)$.
\end{enumerate}
\end{tcolorbox}
\begin{enumerate}[label=(\alph*)]
\item \begin{Proof}
Let $E \subseteq X$. Let $A_{0}, A_{1}, \hdots$ be $\mu^*$-measurable with $\cup_{n=0}^{\infty}A_{n}\supseteq E$ (of course, if no such sets exist, then
$\mu^+(E) = \infty$). Thus by monotonicity and countable subadditivity,
\[ \mu^*(E) \leq \mu^*\left( \cup_{n=0}^{\infty}A_{n} \right) \leq \sum_{n=0}^{\infty}\mu^*(A_{n}). \]
Therefore 
\[ \mu^*(E) \leq \inf\left\{ \sum_{n=0}^{\infty}\mu^*(E_{n}) : \text{each }E_{n} \ \mu^{*}\text{-measurable and } \cup_{n=0}^{\infty}E_{n} \supseteq E \right\} =
\bar{\mu}(E). \]
\end{Proof}

\item \begin{Proof}
Let $E \subseteq X$.

$(\Rightarrow)$ First suppose that $\mu^+(E) = \mu^*(E)$. If $\mu^*(E) = \infty$, take $A = X$. Otherwise, for each $n \in \mathbb{N}$ there exists a
collection $\left\{ A_{n,k} \right\}_{k=0}^{\infty}$ of $\mu^*$-measurable sets such that 
\[ A_{n} := \cup_{k=0}^{\infty}A_{n,k} \supseteq E \text{ and } \sum_{k=0}^{\infty}\mu^*(A_{n,k}) < \mu^{+}(E) + 2^{-n}. \]
Take $A := \cap_{n=0}^{\infty}A_{n}$. As the countable intersection of countable unions of
$\mu^*$-measurable sets, $A$ is $\mu^*$-measurable. Further, since $E \subseteq A \subseteq A_{n}$ for each $n \in \mathbb{N}$,
\[ \mu^*(E) \leq \mu^*(A) \leq \mu^*\left( A_{n} \right) \leq \sum_{k=0}^{\infty}\mu^*(A_{n,k}) < \mu^+(E) + 2^{-n} = \mu^*(E) + 2^{-n}. \]
Therefore $\mu^*(A) = \mu^*(E)$.

$(\Leftarrow)$ Now suppose there exists some $\mu^*$-measurable $A \supseteq E$ such that $\mu^*(A) = \mu^*(E)$. By (i), $\mu^+(E) \geq \mu^*(E)$. But
if we take $A_{0} := A$ and $A_{n} := \emptyset$ for all $n \geq 1$, then 
\[ \mu^{+}(E) \leq \sum_{n=0}^{\infty}\mu^{*}(A_{n}) = \mu^*(A) = \mu^*(E). \]
So $\mu^{+}(E) = \mu^{*}(E)$.
\end{Proof}
\end{enumerate}


\subsection*{5}
\begin{tcolorbox}
Suppose $(X, \mathcal{S}, \mu)$ is a $\sigma$-finite measure space. Suppose $\mathfrak{D} \subseteq \mathcal{S}$ is a disjoint class; i.e. $A\cap B = \emptyset$
whenever $A, B \in \mathfrak{D}$ and $A \neq B$. Prove that for each measurable $E \subseteq X$, $\{D \in \mathfrak{D} : \mu(E\cap D) > 0\}$ is
countable.
\end{tcolorbox}

\begin{Proof}
Since $(X, \mathcal{S}, \mu)$ is $\sigma$-finite, there exists $A_{0}, A_{1}, \hdots \in \mathcal{S}$ such that $X = \cup_{n=0}^{\infty}A_{n}$ and
$\mu(A_{n}) < \infty$ for each $n \in \mathbb{N}$. Without loss of generality we can assume $\left\{ A_{n} \right\}_{n=0}^{\infty}$ is pairwise
disjoint. Let $E \in \mathcal{S}$. Denote $\mathfrak{D}_{n,k} := \left\{ D \in \mathfrak{D} : \mu(E\cap D\cap A_{n}) > 2^{-k} \right\}$ for each $n,k
\in \mathbb{N}$. Note that 
\begin{align*}
\left\{ D \in \mathfrak{D} : \mu(E \cap D) > 0 \right\} & = \cup_{n=0}^{\infty}\left\{ D \in \mathfrak{D} : \mu(E\cap D\cap A_{n}) > 0 \right\} \\
& = \cup_{n=0}^{\infty}\cup_{k=0}^{\infty}\left\{ D \in \mathfrak{D} : \mu(E\cap D\cap A_{n}) > 2^{-k} \right\} \\
& = \cup_{n=0}^{\infty}\cup_{k=0}^{\infty}\mathfrak{D}_{n,k}.
\end{align*}
Therefore it suffices to show that $\mathfrak{D}_{n,k}$ is countable for each $n,k \in \mathbb{N}$. In fact, each $\mathfrak{D}_{n,k}$ must
necessarily be finite. To see this, assume by way of contradiction that there exists some $n', k' \in \mathbb{N}$ such that $\mathfrak{D}_{n',k'}$ is
infinite (without loss of generality, assume $\mathfrak{D}_{n',k'}$ is countably infinite, otherwise we would just consider a countably infinite
subset of $\mathfrak{D}_{n',k'}$). But since $\mathfrak{D}_{n',k'}$ is a disjoint class, 
\[ \mu(A_{n'}) \geq \mu\left(A_{n'}\cap \bigcup_{D\in\mathfrak{D}_{n',k'}}(E\cap D)\right) = \sum_{D\in \mathfrak{D}_{n',k'}}\mu(E\cap D\cap A_{n'}) 
\geq \sum_{D\in\mathfrak{D}_{n',k'}}2^{-k'} = \infty. \]
This is a contradiction. Hence each $\mathfrak{D}_{n,k}$ is finite. Since the countable sum of finite sets is a countable set, we are done.
\end{Proof}












\end{document}
