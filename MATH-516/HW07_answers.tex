\documentclass[12pt]{article}
\usepackage{amsmath}
\usepackage{amsfonts}
\usepackage{parskip}
\usepackage{amsthm}
\usepackage{thmtools}
\usepackage[headheight=15pt]{geometry}
\geometry{a4paper, left=20mm, right=20mm, top=30mm, bottom=30mm}
\usepackage{graphicx}
\usepackage{bm} % for bold font in math mode - command is \bm{text}
\usepackage{enumitem}
\usepackage{fancyhdr}
\usepackage{amssymb} % for stacked arrows and other shit
\pagestyle{fancy}
\usepackage{changepage}
\usepackage{mathcomp}
\usepackage{tcolorbox}

\declaretheoremstyle[headfont=\normalfont]{normal}
\declaretheorem[style=normal]{Theorem}
\declaretheorem[style=normal]{Proposition}
\declaretheorem[style=normal]{Lemma}
\newcounter{ProofCounter}
\newcounter{ClaimCounter}[ProofCounter]
\newcounter{SubClaimCounter}[ClaimCounter]
\newenvironment{Proof}{\stepcounter{ProofCounter}\textsc{Proof.}}{\hfill$\square$}
\newenvironment{claim}[1]{\vspace{1mm}\stepcounter{ClaimCounter}\par\noindent\underline{\bf Claim \theClaimCounter:}\space#1}{}
\newenvironment{claimproof}[1]{\par\noindent\underline{Proof of claim \theClaimCounter:}\space#1}{\hfill $\blacksquare$ Claim \theClaimCounter}
\newenvironment{subclaim}[1]{\stepcounter{SubClaimCounter}\par\noindent\emph{Subclaim \theClaimCounter.\theSubClaimCounter:}\space#1}{}
% \newenvironment{subclaimproof}[1]{\begin{adjustwidth}{2em}{0pt}\par\noindent\emph{Proof of subclaim \theClaimCounter.\theSubClaimCounter:}\space#1}{\hfill
% $\blacksquare$ \emph{Subclaim \theClaimCounter.\theSubClaimCounter}\vspace{5mm}\end{adjustwidth}}
\newenvironment{subclaimproof}[1]{\par\noindent\emph{Proof of subclaim \theClaimCounter.\theSubClaimCounter:}\space#1}{\hfill
$\Diamond$ \emph{Subclaim \theClaimCounter.\theSubClaimCounter}}

\allowdisplaybreaks{}

% chktex-file 3

\title{MATH 516: HW 7}
\author{Evan P. Walsh}
\makeatletter
\makeatother
\lhead{Evan P. Walsh}
\chead{MATH 516: HW 7}
\rhead{\thepage}
\cfoot{}

\begin{document}
\maketitle

\subsection*{1 [RF 18.52]}
\begin{tcolorbox}
  Let $(X,\mathcal{M}, \mu)$ be a finite measure space, $\left\{ E_k \right\}_{k=1}^{n}$ be a collection of measurable sets, and $\left\{ c_k
  \right\}_{k=1}^{n}$ a collection of real numbers. For each $E \in \mathcal{M}$, define
  \[ \nu(E) = \sum_{k=1}^{n}c_k \cdot \mu(E\cap E_k). \]
  Show that $\nu$ is absolutely continuous with respect to $\mu$ and find its Radon-Nikodym derivative $\frac{d\mu}{d\nu}$.
\end{tcolorbox}
\begin{Proof}
  Note that for any $E \in \mathcal{M}$,
  \[ \nu(E) = \sum_{k=1}^{n}c_k \cdot \mu(E\cap E_k) = \int_{E}\sum_{k=1}^{n}c_k\cdot \chi_{E_k}\ d\mu, \]
  and if $\mu(E) = 0$, then $\mu(E\cap E_k) = 0$ for all $1 \leq k \leq n$. Thus,
  \[ \nu(E) = \sum_{k=1}^{n}c_k \cdot \mu(E\cap E_K) = 0. \]
  Hence $\nu \ll \mu$ and $\frac{d\nu}{d\mu} = \sum_{k=1}^{n}c_k\cdot \chi_{E_k}$ a.e. ($\mu$).
\end{Proof}


\newpage
\subsection*{2,3 [RF 18.54]}
\begin{tcolorbox}
  Let $\mu$, $\nu$, and $\lambda$ be $\sigma$-finite measures on the measurable space $(X, \mathcal{M})$.
  \begin{enumerate}[label = (\roman*)]
    \item If $\nu \ll \mu$ and if $f$ is a non-negative function on $X$ that is measurable with respect to $\mathcal{M}$, show that
      \[ \int_{X} f\ d\nu = \int_{X}f\left[ \frac{d\nu}{d\mu} \right]d\mu. \]
    \item If $\nu \ll \mu$ and $\lambda \ll \mu$, show that $\frac{d(\nu + \lambda)}{d\mu} = \frac{d\nu}{d\mu} + \frac{d\lambda}{d\mu}$ a.e. $(\mu)$.
    \item If $\nu \ll \mu \ll \lambda$, show that $\frac{d\nu}{d\lambda} = \frac{d\nu}{d\mu}\cdot \frac{d\mu}{d\lambda}$  a.e. $(\lambda)$
    \item If $\nu \ll \mu$ and $\mu \ll \nu$, show that $\frac{d\nu}{d\mu}\cdot \frac{d\mu}{d\nu} = 1$ a.e. $(\mu)$.
  \end{enumerate}
\end{tcolorbox}
\begin{Proof}
  \begin{enumerate}[label = (\roman*)]
      \item
        Suppose $\nu \ll \mu$. First we will that if $f$ is any non-negative measurable function on $X$, then for any $E \in \mathcal{M}$,
        \begin{equation}
          \int_{E}f\ d\nu = \int_{E}f\left[ \frac{d\nu}{d\mu} \right]d\mu.
          \label{2.1}
        \end{equation}
        Once~\eqref{2.1} is established, part (i) follows with $E = X$. We will show that~\eqref{2.1} holds by the following claims.
        \begin{claim}
          If $f = \chi_{A}$ for some $A \in \mathcal{M}$, then~\eqref{2.1} holds.
        \end{claim}
        \begin{claimproof}
          Note that $\chi_{A}\cdot\chi_{E} = \chi_{A\cap E}$. Thus,
          \begin{align*}
            \int_{E}f\ d\nu = \int_{E}\chi_{A}\ d\nu = \int_{X} \chi_{A}\chi_{E}\ d\nu = \int_{X} \chi_{A\cap E}\ d\nu = \nu(A\cap E)
            = \int_{A\cap E}\frac{d\nu}{d\mu}\ d\mu
            = \int_{E}f\left[ \frac{d\nu}{d\mu} \right]d\mu.
          \end{align*}
        \end{claimproof}
        \begin{claim}
          If $f$ is a non-negative simple function on $X$, then~\eqref{2.1} holds.
        \end{claim}
        \begin{claimproof}
          This follows from Claim 1 and the linearity of the integral.
        \end{claimproof}
        \begin{claim}
          If $f$ is any non-negative measurable function on $X$, then~\eqref{2.1} holds.
        \end{claim}
        \begin{claimproof}
          Let $\left\{ s_n \right\}_{n=0}^{\infty}$ be an increasing sequence of non-negative simple functions such that $s_n \rightarrow f$. By MCT
          and Claim 2,
          \[
            \int_{E}f\ d\nu = \lim_{n\rightarrow\infty}\int_{E}s_n\ d\nu = \lim_{n\rightarrow\infty}\int_{E}s_n\left[ \frac{d\nu}{d\mu} \right]d\mu =
            \int_{E}f\left[ \frac{d\nu}{d\mu} \right]d\mu,
          \]
          where the last equality follows from MCT and the fact that $s_n \cdot \frac{d\nu}{d\mu} \uparrow f\cdot
          \frac{d\nu}{d\mu}$ since $s_n \uparrow f$.

        \end{claimproof}
      \item Now suppose $\nu \ll \mu$ and $\lambda \ll \mu$. Then for any $E \in \mathcal{M}$,
        \[
          (\nu + \lambda)(E) = \nu(E) + \lambda(E) = \int_{E}\frac{d\nu}{d\mu}d\mu + \int_{E}\frac{d\lambda}{d\mu}d\mu = \int_{E}
          \left[ \frac{d\nu}{d\mu} + \frac{d\lambda}{d\mu} \right]d\mu.
        \]
        Hence $\frac{d(\nu + \lambda)}{d\mu} = \frac{d\nu}{d\mu} + \frac{d\lambda}{d\mu}$ a.e. ($\mu$).
      \item Let $E \in \mathcal{M}$. By Equation~\eqref{2.1} (or Claim 3) with $f = \frac{d\nu}{d\mu}$,
        \[
          \int_{E}f\ d\mu = \int_{E}f\left[ \frac{d\mu}{d\lambda} \right]d\lambda, \text{ i.e. } \int_{E}\frac{d\nu}{d\mu}\ d\mu = \int_{E}
          \frac{d\nu}{d\mu}\cdot\frac{d\mu}{d\lambda}\ d\lambda.
        \]
        Thus, $\nu(E) = \int_{E}\frac{d\nu}{d\mu}\cdot \frac{d\mu}{d\lambda}\ d\lambda$ for all $E \in \mathcal{M}$, 
        and so $\frac{d\nu}{d\lambda} = \frac{d\nu}{d\mu}\cdot
        \frac{d\mu}{d\lambda}$ a.e. $(\lambda)$.
      \item We have $\nu \ll \mu \ll \nu$. So by part (iii), $\frac{d\nu}{d\nu} = \frac{d\nu}{d\mu}\cdot \frac{d\mu}{d\nu}$ a.e. ($\nu$). But
        $\frac{d\nu}{d\nu} = 1$ a.e. ($\nu$). Further, a.e. ($\nu$) $\Leftrightarrow$ a.e. ($\mu$) since $\nu \ll \mu$ and $\mu \ll \nu$. Hence part
        (iv) is established.
  \end{enumerate}
\end{Proof}

Note: In all of the above proofs we used the fact that the Radon-Nikydom derivative has to be unique except on a set of measure 0, i.e. % chktex 12
if $\nu \ll \mu$ and $f$ and $g$ are both RN derivatives of $\nu$ with respect to $\mu$, then $f = g$ a.e. ($\mu$). To see why this must be, suppose
otherwise. That is, suppose there exists some measurable $E$ such that $f \neq g$ on $E$ and $\mu(E) > 0$. Without loss of generality suppose $f > g$
on $E$. Then 
\[ \nu(E) = \int_{E}f\  d\mu > \int_{E}g\ d\mu = \nu(E), \]
a contradiction.


\subsection*{4}
\begin{tcolorbox}
  Prove that $\left\{ 0,1 \right\}^{\omega}$ is totally disconnected.
\end{tcolorbox}
\begin{Proof}
  Let $E \subseteq \left\{ 0,1 \right\}^{\omega}$ such that $E$ contains at least two
  unique elements, say $f$ and $g$. Since $f$ and $g$ are unique, there exists a minimum non-negative integer $n_0$ such that $f(n_0) \neq g(n_0)$.
  Let $\sigma_f = (f(0), \hdots, f(n_0))$ and $U = B(\sigma_{f})$. Then define $V$ as
  \[ V := \bigcup \left\{ B(\sigma) : \sigma = (a_0, \hdots, a_{n_0}) \text{ so that } a_i \neq f(i) \text{ for some } 0 \leq i \leq n_0 \right\}. \]
  By definition, $U$ and $V$ are disjoint and open, where $\left\{ 0,1 \right\}^{\omega} = U\cup V \supseteq E$. But since $f \in U$ and $g \in V$, $E
  \not\subseteq U$ and $E \not\subseteq V$. Hence $E$ is disconnected. Therefore we have shown that any subset containing more than 1 element is
  disconnected, so $\left\{ 0,1 \right\}^{\omega}$ is totally disconnected.
\end{Proof}

\newpage
\subsection*{5}
\begin{tcolorbox}
  Prove that $\left\{ 0,1 \right\}^{\omega}$ is compact.
\end{tcolorbox}
\begin{Proof}
  Let $\bm{C} =$ Cantor ternary set in $\mathbb{R}$. $\bm{C}$ is compact since it is closed and bounded as a subset of $\mathbb{R}$. Now, for each $x
  \in \bm{C}$, write $x = \sum_{n=0}^{\infty}\frac{a_n}{3^{n+1}}$ where $a_n \in \left\{ 0,2 \right\}$ for all $n \in \mathbb{N}$, and define
  $\varphi : C \rightarrow \left\{ 0,1 \right\}^{\omega}$ by $\varphi(x) =$ the function $f : \mathbb{N} \rightarrow \left\{ 0,1 \right\}$ such that
  $f(n) = \frac{a_n}{2}$ for all $n \in \mathbb{N}$. To show that $\left\{ 0,1 \right\}^{\omega}$ is compact we will show that $\varphi$ is continuous
  and onto, since the image of compact sets under continuous functions are compact.

  \begin{claim}
    $\varphi$ is onto
  \end{claim}
  \begin{claimproof}
    Let $f \in \left\{ 0,1 \right\}^{\omega}$. Let $x = \sum_{n=0}^{\infty}\frac{2f(n)}{3^{n+1}}$.
  \end{claimproof}

  \begin{claim}
    $d(\varphi(x_0), \varphi(x_1)) \leq 2^{-n+1}$ whenever $x_0, x_1 \in \bm{C}$ such that $|x_0 - x_1| < 3^{-n}$.
  \end{claim}
  \begin{claimproof}
    Let $n_0 \in \mathbb{N}$ and suppose $x_0, x_1 \in \bm{C}$ such that $|x_0 - x_1| < 3^{-n_0}$. Without loss of generality assume $x_0 < x_1$.
    Write $x_0 = \sum_{n=0}^{\infty}\frac{a_n}{3^{n+1}}$ and $x_1 = \sum_{n=0}^{\infty}\frac{b_n}{3^{n+1}}$ where $a_n,b_n \in \left\{ 0,2
    \right\}$ for all $n \in \mathbb{N}$.

    \begin{subclaim}
      $b_{n} = a_{n}$ for all $0 \leq n \leq n_{0} - 1$.
    \end{subclaim}
    \begin{subclaimproof}
      By way of contradiction, assume there exists $0 \leq n< n_{0}$ such that $b_{n} - a_{n} \neq 0$. Let $n' = \min\left\{ 0 \leq n \leq n_0 -1 :
      b_{n} \neq a_{n} \right\}$. Since $b_{n'}, a_{n'} \in \left\{ 0,2 \right\}$ and $x_{1} > x_{0}$ by assumption, $b_{n'} = 2$ and $a_{n'} = 0$. Thus,
      \begin{align*}
        |x_{0} - x_{1}| = x_{1} - x_{0} = \sum_{n=0}^{\infty}\frac{b_{n}}{3^{n+1}} - \sum_{n=0}^{\infty}\frac{a_{n}}{3^{n+1}}
        & = \frac{2}{3^{n'+1}} + \sum_{n=n'+1}^{\infty}\frac{b_{n}}{3^{n+1}} - \sum_{n=n'+1}^{\infty}\frac{a_{n}}{3^{n+1}} \\
        & \geq \frac{2}{3^{n'+1}} - \sum_{n=n'+1}^{\infty}\frac{2}{3^{n+1}} \\
        & = \frac{2}{3^{n'+1}} - \frac{1}{3^{n'+1}} = \frac{1}{3^{n'+1}} \geq \frac{1}{3^{n_{0}}}.
      \end{align*}
      This is a contradiction.
    \end{subclaimproof}

    By Subclaim 2.1,
    \[ d(\varphi(x_0), \varphi(x_1)) = \sum_{n=0}^{\infty}\frac{|a_n - b_n|/2}{2^{n}} = \sum_{n=n_0}^{\infty}\frac{|a_n - b_n|}{2^{n+1}} \leq
    \sum_{n=n_0}^{\infty}\frac{2}{2^{n+1}} = 2^{-n_0+1}. \]
  \end{claimproof}

  By Claim 2, $\varphi$ is continuous. Hence by Claim 1 and Theorem (II)(7)xv, $\left\{ 0,1 \right\}^{\omega}$ is compact. % chktex 36

\end{Proof}

\end{document}
