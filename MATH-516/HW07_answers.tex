\documentclass[12pt]{article}
\usepackage{amsmath}
\usepackage{amsfonts}
\usepackage{parskip}
\usepackage{amsthm}
\usepackage{thmtools}
\usepackage[headheight=15pt]{geometry}
\geometry{a4paper, left=20mm, right=20mm, top=30mm, bottom=30mm}
\usepackage{graphicx}
\usepackage{bm} % for bold font in math mode - command is \bm{text}
\usepackage{enumitem}
\usepackage{fancyhdr}
\usepackage{amssymb} % for stacked arrows and other shit
\pagestyle{fancy}
\usepackage{changepage}
\usepackage{mathcomp}
\usepackage{tcolorbox}

\declaretheoremstyle[headfont=\normalfont]{normal}
\declaretheorem[style=normal]{Theorem}
\declaretheorem[style=normal]{Proposition}
\declaretheorem[style=normal]{Lemma}
\newcounter{ProofCounter}
\newcounter{ClaimCounter}[ProofCounter]
\newcounter{SubClaimCounter}[ClaimCounter]
\newenvironment{Proof}{\stepcounter{ProofCounter}\textsc{Proof.}}{\hfill$\square$}
\newenvironment{claim}[1]{\vspace{1mm}\stepcounter{ClaimCounter}\par\noindent\underline{\bf Claim \theClaimCounter:}\space#1}{}
\newenvironment{claimproof}[1]{\par\noindent\underline{Proof of claim \theClaimCounter:}\space#1}{\hfill $\blacksquare$ Claim \theClaimCounter}
\newenvironment{subclaim}[1]{\stepcounter{SubClaimCounter}\par\noindent\emph{Subclaim \theClaimCounter.\theSubClaimCounter:}\space#1}{}
% \newenvironment{subclaimproof}[1]{\begin{adjustwidth}{2em}{0pt}\par\noindent\emph{Proof of subclaim \theClaimCounter.\theSubClaimCounter:}\space#1}{\hfill
% $\blacksquare$ \emph{Subclaim \theClaimCounter.\theSubClaimCounter}\vspace{5mm}\end{adjustwidth}}
\newenvironment{subclaimproof}[1]{\par\noindent\emph{Proof of subclaim \theClaimCounter.\theSubClaimCounter:}\space#1}{\hfill
$\Diamond$ \emph{Subclaim \theClaimCounter.\theSubClaimCounter}}

\allowdisplaybreaks{}

% chktex-file 3

\title{MATH 516: HW 7}
\author{Evan P. Walsh}
\makeatletter
\makeatother
\lhead{Evan P. Walsh}
\chead{MATH 516: HW 7}
\rhead{\thepage}
\cfoot{}

\begin{document}
\maketitle

\subsection*{1 [RF 18.52]}
\begin{tcolorbox}
  Let $(X,\mathcal{M}, \mu)$ be a finite measure space, $\left\{ E_k \right\}_{k=1}^{n}$ be a collection of measurable sets, and $\left\{ c_k
  \right\}_{k=1}^{n}$ a collection of real numbers. For each $E \in \mathcal{M}$, define
  \[ \nu(E) = \sum_{k=1}^{n}c_k \cdot \mu(E\cap E_k). \]
  Show that $\nu$ is absolutely continuous with respect to $\mu$ and find its Radon-Nikodym derivative $\frac{d\mu}{d\nu}$.
\end{tcolorbox}
\begin{Proof}
  Note that for any $E \in \mathcal{M}$,
  \[ \nu(E) = \sum_{k=1}^{n}c_k \cdot \mu(E\cap E_k) = \int_{E}\sum_{k=1}^{n}c_k\cdot \chi_{E_k}\ d\mu, \]
  and if $\mu(E) = 0$, then $\mu(E\cap E_k) = 0$ for all $1 \leq k \leq n$. Thus,
  \[ \nu(E) = \sum_{k=1}^{n}c_k \cdot \mu(E\cap E_K) = 0. \]
  Hence $\nu \ll \mu$ and $\frac{d\nu}{d\mu} = \sum_{k=1}^{n}c_k\cdot \chi_{E_k}$ a.e. ($\mu$).
\end{Proof}


\newpage
\subsection*{2,3 [RF 18.54]}
\begin{tcolorbox}
  Let $\mu$, $\nu$, and $\lambda$ be $\sigma$-finite measures on the measurable space $(X, \mathcal{M})$.
  \begin{enumerate}[label = (\roman*)]
    \item If $\nu \ll \mu$ and if $f$ is a non-negative function on $X$ that is measurable with respect to $\mathcal{M}$, show that
      \[ \int_{X} f\ d\nu = \int_{X}f\left[ \frac{d\nu}{d\mu} \right]d\mu. \]
    \item If $\nu \ll \mu$ and $\lambda \ll \mu$, show that $\frac{d(\nu + \lambda)}{d\mu} = \frac{d\nu}{d\mu} + \frac{d\lambda}{d\mu}$ a.e. $(\mu)$.
    \item If $\nu \ll \mu \ll \lambda$, show that $\frac{d\nu}{d\lambda} = \frac{d\nu}{d\mu}\cdot \frac{d\mu}{d\lambda}$  a.e. $(\lambda)$
    \item If $\nu \ll \mu$ and $\mu \ll \nu$, show that $\frac{d\nu}{d\mu}\cdot \frac{d\mu}{d\nu} = 1$ a.e. $(\mu)$.
  \end{enumerate}
\end{tcolorbox}
\begin{Proof}
  \begin{enumerate}[label = (\roman*)]
      \item
        Suppose $\nu \ll \mu$. First we will that if $f$ is any non-negative measurable function on $X$, then for any $E \in \mathcal{M}$,
        \begin{equation}
          \int_{E}f\ d\nu = \int_{E}f\left[ \frac{d\nu}{d\mu} \right]d\mu.
          \label{2.1}
        \end{equation}
        Once~\eqref{2.1} is established, part (i) follows with $E = X$. We will show that~\eqref{2.1} holds by the following claims.
        \begin{claim}
          If $f = \chi_{A}$ for some $A \in \mathcal{M}$, then~\eqref{2.1} holds.
        \end{claim}
        \begin{claimproof}
          Note that $\chi_{A}\cdot\chi_{E} = \chi_{A\cap E}$. Thus,
          \begin{align*}
            \int_{E}f\ d\nu = \int_{E}\chi_{A}\ d\nu = \int_{X} \chi_{A}\chi_{E}\ d\nu = \int_{X} \chi_{A\cap E}\ d\nu = \nu(A\cap E)
            = \int_{A\cap E}\frac{d\nu}{d\mu}\ d\mu 
            = \int_{E}f\left[ \frac{d\nu}{d\mu} \right]d\mu.
          \end{align*}
        \end{claimproof}
        \begin{claim}
          If $f$ is a non-negative simple function on $X$, then~\eqref{2.1} holds.
        \end{claim}
        \begin{claimproof}
          This follows from Claim 1 and the linearity of the integral.
        \end{claimproof}
        \begin{claim}
          If $f$ is any non-negative measurable function on $X$, then~\eqref{2.1} holds.
        \end{claim}
        \begin{claimproof}
          Let $\left\{ s_n \right\}_{n=0}^{\infty}$ be an increasing sequence of non-negative simple functions such that $s_n \rightarrow f$. By MCT
          and Claim 2,
          \[
            \int_{E}f\ d\nu = \lim_{n\rightarrow\infty}\int_{E}s_n\ d\nu = \lim_{n\rightarrow\infty}\int_{E}s_n\left[ \frac{d\nu}{d\mu} \right]d\mu =
            \int_{E}f\left[ \frac{d\nu}{d\mu} \right]d\mu,
          \]
          where the last equality follows from MCT and the fact that $s_n \cdot \frac{d\nu}{d\mu} \uparrow f\cdot
          \frac{d\nu}{d\mu}$ since $s_n \uparrow f$.

        \end{claimproof}
      \item Now suppose $\nu \ll \mu$ and $\lambda \ll \mu$. Then for any $E \in \mathcal{M}$,
        \[
          (\nu + \lambda)(E) = \nu(E) + \lambda(E) = \int_{E}\frac{d\nu}{d\mu}d\mu + \int_{E}\frac{d\lambda}{d\mu}d\mu = \int_{E}
          \left[ \frac{d\nu}{d\mu} + \frac{d\lambda}{d\mu} \right]d\mu.
        \]
        Hence $\frac{d(\nu + \lambda)}{d\mu} = \frac{d\nu}{d\mu} + \frac{d\lambda}{d\mu}$ a.e. ($\mu$).
      \item Let $E \in \mathcal{M}$. By Equation~\eqref{2.1} (or Claim 3) with $f = \frac{d\nu}{d\mu}$,
        \[
          \int_{E}f\ d\mu = \int_{E}f\left[ \frac{d\mu}{d\lambda} \right]d\lambda, \text{ i.e. } \int_{E}\frac{d\nu}{d\mu}\ d\mu = \int_{E}
          \frac{d\nu}{d\mu}\cdot\frac{d\mu}{d\lambda}\ d\lambda.
        \]
        Thus, $\nu(E) = \int_{E}\frac{d\nu}{d\mu}\cdot \frac{d\mu}{d\lambda}\ d\lambda$, and so $\frac{d\nu}{d\lambda} = \frac{d\nu}{d\mu}\cdot
        \frac{d\mu}{d\lambda}$ a.e. $(\lambda)$.
      \item We have $\nu \ll \mu \ll \nu$. So by part (iii), $\frac{d\nu}{d\nu} = \frac{d\nu}{d\mu}\cdot \frac{d\mu}{d\nu}$ a.e. ($\nu$). But
        $\frac{d\nu}{d\nu} = 1$ a.e. ($\nu$). Further, a.e. ($\nu$) $\Leftrightarrow$ a.e. ($\mu$) since $\nu \ll \mu$ and $\mu \ll \nu$. Hence part
        (iv) is established.
  \end{enumerate}
\end{Proof}


\subsection*{4}
\begin{tcolorbox}
  Prove that $\left\{ 0,1 \right\}^{\omega}$ is totally disconnected.
\end{tcolorbox}
\begin{Proof}
  Let $E \subseteq \left\{ 0,1 \right\}^{\omega}$ such that $E$ contains at least two
  unique elements, say $f$ and $g$. Since $f$ and $g$ are unique, there exists a minimum non-negative integer $n_0$ such that $f(n_0) \neq g(n_0)$.
  Let $\sigma_f = (f(0), \hdots, f(n_0))$ and $U = B(\sigma_{f})$. Then define $V$ as
  \[ V := \bigcup \left\{ B(\sigma) : \sigma = (a_0, \hdots, a_{n_0}) \text{ so that } a_i \neq f(i) \text{ for some } 0 \leq i \leq n_0 \right\}. \]
  By definition, $U$ and $V$ are disjoint and open, where $\left\{ 0,1 \right\}^{\omega} = U\cup V \supseteq E$. But since $f \in U$ and $g \in V$, $E
  \not\subseteq U$ and $E \not\subseteq V$. Hence $E$ is disconnected. Therefore we have shown that any subset containing more than 1 element is
  disconnected, so $\left\{ 0,1 \right\}^{\omega}$ is totally disconnected.
\end{Proof}

\subsection*{5}
\begin{tcolorbox}
  Prove that $\left\{ 0,1 \right\}^{\omega}$ is compact.
\end{tcolorbox}
\begin{Proof}
  Let $\mathcal{O}$ be an open cover of $\left\{ 0,1 \right\}^{\omega}$. By the Lebesgue Number Lemma, there exists some $K_0 \in \mathbb{N}$ such that
  whenever $f,g \in \left\{ 0,1 \right\}^{\omega}$ and $d(f,g) < 2^{-K_0}$, there exists some $E \in \mathcal{O}$ such that $f,g \in E$. Thus, for any
  $f \in \left\{ 0,1 \right\}^{\omega}$, there exists some $E \in \mathcal{O}$ such that $B_{2^{-K_1}}(f) \subseteq E$, where $K_1 := K_0 + 1$. Now
  let $\left\{ \sigma_{j} \right\}_{j=1}^{2^{K_1}}$ be the set of unique sequences of 0's and 1's of length $K_1$, i.e. $\sigma_{j} = (a_{j,0},
    a_{j,1}, \hdots, a_{j,K_{0}})$, where $a_{j,i} \in \left\{ 0,1 \right\}$ for each $1 \leq j \leq 2^{K_1}$, $0 \leq i \leq K_0$. For each $j \in \left\{
  1, \hdots, 2^{K_1} \right\}$, let $f_j \in \left\{ 0,1 \right\}^{\omega}$ such that 
  \[ f_j(i) = \left\{ \begin{array}{cl}
        a_{j,i} & \text{ if } 0 \leq i \leq K_{0} \\
        0 & \text{ if } i \geq K_{1}. \\
    \end{array} \right. 
  \]
  So for each $j \in \left\{ 1,\hdots, 2^{K_{1}} \right\}$, there exists $E_j \in \mathcal{O}$ such that $B_{2^{-K_{1}}}(f_j) \subseteq E_{j}$.
  But 
  \[ 
    \bigcup_{j=1}^{2^{K_{1}}}E_j \supseteq \bigcup_{j=1}^{2^{K_{1}}}B_{2^{-K_{1}}}(f_j) = \left\{ 0,1 \right\}^{\omega}. 
  \]
  Hence $\left\{ E_1, \hdots, E_{2^{K_{1}}} \right\}$ is a finite subcover, and so $\left\{ 0,1 \right\}^{\omega}$ is compact.
\end{Proof}


\end{document}
