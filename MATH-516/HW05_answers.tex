\documentclass[12pt]{article}
\usepackage{amsmath}
\usepackage{amsfonts}
\usepackage{parskip}
\usepackage{amsthm}
\usepackage{thmtools}
\usepackage[headheight=15pt]{geometry}
\geometry{a4paper, left=20mm, right=20mm, top=30mm, bottom=30mm}
\usepackage{graphicx}
\usepackage{bm} % for bold font in math mode - command is \bm{text}
\usepackage{enumitem}
\usepackage{fancyhdr}
\usepackage{amssymb} % for stacked arrows and other shit
\pagestyle{fancy}
\usepackage{changepage}
\usepackage{mathcomp}
\usepackage{tcolorbox}

\declaretheoremstyle[headfont=\normalfont]{normal}
\declaretheorem[style=normal]{Theorem}
\declaretheorem[style=normal]{Proposition}
\declaretheorem[style=normal]{Lemma}
\newcounter{ProofCounter}
\newcounter{ClaimCounter}[ProofCounter]
\newcounter{SubClaimCounter}[ClaimCounter]
\newenvironment{Proof}{\stepcounter{ProofCounter}\textsc{Proof.}}{\hfill$\square$}
\newenvironment{claim}[1]{\vspace{1mm}\stepcounter{ClaimCounter}\par\noindent\underline{\bf Claim \theClaimCounter:}\space#1}{}
\newenvironment{claimproof}[1]{\par\noindent\underline{Proof of claim \theClaimCounter:}\space#1}{\hfill $\blacksquare$ Claim \theClaimCounter}
\newenvironment{subclaim}[1]{\stepcounter{SubClaimCounter}\par\noindent\emph{Subclaim \theClaimCounter.\theSubClaimCounter:}\space#1}{}
% \newenvironment{subclaimproof}[1]{\begin{adjustwidth}{2em}{0pt}\par\noindent\emph{Proof of subclaim \theClaimCounter.\theSubClaimCounter:}\space#1}{\hfill
% $\blacksquare$ \emph{Subclaim \theClaimCounter.\theSubClaimCounter}\vspace{5mm}\end{adjustwidth}}
\newenvironment{subclaimproof}[1]{\par\noindent\emph{Proof of subclaim \theClaimCounter.\theSubClaimCounter:}\space#1}{\hfill
$\Diamond$ \emph{Subclaim \theClaimCounter.\theSubClaimCounter}}

\title{MATH 516: HW 5}
\author{Evan P. Walsh}
\makeatletter
\let\runauthor\@author
\let\runtitle\@title
\makeatother
\lhead{\runauthor}
\chead{\runtitle}
\rhead{\thepage}
\cfoot{}

\begin{document}
\maketitle


\subsection*{1 [RF 18.2]}
\begin{tcolorbox}
Suppose $(X,\mathcal{M}, \mu)$ is not complete. Let $E$ be a subset of a set of measure zero that does not belong to $M$. Let $f \equiv 0$ on $X$ and $g :=
\chi_{E}$. Show that $f = g$ a.e. on $X$ while $f$ is measurable and $g$ is not.
\end{tcolorbox}
\begin{Proof}
Since $f$ is constant, $f$ is measurable. But $g^{-1}[(1/2,3/2)] = g^{-1}[\{1\}] = E \notin \mathcal{M}$. Thus $g$ is not measurable. Now, by
assumption of the problem, there exists some $A \in \mathcal{M}$ such that $\mu(A) = 0$ and $A \supset E$. So, since $f = g$ on $X - E$, $f = g$ and $X -
A$. Therefore $f = g$ a.e.
\end{Proof}


\newpage
\subsection*{2 [RF 18.13]}
\begin{tcolorbox}
Let $\left\{ f_n \right\}_{n=0}^{\infty}$ be a sequence of real-valued measurable functions on $X$ such that, for each natural number $n$, 
\[ \mu\left\{ x \in X : |f_{n}(x) - f_{n+1}(x)| > 2^{-n} \right\} < 2^{-n}. \]
Show that $\left\{ f_{n} \right\}_{n=0}^{\infty}$ is pointwise convergent a.e. on $X$.
\end{tcolorbox}
\begin{Proof}
Let $A_{n} := \left\{ x \in X : |f_{n}(x) - f_{n+1}(x)| > 2^{-n} \right\}$ for all $n \in \mathbb{N}$. Then 
\[ \sum_{n=0}^{\infty}\mu(A_{n}) \leq \sum_{n=0}^{\infty}2^{-n} < \infty. \]
Thus, by the Borel-Cantelli Lemma, almost all $x \in X$ belong to finitely many $A_{n}$'s. That is, $\mu(E^{c}) = 0$ where $E := \left\{ 
x \in X : x \text{ belongs to finitely many $A_{n}$'s}\right\}$. Then for each $x \in E$ there exists some $N \in \mathbb{N}$ such that  
\begin{equation}
|f_{n}(x) - f_{n+1}(x)| \leq 2^{-n}, 
\label{2.1}
\end{equation}
for all $n \geq N$.
\begin{claim}
$\left\{ f_{n}(x) \right\}_{n=0}^{\infty}$ is Cauchy for all $x \in E$.
\end{claim}
\begin{claimproof}
Let $\epsilon > 0$ and $x \in E$. Then there exists some $N_{1} \in \mathbb{N}$ such that \eqref{2.1} holds for all $n \geq N_{1}$. Further, there exists some
$N_{2} \in \mathbb{N}$ such that $2^{-N_{2}} < \epsilon / 2$. Now set $N := \max\left\{ N_{1},N_{2} \right\}$. Then if $m > n \geq N$,
\begin{align*}
|f_{n}(x) - f_{m}(x)| & \leq |f_{n}(x) - f_{n+1}(x)| + |f_{n+1}(x) - f_{n+2}(x)| + \dots + |f_{m-1}(x) - f_{m}(x)|  \\
& \leq \sum_{k=n}^{\infty}|f_{k}(x) - f_{k+1}(x)| \\
& \leq \sum_{k=n}^{\infty}2^{-k} \\
& = 2^{-n}\sum_{k=0}^{\infty}2^{-k} \\
& = 2^{-n}\cdot 2 < \frac{\epsilon}{2} \cdot 2 = \epsilon.
\end{align*}
Therefore $\left\{ f_{n}(x) \right\}_{n=0}^{\infty}$ is Cauchy for each $x \in E$.
\end{claimproof}

By claim 1, $\left\{ f_{n}(x) \right\}_{n=0}^{\infty}$ converges for each $x \in E$, that is $\left\{ f_{n} \right\}_{n=0}^{\infty}$ converges almost
everywhere since $\mu(E^{c}) = 0$.
\end{Proof}


\newpage
\subsection*{3 [RF 20.5]}
\begin{tcolorbox}
Let $(X,\mathcal{A},\mu) = (\mathbb{N}, \mathcal{P}(\mathbb{N}), c)$, where $c$ is the counting measure. Define $f : \mathbb{N} \times \mathbb{N}
\rightarrow \mathbb{R}$ by setting 
\[ f(x,y) := \left\{ \begin{array}{cl}
2 - 2^{-x} & \text{ if } x = y \\
-2 + 2^{-x} & \text{ if } x = y + 1 \\
0 & \text{ otherwise. }
\end{array} \right.
\]
Show that $f$ is measurable with respect to the product measure $c\times c$. Also show that 
\[ \int_{\mathbb{N}}\left[ \int_{\mathbb{N}}f(m,n)\ dc(m) \right]\ dc(n) \neq \int_{\mathbb{N}}\left[ \int_{\mathbb{N}}f(m,n)\ dc(n) \right]\ dc(m).\]
Is this a contradiction of either Fubini's or Tonelli's Theorem?
\end{tcolorbox}
\begin{claim}
$f$ is measurable
\end{claim}
\begin{claimproof}
Let $\mathcal{S}$ denote the $\sigma$-algebra generated by the measurable rectangles.
It suffices to show that $\mathcal{S}$ is equal to $\mathcal{P}(\mathbb{N}\times \mathbb{N})$, which implies that any function defined over
$\mathbb{N}\times\mathbb{N}$ is measurable. Clearly $\mathcal{S} \subseteq
\mathcal{P}(\mathbb{N}\times\mathbb{N})$, so it remains to show that $\mathcal{S} \supseteq \mathcal{P}(\mathbb{N}\times\mathbb{N})$. To that end, let
$E \subseteq \mathbb{N}\times\mathbb{N}$. Then $E$ must be countable since $\mathbb{N}\times \mathbb{N}$ is countable. Hence $E$ can be expressed as a
countable union of pairs $(m,n) \in \mathbb{N}\times\mathbb{N}$. Since all such pairs $(m,n)$ are measurable rectangles, $E \in \mathcal{S}$.
\end{claimproof}

Now we can compute the following integrals:
\begin{align*}
\text{For each $n \in \mathbb{N}$, } \int_{\mathbb{N}}f(m,n)\ dc(m) & = \sum_{m=0}^{\infty}f(m,n) \\
& = f(n,n) + f(n+1,n) = 2 - 2^{-n} - 2 + 2^{-(n+1)} = -2^{-(n+1)}, \\
\text{for $m \geq 1$, } \int_{\mathbb{N}}f(m,n)\ dc(n) & = \sum_{n=0}^{\infty}f(m,n) \\
& = f(m,m-1) + f(m,m) = - 2 + 2^{-m} + 2 - 2^{-m} = 0, \\
\text{for $m = 0$, } \int_{\mathbb{N}}f(m,n)\ dc(n) & = \sum_{n=0}^{\infty}f(m,n) = f(0,0) = 1.
\end{align*}
Therefore,
\[
\int_{\mathbb{N}}\int_{\mathbb{N}}f(m,n)\ dc(m)\ dc(n) = \sum_{n=0}^{\infty}-2^{-(n+1)} = -1 \neq 1 = 1 + \sum_{m=1}^{\infty}0 = 
\int_{\mathbb{N}}\int_{\mathbb{N}}f(m,n)\ dc(n)\ dc(m).
\]
Note that Tonelli's Theorem is not relevant since $f(m,n) \not\geq 0$. However, Fubini's Theorem fails as well since $f$ is not integrable over
$\mathbb{N}\times \mathbb{N}$. To see this, we can apply Tonelli's Theorem to $|f|$:
\begin{align*}
\int_{\mathbb{N}\times\mathbb{N}}|f|\ d(c\times c) = \int_{\mathbb{N}}\int_{\mathbb{N}}f(m,n)\ dc(m)\ dc(n) & = \int_{\mathbb{N}}\left[|2 - 2^{-n}|
+ |-2 + 2^{-(n+1)}|\right]dc(n) \\
& \geq \int_{\mathbb{N}}2\ dc(n) = \infty. 
\end{align*}




\newpage
\subsection*{4}
\begin{tcolorbox}
Suppose $\mu(X) = 1$ and $f : X \rightarrow [0,\infty)$ is measurable. Set 
\[ A := \int_{X}f\ d\mu. \]
Prove that 
\[ \sqrt{1 + A^{2}} \leq \int_{X}\sqrt{1  +f^{2}}\ d\mu \leq 1 + A. \]
\end{tcolorbox}
\begin{Proof}
First note that since $f \geq 0$, $\sqrt{1 + f^{2}} \leq 1 + f$. Thus,
\begin{equation}
\int_{X}\sqrt{1 + f^{2}}d\mu \leq \int_{X}(1 + f)d\mu = 1 + \int_{X}fd\mu = 1 + A.
\label{4.2}
\end{equation}
If $\left\|\sqrt{1 + f^{2}}\right\|_{1} = \infty$, then the rest of the proof is trivial. Thus, assume 
\[ \int_{X}\left|\sqrt{1 + f^{2}}\right|\ d\mu = \int_{X}\sqrt{1 + f^{2}}\ d\mu  < \infty.\] 
Therefore $\|f\|_{1} < \infty$ since $f \leq \sqrt{1 + f^{2}}$.
Let $\varphi : [0,\infty) \rightarrow \mathbb{R}$ be defined by $\varphi(y) := \sqrt{1 + y^{2}}$ for all $y \geq 0$. $\phi$ is convex over $[0,\infty)$,
and by assumption $f, \varphi\circ f \in \mathcal{L}^{1}(\mu)$. Thus, by Jensen's inequality,
\begin{equation}
\sqrt{1 + A^{2}} = \varphi\left( \int_{X}fd\mu \right) \leq \int_{X}\varphi\circ f d\mu = \int_{X}\sqrt{1 + f^{2}}\ d\mu.
\label{4.3}
\end{equation}
So by \eqref{4.2} and \eqref{4.3} we are done.
\end{Proof}



\newpage
\subsection*{5}
\begin{tcolorbox}
Suppose $f : X \rightarrow [0, \infty)$ is measurable, and set 
\[ c := \int_{X}f\ d\mu. \]
Suppose $0 < c < \infty$. Prove that 
\[ \lim_{n\rightarrow\infty}\int_{X}n\ln[1 + (f/n)^{\alpha}]\ d\mu = \left\{ \begin{array}{cl}
\infty & : 0 < \alpha < 1 \\
c & : \alpha = 1 \\
0 & : 1 < \alpha < \infty.
\end{array} \right.
\]
\end{tcolorbox}



\begin{Proof}
Let $g_{n}(x) := n\ln\left[ 1 + \left( \frac{x}{n} \right)^{\alpha} \right]$ for all $n \in \mathbb{N}$ and $x \geq 0$. Then for all $\omega \in X$,
\begin{equation}
\lim_{n\rightarrow\infty}g_{n}(f(\omega)) = \lim_{n\rightarrow\infty}\ln\left[ 1 + \left( \frac{f(\omega)^{\alpha}}{n^{\alpha - 1}} \right)\left( \frac{1}{n} \right)
 \right]^{n} = \left\{ \begin{array}{cl}
\infty & : 0 < \alpha < 1 \\
f(\omega) & : \alpha = 1 \\
0 & : 1 < \alpha < \infty.
\end{array} \right.
\label{5.1}
\end{equation}
Now, for all $a \geq 1$, $b \geq 0$, we have $1 + b^{a} \leq (1 + b)^{a}$ and $\ln(1 + b) \leq b$. Thus 
\[ \ln(1 + b^{a}) \leq \ln(1 + b)^{a} = a \ln(1 + b) \leq a b. \]
Hence, for $\alpha \geq 1$ and $n \in \mathbb{N}$,
\[ g_{n}(f(\omega)) \leq n\alpha\frac{f(\omega)}{n} = \alpha f(x). \]
Since $\alpha f$ is integrable, we can apply the Dominated Convergence Theorem to $\left\{ g_{n}\circ f \right\}_{n=0}^{\infty}$ when $\alpha \geq 1$. 
Thus, by \eqref{5.1},
\[ \lim_{n\rightarrow\infty}\int_{X}g_{n}\circ f\ d\mu = \left\{\begin{array}{ll}
\int_{X}f\ d\mu = c & : \alpha = 1,\\ \\
\int_{X}0\ d\mu = 0 & : 1 < \alpha < \infty.
\end{array} \right.
\]
For $0 < \alpha < 1$ we can apply Fatou's Lemma. Therefore by \eqref{5.1} 
\[ \liminf_{n\rightarrow\infty}\int_{X}g_{n}\circ f\ d\mu \geq \int_{X}\infty\ d\mu = \infty. \]
\end{Proof}




\end{document}

