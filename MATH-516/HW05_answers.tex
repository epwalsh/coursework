\documentclass[12pt]{article}
\usepackage{amsmath}
\usepackage{amsfonts}
\usepackage{parskip}
\usepackage{amsthm}
\usepackage{thmtools}
\usepackage[headheight=15pt]{geometry}
\geometry{a4paper, left=20mm, right=20mm, top=30mm, bottom=30mm}
\usepackage{graphicx}
\usepackage{bm} % for bold font in math mode - command is \bm{text}
\usepackage{enumitem}
\usepackage{fancyhdr}
\usepackage{amssymb} % for stacked arrows and other shit
\pagestyle{fancy}
\usepackage{changepage}
\usepackage{mathcomp}
\usepackage{tcolorbox}

\declaretheoremstyle[headfont=\normalfont]{normal}
\declaretheorem[style=normal]{Theorem}
\declaretheorem[style=normal]{Proposition}
\declaretheorem[style=normal]{Lemma}
\newcounter{ProofCounter}
\newcounter{ClaimCounter}[ProofCounter]
\newcounter{SubClaimCounter}[ClaimCounter]
\newenvironment{Proof}{\stepcounter{ProofCounter}\textsc{Proof.}}{\hfill$\square$}
\newenvironment{claim}[1]{\vspace{1mm}\stepcounter{ClaimCounter}\par\noindent\underline{\bf Claim \theClaimCounter:}\space#1}{}
\newenvironment{claimproof}[1]{\par\noindent\underline{Proof of claim \theClaimCounter:}\space#1}{\hfill $\blacksquare$ Claim \theClaimCounter}
\newenvironment{subclaim}[1]{\stepcounter{SubClaimCounter}\par\noindent\emph{Subclaim \theClaimCounter.\theSubClaimCounter:}\space#1}{}
% \newenvironment{subclaimproof}[1]{\begin{adjustwidth}{2em}{0pt}\par\noindent\emph{Proof of subclaim \theClaimCounter.\theSubClaimCounter:}\space#1}{\hfill
% $\blacksquare$ \emph{Subclaim \theClaimCounter.\theSubClaimCounter}\vspace{5mm}\end{adjustwidth}}
\newenvironment{subclaimproof}[1]{\par\noindent\emph{Proof of subclaim \theClaimCounter.\theSubClaimCounter:}\space#1}{\hfill
$\Diamond$ \emph{Subclaim \theClaimCounter.\theSubClaimCounter}}

\title{MATH 516: HW 5}
\author{Evan P. Walsh}
\makeatletter
\let\runauthor\@author
\let\runtitle\@title
\makeatother
\lhead{\runauthor}
\chead{\runtitle}
\rhead{\thepage}
\cfoot{}

\begin{document}
\maketitle


\subsection*{1 [RF 18.2]}
\begin{tcolorbox}
Suppose $(X,\mathcal{M}, \mu)$ is not complete. Let $E$ be a subset of measure zero that does not belong to $M$. Let $f \equiv 0$ on $X$ and $g :=
\chi_{E}$. Show that $f = g$ a.e. on $X$ while $f$ is measurable and $g$ is not.
\end{tcolorbox}


\newpage
\subsection*{4}
\begin{tcolorbox}
Suppose $\mu(X) = 1$ and $f : X \rightarrow [0,\infty)$ is measurable. Set 
\[ A := \int_{X}f\ d\mu. \]
Prove that 
\[ \sqrt{1 + A^{2}} \leq \int_{X}\sqrt{1  +f^{2}}\ d\mu \leq 1 + A. \]
\end{tcolorbox}
\begin{Proof}
First note that since $f \geq 0$, $\sqrt{1 + f^{2}} \leq 1 + f$. Thus,
\begin{equation}
\int_{X}\sqrt{1 + f^{2}}d\mu \leq \int_{X}(1 + f)d\mu = 1 + \int_{X}fd\mu = 1 + A.
\label{4.2}
\end{equation}
If $\left\|\sqrt{1 + f^{2}}\right\|_{1} = \infty$, then the rest of the proof is trivial. Thus, assume 
\[ \int_{X}\left|\sqrt{1 + f^{2}}\right|\ d\mu = \int_{X}\sqrt{1 + f^{2}}\ d\mu  < \infty.\] 
Therefore $\|f\|_{1} < \infty$ since $f \leq \sqrt{1 + f^{2}}$.
Let $\varphi : [0,\infty) \rightarrow \mathbb{R}$ be defined by $\varphi(y) := \sqrt{1 + y^{2}}$ for all $y \geq 0$. $\phi$ is convex over $[0,\infty)$,
and by assumption $f, \varphi\circ f \in \mathcal{L}^{1}(\mu)$. Thus, by Jensen's inequality,
\begin{equation}
\sqrt{1 + A^{2}} = \varphi\left( \int_{X}fd\mu \right) \leq \int_{X}\varphi\circ f d\mu = \int_{X}\sqrt{1 + f^{2}}\ d\mu.
\label{4.3}
\end{equation}
So by \eqref{4.2} and \eqref{4.3} we are done.
\end{Proof}



\newpage
\subsection*{5}
\begin{tcolorbox}
Suppose $f : X \rightarrow [0, \infty)$ is measurable, and set 
\[ c := \int_{X}f\ d\mu. \]
Suppose $0 < c < \infty$. Prove that 
\[ \lim_{n\rightarrow\infty}\int_{X}n\ln[1 + (f/n)^{\alpha}]\ d\mu = \left\{ \begin{array}{cl}
\infty & : 0 < \alpha < 1 \\
c & : \alpha = 1 \\
0 & : 1 < \alpha < \infty.
\end{array} \right.
\]
\end{tcolorbox}



\begin{Proof}
Let $g_{n}(x) := n\ln\left[ 1 + \left( \frac{x}{n} \right)^{\alpha} \right]$ for all $n \in \mathbb{N}$ and $x \geq 0$. Then for all $\omega \in X$,
\begin{equation}
\lim_{n\rightarrow\infty}g_{n}(f(\omega)) = \lim_{n\rightarrow\infty}\ln\left[ 1 + \left( \frac{f(\omega)^{\alpha}}{n^{\alpha - 1}} \right)\left( \frac{1}{n} \right)
 \right]^{n} = \left\{ \begin{array}{cl}
\infty & : 0 < \alpha < 1 \\
f(\omega) & : \alpha = 1 \\
0 & : 1 < \alpha < \infty.
\end{array} \right.
\label{5.1}
\end{equation}
Now, for all $a \geq 1$, $b \geq 0$, we have $1 + b^{a} \leq (1 + b)^{a}$ and $\ln(1 + b) \leq b$. Thus 
\[ \ln(1 + b^{a}) \leq \ln(1 + b)^{a} = a \ln(1 + b) \leq a b. \]
Hence, for $\alpha \geq 1$ and $n \in \mathbb{N}$,
\[ g_{n}(f(\omega)) \leq n\alpha\frac{f(\omega)}{n} = \alpha f(x). \]
Since $\alpha f$ is integrable, we can apply the Dominated Convergence Theorem to $\left\{ g_{n}\circ f \right\}_{n=0}^{\infty}$ when $\alpha \geq 1$. 
Thus, by \eqref{5.1},
\[ \lim_{n\rightarrow\infty}\int_{X}g_{n}\circ f\ d\mu = \left\{\begin{array}{ll}
\int_{X}f\ d\mu = c & : \alpha = 1,\\ \\
\int_{X}0\ d\mu = 0 & : 1 < \alpha < \infty.
\end{array} \right.
\]
For $0 < \alpha < 1$ we can apply Fatou's Lemma. Therefore by \eqref{5.1} 
\[ \liminf_{n\rightarrow\infty}\int_{X}g_{n}\circ f\ d\mu \leq \int_{X}\infty\ d\mu = \infty. \]
\end{Proof}




\end{document}

