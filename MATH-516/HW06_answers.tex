\documentclass[12pt]{article}
\usepackage{amsmath}
\usepackage{amsfonts}
\usepackage{parskip}
\usepackage{amsthm}
\usepackage{thmtools}
\usepackage[headheight=15pt]{geometry}
\geometry{a4paper, left=20mm, right=20mm, top=30mm, bottom=30mm}
\usepackage{graphicx}
\usepackage{bm} % for bold font in math mode - command is \bm{text}
\usepackage{enumitem}
\usepackage{fancyhdr}
\usepackage{amssymb} % for stacked arrows and other shit
\pagestyle{fancy}
\usepackage{changepage}
\usepackage{mathcomp}
\usepackage{tcolorbox}

\declaretheoremstyle[headfont=\normalfont]{normal}
\declaretheorem[style=normal]{Theorem}
\declaretheorem[style=normal]{Proposition}
\declaretheorem[style=normal]{Lemma}
\newcounter{ProofCounter}
\newcounter{ClaimCounter}[ProofCounter]
\newcounter{SubClaimCounter}[ClaimCounter]
\newenvironment{Proof}{\stepcounter{ProofCounter}\textsc{Proof.}}{\hfill$\square$}
\newenvironment{claim}[1]{\vspace{1mm}\stepcounter{ClaimCounter}\par\noindent\underline{\bf Claim \theClaimCounter:}\space#1}{}
\newenvironment{claimproof}[1]{\par\noindent\underline{Proof of claim \theClaimCounter:}\space#1}{\hfill $\blacksquare$ Claim \theClaimCounter}
\newenvironment{subclaim}[1]{\stepcounter{SubClaimCounter}\par\noindent\emph{Subclaim \theClaimCounter.\theSubClaimCounter:}\space#1}{}
% \newenvironment{subclaimproof}[1]{\begin{adjustwidth}{2em}{0pt}\par\noindent\emph{Proof of subclaim \theClaimCounter.\theSubClaimCounter:}\space#1}{\hfill
% $\blacksquare$ \emph{Subclaim \theClaimCounter.\theSubClaimCounter}\vspace{5mm}\end{adjustwidth}}
\newenvironment{subclaimproof}[1]{\par\noindent\emph{Proof of subclaim \theClaimCounter.\theSubClaimCounter:}\space#1}{\hfill
$\Diamond$ \emph{Subclaim \theClaimCounter.\theSubClaimCounter}}

\allowdisplaybreaks

\title{MATH 516: HW 06}
\author{Evan P. Walsh}
\makeatletter
\let\runauthor\@author
\let\runtitle\@title
\makeatother
\lhead{\runauthor}
\chead{\runtitle}
\rhead{\thepage}
\cfoot{}

\begin{document}
\maketitle

\subsection*{1 [Lemma (I)(6)xii Claim 1]}
\begin{Proof}
By way of contradiction, suppose $k_{n+1} \leq k_{n}$. Set $E_{n'} := E_{n+1}\cup E_{n}$. Note that $E_{n'}$ is a subset of $E$ and is 
disjoint from $\cup_{k < n}E_{k}$. Further, since $E_{n+1}\cap E_{n} = \emptyset$ by construction,
\[ \mu(E_{n'}) = \mu(E_{n+1}) + \mu(E_{n}) \geq 2^{-k_{n+1}} \geq 2\cdot 2^{-k_{n+1}} = 2^{-(k_{n+1}-1)} = 2^{-k_{n}'}, \]
where $k_{n}' := k_{n+1} - 1 < k_{n}$. By this is a contradiction since $k_{n}$ was chosen to be the smallest integer such that there exists a 
measurable set $H \subseteq E - \cup_{k < n}E_{k}$ with $\mu(E) \geq 2^{-k_{n}}$.
\end{Proof}



\subsection*{2 [Lemma (I)(6)xii Claim 2]}
\begin{Proof}
By way of contradiction, suppose there exists a measurable $H' \subseteq F$ such that $\mu(H') > 0$. Then there exists some $n_{0} \in \mathbb{N}$ such that 
$\mu(H') \geq 2^{-n_{0}}$. But by claim 1, there exists a positive integer $n$ and smallest integer $k_{n} > n_{0}$ for which there exists a measurable subset $H \subseteq 
E - \cup_{j < k_{n}}E_{j}$, where each $E_{j}$, $1 \leq j < n$, have been chosen as in the beginning of the proof of this lemma. This is a contradiction.
\end{Proof}



\subsection*{3 [Theorem (I)(6)xxix Claim 3]}




\subsection*{4 [Theorem (I)(6)xxix Claim 4]}




\subsection*{5 [Theorem (I)(6)xxix Claim 5]}


\end{document}

