\documentclass[12pt]{article}
\usepackage{amsmath}
\usepackage{amsfonts}
\usepackage{parskip}
\usepackage{amsthm}
\usepackage{thmtools}
\usepackage[headheight=15pt]{geometry}
\geometry{a4paper, left=20mm, right=20mm, top=30mm, bottom=30mm}
\usepackage{graphicx}
\usepackage{bm} % for bold font in math mode - command is \bm{text}
\usepackage{enumitem}
\usepackage{fancyhdr}
\usepackage{amssymb} % for stacked arrows and other shit
\pagestyle{fancy}
\usepackage{changepage}
\usepackage{mathcomp}
\usepackage{tcolorbox}

\declaretheoremstyle[headfont=\normalfont]{normal}
\declaretheorem[style=normal]{Theorem}
\declaretheorem[style=normal]{Proposition}
\declaretheorem[style=normal]{Lemma}
\newcounter{ProofCounter}
\newcounter{ClaimCounter}[ProofCounter]
\newcounter{SubClaimCounter}[ClaimCounter]
\newenvironment{Proof}{\stepcounter{ProofCounter}\textsc{Proof.}}{\hfill$\square$}
\newenvironment{claim}[1]{\vspace{1mm}\stepcounter{ClaimCounter}\par\noindent\underline{\bf Claim \theClaimCounter:}\space#1}{}
\newenvironment{claimproof}[1]{\par\noindent\underline{Proof of claim \theClaimCounter:}\space#1}{\hfill $\blacksquare$ Claim \theClaimCounter}
\newenvironment{subclaim}[1]{\stepcounter{SubClaimCounter}\par\noindent\emph{\underline{Subclaim \theClaimCounter.\theSubClaimCounter:}}\space#1}{}
% \newenvironment{subclaimproof}[1]{\begin{adjustwidth}{2em}{0pt}\par\noindent\emph{Proof of subclaim \theClaimCounter.\theSubClaimCounter:}\space#1}{\hfill
% $\blacksquare$ \emph{Subclaim \theClaimCounter.\theSubClaimCounter}\vspace{5mm}\end{adjustwidth}}
\newenvironment{subclaimproof}[1]{\par\noindent\emph{\underline{Proof of subclaim \theClaimCounter.\theSubClaimCounter:}}\space#1}{\hfill
$\Diamond$ \emph{Subclaim \theClaimCounter.\theSubClaimCounter}}

\allowdisplaybreaks

\title{MATH 516: HW 06}
\author{Evan P. Walsh}
\makeatletter
\let\runauthor\@author
\let\runtitle\@title
\makeatother
\lhead{\runauthor}
\chead{\runtitle}
\rhead{\thepage}
\cfoot{}

\begin{document}
\maketitle

\subsection*{1 [Lemma (I)(6)xii Claim 1]}
\begin{tcolorbox}
$k_{n+1} > k_{n}$.
\end{tcolorbox}
\stepcounter{ClaimCounter}
\begin{claimproof}
By way of contradiction, suppose $k_{n+1} \leq k_{n}$. Set $E_{n'} := E_{n+1}\cup E_{n}$. Note that $E_{n'}$ is a subset of $E$ and is 
disjoint from $\cup_{k < n}E_{k}$. Further, since $E_{n+1}\cap E_{n} = \emptyset$ by construction,
\[ \mu(E_{n'}) = \mu(E_{n+1}) + \mu(E_{n}) \geq 2^{-k_{n+1}} \geq 2\cdot 2^{-k_{n+1}} = 2^{-(k_{n+1}-1)} = 2^{-k_{n}'}, \]
where $k_{n}' := k_{n+1} - 1 < k_{n}$. By this is a contradiction since $k_{n}$ was chosen to be the smallest integer such that there exists a 
measurable set $H \subseteq E - \cup_{k < n}E_{k}$ with $\mu(E) \geq 2^{-k_{n}}$.
\end{claimproof}



\subsection*{2 [Lemma (I)(6)xii Claim 2]}
\begin{tcolorbox}
$F$ is negative.
\end{tcolorbox}
\stepcounter{ClaimCounter}
\begin{claimproof}
By way of contradiction, suppose there exists a measurable $H' \subseteq F$ such that $\mu(H') > 0$. Then there exists some $n_{0} \in \mathbb{N}$ such that 
$\mu(H') \geq 2^{-n_{0}}$. But by claim 1, there exists a positive integer $n$ and smallest integer $k_{n} > n_{0}$ for which there exists a measurable subset $H \subseteq 
E - \cup_{j < k_{n}}E_{j}$, where each $E_{j}$, $1 \leq j < n$, have been chosen as in the beginning of the proof of this lemma. This is a contradiction.
\end{claimproof}


\newpage
\subsection*{3 [Theorem (I)(6)xxix Claim 3]}
\begin{tcolorbox}
If $A,B \in \mathcal{S}$ such that $A\cap B = \emptyset$, then $\mu_0(A\cup B) = \mu_0(A) + \mu_0(B)$.
\end{tcolorbox}
\stepcounter{ClaimCounter}
\begin{claimproof}
\begin{subclaim}
Suppose $\psi \in \mathcal{N}$, where $\psi = \sum_{i=1}^{n}c_{i}\chi_{E_{i}}$ such that $c_{i} \in \mathbb{R}$ and $\nu(E_{i}) = 0$ for all $1 \leq
i \leq n$. Suppose $A\in \mathcal{S}$. Let $E_{1}^{*} := E_{1}$ and $E_{i}^{*} := E_{i} - \cup_{j < i}E_{i}^{*}$ for all $1 < i \leq
n$. Let $\psi^{*} := \sum_{i=1}^{n}\chi_{E_{i}^{*}}$.
Then $\chi_{A}\psi^{*} \in \mathcal{N}$ and $|\chi_{A} - \psi|^{2} \geq |\chi_{A} - \chi_{A}\psi^*|^2$.
\end{subclaim}
\begin{subclaimproof}
The first part of the subclaim follows from the fact that $\chi_{A}\chi_{E_{i}^*} = \chi_{A\cap E_{i}^{*}}$, where $\nu(A\cap E_{i}^{*}) = 0$ 
for all $1 \leq i \leq n$ since $A\cap E_{i}^* \subseteq E_{i}$. Now suppose
$\omega \in X$. If $\omega \notin \cup_{i=1}^{n}E_{i}$, then $\omega \notin \cup_{i=1}^{n}A\cap E_{i}^{*}$, so
\[ |\chi_{A}(\omega) - \psi(\omega)|^{2} = |\chi_{A}(\omega)|^{2} = |\chi_{A}(\omega) - \chi_{A}(\omega)\psi^{*}(\omega)|^{2}. \]
If $\omega \in \cup_{i=1}^{n}E_{i}$, then there is at most one $i \in \left\{ 1, \hdots, n \right\}$ such that $\omega \in A\cap E_{i}^{*}$ since 
$\left\{ A\cap E_{i}^{*} \right\}_{i=1}^{n}$ is pairwise disjoint. Clearly one such $i$ exists only if $\omega \in A$. In either case,
\[ |\chi_{A}(\omega) - \chi_{A}(\omega)\psi^*(\omega)|^{2} = 0 \leq |\chi_{A}(\omega) - \psi(\omega)|^{2}. \]
\end{subclaimproof}

Now suppose $A,B \in \mathcal{S}$ such that $A\cap B = \emptyset$.
\begin{subclaim}
$\mu_0(A\cup B) \leq \mu_0(A) + \mu_0(B)$.
\end{subclaim}
\begin{subclaimproof}
Let $\psi_{1}, \psi_{2} \in \mathcal{N}$. Define $\psi_{1}^{*}, \psi_{2}^{*}$ as in Subclaim 3.1. Then by Subclaim 3.1, $\chi_{A}\psi_{1}^{*} +
\chi_{B}\psi_{2}^{*} \in \mathcal{N}$ and 
\begin{align*}
\mu_0(A\cup B) & = \inf_{\psi \in \mathcal{N}}\int |\chi_{A\cup B} - \psi|^{2}d\mu \\
& \leq \int|\chi_{A\cup B} - (\chi_{A}\psi_{1}^{*} + \chi_{B}\psi_{2}^{*})|^{2}d\mu \\
& = \int|\chi_{A} - \chi_{A}\psi_{1}^{*} + \chi_{B} - \chi_{B}\psi_{2}^{*}|d\mu \\
& = \int_{A}|\chi_{A} - \chi_{A}\psi_{1}^{*} + \chi_{B} - \chi_{B}\psi_{2}^{*}|d\mu + 
\int_{B}|\chi_{A} - \chi_{A}\psi_{1}^{*} + \chi_{B} - \chi_{B}\psi_{2}^{*}|d\mu \\
& = \int_{A}|\chi_{A} - \chi_{A}\psi_{1}^{*}|d\mu + \int_{B}|\chi_{B} - \chi_{B}\psi_{2}^{*}|d\mu \\
& \leq \int|\chi_{A} - \psi_{1}|d\mu + \int|\chi_{B} - \psi_{2}|d\mu \\
\end{align*}
Therefore 
\[ \mu_0(A\cap B) \leq \inf_{\psi\in \mathcal{N}}\int|\chi_{A} - \psi|^{2}d\mu + \inf_{\psi\in\mathcal{N}}\int|\chi_{B} - \psi|^{2}d\mu =
\mu_0(A) + \mu_0(B). \]
\end{subclaimproof}

\begin{subclaim}
$\mu_0(A) + \mu_0(B) \leq \mu_0(A\cup B)$.
\end{subclaim}
\begin{subclaimproof}
Let $\psi_{1} \in \mathcal{N}$ and define $\psi_{1}^{*}$ as in Subclaim 3.1. Then by Subclaim 3.1, $\chi_{A}\psi_{1}^{*}, \chi_{B}\psi_{1}^{*}$, and
$\chi_{A}\psi_{1}^{*} + \chi_{B}\psi_{1}^{*} = \chi_{A\cup B}\psi_{1}^{*} \in \mathcal{N}$, and 
\begin{align*}
\mu_0(A) + \mu_0(B) & = \inf_{\psi\in\mathcal{N}}\int|\chi_{A} - \psi|^{2}d\mu + \inf_{\psi\in\mathcal{N}}\int|\chi_{B} - \psi|^{2}d\mu \\
& \leq \int|\chi_{A} - \chi_{A}\psi_{1}^{*}|^{2}d\mu + \int|\chi_{B} - \chi_{B}\psi_{1}^{*}|^{2}d\mu \\
& = \int_{A}|\chi_{A} - \chi_{A}\psi_{1}^{*}|^{2}d\mu + \int_{B}|\chi_{B} - \chi_{B}\psi_{1}^{*}|^{2}d\mu \\
& = \int_{A\cup B}|\chi_{A} - \chi_{A}\psi_{1}^{*} + \chi_{B} - \chi_{B}\psi_{1}^{*}|^{2}d\mu \\
& = \int|\chi_{A\cup B} - \chi_{A\cup B}\psi_{1}^{*}|^{2}d\mu \\
& \leq \int|\chi_{A\cup B} - \psi_{1}|^{2}d\mu \\
\end{align*}
Therefore $\mu_0(A) + \mu_(B) \leq \inf_{\psi\in\mathcal{N}}\int|\chi_{A\cup B} - \psi|^{2}d\mu = \mu_0(A\cup B)$.
\end{subclaimproof}

By subclaims 3.2 and 3.3, we are done.
\end{claimproof}


\newpage
\subsection*{4 [Theorem (I)(6)xxix Claim 4]}
\begin{tcolorbox}
$\mu_0$ is a measure.
\end{tcolorbox}
\stepcounter{ClaimCounter}
\begin{claimproof}
\begin{subclaim}
If $\left\{ A_{n} \right\}_{n=0}^{\infty}$ is a sequence of measurable sets such that $A_{n} \supseteq A_{n+1}$ for all $n \in \mathbb{N}$, then 
$\lim_{n\rightarrow\infty}\mu_0(A_{n}) = \mu_0(A)$, where $A := \cap_{n=0}^{\infty}A_{n}$.
\end{subclaim}
\begin{subclaimproof}
Suppose $\left\{ A_{n} \right\}_{n=0}^{\infty}$ is a descending sequence of measurable sets. Let $\psi_{1} \in \mathcal{N}$. Clearly $\psi_{1}$ 
must be bounded. Therefore $|\chi_{A_{n}} - \psi_{1}|^{2} \leq M$, for some $M > 0$, 
and $\lim_{n\rightarrow\infty}|\chi_{A_{n}} - \psi_{1}|^{2} = |\chi_{A} - \psi_{1}|^{2}$. But the constant funtion $g \equiv M$ is integrable with respect to $\mu$
since $\mu$ is finite. Hence we can apply the DCT, and so
\[ \liminf_{n\rightarrow\infty}\inf_{\psi\in\mathcal{N}}\int|\chi_{A_{N}} - \psi|^{2}d\mu \leq \liminf_{n\rightarrow\infty}\int|\chi_{A_{n}} -
\psi_{1}|^{2}d\mu = \int|\chi_{A} - \psi_{1}|^{2}d\mu. \]
Now by taking the infimum over all $\psi \in \mathcal{N}$ on the right hand side of the above inequality, we get 
\begin{equation}
\liminf_{n\rightarrow\infty}\mu_0(A_{n}) \leq \mu_0(A).
\label{4.1}
\end{equation}
On the other hand, $\mu_0(A_n) = \mu_0(A) + \mu_0(A_n - A) \geq \mu_0(A)$ for all $n \in \mathbb{N}$ by Claim 3. Therefore 
\begin{equation}
\limsup_{n\rightarrow\infty}\mu_0(A_{n}) \geq \mu_0(A).
\label{4.2}
\end{equation}
So by \eqref{4.1} and \eqref{4.2}, we are done.
\end{subclaimproof}

\begin{subclaim}
If $\left\{ A_{n} \right\}_{n=0}^{\infty}$ is a sequence of measurable sets such that $A_{n} \subseteq A_{n+1}$ for all $n \in \mathbb{N}$, then 
$\lim_{n\rightarrow\infty}\mu_0(A_{n}) = \mu_0(A)$, where $A := \cup_{n=0}^{\infty}A_{n}$.
\end{subclaim}
\begin{subclaimproof}
Suppose $\left\{ A_n \right\}_{n=0}^{\infty}$ is an ascending sequence of measurable sets. Then $A^{c} = \cap_{n=0}^{\infty}A_{n}^{c}$, where $\left\{
A_{n}^{c} \right\}_{n=0}^{\infty}$ is a descending sequence of measurable sets. So by Subclaim 4.1, and Claim 3,
\[ \lim_{n\rightarrow\infty}\mu_0(A_n) = \lim_{n\rightarrow\infty}\mu_0(X) - \mu_0(A_{n}^{c}) = \mu_0(X) - \lim_{n\rightarrow\infty}\mu_0(A_{n}^{c}) = 
\mu_0(X) - \mu_0(A^{c}) = \mu_0(A). \]
\end{subclaimproof}

By Subclaim 4.2, Claim 3, and the result from Homework 1: Question 3, $\mu_0$ is a measure.

\end{claimproof}



\newpage
\subsection*{5 [Theorem (I)(6)xxix Claim 5]}
\begin{tcolorbox}
$\mu_{1}\perp \nu$.
\end{tcolorbox}
\stepcounter{ClaimCounter}
\begin{claimproof}
\begin{subclaim}
If $\theta$ is a measure such that $\theta \leq \mu$ and $\theta \ll \nu$, then $\theta \leq \mu_0$.
\end{subclaim}
\begin{subclaimproof}
Let $A \in \mathcal{S}$ and $\psi \in \mathcal{N}$. Since $\theta \ll \nu$, $\theta\left(\left\{ \omega \in X : |\psi(\omega)| > 0) \right\}\right) =
0$. Hence,
\[ \theta(A) = \int \chi_{A}d\theta = \int|\chi_{A}|^{2}d\theta = \int|\chi_{A} - \psi|^{2}d\theta \leq \int|\chi_{A} - \psi|^{2}d\mu. \]
Thus,
\[ \theta(A) \leq \inf_{\psi\in\mathcal{N}}\int|\chi_{A} - \psi|^{2}d\mu = \mu_0(A). \]
\end{subclaimproof}

Now suppose $\eta$ is a measure such that $\eta \leq \mu_{1}, \nu$. Consider the measure $\theta := \mu_0 + \eta$. Then 
\[ \theta = \mu_0 + \eta = \mu - \mu_1 + \eta \leq \mu, \]
and if $\nu(A) = 0$, then $\theta(A) = \mu_0(A) + \eta(A) = 0$. Hence, by Subclaim 1,
$\theta = \mu_0 + \eta \leq \mu_0$, which implies $\eta \equiv 0$. Thus, by Lemma (I)(6)xxvii, $\mu_1 \perp \nu$.
\end{claimproof}


\end{document}

