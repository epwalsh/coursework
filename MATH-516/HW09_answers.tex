\documentclass[12pt]{article}
\usepackage{amsmath}
\usepackage{amsfonts}
\usepackage{parskip}
\usepackage{amsthm}
\usepackage{thmtools}
\usepackage[headheight=15pt]{geometry}
\geometry{a4paper, left=20mm, right=20mm, top=30mm, bottom=30mm}
\usepackage{graphicx}
\usepackage{bm} % for bold font in math mode - command is \bm{text}
\usepackage{enumitem}
\usepackage{fancyhdr}
\usepackage{amssymb} % for stacked arrows and other shit
\pagestyle{fancy}
\usepackage{changepage}
\usepackage{mathcomp}
\usepackage{tcolorbox}

\declaretheoremstyle[headfont=\normalfont]{normal}
\declaretheorem[style=normal]{Theorem}
\declaretheorem[style=normal]{Proposition}
\declaretheorem[style=normal]{Lemma}
\newcounter{ProofCounter}
\newcounter{ClaimCounter}[ProofCounter]
\newcounter{SubClaimCounter}[ClaimCounter]
\newenvironment{Proof}{\stepcounter{ProofCounter}\textsc{Proof.}}{\hfill$\square$}
\newenvironment{claim}[1]{\vspace{1mm}\stepcounter{ClaimCounter}\par\noindent\underline{\bf Claim \theClaimCounter:}\space#1}{}
\newenvironment{claimproof}[1]{\par\noindent\underline{Proof of claim \theClaimCounter:}\space#1}{\hfill $\blacksquare$ Claim \theClaimCounter}
\newenvironment{subclaim}[1]{\stepcounter{SubClaimCounter}\par\noindent\emph{Subclaim \theClaimCounter.\theSubClaimCounter:}\space#1}{}
% \newenvironment{subclaimproof}[1]{\begin{adjustwidth}{2em}{0pt}\par\noindent\emph{Proof of subclaim \theClaimCounter.\theSubClaimCounter:}\space#1}{\hfill
% $\blacksquare$ \emph{Subclaim \theClaimCounter.\theSubClaimCounter}\vspace{5mm}\end{adjustwidth}}
\newenvironment{subclaimproof}[1]{\par\noindent\emph{Proof of subclaim \theClaimCounter.\theSubClaimCounter:}\space#1}{\hfill
$\Diamond$ \emph{Subclaim \theClaimCounter.\theSubClaimCounter}}

\allowdisplaybreaks{}

% chktex-file 3

\title{MATH 516: HW 9}
\author{Evan P. Walsh}
\makeatletter
\makeatother
\lhead{Evan P. Walsh}
\chead{MATH 516: HW 9}
\rhead{\thepage}
\cfoot{}

\begin{document}
\maketitle

\subsection*{1}
\begin{tcolorbox}
  Prove Subclaim 3.a in the proof of the Open Mapping Theorem.
\end{tcolorbox}
\begin{Proof}
  From Claim 2, we know that $B_{1}(\bm{0}) \subseteq \overline{T[B_{2k/r}(\bm{0})]}$. Let $w \in \mathcal{B}$ and $R := \frac{2k}{r}\|w\|$. Thus, 
  by linearity,
  \[ B_{\|w\|}(\bm{0}) \subseteq \overline{T[B_{R}(\bm{0})]}, \ \text{and so} \ \overline{B_{\|w\|}(\bm{0})} \subseteq \overline{T[B_{R}(\bm{0})]}.\]
  But this implies that for any $\epsilon > 0$, there exists some $y \in T[B_{R}(\bm{0})]$ such that 
  \[ y \in \overline{B_{\|w\|}(\bm{0})} \cap B_{\epsilon}(w). \]
  Hence there exists some $v \in B_{R}(\bm{0})$ with $T(v) = y$. So 
  $T(v) \in \overline{B_{\|w\|}(\bm{0})} \cap B_{\epsilon}(w)$, and thus $\|w - T(v)\| < \epsilon$.
\end{Proof}

\newpage
\subsection*{2 [RF 13.6]}
\begin{tcolorbox}
  Let $X$ be a normed linear space.
  \begin{enumerate}[label = (\roman*)]
    \item Let $\left\{ x_n \right\}_{n=0}^{\infty}$ and $\left\{ y_{n} \right\}_{n=0}^{\infty}$ be sequences in $X$ such that $x_n \rightarrow x$ and
      $y_n \rightarrow y$. Show that for any real numbers $\alpha$ and $\beta$, $\alpha x_n + \beta y_n \rightarrow \alpha x + \beta y$.
    \item Use (i) to show that if $Y$ is a subspace of $X$, then its closure $\overline{Y}$ is also a linear subspace of $X$.
  \end{enumerate}
\end{tcolorbox}
\begin{Proof}
  \begin{enumerate}[label = (\roman*)]
    \item Let $\epsilon > 0$. Then there exists $N_{1} \in \mathbb{N}$ such that $\| x_n - x\| < \epsilon / 2\alpha$ whenever $n \geq N_1$ and $N_2 \in \mathbb{N}$ 
      such that $\|y_n - y\| < \epsilon / 2\beta$ whenever $n \geq N_2$. Let $N = \max\{N_1, N_2\}$. Then for $n \geq N$,
      \begin{align*}
        \|(\alpha x_n + \beta y_n) - (\alpha x + \beta y)\| \leq \|\alpha x_n - \alpha x\| + \|\beta y_n - \beta y\| & = \alpha \|x_n - x\| + \beta
        \|y_n - y\| \\
        & < \alpha \cdot \frac{\epsilon}{2\alpha} + \beta \frac{\epsilon}{2\beta} = \epsilon.
      \end{align*}
      Hence $\lim_{n\rightarrow\infty}\|(\alpha x_n + \beta y_n) - (\alpha x + \beta y)\| = 0$, i.e. $\alpha x_n + \beta y_n \rightarrow \alpha x +
      \beta y$.
    \item Suppose $Y$ is a subspace of $X$. Let $x,y \in \overline{Y}$ and $\alpha, \beta \in \mathbb{R}$. We need to show that $\alpha x + \beta y
      \in \overline{Y}$. Well, since $x,y \in \overline{Y}$, there exists sequence $\left\{ x_n \right\}_{n=0}^{\infty}$ and $\left\{ y_n
      \right\}_{n=0}^{\infty}$ such that $x_n,y_n \in Y$ for all $n \in \mathbb{N}$. Since $Y$ is a subspace, $\alpha x_n + \beta y_n \in Y$ for all $n
      \in \mathbb{N}$. But by part (i), $\lim_{n\rightarrow \infty}\alpha x_n + \beta y_n = \alpha x + \beta y$. Thus $\alpha x + \beta y \in
      \overline{Y}$ by definition of closure.
  \end{enumerate}
\end{Proof}


\subsection*{3 [RF 13.7]}
\begin{tcolorbox}
  Show that the set $\mathcal{P}$ of all polynomials on $[a,b]$ is a linear space. For $\mathcal{P}$ considered as a subset of the normed linear
  space $C[a,b]$ and $L^{1}[a,b]$, show that $\mathcal{P}$ fails to be closed.
\end{tcolorbox}
\begin{Proof}
  Trivially $0 \in \mathcal{P}$. Now let $P$ and $Q$ be two polynomials in $\mathcal{P}$ and $\alpha,\beta \in \mathbb{R}$. Then we can write 
  $P(x) = a_n x^{n} + a_{n-1}x^{n-1} + \cdots + a_{0}$ and $Q(x) = b_{m} x^{m} + b_{m-1}x^{m-1} + \cdots + b_{0}$
  for all $x \in [a,b]$. Without loss of generality assume $n \geq m$. Then,
  \[ 
    R(x) := \alpha P(x) + \beta Q(x) = \alpha a_n x^{n} + \cdots + (\alpha a_{m} + \beta b_{m})x^{m} + (\alpha a_{m-1} + \beta b_{m-1})x^{m-1} + \cdots +
    (\alpha a_0 + \beta b_0)
  \]
  for all $x \in [a,b]$. Since $R$ is also a polynomial, $R \in \mathcal{P}$ and hence $\mathcal{P}$ is a linear space. However, if we define $S_k(x) :=
  \sum_{n=0}^{k}\frac{x^{n}}{n!}$ for all $k \in \mathbb{N}$, $x \in [a,b]$, then each $S_k \in \mathcal{P}$ and 
  \[ 
    \lim_{k\rightarrow\infty}S_k(x) = \lim_{k\rightarrow\infty}\sum_{n=0}^{k}\frac{x^{n}}{n!} = \exp(x),
  \]
  for all $x \in [a,b]$. $e^{x}$ is well-defined and continuous (therefore integrable on $[a,b]$ as well), yet is not a polynomial since it is 
  defined in terms of a power series with infinitely many
  terms. Thus we conclude that $\mathcal{P}$ is not closed as a subspace of either $C[a,b]$ or $L^{1}[a,b]$.
\end{Proof}


\subsection*{4 [RF 13.13]}
\begin{tcolorbox}
  Let $X$ be a Banach space and $T \in \mathcal{L}(X,X)$ have $\|T\| < 1$. Use the Contraction Mapping Principle to show that $I - T \in
  \mathcal{L}(X,X)$ is one-to-one and onto.
\end{tcolorbox}
\begin{Proof}
  Let $S := I - T$. 
  \begin{claim}
    $T(w) = w$ if and only if $w = \bm{0}$.
  \end{claim}
  \begin{claimproof}
    Since 
    \[ \|T(u) - T(v)\| = \|T(u-v)\| \leq \|T\|\cdot \|u - v\| \]
    for any $u, v \in X$, $T$ is a contraction map. Hence, by the contraction mapping principle there exists a unique $w \in X$ such that $T(w) = w$.
    But since $T(\bm{0}) = \bm{0}$, $w = \bm{0}$.
  \end{claimproof}

  \begin{claim}
    $S$ is one-to-one.
  \end{claim}
  \begin{claimproof}
    Suppose $u,v \in X$ and $S(u) = S(v)$. Then $u - T(u) = v - T(v)$, so 
    \[
      u-v = T(u) - T(v) = T(u-v). 
    \]
    Hence, by Claim 1, $u - v = \bm{0}$.
  \end{claimproof}

  We will use the notation $T^{n}(u) = \underbrace{T\circ T\circ \cdots \circ T}_{n\ \text{times}}(u)$, for $n \geq 1$, $u \in X$. Then
  \begin{equation}
    \|T^{n}(u)\| \leq \|T\|\cdot \|T^{n-1}(u)\| \leq \|T\|^{2}\cdot \|T^{n-2}(u)\| \leq \cdots \leq \|T\|^{n}\|u\|.
    \label{4.1}
  \end{equation}
  Thus, for any $u \in X$, the series 
  $H(u) := u + \sum_{n=1}^{\infty}T^{n}(u)$
  is absolutely convergent since 
  \[
    \sum_{n=1}^{\infty}\|T^{n}(u)\| \leq \sum_{n=1}^{\infty}\|T\|^{n}\cdot \|u\| = \|u\| \sum_{n=1}^{\infty}\|T\|^{n} = \|u\|\frac{\|T\|}{1 - \|T\|} <
    \infty
  \]
  by~\eqref{4.1} and the fact that $\|T\| < 1$. Hence $H(u)$ is well-defined since $X$ is a Banach space.

  \begin{claim}
    For any $u \in X$, $S(H(u)) = u$, and therefore $S$ is onto.
  \end{claim}
  \begin{claimproof}
    Let $u \in X$. Then, by continuity and linearity,
    \begin{align*}
      \left\|u - S(H(u))\right\| = \left\|u - \lim_{k\rightarrow\infty}S\left( u + \sum_{n=1}^{k}T^{n}(u) \right)\right\| 
      & = \left\|u - S(u) + \lim_{k\rightarrow\infty}\sum_{n=1}^{k} S(T^{n}(u))\right\| \\
      & = \left\|T(u) + \lim_{k\rightarrow\infty}\sum_{n=1}^{k}(T^{n}(u) - T^{n+1}(u))\right\| \\
      & = \left\|\lim_{k\rightarrow\infty}T^{k+1}(u)\right\| = 0,\\
    \end{align*}
    where the last equality arises from $\|T^{k+1}(u)\| \leq \|T\|^{k+1}\cdot \|u\| \rightarrow 0$ as $k \rightarrow\infty$.
  \end{claimproof}

  By Claims 2 and 3, we are done.
\end{Proof}

\subsection*{5}
\begin{tcolorbox}
  Suppose $\mathcal{B}$ is a Banach space and $T : \mathcal{B} \rightarrow \mathcal{B}$ is a bounded linear endomorphism (i.e.\ is linear % chktex 12
  and surjective). Suppose $\left\{ y_n \right\}_{n=0}^{\infty}$ converges in $\mathcal{B}$. Prove that there exists $x_0, x_1, \hdots \in \mathcal{B}$
  such that $T(x_n) = y_n$ for all $n$ and $\lim_{n\rightarrow\infty}x_n$ exists.
\end{tcolorbox}
\begin{Proof}
  Since $T$ is surjective, $T[\mathcal{B}] = \mathcal{B}$, which is closed. Hence,
  by Theorem 8 of Section 13.4 (page 263) in Royden \& Fitzpatrick, there exists a constant $M > 0$ so that
  for each $y \in \mathcal{B}$, there exists $x \in \mathcal{B}$ such that $T(x) = y$ and $\|x\| \leq M\|y\|$. 
  Let $y := \lim_{n\rightarrow\infty}y_{n}$ and $x \in \mathcal{B}$ such that $T(x) = y$. 
  Thus, for each $n \in \mathbb{N}$, we can choose $x_{n}^{*} \in \mathcal{B}$ such that 
  $T(x_{n}^{*}) = y_n - y$ and $\|x_{n}^{*}\| \leq M\|y_n - y\|$. Then let $x_n := x^{*} + x$. So
  \[ T(x_n) = T(x_{n}^{*}) + T(x) = y_n - y + y = y_n, \]
  and 
  \[ \|x_n - x\| = \|x_{n}^{*}\| \leq M\|y_n - y\|. \]
  Therefore $\|x_n - x\| \rightarrow 0$ since $\|y_n - y\| \rightarrow 0$, so $x_n \rightarrow x$.
  % Now, let $x_0 \in \mathcal{B}$ such
  % that $T(x_0) = y_0$. By way of induction, suppose $x_{n}$ has been chosen. By the previous remark, we can choose $x^{*} \in \mathcal{B}$ such that 
  % $T(x^{*}) = y_{n+1} - y_{n}$ and $\|x^{*}\| \leq M\|y_{n+1} - y_{n}\|$. Then let $x_{n+1} = x^{*} + x_{n}$. Hence,
  % \begin{equation}
    % \|x_{n+1} - x_{n}\| = \|x^{*}\| \leq M\|y_{n+1} - y_{n}\|.
    % \label{5.1}
  % \end{equation}

  % \begin{claim}
    % Every subsequence of $\left\{ x_n \right\}_{n=0}^{\infty}$ has a further subsequence that converges.
  % \end{claim}
  % \begin{claimproof}
    % Let $\left\{ n_j \right\}_{j=0}^{\infty}$ be a subsequence of $\mathbb{N}$.
    % Let $\epsilon > 0$. Since $\left\{ y_n \right\}_{n=0}^{\infty}$ converges, we can choose a further 
    % subsequence $\left\{ n_{j_k} \right\}_{k=0}^{\infty}$ such
    % that $\|y_{n_{j_{k+1}}} - y_{n_{j_{k}}}\| < M^{-1}2^{-k}$ for each $k \in \mathbb{N}$.
  % \end{claimproof}
  
  % \begin{claim}
    % Every convergent subsequence of $\left\{ x_{n} \right\}_{n=0}^{\infty}$ converges to the same point.
  % \end{claim}
  % \begin{claimproof}
    % By way of contradiction suppose $\left\{ x_{n_j}\right\}_{j=0}^{\infty}$ and $\left\{ x_{k_j} \right\}_{j=0}^{\infty}$ are two subsequences that
    % converge to separate limits $x'$ and $x''$, respectively. Let $0 < \epsilon < \frac{1}{4}\|x' - x''\|$. 
    % Let $N_{1} \in \mathbb{N}$ such that $\|x_{n_j} - x'\| < \epsilon$ whenever $j \geq N_{1}$ and let $N_{2} \in \mathbb{N}$ such that 
    % $\|x_{k_j} - x''\| < \epsilon$ whenever $j \geq N_{2}$. Since $\left\{ y_n \right\}_{n=0}^{\infty}$ converges, 
    % there exists $N_{3} \in \mathbb{N}$ such that $\|y_n - y_m\| < \epsilon$ whenever $m,n \geq N_{3}$. Let $N := \max_{1\leq i\leq
    % 3}\{N_{i}\}$. Since $n_j,k_j \geq j$ for all $j \in \mathbb{N}$, $j\geq N$ implies $n_j,k_j \geq N$.
    % Thus $\|x_{n_j} - x_{k_j}\| < \epsilon$ whenever $j \geq N$. So for $j \geq N$, we have
    % \[ \|x' - x''\| \leq \|x' - x_{n_j}\| + \|x_{n_j} - x_{k_j}\| + \|x_{k_j} - x''\| < \epsilon + \epsilon + \epsilon < 4 \epsilon = \|x' - x''\|. \]
    % This is a contradiction. Therefore every convergence subsequence must converge to the same point.
  % \end{claimproof}

  % By Claims 1 and 2, every subsequence of $\left\{ x_{n} \right\}_{n=0}^{\infty}$ has a further subsequence that converges to the same limit. 
  % Hence $\left\{ x_{n} \right\}_{n=0}^{\infty}$ converges.
\end{Proof}


\end{document}

