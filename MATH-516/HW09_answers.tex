\documentclass[12pt]{article}
\usepackage{amsmath}
\usepackage{amsfonts}
\usepackage{parskip}
\usepackage{amsthm}
\usepackage{thmtools}
\usepackage[headheight=15pt]{geometry}
\geometry{a4paper, left=20mm, right=20mm, top=30mm, bottom=30mm}
\usepackage{graphicx}
\usepackage{bm} % for bold font in math mode - command is \bm{text}
\usepackage{enumitem}
\usepackage{fancyhdr}
\usepackage{amssymb} % for stacked arrows and other shit
\pagestyle{fancy}
\usepackage{changepage}
\usepackage{mathcomp}
\usepackage{tcolorbox}

\declaretheoremstyle[headfont=\normalfont]{normal}
\declaretheorem[style=normal]{Theorem}
\declaretheorem[style=normal]{Proposition}
\declaretheorem[style=normal]{Lemma}
\newcounter{ProofCounter}
\newcounter{ClaimCounter}[ProofCounter]
\newcounter{SubClaimCounter}[ClaimCounter]
\newenvironment{Proof}{\stepcounter{ProofCounter}\textsc{Proof.}}{\hfill$\square$}
\newenvironment{claim}[1]{\vspace{1mm}\stepcounter{ClaimCounter}\par\noindent\underline{\bf Claim \theClaimCounter:}\space#1}{}
\newenvironment{claimproof}[1]{\par\noindent\underline{Proof of claim \theClaimCounter:}\space#1}{\hfill $\blacksquare$ Claim \theClaimCounter}
\newenvironment{subclaim}[1]{\stepcounter{SubClaimCounter}\par\noindent\emph{Subclaim \theClaimCounter.\theSubClaimCounter:}\space#1}{}
% \newenvironment{subclaimproof}[1]{\begin{adjustwidth}{2em}{0pt}\par\noindent\emph{Proof of subclaim \theClaimCounter.\theSubClaimCounter:}\space#1}{\hfill
% $\blacksquare$ \emph{Subclaim \theClaimCounter.\theSubClaimCounter}\vspace{5mm}\end{adjustwidth}}
\newenvironment{subclaimproof}[1]{\par\noindent\emph{Proof of subclaim \theClaimCounter.\theSubClaimCounter:}\space#1}{\hfill
$\Diamond$ \emph{Subclaim \theClaimCounter.\theSubClaimCounter}}

\allowdisplaybreaks{}

% chktex-file 3

\title{MATH 516: HW 9}
\author{Evan P. Walsh}
\makeatletter
\makeatother
\lhead{Evan P. Walsh}
\chead{MATH 516: HW 9}
\rhead{\thepage}
\cfoot{}

\begin{document}
\maketitle

\newpage
\subsection*{2 [RF 13.6]}
\begin{tcolorbox}
  Let $X$ be a normed linear space.
  \begin{enumerate}[label = (\roman*)]
    \item Let $\left\{ x_n \right\}_{n=0}^{\infty}$ and $\left\{ y_{n} \right\}_{n=0}^{\infty}$ be sequences in $X$ such that $x_n \rightarrow x$ and
      $y_n \rightarrow y$. Show that for any real numbers $\alpha$ and $\beta$, $\alpha x_n + \beta y_n \rightarrow \alpha x + \beta y$.
    \item Use (i) to show that if $Y$ is a subspace of $X$, then its closure $\overline{Y}$ is also a linear subspace of $X$.
  \end{enumerate}
\end{tcolorbox}
\begin{Proof}
  \begin{enumerate}[label = (\roman*)]
    \item Let $\epsilon > 0$. Then there exists $N_{1} \in \mathbb{N}$ such that $\| x_n - x\| < \epsilon / 2\alpha$ whenever $n \geq N_1$ and $N_2 \in \mathbb{N}$ 
      such that $\|y_n - y\| < \epsilon / 2\beta$ whenever $n \geq N_2$. Let $N = \max\{N_1, N_2\}$. Then for $n \geq N$,
      \begin{align*}
        \|(\alpha x_n + \beta y_n) - (\alpha x + \beta y)\| \leq \|\alpha x_n - \alpha x\| + \|\beta y_n - \beta y\| & = \alpha \|x_n - x\| + \beta
        \|y_n - y\| \\
        & < \alpha \cdot \frac{\epsilon}{2\alpha} + \beta \frac{\epsilon}{2\beta} = \epsilon.
      \end{align*}
      Hence $\lim_{n\rightarrow\infty}\|(\alpha x_n + \beta y_n) - (\alpha x + \beta y)\| = 0$, i.e. $\alpha x_n + \beta y_n \rightarrow \alpha x +
      \beta y$.
    \item Suppose $Y$ is a subspace of $X$. Let $x,y \in \overline{Y}$ and $\alpha, \beta \in \mathbb{R}$. We need to show that $\alpha x + \beta y
      \in \overline{Y}$. Well, since $x,y \in \overline{Y}$, there exists sequence $\left\{ x_n \right\}_{n=0}^{\infty}$ and $\left\{ y_n
      \right\}_{n=0}^{\infty}$ such that $x_n,y_n \in Y$ for all $n \in \mathbb{N}$. Since $Y$ is a subspace, $\alpha x_n + \beta y_n \in Y$ for all $n
      \in \mathbb{N}$. But by part (i), $\lim_{n\rightarrow \infty}\alpha x_n + \beta y_n = \alpha x + \beta y$. Thus $\alpha x + \beta y \in
      \overline{Y}$ by definition of closure.
  \end{enumerate}
\end{Proof}


\subsection*{3 [RF 13.7]}
\begin{tcolorbox}
  Show that the set $\mathcal{P}$ of all polynomials on $[a,b]$ is a linear space. For $\mathcal{P}$ considered as a subset of the normed linear
  space $C[a,b]$ and $L^{1}[a,b]$, show that $\mathcal{P}$ fails to be closed.
\end{tcolorbox}
\begin{Proof}
  Trivially $0 \in \mathcal{P}$. Now let $P$ and $Q$ be two polynomials in $\mathcal{P}$ and $\alpha,\beta \in \mathbb{R}$.. Then we can write 
  $P(x) = a_n x^{n} + a_{n-1}x^{n-1} + \cdots + a_{0}$ and $Q(x) = b_{m} x^{m} + b_{m-1}x^{m-1} + \cdots + b_{0}$
  for all $x \in [a,b]$. Without loss of generality assume $n \geq m$. Then,
  \[ 
    R(x) := \alpha P(x) + \beta Q(x) = \alpha a_n x^{n} + \cdots + (\alpha a_{m} + \beta b_{m})x^{m} + (\alpha a_{m-1} + \beta b_{m-1})x^{m-1} + \cdots +
    (\alpha a_0 + \beta b_0)
  \]
  for all $x \in [a,b]$. Since $R$ is also a polynomial, $R \in \mathcal{P}$ and hence $\mathcal{P}$ is a linear space. However, if we define $S_k(x) :=
  \sum_{n=0}^{k}\frac{x^{n}}{n!}$ for all $k \in \mathbb{N}$, $x \in [a,b]$, then each $S_k \in \mathcal{P}$ and 
  \[ 
    \lim_{k\rightarrow\infty}S_k(x) = \lim_{k\rightarrow\infty}\sum_{n=0}^{k}\frac{x^{n}}{n!} = \exp(x),
  \]
  for all $x \in [a,b]$. $e^{x}$ is well-defined and continuous (therefore integrable on $[a,b]$ as well), yet is not a polynomial since it is 
  defined in terms of a power series with infinitely many
  terms. Thus we conclude that $\mathcal{P}$ is not closed as a subspace of either $C[a,b]$ or $L^{1}[a,b]$.
\end{Proof}


\end{document}

