\documentclass[12pt]{article}
\usepackage{amsmath}
\usepackage{amsfonts}
\usepackage{parskip}
\usepackage{amsthm}
\usepackage{thmtools}
\usepackage[headheight=15pt]{geometry}
\geometry{a4paper, left=20mm, right=20mm, top=30mm, bottom=30mm}
\usepackage{graphicx}
\usepackage{bm} % for bold font in math mode - command is \bm{text}
\usepackage{enumitem}
\usepackage{fancyhdr}
\usepackage{amssymb} % for stacked arrows and other shit
\pagestyle{fancy}
\usepackage{changepage}
\usepackage{mathcomp}
\usepackage{tcolorbox}

\declaretheoremstyle[headfont=\normalfont]{normal}
\declaretheorem[style=normal]{Theorem}
\declaretheorem[style=normal]{Proposition}
\declaretheorem[style=normal]{Lemma}
\newcounter{ProofCounter}
\newcounter{ClaimCounter}[ProofCounter]
\newcounter{SubClaimCounter}[ClaimCounter]
\newenvironment{Proof}{\stepcounter{ProofCounter}\textit{Proof.}}{\hfill$\square$}
\newenvironment{claim}[1]{\vspace{1mm}\stepcounter{ClaimCounter}\par\noindent\underline{\bf Claim \theClaimCounter:}\space#1}{}
\newenvironment{claimproof}[1]{\par\noindent\underline{Proof of claim \theClaimCounter:}\space#1}{\hfill $\blacksquare$ Claim \theClaimCounter}
\newenvironment{subclaim}[1]{\stepcounter{SubClaimCounter}\par\noindent\emph{Subclaim \theClaimCounter.\theSubClaimCounter:}\space#1}{}
% \newenvironment{subclaimproof}[1]{\begin{adjustwidth}{2em}{0pt}\par\noindent\emph{Proof of subclaim \theClaimCounter.\theSubClaimCounter:}\space#1}{\hfill
% $\blacksquare$ \emph{Subclaim \theClaimCounter.\theSubClaimCounter}\vspace{5mm}\end{adjustwidth}}
\newenvironment{subclaimproof}[1]{\par\noindent\emph{Proof of subclaim \theClaimCounter.\theSubClaimCounter:}\space#1}{\hfill
$\Diamond$ \emph{Subclaim \theClaimCounter.\theSubClaimCounter}}

\title{MATH 516: HW 1}
\author{Evan P. Walsh}
\makeatletter
\let\runauthor\@author
\let\runtitle\@title
\makeatother
\lhead{\runauthor}
\chead{\runtitle}
\rhead{\thepage}
\cfoot{}

\begin{document}
% \maketitle

\subsection*{1}
\begin{tcolorbox}
Let $\mathcal{A} := \left\{ E \subseteq [0,1] : \chi_{E} \text{ is Riemann integrable} \right\}$. Prove that $\mathcal{A}$ is an algebra but not a
$\sigma$-algebra.
\end{tcolorbox}

% {\bf Solution:}

\begin{Lemma}
If $s_{1}, s_{2}$ are step functions on $[a,b]$, then $\max\left\{ s_{1}, s_{2} \right\}$ is also a step function on $[a,b]$.
\end{Lemma}
\begin{Proof}
Suppose $P_{1} := (x_{0}, \hdots, x_{n})$ and $P_{2} := (y_{0}, \hdots, y_{m})$ are partitions of $[0,1]$ such that $s_{1}$ is constant on $(x_{i-1},
x_{i})$ and $s_{2}$ is constant on $(y_{j-1}, y_{j})$ for all $1\leq i \leq n$, $1 \leq j \leq m$. Let $P$ be the partition formed by the union of
$P_{1}$ and $P_{2}$. Then it is clear that $\max\left\{ s_{i} \right\}_{i=1,2}$ is constant over each subinterval of $P$. Hence it is a step function.
\end{Proof}

Now we will prove that $\mathcal{A}$ is an algebra.

\begin{Proof}
Clearly $\chi_{[0,1]}$ is Riemann integrable, so $[0,1] \in \mathcal{A}$. Now suppose $E \in \mathcal{A}$. Then 
\[ \chi_{E^{c}}(x) = 1 - \chi_{E}(x) = \chi_{[0,1]}(x) - \chi_{E}(x),\]
for all $x \in [0,1]$. Therefore $\chi_{E^{c}}$ is Riemann integrable since $\chi_{[0,1]}$ and $\chi_{E}$ are Riemann integrable. So $\mathcal{A}$ is
closed under complementation. Lastly, it suffices to show that $\mathcal{A}$ is closed under finite unions. So suppose $E_{1}, E_{2} \in \mathcal{A}$
and let $\epsilon > 0$. Since $\chi_{E_{1}}$ and $\chi_{E_{2}}$ are Riemann integrable, there are step functions $\varphi_{1}, \psi_{1}, \varphi_{2},
\psi_{2}$ over $[0,1]$ for which $\psi_{1} \leq \chi_{E_{1}} \leq \varphi_{1}$, $\psi_{2} \leq \chi_{E_{2}} \leq \varphi_{2}$ and such that
\begin{equation}
\int (\varphi_{1} - \psi_{1}) < \epsilon / 2 \qquad \text{and}\qquad \int (\varphi_{2} - \psi_{2}) < \epsilon / 2.
\label{1.1}
\end{equation}
Let $\varphi := \max\left\{ \varphi_{1}, \varphi_{2} \right\}$ and $\psi:= \max\left\{ \psi_{1}, \psi_{2} \right\}$. By Lemma 1 $\varphi$ and $\psi$
are step functions. Further, for each $x \in [0,1]$,
\begin{equation}
\psi(x) \leq \chi_{E_{1} \cup E_{2}}(x) = \max\left\{ \chi_{E_{1}}(x), \chi_{E_{2}}(x) \right\} \leq \varphi(x).
\label{1.2}
\end{equation}
Now note that 
\[ \max\left\{ \varphi_{1}(x), \varphi_{2}(x) \right\} - \max\left\{ \psi_{1}(x), \psi_{2}(x) \right\} \leq \varphi_{1}(x) - \psi_{1}(x) +
\varphi_{2}(x) -\psi_{2}(x)\] 
since $\varphi_{1}(x) \geq \psi_{1}(x)$ and $\varphi_{2}(x) \geq \psi_{2}(x)$. Thus, by \eqref{1.1}, we have 
\begin{equation}
\int (\varphi - \psi) \leq \int (\varphi_{1} - \psi_{1} + \varphi_{2} - \psi_{2}) = \int (\varphi_{1} - \psi_{1}) + \int (\varphi_{2} - \psi_{2}) <
\epsilon. 
\label{1.3}
\end{equation}
So by \eqref{1.2} and \eqref{1.3}, $\chi_{E_{1}\cup E_{2}}$ is Riemann integrable, and therefore $E_{1} \cup E_{2} \in \mathcal{A}$.
\end{Proof}

To see that $\mathcal{A}$ is not a $\sigma$-algebra, consider first that $\chi_{\{x\}}$ is Riemann integrable for any $x \in [0,1]$. Therefore $q \in
\mathcal{A}$ for, say, all rational $q \in [0,1]$. However, $\mathbb{Q}\cap [0,1]$, a countable union of singletons belonging to $\mathcal{A}$, is
clearly not in $\mathcal{A}$ since we have shown the Dirichlet function to not be Riemann integrable.


\newpage
\subsection*{2}
\begin{tcolorbox}
Give an example of a monotone class that is not a $\sigma$-algebra.
\end{tcolorbox}

Let $\mathcal{M} = \{\mathbb{N}\} \cup \left\{ \{0,\hdots,n\} : n \in \mathbb{N} \right\}$. Then $\mathcal{M}$ is closed under
countable unions and countable intersections, and therefore is a monotone class. To see this, let $\left\{ A_{n} \right\}_{n=0}^{\infty}\subseteq
\mathcal{M}$. If $A := \cup_{n=0}^{\infty}A_{n}$ is bounded, then 
\[ A = \left\{ 0, \hdots, \max\{n \in \cup A\} \right\} \in \mathcal{M}. \]
If $\cup_{n=0}^{\infty}A_{n}$ is unbouded, then $A = \mathbb{N} \in \mathcal{M}$. On the other hand, if $B := \cap_{n=0}^{\infty}A_{n}$, then 
$B$ must be bounded, so 
\[ B = \{0,\hdots, \max\{n \in B\}\}. \]
However, $\{1,2\}^{c} = \{3, 4, 5, \hdots\}$, for example, is not in
$\mathcal{M}$, so $\mathcal{M}$ is not closed under complementation.



\subsection*{3 [RF 17.2]}
\begin{tcolorbox}
Let $\mathcal{M}$ be a $\sigma$-algebra of subsets of a set $X$ and the set function $\mu : \mathcal{M} \rightarrow [0, \infty)$ be finitely additive.
Prove that $\mu$ is a measure if and only if whenever $\left\{ A_{n} \right\}_{n=0}^{\infty}$ is an ascending sequence of sets in $\mathcal{M}$, then 
\[ \mu\left( \cup_{n=0}^{\infty}A_{n} \right) = \lim_{n\rightarrow\infty}\mu(A_{n}). \]
\end{tcolorbox}


\begin{Proof}
$(\Rightarrow)$ Apply Proposition (I)(1)xix.

$(\Leftarrow)$ First note that for any $E \in \mathcal{M}$, $\mu(E) = \mu(E) + \mu(\emptyset)$ by finite additivity, so $\mu(\emptyset) = 0$. It
remains to show that $\mu$ is countably additive. Suppose $\left\{ E_{k} \right\}_{k=0}^{\infty} \subseteq \mathcal{M}$ is pairwise disjoint. Set
$A_{n} := \cup_{k=0}^{n}E_{k}$. So $A_{n} \subseteq A_{n+1}$. Thus,
\[ \mu\left( \cup_{k=0}^{\infty}E_{k} \right) = \mu\left( \cup_{n=0}^{\infty}A_{n} \right) = \lim_{n\rightarrow\infty}\mu(A_{n}) \stackrel{*}{=}
\lim_{n\rightarrow\infty}\sum_{k=0}^{n}\mu(E_{k}) = \sum_{k=0}^{\infty}\mu(E_{k}), \]
where $(*)$ is due to finite additivity.
\end{Proof}


\subsection*{4}
\begin{tcolorbox}
Let $\mathcal{S}$ denote the $\sigma$-algebra of all Legesgue measurable subsets of $[0,1]$ and let $\mu$ denote Lebesgue measure on $[0,1]$. Suppose
$A$ is a measurable subset of $[0,1]$. For each $\alpha \in [0,1]$, let $f(\alpha) := \mu([0,\alpha]\cap A)$. Prove that $f$ is continuous.
\end{tcolorbox}

Actually it turns out $f$ is Lipschitz continuous.

\begin{Proof}
Let $t_{0}, t_{1} \in [0,1]$ and assume without loss of generality that $t_{0} \leq t_{1}$. Then using monotonicity and the excision property of
measures,
\[ |f(t_{0}) - f(t_{1})| = \mu(A\cap[0,t_{1}]) - \mu(A\cap[0,t_{0}]) = \mu(A\cap[t_{0},t_{1}]) \leq \mu([t_{0},t_{1}]) = |t_{0} - t_{1}|. \]
So $f$ is Lipschitz continuous with Lipschitz constant $c = 1$.
\end{Proof}






\subsection*{5}
\begin{tcolorbox}
Let $\mathcal{S}$ denote the $\sigma$-algebra of all Legesgue measurable subsets of $[0,1]$ and let $\mu$ denote Lebesgue measure on $[0,1]$. Suppose
$A$ is a measurable subset of $[0,1]$ and $r < \mu(A)$. Prove that there is a measurable $B \subseteq A$ such that $\mu(B) = r$.
\end{tcolorbox}


\begin{Proof}
Define the function $f : [0,1] \rightarrow [0,1]$ by 
\[ f(t) := \mu(A \cap [0,t]) \]
for all $t \in [0,1]$. Clearly $f$ is monotone increasing and $0 = \mu(\emptyset) = f(0) < r < f(1) = \mu(A)$. By question 4, $f$ is
continuous. Hence, ran$(f) = [0, \mu(A)] \ni r$, so by the intermediate value theorem there exists some $t' \in [0,1]$ such that $r = f(t') = \mu(A \cap [0,t'])$. So
$A \supset B := A \cap [0,t']$ has measure $r$.
\end{Proof}

\subsection*{6 [Extra Credit]}
\begin{tcolorbox}
Let $\mathcal{S}$ denote the $\sigma$-algebra of all Lebesgue measurable subsets of $[0,1]$. Let $m$ denote the Lebesgue measure, and let $\mu :
\mathcal{S} \rightarrow \mathbb{R}$ denote a measure so that $\mu(A) = m(A)$ whenever $\mathcal{A} \in \mathcal{S}$ and $m(A) = 1/2$. Prove that
$\mu(A) = m(A)$ whenever $A \in \mathcal{S}$.
\end{tcolorbox}

\begin{Proof}
Define the function $F_{\mu} : [0,1] \rightarrow \mathbb{R}$ by $F_{\mu}(t) := \mu([0,t])$ for all $t \in [0,1]$, i.e. $F_{\mu}$ is the 
cumulative distribution function (cdf) of $\mu$. Similarly, let $F_{m}$ be the cdf of $m$. It's easy enough to show that $F_{m}$ goes from 0 to 1 with
constant derivative of 1. We will show that $F_{\mu} \equiv F_{m}$ and use that to prove that $\mu$ and $m$ agree on all $E \in \mathcal{S}$.

\begin{claim}
$\mu$ and $m$ agree on all sets $E \in \mathcal{S}$ such that $m(E) = 2^{-(n+1)}$, for some $n \in \mathbb{N}$.
\end{claim}
\begin{claimproof}
Let $n \geq 1$. Choose a measurable set $E_{0} \subset [0,1/2)$ such that $m(E_{0}) = 1/2 - 2^{-(n+1)}$. By monotonicity,
\[ r := \mu(E_{0}) \leq 1/2. \]
Thus, if $E \subseteq [1/2,1]$ such that $m(E) = 2^{-(n+1)}$, then 
\[ m(E_{0} \cup E) = m(E_{0}) + m(E) = 1/2 = \mu(E_{0} \cup E) = \mu(E_{0}) + \mu(E), \] 
so $\mu(E) = 1/2 - r$. Now, if we partition $[1/2,1]$ into $2^{n}$ pairwise disjoint sets $\left\{ E_{1}, \hdots , E_{2^{n}} \right\}$ such that
$\mu(E_{k}) = 2^{-(n+1)}$ for all $1 \leq k \leq 2^{n}$, then since $\cup_{k=1}^{2^{n}}E_{k} = [1/2,1]$, we have 
\[ 1/2 = \sum_{k=1}^{2^{n}}\mu(E_{k}) = 2^{n}(1/2 - r), \]
which implies $r = 1/2 - 2^{-(n+1)}$. Therefore $\mu(E) = 2^{-(n+1)}$ for all $E \subseteq [1/2,1]$ such that $m(E) = 2^{-(n+1)}$. A similar argument
holds for sets $E \in \mathcal{S}$ such that $E \subseteq [0,1/2]$ or $E \subseteq [1/2,3/4]$. Therefore we can conclude that $\mu(E) = m(E)$ whenever
$m(E) = 2^{-(n+1)}$ for some $n \in \mathbb{N}$.
\end{claimproof}

\begin{claim}
For every $1/2 > |h|  > 0$, there exists $c_{h} \geq 0$ such that $F_{\mu}(t) - F_{\mu}(t - h) \equiv c_{h}$ for every $t \in
[1/2,1]$. Similarly, there exists $d_{h} \geq 0$ such that $F_{\mu}(t+h) - F_{\mu}(t) \equiv d_{h}$ for every $t \in [0,1/2]$.
\end{claim}
\begin{claimproof}
Without loss of generality, suppose $1/2 > h > 0$. 
Let $E_{0} := [0, 1/2 - h]$. Then for any $t \in [1/2,1]$,
\[ 1/2 = \mu(E_{0} \cup (t-h, t]) = \mu(E_{0}) + \mu( (t-h, t]) = \mu(E_{0}) + F_{\mu}(t) - F_{\mu}(t-h). \]
Therefore $F_{\mu}(t) - F_{\mu}(t-h) \equiv c_{h} := 1/2 - \mu(E_{0})$. A similar argument shows that this holds when $t \in [0,1/2]$.
\end{claimproof}

\begin{claim}
$F_{\mu}$ is differentiable, and therefore continuous, on $[0,1]$ with constant derivative of $1$.
\end{claim}
\begin{claimproof}
Since $F_{\mu}$ is monotone increasing (this is clear by its definition), $F_{\mu}'$ exists a.e. ($m$). Now suppose $t_{0} \in [0,1]$ such that
$F_{\mu}(t_{0})$ exists. Without loss of generality assume $t_{0} \neq 1$. Then by claim 1,
\[ F_{\mu}'(t_{0}) = \lim_{n\rightarrow\infty}\frac{F_{\mu}(t_{0} + 2^{-(n+1)}) - F_{\mu}(t_{0})}{2^{-(n+1)}} = \frac{\mu(
(t_{0},t_{0}+2^{-(n+1)}])}{2^{-(n+1)}} = 1. \]
Therefore $F_{\mu}'(t) \equiv 1$ wherever the derivative exists. But now we will show that the derivative exists everywhere. So let $t \in [0,1]$ and
suppose $\left\{ t_{n} \right\}_{n=0}^{\infty} \subset [0,1]$ such that $t_{n} \rightarrow t$. If $t \geq 1/2$, let $t' \geq 1/2$ so that
$F_{\mu}(t')$ exists. Otherwise let $t' < 1/2$ so that $F_{\mu}(t')$ exists. By claim 2,
\[ F_{\mu}(t) - F_{\mu}(t_{n}) = F_{\mu}(t') - F_{\mu}(t' - t + t_{n}), \]
since $t - t_{n} = t' - (t' - t + t_{n})$. Thus, 
\begin{equation}
\lim_{n\rightarrow\infty}\frac{F_{\mu}(t) - F_{\mu}(t_{n})}{t_{n} - t} = \lim_{n\rightarrow\infty}\frac{F_{\mu}(t') - F_{\mu}(t' - t + t_{n})}{t_{n} -
t} = F_{\mu}'(t') = 1.
\label{6.1}
\end{equation}
Since \eqref{6.1} holds for any sequence $\left\{ t_{n} \right\}_{n=0}^{\infty}$ such that $t_{n} \rightarrow t$, $F_{\mu}'(t) = 1$.
\end{claimproof}

\begin{claim}
$F_{\mu} \equiv F_{m}$.
\end{claim}
\begin{claimproof}
Follows by claim 3 since $0 = F_{\mu}(0) = F_{m}(0)$ and $1 \equiv F_{\mu}' \equiv F_{m}'$.
\end{claimproof}

\begin{claim}
$\mu\left( \left\{ x \right\} \right) = 0$ for all $x \in [0,1]$.
\end{claim}
\begin{claimproof}
Suppose $x \in [0,1]$ and without loss of generality assume $x \neq 0$. Then by the continuity of measure and the continuity of $F_{\mu}$ (claim 3),
\begin{align*}
\mu\left( \left\{ x \right\} \right) = \lim_{n\rightarrow\infty}\mu\left( (x-2^{-n},x] \right) = \lim_{n\rightarrow\infty}\mu([0,x]) -
\mu([0,x-2^{-n}]) & = \lim_{n\rightarrow\infty}F_{\mu}(x) - F_{\mu}(x - 2^{-n}) \\
& = F_{\mu}(x) - F_{\mu}(x) = 0. 
\end{align*}
\end{claimproof}

\begin{claim}
$\mu(I) = m(I)$ for every interval $I \subseteq [0,1]$.
\end{claim}
\begin{claimproof}
Suppose $0 \leq a \leq b \leq 1$. Then 
\[ m\left( (a,b] \right) = F_{m}(b) - F_{m}(a) = F_{\mu}(b) - F_{\mu}(a) = \mu\left( (a,b] \right).\]
By claim 5, $\mu( (a,b] ) = \mu( (a,b) ) = \mu( [a,b) ) = \mu([a,b])$. Therefore $\mu$ and $m$ agree on every interval.
\end{claimproof}

By claim 6, $\mu$ and $m$ agree on every open set, since an open set is the disjoint union of countably many open intervals. Therefore $\mu$ and $m$
also agree on every closed set, since a closed set is the complement of an open set and $\mu([0,1]) = m([0,1]) = 1$.

Now suppose $E$ is any measurable subset of $[0,1]$ and let $\epsilon > 0$. By a theorem from 515, there exists a
an open set $\mathcal{O}$ and closed set $F$ such that $\mathcal{O} \supseteq E \supseteq F$, while $m(\mathcal{O}) - m(E) < \epsilon / 2$ and $m(E) - m(F)
< \epsilon / 2$. But by monotonicity of $\mu$, $\mu(\mathcal{O}) \geq \mu(E) \geq \mu(F)$, which implies $m(\mathcal{O}) \geq \mu(E) \geq m(F)$ since
$\mu(\mathcal{O}) = m(\mathcal{O})$ and $\mu(F) = m(F)$. Therefore 
\[ |\mu(E) - m(E)| \leq |m(\mathcal{O}) - m(F)| \leq |m(\mathcal{O}) - m(E)| + |m(E) - m(F)| < \epsilon. \]
Since $\epsilon > 0$ was arbitrary, $\mu(E) = m(E)$.
\end{Proof}











\end{document}

