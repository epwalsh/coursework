\documentclass[12pt]{article}
\usepackage{amsmath}
\usepackage{amsfonts}
\usepackage{parskip}
\usepackage{amsthm}
\usepackage{thmtools}
\usepackage[headheight=15pt]{geometry}
\geometry{a4paper, left=20mm, right=20mm, top=30mm, bottom=30mm}
\usepackage{graphicx}
\usepackage{bm} % for bold font in math mode - command is \bm{text}
\usepackage{enumitem}
\usepackage{fancyhdr}
\usepackage{amssymb} % for stacked arrows and other shit
\pagestyle{fancy}
\usepackage{changepage}
\usepackage{mathcomp}
\usepackage{tcolorbox}

\declaretheoremstyle[headfont=\normalfont]{normal}
\declaretheorem[style=normal]{Theorem}
\declaretheorem[style=normal]{Proposition}
\declaretheorem[style=normal]{Lemma}
\newcounter{ProofCounter}
\newcounter{ClaimCounter}[ProofCounter]
\newcounter{SubClaimCounter}[ClaimCounter]
\newenvironment{Proof}{\stepcounter{ProofCounter}\textit{Proof.}}{\hfill$\square$}
\newenvironment{claim}[1]{\vspace{1mm}\stepcounter{ClaimCounter}\par\noindent\underline{\bf Claim \theClaimCounter:}\space#1}{}
\newenvironment{claimproof}[1]{\par\noindent\underline{Proof of claim \theClaimCounter:}\space#1}{\hfill $\blacksquare$ Claim \theClaimCounter}
\newenvironment{subclaim}[1]{\stepcounter{SubClaimCounter}\par\noindent\emph{Subclaim \theClaimCounter.\theSubClaimCounter:}\space#1}{}
% \newenvironment{subclaimproof}[1]{\begin{adjustwidth}{2em}{0pt}\par\noindent\emph{Proof of subclaim \theClaimCounter.\theSubClaimCounter:}\space#1}{\hfill
% $\blacksquare$ \emph{Subclaim \theClaimCounter.\theSubClaimCounter}\vspace{5mm}\end{adjustwidth}}
\newenvironment{subclaimproof}[1]{\par\noindent\emph{Proof of subclaim \theClaimCounter.\theSubClaimCounter:}\space#1}{\hfill
$\Diamond$ \emph{Subclaim \theClaimCounter.\theSubClaimCounter}}

\allowdisplaybreaks

\title{MATH 516: HW 1 - 5 Questions}
\author{Evan P. Walsh}
\makeatletter
\let\runauthor\@author
\let\runtitle\@title
\makeatother
\lhead{\runauthor}
\chead{\runtitle}
\rhead{\thepage}
\cfoot{}

\begin{document}
% \maketitle


\subsection*{1.1}
\begin{tcolorbox}
Let $\mathcal{A} := \left\{ E \subseteq [0,1] : \chi_{E} \text{ is Riemann integrable} \right\}$. Prove that $\mathcal{A}$ is an algebra but not a
$\sigma$-algebra.
\end{tcolorbox}

\subsection*{1.2}
\begin{tcolorbox}
Give an example of a monotone class that is not a $\sigma$-algebra.
\end{tcolorbox}

\subsection*{1.3 [RF 17.2]}
\begin{tcolorbox}
Let $\mathcal{M}$ be a $\sigma$-algebra of subsets of a set $X$ and the set function $\mu : \mathcal{M} \rightarrow [0, \infty)$ be finitely additive.
Prove that $\mu$ is a measure if and only if whenever $\left\{ A_{n} \right\}_{n=0}^{\infty}$ is an ascending sequence of sets in $\mathcal{M}$, then 
\[ \mu\left( \cup_{n=0}^{\infty}A_{n} \right) = \lim_{n\rightarrow\infty}\mu(A_{n}). \]
\end{tcolorbox}


\subsection*{1.4}
\begin{tcolorbox}
Let $\mathcal{S}$ denote the $\sigma$-algebra of all Legesgue measurable subsets of $[0,1]$ and let $\mu$ denote Lebesgue measure on $[0,1]$. Suppose
$A$ is a measurable subset of $[0,1]$. For each $\alpha \in [0,1]$, let $f(\alpha) := \mu([0,\alpha]\cap A)$. Prove that $f$ is continuous.
\end{tcolorbox}


\subsection*{1.5}
\begin{tcolorbox}
Let $\mathcal{S}$ denote the $\sigma$-algebra of all Legesgue measurable subsets of $[0,1]$ and let $\mu$ denote Lebesgue measure on $[0,1]$. Suppose
$A$ is a measurable subset of $[0,1]$ and $r < \mu(A)$. Prove that there is a measurable $B \subseteq A$ such that $\mu(B) = r$.
\end{tcolorbox}


\subsection*{1.6}
\begin{tcolorbox}
Let $\mathcal{S}$ denote the $\sigma$-algebra of all Lebesgue measurable subsets of $[0,1]$. Let $m$ denote the Lebesgue measure, and let $\mu :
\mathcal{S} \rightarrow \mathbb{R}$ denote a measure so that $\mu(A) = m(A)$ whenever $\mathcal{A} \in \mathcal{S}$ and $m(A) = 1/2$. Prove that
$\mu(A) = m(A)$ whenever $A \in \mathcal{S}$.
\end{tcolorbox}


\subsection*{2.1 [RF 17.19]}
\begin{tcolorbox}
Show that any measure induced by an outer measure is complete.
\end{tcolorbox}


\subsection*{2.2 [RF 17.7 i]}
\begin{tcolorbox}
Let $(X,\mathcal{M})$ be a measurable space. Verify that if $\mu$ and $\nu$ are measures defined on $\mathcal{M}$, then the set function $\lambda$ defined on
$\mathcal{M}$ by $\lambda(E) := \mu(E) + \nu(E)$ also is a measure. We denote $\lambda$ by $\mu + \nu$.
\end{tcolorbox}

\subsection*{2.3 [RF 17.7 ii]}
\begin{tcolorbox}
Let $(X,\mathcal{M})$ be a measurable space. Verify that if $\mu$ and $\nu$ are measures on $\mathcal{M}$ and $\mu \geq \nu$, then there is a measure
$\lambda$ on $\mathcal{M}$ for which $\mu = \nu + \lambda$.
\end{tcolorbox}


\subsection*{2.4 [RF 17.7 iii]}
\begin{tcolorbox}
Let $(X,\mathcal{M})$ be a measurable space. Verify that if $\nu$ is $\sigma$-finite, the measure $\lambda$ as in 3 is unique.
\end{tcolorbox}


\subsection*{2.5 [RF 17.7 iv]}
\begin{tcolorbox}
Let $(X,\mathcal{M})$ be a measurable space. Show that in general the measure $\lambda$ as in 3 need not be unique but that there is always a smallest
such $\lambda$.
\end{tcolorbox}


\newpage
\subsection*{3.1}
\begin{tcolorbox}
Define a measure on $\left\{ 0,1 \right\}^{\omega}$ to model a biased coin.
\end{tcolorbox}


\subsection*{3.2}
\begin{tcolorbox}
Suppose $X := \mathbb{N}$ and $\mathcal{M} := \{\emptyset\}$ and $p(\emptyset) = 0$. Describe $\mu_{p}^{*}$.
\end{tcolorbox}


\subsection*{3.3 [RF 17.36]}
\begin{tcolorbox}
Let $\mu$ be a finite premeasure on an algebra $\mathcal{S}$, and $\mu^{*}$ the induced outer measure. Show that a subset $E$ of $X$ is
$\mu^*$-measurable if and only if for each $\epsilon > 0$ there exists a set $A \in \mathcal{S}_{\delta}$ with $A\subseteq E$ such that $\mu^*(E-A) <
\epsilon$.
\end{tcolorbox}


\subsection*{3.4 [RF 17.34]}
\begin{tcolorbox}
If we start with an outer measure $\mu^{*}$ on $2^{X}$ and form the induced measure $\bar{\mu}$ on the $\mu^{*}$-measurable sets, we can view
$\bar{\mu}$ as a set function and denote by $\mu^{+}$ the outer measure induced by $\bar{\mu}$.
\begin{enumerate}[label=(\alph*)]
\item Show that for each set $E \subseteq X$ we have $\mu^{+}(E) \geq \mu^{*}(E)$.
\item For a given set $E$, show that $\mu^{+}(E) = \mu^*(E)$ if and only if there is a $\mu^*$-measurable set $A \supseteq E$ with $\mu^*(A) =
\mu^*(E)$.
\end{enumerate}
\end{tcolorbox}

\subsection*{3.5}
\begin{tcolorbox}
Suppose $(X, \mathcal{S}, \mu)$ is a $\sigma$-finite measure space. Suppose $\mathfrak{D} \subseteq \mathcal{S}$ is a disjoint class; i.e. $A\cap B = \emptyset$
whenever $A, B \in \mathfrak{D}$ and $A \neq B$. Prove that for each measurable $E \subseteq X$, $\{D \in \mathfrak{D} : \mu(E\cap D) > 0\}$ is
countable.
\end{tcolorbox}



\newpage
\subsection*{4.1 [RF 17.6]}
\begin{tcolorbox}
Let $(X, \mathcal{M}, \mu)$ be a measure space and $X_{0}$ belong to $\mathcal{M}$. Define $M_{0}$ to be the collection of sets in $\mathcal{M}$ that
are subsets of $X_{0}$ and $\mu_{0}$ the restriction of $\mu$ to $\mathcal{M}_{0}$. Show that $(X_{0}, \mathcal{M}_{0}, \mu_{0})$ is a measure space.
\end{tcolorbox}

\subsection*{4.2 [RF 17.4]}
\begin{tcolorbox}
Let $\left\{ (X_{\lambda}, \mathcal{M}_{\lambda}, \mu_{\lambda}) \right\}_{\lambda \in \Lambda}$ be a collection of measure spaces parametrized by the
set $\Lambda$. Assume the collection $\left\{ X_{\lambda} \right\}_{\lambda\in\Lambda}$ is pairwise disjoint. Then let $X := \cup_{\lambda \in
\Lambda}X_{\lambda}$, $\mathcal{B} := \left\{ B \subset X : B\cap X_{\lambda} \in \mathcal{M}_{\lambda}\ \forall \ \lambda \in \Lambda \right\}$, and
$\mu(B) := \sum_{\lambda\in\Lambda}\mu_{\lambda}(B\cap X_{\lambda})$ for each $B\in\mathcal{B}$. Then 
\begin{enumerate}[label=(\roman*)]
\item Show that $\mathcal{B}$ is a $\sigma$-algebra.
\item Show that $\mu$ is a measure.
\item Show that $\mu$ is $\sigma$-finite if and only if all but a countable number of the measures $\mu_{\lambda}$ have $\mu(X_{\lambda}) = 0$ and the
remaining are $\sigma$-finite.
\end{enumerate}
\end{tcolorbox}


\subsection*{4.3 [RF 18.19]}
\begin{tcolorbox}
Show that if $f$ is a non-negative measurable function on $X$, then 
\[ \int_{X} f\ d\mu = 0 \ \ \text{ if and only if $f = 0$ a.e. on $X$.} \]
\end{tcolorbox}

\subsection*{4.4 [RF 18.22]}
\begin{tcolorbox}
Suppose $f$ and $g$ are non-negative measurable functions on $X$ for which $f^{2}$ and $g^{2}$ are integrable over $X$ with respect to $\mu$. Show
that $f\cdot g$ is also integrable over $X$ with respect to $\mu$.
\end{tcolorbox}



\subsection*{4.5}
\begin{tcolorbox}
Suppose $\Omega := (X, \mathcal{S}, \mu)$ is a $\sigma$-finite measure space. Use the results from questions 1 and 2 to show that $\Omega$ can be written as a
countable sum of finite measures.
\end{tcolorbox}




\subsection*{5.1 [RF 18.2]}
\begin{tcolorbox}
Suppose $(X,\mathcal{M}, \mu)$ is not complete. Let $E$ be a subset of a set of measure zero that does not belong to $M$. Let $f \equiv 0$ on $X$ and $g :=
\chi_{E}$. Show that $f = g$ a.e. on $X$ while $f$ is measurable and $g$ is not.
\end{tcolorbox}


\subsection*{5.2 [RF 18.13]}
\begin{tcolorbox}
Let $\left\{ f_n \right\}_{n=0}^{\infty}$ be a sequence of real-valued measurable functions on $X$ such that, for each natural number $n$, 
\[ \mu\left\{ x \in X : |f_{n}(x) - f_{n+1}(x)| > 2^{-n} \right\} < 2^{-n}. \]
Show that $\left\{ f_{n} \right\}_{n=0}^{\infty}$ is pointwise convergent a.e. on $X$.
\end{tcolorbox}

\subsection*{5.3 [RF 20.5]}
\begin{tcolorbox}
Let $(X,\mathcal{A},\mu) = (\mathbb{N}, \mathcal{P}(\mathbb{N}), c)$, where $c$ is the counting measure. Define $f : \mathbb{N} \times \mathbb{N}
\rightarrow \mathbb{R}$ by setting 
\[ f(x,y) := \left\{ \begin{array}{cl}
2 - 2^{-x} & \text{ if } x = y \\
-2 + 2^{-x} & \text{ if } x = y + 1 \\
0 & \text{ otherwise. }
\end{array} \right.
\]
Show that $f$ is measurable with respect to the product measure $c\times c$. Also show that 
\[ \int_{\mathbb{N}}\left[ \int_{\mathbb{N}}f(m,n)\ dc(m) \right]\ dc(n) \neq \int_{\mathbb{N}}\left[ \int_{\mathbb{N}}f(m,n)\ dc(n) \right]\ dc(m).\]
Is this a contradiction of either Fubini's or Tonelli's Theorem?
\end{tcolorbox}


\subsection*{5.4}
\begin{tcolorbox}
Suppose $\mu(X) = 1$ and $f : X \rightarrow [0,\infty)$ is measurable. Set 
\[ A := \int_{X}f\ d\mu. \]
Prove that 
\[ \sqrt{1 + A^{2}} \leq \int_{X}\sqrt{1  +f^{2}}\ d\mu \leq 1 + A. \]
\end{tcolorbox}



\subsection*{5.5}
\begin{tcolorbox}
Suppose $f : X \rightarrow [0, \infty)$ is measurable, and set 
\[ c := \int_{X}f\ d\mu. \]
Suppose $0 < c < \infty$. Prove that 
\[ \lim_{n\rightarrow\infty}\int_{X}n\ln[1 + (f/n)^{\alpha}]\ d\mu = \left\{ \begin{array}{cl}
\infty & : 0 < \alpha < 1 \\
c & : \alpha = 1 \\
0 & : 1 < \alpha < \infty.
\end{array} \right.
\]
\end{tcolorbox}



\end{document}

