\documentclass[12pt]{article}
\usepackage{amsmath}
\usepackage{amsfonts}
\usepackage{parskip}
\usepackage{amsthm}
\usepackage{thmtools}
\usepackage[headheight=15pt]{geometry}
\geometry{a4paper, left=20mm, right=20mm, top=30mm, bottom=30mm}
\usepackage{graphicx}
\usepackage{bm} % for bold font in math mode - command is \bm{text}
\usepackage{enumitem}
\usepackage{fancyhdr}
\usepackage{amssymb} % for stacked arrows and other shit
\pagestyle{fancy}
\usepackage{changepage}
\usepackage{mathcomp}
\usepackage{tcolorbox}

\declaretheoremstyle[headfont=\normalfont]{normal}
\declaretheorem[style=normal]{Theorem}
\declaretheorem[style=normal]{Proposition}
\declaretheorem[style=normal]{Lemma}
\newcounter{ProofCounter}
\newcounter{ClaimCounter}[ProofCounter]
\newcounter{SubClaimCounter}[ClaimCounter]
\newenvironment{Proof}{\stepcounter{ProofCounter}\textsc{Proof.}}{\hfill$\square$}
\newenvironment{claim}[1]{\vspace{1mm}\stepcounter{ClaimCounter}\par\noindent\underline{\bf Claim \theClaimCounter:}\space#1}{}
\newenvironment{claimproof}[1]{\par\noindent\underline{Proof of claim \theClaimCounter:}\space#1}{\hfill $\blacksquare$ Claim \theClaimCounter}
\newenvironment{subclaim}[1]{\stepcounter{SubClaimCounter}\par\noindent\emph{Subclaim \theClaimCounter.\theSubClaimCounter:}\space#1}{}
% \newenvironment{subclaimproof}[1]{\begin{adjustwidth}{2em}{0pt}\par\noindent\emph{Proof of subclaim \theClaimCounter.\theSubClaimCounter:}\space#1}{\hfill
% $\blacksquare$ \emph{Subclaim \theClaimCounter.\theSubClaimCounter}\vspace{5mm}\end{adjustwidth}}
\newenvironment{subclaimproof}[1]{\par\noindent\emph{Proof of subclaim \theClaimCounter.\theSubClaimCounter:}\space#1}{\hfill
$\Diamond$ \emph{Subclaim \theClaimCounter.\theSubClaimCounter}}

\allowdisplaybreaks{}

% chktex-file 3
% chktex-file 8

\title{}
\author{Evan P. Walsh}
\makeatletter
\makeatother
\lhead{Evan P. Walsh}
\chead{Final Review: HW 7 - 9}
\rhead{\thepage}
\cfoot{}

\begin{document}
% \maketitle




\subsection*{7.1 [RF 18.52]}
\begin{tcolorbox}
  Let $(X,\mathcal{M}, \mu)$ be a finite measure space, $\left\{ E_k \right\}_{k=1}^{n}$ be a collection of measurable sets, and $\left\{ c_k
  \right\}_{k=1}^{n}$ a collection of real numbers. For each $E \in \mathcal{M}$, define
  \[ \nu(E) = \sum_{k=1}^{n}c_k \cdot \mu(E\cap E_k). \]
  Show that $\nu$ is absolutely continuous with respect to $\mu$ and find its Radon-Nikodym derivative $\frac{d\mu}{d\nu}$.
\end{tcolorbox}


\subsection*{7.2,3 [RF 18.54]}
\begin{tcolorbox}
  Let $\mu$, $\nu$, and $\lambda$ be $\sigma$-finite measures on the measurable space $(X, \mathcal{M})$.
  \begin{enumerate}[label = (\roman*)]
    \item If $\nu \ll \mu$ and if $f$ is a non-negative function on $X$ that is measurable with respect to $\mathcal{M}$, show that
      \[ \int_{X} f\ d\nu = \int_{X}f\left[ \frac{d\nu}{d\mu} \right]d\mu. \]
    \item If $\nu \ll \mu$ and $\lambda \ll \mu$, show that $\frac{d(\nu + \lambda)}{d\mu} = \frac{d\nu}{d\mu} + \frac{d\lambda}{d\mu}$ a.e. $(\mu)$.
    \item If $\nu \ll \mu \ll \lambda$, show that $\frac{d\nu}{d\lambda} = \frac{d\nu}{d\mu}\cdot \frac{d\mu}{d\lambda}$  a.e. $(\lambda)$
    \item If $\nu \ll \mu$ and $\mu \ll \nu$, show that $\frac{d\nu}{d\mu}\cdot \frac{d\mu}{d\nu} = 1$ a.e. $(\mu)$.
  \end{enumerate}
\end{tcolorbox}


\subsection*{7.4}
\begin{tcolorbox}
  Prove that $\left\{ 0,1 \right\}^{\omega}$ is totally disconnected.
\end{tcolorbox}

\subsection*{7.5}
\begin{tcolorbox}
  Prove that $\left\{ 0,1 \right\}^{\omega}$ is compact.
\end{tcolorbox}


\newpage
\subsection*{8.1}
\begin{tcolorbox}
  Prove Claim 1 from proof of Theorem (10)ii. % chktex 36
\end{tcolorbox}

\subsection*{8.2}
\begin{tcolorbox}
  Prove Claim 2 from proof of Theorem (10)ii. % chktex 36
\end{tcolorbox}

\subsection*{8.3}
\begin{tcolorbox}
  Prove Claim 3 from proof of Theorem (10)ii.  % chktex 36
\end{tcolorbox}


\subsection*{8.4}
\begin{tcolorbox}
  Prove that there is no one-to-one map from $\mathbb{R}^{2}$ into $\mathbb{R}$.
\end{tcolorbox}


\subsection*{8.5}
\begin{tcolorbox}
  Suppose $(X,\mathcal{S}, \mu)$ is a measure space and $\mu(X) < \infty$. When $A,B \in \mathcal{S}$, let $d(A,B) = \mu(A\triangle B)$. Prove that 
  $(\mathcal{S}, d)$ is a complete metric space (identify sets with symmetric difference of 0).
\end{tcolorbox}


\newpage
\subsection*{9.1}
\begin{tcolorbox}
  Prove Subclaim 3.a in the proof of the Open Mapping Theorem.
\end{tcolorbox}

\subsection*{9.2 [RF 13.6]}
\begin{tcolorbox}
  Let $X$ be a normed linear space.
  \begin{enumerate}[label = (\roman*)]
    \item Let $\left\{ x_n \right\}_{n=0}^{\infty}$ and $\left\{ y_{n} \right\}_{n=0}^{\infty}$ be sequences in $X$ such that $x_n \rightarrow x$ and
      $y_n \rightarrow y$. Show that for any real numbers $\alpha$ and $\beta$, $\alpha x_n + \beta y_n \rightarrow \alpha x + \beta y$.
    \item Use (i) to show that if $Y$ is a subspace of $X$, then its closure $\overline{Y}$ is also a linear subspace of $X$.
  \end{enumerate}
\end{tcolorbox}

\subsection*{9.3 [RF 13.7]}
\begin{tcolorbox}
  Show that the set $\mathcal{P}$ of all polynomials on $[a,b]$ is a linear space. For $\mathcal{P}$ considered as a subset of the normed linear
  space $C[a,b]$ and $L^{1}[a,b]$, show that $\mathcal{P}$ fails to be closed.
\end{tcolorbox}


\subsection*{9.4 [RF 13.13]}
\begin{tcolorbox}
  Let $X$ be a Banach space and $T \in \mathcal{L}(X,X)$ have $\|T\| < 1$. Use the Contraction Mapping Principle to show that $I - T \in
  \mathcal{L}(X,X)$ is one-to-one and onto.
\end{tcolorbox}

\subsection*{9.5}
\begin{tcolorbox}
  Suppose $\mathcal{B}$ is a Banach space and $T : \mathcal{B} \rightarrow \mathcal{B}$ is a bounded linear endomorphism (i.e.\ is linear % chktex 12
  and surjective). Suppose $\left\{ y_n \right\}_{n=0}^{\infty}$ converges in $\mathcal{B}$. Prove that there exists $x_0, x_1, \hdots \in \mathcal{B}$
  such that $T(x_n) = y_n$ for all $n$ and $\lim_{n\rightarrow\infty}x_n$ exists.
\end{tcolorbox}


\end{document}
