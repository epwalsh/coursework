\documentclass[12pt]{article}
\usepackage{amsmath}
\usepackage{amsfonts}
\usepackage{parskip}
\usepackage{amsthm}
\usepackage{thmtools}
\usepackage[headheight=15pt]{geometry}
\geometry{a4paper, left=20mm, right=20mm, top=30mm, bottom=30mm}
\usepackage{graphicx}
\usepackage{bm} % for bold font in math mode - command is \bm{text}
\usepackage{enumitem}
\usepackage{fancyhdr}
\usepackage{amssymb} % for stacked arrows and other shit
\pagestyle{fancy}
\usepackage{changepage}
\usepackage{mathcomp}
\usepackage{tcolorbox}

\declaretheoremstyle[headfont=\normalfont]{normal}
\declaretheorem[style=normal]{Theorem}
\declaretheorem[style=normal]{Proposition}
\declaretheorem[style=normal]{Lemma}
\newcounter{ProofCounter}
\newcounter{ClaimCounter}[ProofCounter]
\newcounter{SubClaimCounter}[ClaimCounter]
\newenvironment{Proof}{\stepcounter{ProofCounter}\textsc{Proof.}}{\hfill$\square$}
\newenvironment{claim}[1]{\vspace{1mm}\stepcounter{ClaimCounter}\par\noindent\underline{\bf Claim \theClaimCounter:}\space#1}{}
\newenvironment{claimproof}[1]{\par\noindent\underline{Proof of claim \theClaimCounter:}\space#1}{\hfill $\blacksquare$ Claim \theClaimCounter}
\newenvironment{subclaim}[1]{\stepcounter{SubClaimCounter}\par\noindent\emph{Subclaim \theClaimCounter.\theSubClaimCounter:}\space#1}{}
% \newenvironment{subclaimproof}[1]{\begin{adjustwidth}{2em}{0pt}\par\noindent\emph{Proof of subclaim \theClaimCounter.\theSubClaimCounter:}\space#1}{\hfill
% $\blacksquare$ \emph{Subclaim \theClaimCounter.\theSubClaimCounter}\vspace{5mm}\end{adjustwidth}}
\newenvironment{subclaimproof}[1]{\par\noindent\emph{Proof of subclaim \theClaimCounter.\theSubClaimCounter:}\space#1}{\hfill
$\Diamond$ \emph{Subclaim \theClaimCounter.\theSubClaimCounter}}

\title{MATH 516: HW 4}
\author{Evan P. Walsh}
\makeatletter
\let\runauthor\@author
\let\runtitle\@title
\makeatother
\lhead{\runauthor}
\chead{\runtitle}
\rhead{\thepage}
\cfoot{}

\begin{document}
\maketitle

\subsection*{1 [RF 17.6]}
\begin{tcolorbox}
Let $(X, \mathcal{M}, \mu)$ be a measure space adn $X_{0}$ belong to $\mathcal{M}$. Define $M_{0}$ to be the collection of sets in $\mathcal{M}$ that
are subsets of $X_{0}$ and $\mu_{0}$ the restriction of $\mu$ to $\mathcal{M}_{0}$. Show that $(X_{0}, \mathcal{M}_{0}, \mu_{0})$ is a measure space.
\end{tcolorbox}
\begin{Proof}

\begin{claim}
$\mathcal{M}_{0}$ is a $\sigma$-algebra.
\end{claim}
\begin{claimproof}
Clearly $\emptyset \in \mathcal{M}_{0}$. Now suppose $E \in \mathcal{M}_{0}$. Then $E \in \mathcal{M}$, so $X_{0} - E \in \mathcal{M}$, and since
$X_{0} - E \subseteq X_{0}$, $X_{0} - E \in \mathcal{M}_{0}$. Lastly, assume $E_{0}, E_{1}, \hdots \in \mathcal{M}_{0}$. Since $E_{n} \in \mathcal{M}$
for each $n \in \mathbb{N}$, $\cup_{n=0}^{\infty}E_{n} \in \mathcal{M}$, and since $E_{n} \subseteq X_{0}$ for all $n \in \mathbb{N}$,
$\cup_{n=0}^{\infty}E_{n}\subseteq X_{0}$. Therefore $\cup_{n=0}^{\infty}E_{n} \in \mathcal{M}_{0}$.
\end{claimproof}

\begin{claim}
$\mu_{0}$ is a measure on $(X_{0}, \mathcal{M}_{0})$.
\end{claim}
\begin{claimproof}
Clearly $\mu_{0}(\emptyset) = \mu(\emptyset) = 0$. Now suppose $\left\{ E_{n} \right\}_{n=0}^{\infty}$ is a pairwise disjoint sequence of sets in
$\mathcal{M}_{0}$. Since $\left\{ E_{n} \right\}_{n=0}^{\infty}$ is also a pairwise disjoint sequence of sets in $\mathcal{M}$,
\[ \mu_{0}\left( \cup_{n=0}^{\infty}E_{n} \right) = \mu\left( \cup_{n=0}^{\infty}E_{n} \right) = \sum_{n=0}^{\infty}\mu(E_{n}) =
\sum_{n=0}^{\infty}\mu_{0}(E_{n}). \]
\end{claimproof}

\end{Proof}

\newpage
\subsection*{2 [RF 17.4]}
\begin{tcolorbox}
Let $\left\{ (X_{\lambda}, \mathcal{M}_{\lambda}, \mu_{\lambda}) \right\}_{\lambda \in \Lambda}$ be a collection of measure spaces parametrized by the
set $\Lambda$. Assume the collection $\left\{ X_{\lambda} \right\}_{\lambda\in\Lambda}$ is pairwise disjoint. Then let $X := \cup_{\lambda \in
\Lambda}X_{\lambda}$, $\mathcal{B} := \left\{ B \subset X : B\cap X_{\lambda} \in \mathcal{M}_{\lambda}\ \forall \ \lambda \in \Lambda \right\}$, and
$\mu(B) := \sum_{\lambda\in\Lambda}\mu_{\lambda}(B\cap X_{\lambda})$ for each $B\in\mathcal{B}$. Then 
\begin{enumerate}[label=(\roman*)]
\item Show that $\mathcal{B}$ is a $\sigma$-algebra.
\item Show that $\mu$ is a measure.
\item Show that $\mu$ is $\sigma$-finite if and only if all but a countable number of the measures $\mu_{\lambda}$ have $\mu(X_{\lambda}) = 0$ and the
remaining are $\sigma$-finite.
\end{enumerate}
\end{tcolorbox}
\begin{enumerate}[label=(\roman*)]
\item \begin{Proof}
First note that $\emptyset \in \mathcal{B}$ since $\emptyset\cap X_{\lambda} = \emptyset \in \mathcal{M}_{\lambda}$ for all $\lambda \in \Lambda$. Now suppose $B
\in \mathcal{B}$. Then for each $\lambda \in \Lambda$, $(X - B) \cap X_{\lambda} = X_{\lambda} - B \in \mathcal{M}_{\lambda}$ since $B \cap
X_{\lambda} \in \mathcal{M}_{\lambda}$ and $\mathcal{M}_{\lambda}$ is closed under complementation. Therefore $X - B \in \mathcal{B}$. Now suppose
$B_{0}, B_{1}, \hdots \in \mathcal{B}$. Then for each $\lambda \in \Lambda$, $\left( \cup_{n=0}^{\infty}B_{n} \right)\cap X_{\lambda} =
\cup_{n=0}^{\infty}(B_{n} \cap X_{\lambda}) \in \mathcal{M}_{\lambda}$ since $B_{n} \cap X_{\lambda} \in \mathcal{M}_{\lambda}$ and
$\mathcal{M}_{\lambda}$ is closed under countable unions. Therefore $\cup_{n=0}^{\infty}B_{n} \in \mathcal{B}$.
\end{Proof}

\item \begin{Proof}
Since $\mu_{\lambda}(\emptyset) = 0$ for all $\lambda \in \Lambda$, $\mu(\emptyset) = \sum_{\lambda\in\Lambda}\mu_{\lambda}(\emptyset \cap
X_{\lambda}) = \sum_{\lambda\in\Lambda}\mu_{\lambda}(\emptyset) = 0$. Now suppose $B_{0}, B_{1}, \hdots \in \mathcal{B}$ such that $\left\{ 
B_{n}\right\}_{n=0}^{\infty}$ is pairwise disjoint. By the countable additivity of each $\mu_{\lambda}$,
\[ \mu_{\lambda}\left( X_{\lambda}\cap \bigcup_{n=0}^{\infty}B_{n} \right) = \sum_{n=0}^{\infty}\mu_{\lambda}(X_{\lambda}\cap B_{n}). \]
Therefore 
\begin{align}
\mu\left( \cup_{n=0}^{\infty}B_{n} \right) & = \sup\left\{ \sum_{\lambda\in F}\mu_{\lambda}\left( X_{\lambda}\cap\bigcup_{n=0}^{\infty} B_{n}\right) 
: F \subseteq \Lambda \text{ is finite}\right\}  \nonumber \\
& = \sup\left\{ \sum_{\lambda\in F}\sum_{n=0}^{\infty}\mu_{\lambda}(X_{\lambda}\cap B_{n}) : 
F \subseteq \Lambda \text{ is finite} \right\} \nonumber \\
& = \sup\left\{ \sum_{n=0}^{\infty}\sum_{\lambda\in F}\mu_{\lambda}(X_{\lambda}\cap B_{n}) : 
F \subseteq \Lambda \text{ is finite} \right\}. \label{2.0}
\end{align}
\begin{claim}
\[ \sup\left\{ \sum_{n=0}^{\infty}\sum_{\lambda\in F}\mu_{\lambda}(X_{\lambda}\cap B_{n}) : 
F \subseteq \Lambda \text{ is finite} \right\} \leq \sum_{n=0}^{\infty}\sup\left\{ \sum_{\lambda\in F}\mu_{\lambda}(X_{\lambda}\cap B_{n}) \right\}. \]
\end{claim}
\begin{claimproof}
Trivial.
\end{claimproof}
\begin{claim}
\[ \sup\left\{ \sum_{n=0}^{\infty}\sum_{\lambda\in F}\mu_{\lambda}(X_{\lambda}\cap B_{n}) : 
F \subseteq \Lambda \text{ is finite} \right\} \geq \sum_{n=0}^{\infty}\sup\left\{ \sum_{\lambda\in F}\mu_{\lambda}(X_{\lambda}\cap B_{n}) \right\}. \]
\end{claim}
\begin{claimproof}
For each $n \in \mathbb{N}$, let $F_{n} \subseteq \Lambda$ be finite. Let $F := \cup_{n=0}^{\infty}F_{n}$. 
Since $\mu_{\lambda} \geq 0$ for all $\lambda \in \Lambda$, $\sum_{\lambda\in F}\mu_{\lambda}(X_{\lambda}\cap B_{n}) \geq \sum_{\lambda\in
F_{n}}\mu_{\lambda}(X_{\lambda}\cap B_{n})$ for all $n \in \mathbb{N}$. Therefore 
\begin{equation}
\sum_{n=0}^{\infty}\sum_{\lambda\in F}\mu_{\lambda}(X_{\lambda}\cap B_{n}) \geq \sum_{n=0}^{\infty}\sum_{\lambda\in F_{n}}\mu_{\lambda}(X_{\lambda}\cap B_{n}).
\label{2.1}
\end{equation}
Hence, for any sequence $\left\{ F_{n} \right\}_{n=0}^{\infty}$ of finite subsets of $\Lambda$, we can find another finite $F \subseteq \Lambda$ such that 
\eqref{2.1} is satisfied. This proves claim 2.
\end{claimproof}

By \eqref{2.0} and claims 1 and 2,
\[ \mu\left( \cup_{n=0}^{\infty}B_{n} \right) = \sum_{n=0}^{\infty}\sup\left\{ \sum_{\lambda\in\Lambda}\mu_{\lambda}(X_{\lambda}\cap B_{n}) : 
F\subseteq \Lambda\text{ is finite} \right\} = \sum_{n=0}^{\infty}\mu\left( B_{n} \right). \]
\end{Proof}

\item \begin{Proof}
$(\Rightarrow)$ Suppose $\mu$ is $\sigma$-finite. Then there exists $A_{0}, A_{1}, \hdots \in \mathcal{B}$ such that $X = \cup_{n=0}^{\infty}A_{n}$
and $\mu(A_{n}) < \infty$ for all $n \in \mathbb{N}$. Therefore $\mu_{\lambda}(X_{\lambda}\cap A_{n}) < \infty$ for all $\lambda \in \Lambda$ and $n
\in \mathbb{N}$. Since $X_{\lambda} = \cup_{n=0}^{\infty}(X_{\lambda}\cap A_{n})$, this means that each $\mu_{\lambda}$ is $\sigma$-finite. Now note
that $\left\{ X_{\lambda} \right\}_{\lambda\in\Lambda}$ is a disjoint class. Thus, by problem 5 on homework 2, for each $n \in \mathbb{N}$ there is
only a countable number of $\lambda$'s such that $\mu(A_{n}\cap X_{\lambda}) = \mu_{\lambda}(A_{n}\cap X_{\lambda}) > 0$. So all together there is
only a countable number of $\lambda$'s such that $\mu_{\lambda}(A_{n}\cap X_{\lambda}) > 0$ for all $n\in\mathbb{N}$.
Since $\mu_{\lambda}(X_{\lambda}) = \sum_{n=0}^{\infty}\mu_{\lambda}(X_{\lambda}\cap A_{n})$, this means that all but a countable number of
$\lambda$'s satisfy $\mu_{\lambda}(X_{\lambda}) = 0$.

$(\Leftarrow)$ Now suppose that all but a countable number of the measures $\mu_{\lambda}$ satisfy $\mu_{\lambda}(X_{\lambda}) = 0$ and the remainder
are $\sigma$-finite. Then there exists a countable subset of $\Lambda$, call it $\mathcal{G}$, such that $\mu_{\lambda}$ is $\sigma$-finite for each
$\lambda \in \mathcal{G}$. Therefore for each $\lambda \in \mathcal{G}$, there exists $A_{\lambda,0}, A_{\lambda,1}, \hdots \in \mathcal{M}_{\lambda}$
such that $X_{\lambda} = \cup_{n=0}^{\infty}A_{\lambda,n}$ and $\mu_{\lambda}(A_{\lambda,n}) < \infty$ for all $n \in\mathbb{N}$. But then $X =
\cup_{\lambda\in \mathcal{G}}\cup_{n\in\mathbb{N}}A_{\lambda,n}$ and $\mu(A_{\lambda,n}) = \mu_{\lambda}(A_{\lambda,n}) < \infty$ for all
$\lambda\in\mathcal{G}$ and $n \in \mathbb{N}$. Thus $\mu$ is $\sigma$-finite.
\end{Proof}
\end{enumerate}

\end{document}

