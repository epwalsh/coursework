\documentclass[12pt]{article}
\usepackage{amsmath}
\usepackage{amsfonts}
\usepackage{parskip}
\usepackage{amsthm}
\usepackage{thmtools}
\usepackage[headheight=15pt]{geometry}
\geometry{a4paper, left=20mm, right=20mm, top=30mm, bottom=30mm}
\usepackage{graphicx}
\usepackage{bm} % for bold font in math mode - command is \bm{text}
\usepackage{enumitem}
\usepackage{fancyhdr}
\usepackage{amssymb} % for stacked arrows and other shit
\pagestyle{fancy}
\usepackage{changepage}
\usepackage{mathcomp}
\usepackage{tcolorbox}

\declaretheoremstyle[headfont=\normalfont]{normal}
\declaretheorem[style=normal]{Theorem}
\declaretheorem[style=normal]{Proposition}
\declaretheorem[style=normal]{Lemma}
\newcounter{ProofCounter}
\newcounter{ClaimCounter}[ProofCounter]
\newcounter{SubClaimCounter}[ClaimCounter]
\newenvironment{Proof}{\stepcounter{ProofCounter}\textsc{Proof.}}{\hfill$\square$}
\newenvironment{claim}[1]{\vspace{1mm}\stepcounter{ClaimCounter}\par\noindent\underline{\bf Claim \theClaimCounter:}\space#1}{}
\newenvironment{claimproof}[1]{\par\noindent\underline{Proof of claim \theClaimCounter:}\space#1}{\hfill $\blacksquare$ Claim \theClaimCounter}
\newenvironment{subclaim}[1]{\stepcounter{SubClaimCounter}\par\noindent\emph{Subclaim \theClaimCounter.\theSubClaimCounter:}\space#1}{}
% \newenvironment{subclaimproof}[1]{\begin{adjustwidth}{2em}{0pt}\par\noindent\emph{Proof of subclaim \theClaimCounter.\theSubClaimCounter:}\space#1}{\hfill
% $\blacksquare$ \emph{Subclaim \theClaimCounter.\theSubClaimCounter}\vspace{5mm}\end{adjustwidth}}
\newenvironment{subclaimproof}[1]{\par\noindent\emph{Proof of subclaim \theClaimCounter.\theSubClaimCounter:}\space#1}{\hfill
$\Diamond$ \emph{Subclaim \theClaimCounter.\theSubClaimCounter}}

\allowdisplaybreaks{}

% chktex-file 3

\title{STAT 520: Assignment 1}
\author{Evan P. Walsh}
\makeatletter
\makeatother
\lhead{Evan P. Walsh}
\chead{STAT 520: Assignment 1}
\rhead{\thepage}
\cfoot{}

\begin{document}
\maketitle


\section*{Part I: Random Variables}

\subsection*{1}
\begin{enumerate}[label=(\alph*)]
  \item Since we are interested in the presence of mayflies at sampling locations and how that is related to the primary habitat category and sediment
    type, I would define binary random variables that are associated with whether or not mayflies were found in specific samples.
    Let $X_{i,j,k} \in \{0,1\}$ indicate whether or not mayflies were found during the $i^{th}$ year since 1992, at the $j^{th}$ reach of the Upper
    Mississippi River, and at the $k^{th}$ sampling location, where $0 \leq i \leq 12$, $1 \leq j \leq 6$, and $1 \leq k \leq n_{i,j}$.
    Here 0 and 1 represent the absense or presence of mayflies, respectively.

    We could treat the primary habitat category and sediment type as non-random covariates. Depending on how far apart the samples were in time and
    space, it could be reasonable to treat the random variables as independent across sampling locations.

  \item To assess how the abundance of mayflies has changed over the period from 1992 to 2004 we could aggregate the variables defined in part (a)
    over $k$, the sampling locations. This is assuming that we are interested in examining the six reaches of the Upper Mississippi River separately, otherwise we could also
    aggregate over $j$ as well. Thus we define 
    \[
      Y_{i,j} := \frac{1}{n_{j}}\sum_{k=1}^{n_{j}}X_{i,j,k}. 
    \]
    Note that we use the mean as the aggregation function
    as opposed to the sum since the total number of samples taken $n_{j}$ may have differed across year and reach.

    The support of each $Y_{i,j}$ is technically discrete since the number of samples was finite, although it's probably reasonable to treat them as
    continuous for modelling purposes since each $n_{j}$ was approximately 120 -- a relatively large number. Thus the support of $Y_{i,j}$ would be
    the interval $[0, 1]$.

    Let $Y_{j} := (Y_{0,j}, \dots, Y_{12, j})$, the time series data for the $j^{th}$ reach of the river. It might make sense to assume
    $\{Y_j\}_{j=1}^{6}$ is an independent family of random variables, although it would make less sense to assume that $Y_{0,j}, \dots, Y_{12,j}$ are
    independent for fixed $j$.
\end{enumerate}


\subsection*{2}
In this study there were 2 groups and 12 pregnant sows in each group -- thus 12 litters across 2 groups. Let $X_{i,j,k} \in \{0,1\}$ be a random
variable associated with whether or not
the $k^{th}$ piglet in the $j^{th}$ litter in the $i^{th}$ group was born normal, for $i \in \{1, 2\}$, $j \in \{1, \dots, 12\}$, and $k \in \{1,
\dots, n_{j}\}$. Here $X_{i,j,k} = 1$ represents the event that the piglet was born normal. Similarly we define $Y_{i,j,k}$ and $Z_{i,j,k}$ to represent whether or not 
the $k^{th}$ piglet in the $j^{th}$ litter in the $i^{th}$ group was born weak or still, respectively. Obviously for fixed $i, j, k$, the random
variables $X_{i,j,k}, Y_{i,j,k}$, and $Z_{i,j,k}$ are not independent since one and only one of them will assume a value of 1. However, since sows
were separated, it would be reasonable to assume that the triplets $(X_{i,j,k}, Y_{i,j,k}, Z_{i,j,k})$ are independent across litters.



\subsection*{3}
For the direct observation part of the study we could define $X_{0}$ and $X_{1}$ to be associated with the number of visits and captures of the wasps 
during the observation time, respectively. Both $X_0$ and $X_1$ would have support in the non-negative integers.
Clearly these variables are not independent, as $X_1$ cannot exceed $X_0$. The conditional support of $X_1$ given $X_0 = x$ would be $\{0,\dots, x\}$.

For the indirect observation part of the study we could define $Y$ to be associated with the number of ingested wasps during the 2 week period of
observation. Then $Y$ would also have support in the non-negative integers. 


\newpage
\section*{Part II: Exponential Families}

\subsection*{1}
We can rewrite $f(y|\alpha, \beta)$ as 
\begin{align*}
  f(y|\alpha,\beta) & = \frac{\Gamma(\alpha + \beta)}{\Gamma(\alpha)\Gamma(\beta)}y^{\alpha - 1}(1-y)^{\beta - 1} \\
  & = \exp\left\{ \log\left[ \Gamma(\alpha + \beta) \right] - \log\left[ \Gamma(\alpha) \right] - \log\left[ \Gamma(\beta) \right] + (\alpha-1)\log(y)
  + (\beta - 1)\log(1-y)\right\} \\
  & = \exp\big\{ \alpha \log(y) + \beta\log(1-y) - \left[ \log\left\{ \Gamma(\alpha) \right\} + \log\left\{ \Gamma(\beta)\right\} - \log\left\{ \Gamma(\alpha
  + \beta) \right\} \right] \\
  & \qquad\qquad - \log(y) - \log(1-y) \big\} \\
  & = \exp\big\{ \theta_{1} \log(y) + \theta_{2}\log(1-y) - \left[ \log\left\{ \Gamma(\theta_{1}) \right\} + \log\left\{ \Gamma(\theta_{2})\right\} - \log\left\{ \Gamma(\theta_{1}
  + \theta_{2}) \right\} \right] \\
  & \qquad\qquad - \log(y) - \log(1-y) \big\},
\end{align*}
where $\theta_{1} := \alpha$ and $\theta_2 := \beta$. Thus, using property 5 of exponential families on page 32 of our notes, 
\begin{align*}
  E\left\{ \log(y) \right\} = \frac{\partial}{\partial\theta_{1}}\left[ \log\left\{ \Gamma(\theta_1) \right\} + \log\left\{ \Gamma(\theta_2)
  \right\} - \log\left\{ \Gamma(\theta_1 + \theta_2 \right\} \right]
  & = \frac{\Gamma'(\theta_1)}{\Gamma(\theta_1)} - \frac{\Gamma'(\theta_1 + \theta_2)}{\Gamma(\theta_1 + \theta_2)} \\
  & = \frac{\Gamma'(\alpha)}{\Gamma(\alpha)} - \frac{\Gamma'(\alpha + \beta)}{\Gamma(\alpha + \beta)} \\
\end{align*}


\subsection*{2}
We can rewrite $f(y|\lambda)$ as 
\begin{align*}
  f(y|\lambda) & = \exp\left\{ y\log(y) - \log(y!) - \lambda \right\} \\
  & = \exp\left\{ y\log(y) - \lambda - \log(y!) \right\} \\
  & = \exp\left\{ \theta y - e^{\theta} - \log(y!) \right\},
\end{align*}
where $\theta := \log(y)$. Thus, using property 5 of exponential families on page 32 of our notes,
\[
  EY = \frac{\partial}{\partial \theta}e^{\theta} = e^{\theta} = \lambda. 
\]


\subsection*{3}
First we derive the pdf of $W$. Let $w := g(y) := y^{-1}$. Then $y = g^{-1}(w) = w^{-1}$. So,
\begin{align*}
  f_{W}(w|\alpha,\beta) = f_{Y}(g^{-1}(w))\cdot \left| \frac{-1}{w^{2}} \right| & = f_{Y}(w^{-1})\cdot w^{-2} \\
  & = \frac{\beta^{\alpha}}{\Gamma(\alpha)}\left( \frac{1}{w} \right)^{\alpha - 1}\exp\left[ -\frac{\beta}{\alpha} \right]\frac{1}{w^{2}} \\
  & = \frac{\beta^{\alpha}}{\Gamma(\alpha)}w^{-\alpha - 1}\exp\left( -\frac{\beta}{w} \right), \ \ w \in (0, \infty).
\end{align*}
Now we can rewrite $f_{W}(w|\alpha,\beta)$ as 
\begin{align*}
  f_{W}(w|\alpha,\beta) & = \exp\left\{ -\frac{\beta}{w} + \alpha\log(\beta) - \log\left[ \Gamma(\alpha) \right] - (\alpha + 1)\log(w) \right\} \\
  & = \exp\left\{ -\frac{\beta}{w} - \alpha\log(w) - \left[ \log\left\{ \Gamma(\alpha) \right\} - \alpha\log(\beta) \right] - \log(w) \right\} \\
  & = \exp\left\{ \theta_1 w^{-1} + \theta_2\log(w) - \left[ \log\left\{ \Gamma(-\theta_2) \right\} + \theta_2\log(-\theta_1) \right] - \log(w) \right\},
\end{align*}
where $\theta_1 := -\beta$ and $\theta_2 := -\alpha$. Thus, using property 5 of exponential families on page 32 of the notes,
\begin{align*}
  E\left( \frac{1}{W} \right) & = \frac{\partial}{\partial \theta_1}\left[ \log\left\{ \Gamma(-\theta_2) \right\} + \theta_2\log(-\theta_1) \right] \\
  & = \frac{\theta_2}{\theta_1} = \frac{\alpha}{\beta}, \ \ \text{ and } \\
  E\left[ \log(W) \right] & =  \frac{\partial}{\partial \theta_2}\left[ \log\left\{ \Gamma(-\theta_2) \right\} + \theta_2\log(-\theta_1) \right] \\
  & = \log(-\theta_1) - \frac{\Gamma'(-\theta_2)}{\Gamma(\theta_2)} \\
  & = \log(\beta) - \frac{\Gamma'(\alpha)}{\Gamma(\alpha)}.
\end{align*}


\subsection*{Bonus}
Note that $\phi(x) := x^{-1}$ is convex on $(0,\infty)$. Further, $E|Y|$ and $E|W| = E(|Y|^{1})$ are finite. Thus, by Jensen's inequality,
\[
  \phi\left( EY \right) \leq E\phi(Y), \ \text{ i.e. } \ \frac{1}{EY} \leq E\left( \frac{1}{Y} \right).
\]
So the expected value of $W$ is greater than or equal to $(EY)^{-1}$.


\end{document}

