\documentclass[12pt]{article}
\usepackage{amsmath}
\usepackage{amsfonts}
\usepackage{parskip}
\usepackage{amsthm}
\usepackage{thmtools}
\usepackage[headheight=15pt]{geometry}
\geometry{a4paper, left=20mm, right=20mm, top=30mm, bottom=30mm}
\usepackage{graphicx}
\usepackage{bm} % for bold font in math mode - command is \bm{text}
\usepackage{enumitem}
\usepackage{fancyhdr}
\usepackage{amssymb} % for stacked arrows and other shit
\pagestyle{fancy}
\usepackage{changepage}
\usepackage{mathcomp}

\declaretheoremstyle[headfont=\normalfont]{normal}
\declaretheorem[style=normal]{Theorem}
\declaretheorem[style=normal]{Proposition}
\declaretheorem[style=normal]{Lemma}
\newcounter{ProofCounter}
\newcounter{ClaimCounter}[ProofCounter]
\newcounter{SubClaimCounter}[ClaimCounter]
\newenvironment{Proof}{\stepcounter{ProofCounter}\textit{Proof.}}{\hfill$\square$}
\newenvironment{claim}[1]{\vspace{3mm}\stepcounter{ClaimCounter}\par\noindent\underline{\bf Claim \theClaimCounter:}\space#1}{}
\newenvironment{claimproof}[1]{\par\noindent\underline{Proof of claim \theClaimCounter:}\space#1}{\hfill $\blacksquare$ Claim \theClaimCounter}
\newenvironment{subclaim}[1]{\stepcounter{SubClaimCounter}\par\noindent\emph{Subclaim \theClaimCounter.\theSubClaimCounter:}\space#1}{}
% \newenvironment{subclaimproof}[1]{\begin{adjustwidth}{2em}{0pt}\par\noindent\emph{Proof of subclaim \theClaimCounter.\theSubClaimCounter:}\space#1}{\hfill
% $\blacksquare$ \emph{Subclaim \theClaimCounter.\theSubClaimCounter}\vspace{5mm}\end{adjustwidth}}
\newenvironment{subclaimproof}[1]{\par\noindent\emph{Proof of subclaim \theClaimCounter.\theSubClaimCounter:}\space#1}{\hfill
$\Diamond$ \emph{Subclaim \theClaimCounter.\theSubClaimCounter}}

\title{STAT 641: HW 9}
\author{Evan P. Walsh}
\makeatletter
\let\runauthor\@author
\let\runtitle\@title
\makeatother
\lhead{\runauthor}
\chead{\runtitle}
\rhead{\thepage}
\cfoot{}

\begin{document}
\maketitle

\section*{5.1}
Suppose $\mu_{1}$ and $\mu_{2}$ are finite measures\footnote{Actually we don't need this assumption. In fact, even the assumption that $\mu_{1},
\mu_{2}$ are $\sigma$-finite is not necessary.} on $(\Omega_{1}, \mathcal{F}_{1})$ and $(\Omega_{2}, \mathcal{F}_{2})$, respectively. Then
$\mu_{12}(\cdot)$ and $\mu_{21}(\cdot)$, defined on page 149, are measures on $(\Omega_{1}\times \Omega_{2}, \mathcal{F}_{1}\otimes \mathcal{F}_{2})$.

\subsection*{Solution}
\begin{Proof}
Suppose $\mu_{1}$ and $\mu_{2}$ are as above.

\begin{claim}
$\mu_{12}, \mu_{21} \geq 0$.
\end{claim}
\begin{claimproof}
Since $\mu_{1}, \mu_{2} \geq 0$, 
\begin{align*}
\mu_{12}(A) & = \int_{\Omega_{1}}\mu_{2}(A_{1\omega_{1}})d\mu_{\omega_{1}} \geq 0 \text{ and } \\
\mu_{21}(A) & = \int_{\Omega_{2}}\mu_{1}(A_{2\omega_{2}})d\mu_{\omega_{2}} \geq 0 \ \forall \ A \in \mathcal{F}_{1} \otimes \mathcal{F}_{2}.
\end{align*}
\end{claimproof}

\begin{claim}
$\mu_{12}(\emptyset) = \mu_{21}(\emptyset) = 0$.
\end{claim}
\begin{claimproof}
Let $A := \emptyset$. Clearly $A_{1\omega_{1}} = \emptyset$ for all $\omega_{1} \in \Omega_{1}$ and $A_{2\omega_{2}} = \emptyset$ for all $\omega_{2}
\in \Omega_{2}$. Therefore $\mu_{2}(A_{1\omega_{1}}) = 0 = \mu_{1}(A_{2\omega_{2}})$. Thus,
\begin{align*}
\mu_{12}(A) & = \int_{\Omega_{1}}\mu_{2}(A_{1\omega_{1}})d\mu_{1}(\omega_{1}) = \int_{\Omega_{1}}0d\mu_{1} = 0, \text{ and } \\
\mu_{21}(A) & = \int_{\Omega_{2}}\mu_{1}(A_{2\omega_{2}})d\mu_{2}(\omega_{2}) = \int_{\Omega_{2}}0d\mu_{2} = 0. 
\end{align*}
\end{claimproof}

\begin{claim}
If $A := \cup_{n=0}^{\infty}A_{n}$, where $\left\{ A_{n} \right\}_{n=0}^{\infty} \subseteq \mathcal{F}_{1} \otimes \mathcal{F}_{2}$ is pairwise
disjoint, then 
\[ \mu_{12}\left( \cup_{n=0}^{\infty}A_{n} \right) = \sum_{n=0}^{\infty}\mu_{12}(A_{n})\qquad \text{ and } \qquad \mu_{21}\left( \cup_{n=0}^{\infty}A_{n}
\right) = \sum_{n=0}^{\infty}\mu_{21}(A_{n}). \]
\end{claim}
\begin{claimproof}
Fix $\omega_{1} \in \Omega_{1}$. Let $A_{1\omega_{1}}^{(n)} := \left\{ \omega_{2} \in \Omega_{2} : (\omega_{1}, \omega_{2}) \in A_{n} \right\}$. Then,
\begin{align}
A_{1\omega} := \left\{ \omega_{2} \in \Omega_{2} : (\omega_{1}, \omega_{2}) \in A \right\} & = \left\{ \omega_{2} \in \Omega_{2} : (\omega_{1}, \omega_{2}) \in \cup_{n=0}^{\infty}A_{n} \right\} \nonumber \\
& = \bigcup_{n=0}^{\infty}\left\{ \omega_{2} \in \Omega_{2} : (\omega_{1}, \omega_{2}) \in A_{n} \right\} \nonumber \\
& = \bigcup_{n=0}^{\infty}A_{1\omega_{1}}^{(n)}.
\label{1.1}
\end{align}
Similarly, for a fixed $\omega_{2} \in \Omega_{2}$, define $A_{2\omega_{2}}^{(n)} := \left\{ \omega_{1} \in \Omega_{1} : (\omega_{1}, \omega_{2} \in A
\right\}$. Then,
\begin{equation}
A_{2\omega_{2}} = \bigcup_{n=0}^{\infty}A_{2\omega_{2}}^{(n)}.
\label{1.2}
\end{equation}

\begin{subclaim}
$\left\{ A_{1\omega_{1}}^{(n)} \right\}_{n=0}^{\infty}$ and $\left\{ A_{2\omega_{2}}^{(n)} \right\}_{n=0}^{\infty}$ are pairwise disjoint sequences
for all $\omega_{1} \in \Omega_{1}$, $\omega_{2} \in \Omega_{2}$, respectively.
\end{subclaim}
\begin{subclaimproof}
By way of contradiction, suppose there exists $\omega_{1}' \in \Omega_{1}$ such that $\left\{ A_{1\omega_{1}'}^{(n)} \right\}_{n=0}^{\infty}$ is not pairwise disjoint. Then there exists $i,j \in
\mathbb{N}$, with $i \neq j$, such that $A_{1\omega_{1}'}^{(i)} \cap A_{1\omega_{1}'}^{(j)} \neq \emptyset$. Then there exists $\omega_{2}' \in
A_{1\omega_{1}'}^{(i)} \cap A_{1\omega_{1}'}^{(j)}$. But then $(\omega_{1}', \omega_{2}') \in A_{i}$ and $(\omega_{1}', \omega_{2}') \in A_{j}$. So
$A_{i}\cap A_{j} \neq \emptyset$. This is a contradiction. By the same argument, we see that $\left\{ A_{2\omega_{2}}^{(n)} \right\}_{n=0}^{\infty}$ must be
pairwise disjoint.
\end{subclaimproof}

Therefore, 
\begin{align*}
\mu_{12}(A) = \int_{\Omega_{1}}\mu_{2}(A_{1\omega_{1}})d\mu_{1}(\omega_{1}) & \stackrel{\eqref{1.1}}{=} \int_{\Omega_{1}}\mu_{2}\left(
\cup_{n=0}^{\infty}A_{1\omega_{1}}^{(n)} \right)d\mu_{1}(\omega_{1}) \\
\text{(subclaim 3.1 and since $\mu_{2}$ is a measure) } & = \int_{\Omega_{1}}\left[ \sum_{n=0}^{\infty}\mu_{2}\left( A_{1\omega_{1}}^{(n)}
\right) \right]d\mu_{1}(\omega_{1}) \\
\text{(Corollary 2.3.5) } & = \sum_{n=0}^{\infty}\int_{\Omega_{1}}\mu_{2}\left( A_{1\omega_{1}}^{(n)} \right)d\mu_{1}(\omega_{1}) \\
& = \sum_{n=0}^{\infty}\mu_{12}(A_{n}).
\end{align*}
Similarly, $\mu_{21}(A) = \sum_{n=0}^{\infty}\mu_{21}(A_{n})$.
\end{claimproof}

\end{Proof}



\newpage 
\section*{5.6}
Suppose $f$ is a non-negative measurable function over the $\sigma$-finite measure space $(\Omega, \mathcal{F}, \mu)$. Then 
\[ \int_{\Omega}fd\mu = \int_{[0,\infty)}\mu\left( \left\{ f\geq t \right\} \right)dm(t), \]
where $m$ is the Lebesgue measure.

\subsection*{Solution}
\begin{Proof}
Let $\bm{C} := \left\{ A_{1} \times A_{2} : A_{1} \in \mathcal{F}, A_{2} \in \mathcal{B}(\mathbb{R}_+) \right\}$. Let $\phi, \psi, g : \Omega\times
\mathbb{R}_{+} \rightarrow \mathbb{R}$ be defined by $\phi(\omega, t) := f(\omega)$, $\psi(\omega, t) := t$, and $g(\omega, t) := 
\mathbb{I}(\phi(\omega, t) \geq \psi(\omega, t))$ for all $(\omega, t) \in \Omega \times
\mathbb{R}_{+}$.

\begin{claim}
$g$ is $(\mathcal{F}\otimes \mathcal{B}(\mathbb{R}_{+}), \mathcal{B}(\mathbb{R}))$-measurable.
\end{claim}
\begin{claimproof}
\begin{subclaim}
$\phi$ is measurable.
\end{subclaim}
\begin{subclaimproof}
Let $A \in \mathcal{B}(\mathbb{R})$. Then 
\begin{align*}
\phi^{-1}[A] = \left\{ (\omega, t) \in \Omega\times \mathbb{R}_{+} : \phi(\omega, t) \in A \right\} & = \left\{ (\omega, t) \in \Omega\times
\mathbb{R}_{+} : f(\omega) \in A \wedge t \in \mathbb{R}_{+} \right\} \\
\text{(since $f$ is measurable) } & = f^{-1}[A] \times \mathbb{R}_{+} \in \bm{C} \subseteq \sigma\langle\bm{C}\rangle = \mathcal{F}\otimes
\mathcal{B}(\mathbb{R}_{+}).
\end{align*}
\end{subclaimproof}

\begin{subclaim}
$\psi$ is measurable.
\end{subclaim}
\begin{subclaimproof}
Let $A \in \mathcal{B}(\mathbb{R})$. Then 
\begin{align*}
\psi^{-1}[A] = \left\{ (\omega, t) \in \Omega \times \mathbb{R}_{+}  :\psi(\omega, t) \in A \right\} & = \left\{ 
(\omega, t) \in \Omega\times \mathbb{R}_{+} : \omega \in \Omega \wedge t \in A \right\} \\
& = \Omega \times [0,\infty) \in \bm{C} \subseteq \sigma\langle\bm{C}\rangle = \mathcal{F}\otimes\mathcal{B}(\mathbb{R}_{+}).
\end{align*}
\end{subclaimproof}

From subclaims 1.1 and 1.2 it follows that $\phi - \psi$ is measurable, and therefore $g = \mathbb{I}(\phi - \psi \geq 0)$ is measurable.
\end{claimproof}

By claim 1, we can integrate $g$. Thus, note that 
\begin{equation}
\int_{\mathbb{R}_{+}} g(\omega, t)dm(t) = \int_{\mathbb{R}_{+}}\mathbb{I}(t \leq f(\omega))dm(t) = m([0, f(\omega)]) = f(\omega),
\label{2.1}
\end{equation}
and 
\begin{equation}
\int_{\Omega}g(\omega, t)d\mu(\omega) = \int_{\Omega}\mathbb{I}(f(\omega) \geq t)d\mu(\omega) = \mu(\left\{ \omega \in \Omega : f(\omega) \geq t
\right\}).
\label{2.2}
\end{equation}
Now, clearly $g \geq 0$, so we can apply Tonelli's Theorem to \eqref{2.1} and \eqref{2.2}. Therefore 
\[ \int_{\mathbb{R}_{+}}\mu\left( \left\{ \omega \in \Omega : f(\omega) \geq t \right\} \right)dm(t) = \int_{\Omega}f(\omega)d\mu(\omega). \]
\end{Proof}


\newpage 
\section*{5.10}
Show that $\mathcal{I} := \int_{0}^{\infty}e^{-x^{2}/ 2}dm(x) = \sqrt{\frac{\pi}{2}}$.

\subsection*{Solution}
\begin{Proof}
By Tonelli's Theorem,
\begin{align*}
\mathcal{I}^{2} = \left( \int_{\mathbb{R}_{+}}e^{-x^{2}/2}dm(x) \right)\left( \int_{\mathbb{R}_{+}}e^{-y^{2}/2}dm(y) \right) 
& = \int_{\mathbb{R}_{+}}\int_{\mathbb{R}_{+}}e^{-(x^{2} + y^{2}) / 2}dm(x)dm(y) \\
& = \int_{\mathbb{R}_{+}\times \mathbb{R}_{+}}e^{-(x^{2} + y^{2})/2}d(m\times m).
\end{align*}
Now we will use the change of variables $x := r\cos \theta, y := r\sin \theta$, for $0 < r < \infty$ and $0 < \theta < \frac{\pi}{2}$. Thus 
\[ J = \det \begin{bmatrix}
\cos \theta & -r\sin \theta \\
\sin \theta & r\cos \theta \\ 
\end{bmatrix} = r\cos^{2}\theta + r\sin^{2}\theta = r. \]
Hence, by Theorem 4.4.6,
\begin{align*}
\int_{\mathbb{R}_{+} \times \mathbb{R}_{+}}e^{-(x^{2} + y^{2})/2}d(m\times m) & = \int_{(0,\pi/2)\times \mathbb{R}_{+}} re^{-r^{2}/2}d(m\times m) \\
\text{(Tonelli's) } & = \int_{(0,\pi/2)}\int_{\mathbb{R}_{+}}re^{-r^{2}/2}dm(r)dm(\theta)  \\
& = \int_{0}^{\pi/2}1 dm(\theta) \\
& = \frac{\pi}{2}.
\end{align*}
Therefore $\mathcal{I} = \sqrt{\frac{\pi}{2}}$.

\end{Proof}



\newpage 
\section*{5.11}
Suppose $\mu$ is a finite measure of $(\mathbb{R}, \mathcal{B}(\mathbb{R}))$ and $f,g : \mathbb{R} \rightarrow \mathbb{R}_{+}$ are nondecreasing and
$\mathcal{B}(\mathbb{R})$ measurable. Show that 
\[ \mu(\mathbb{R}) \int fgd\mu \geq \left( \int fd\mu \right)\left( \int gd\mu \right). \]

\subsection*{Solution}
\begin{Proof}
Let $h(x_{1}, x_{2}) := [f(x_{1} - f(x_{2})][g(x_{1}) - g(x_{2})]$, for all $(x_{1}, x_{2}) \in \mathbb{R}^{2}$.

\begin{claim}
$h(\cdot, \cdot) \geq 0$.
\end{claim}
\begin{claimproof}
Let $(x_{1}, x_{2}) \in \mathbb{R}^{2}$. If $x_{1} < x_{2}$, then $f(x_{1}) \leq f(x_{2})$ and $g(x_{1}) \leq g(x_{2})$. Therefore $f(x_{1}) -
f(x_{2}), g(x_{1}) - g(x_{2}) \leq 0$. Thus $h(x_{1}, x_{2}) \geq 0$. On the other hand, if $x_{1} \geq x_{2}$, then $f(x_{1}) - f(x_{2}), g(x_{1}) -
g(x_{2}) \geq 0$. So $h(x_{1}, x_{2}) \geq 0$.
\end{claimproof}

By claim 1 and Tonelli's Theorem,
\begin{align*}
0 & \leq \int_{\mathbb{R}^{2}}h(x_{1}, x_{2})d(\mu\times\mu) \\
& = \int_{\mathbb{R}^{2}}\left[ f(x_{1})g(x_{1}) - f(x_{1})g(x_{2}) - f(x_{2})g(x_{1}) + f(x_{2})g(x_{2}) \right]d(\mu(x_{1})\times\mu(x_{2})) \\
& = \int_{\mathbb{R}}\int_{\mathbb{R}} \left[ f(x_{1})g(x_{1}) - f(x_{1})g(x_{2}) - f(x_{2})g(x_{1}) + f(x_{2})g(x_{2})
\right] d\mu(x_{1})d\mu(x_{2}) \\
& = 2\mu(\mathbb{R})\int_{\mathbb{R}}fgd\mu - 2\left( \int_{\mathbb{R}}fd\mu \right)\left( \int_{\mathbb{R}}gd\mu \right)
\end{align*}
Therefore 
\[ \left( \int_{\mathbb{R}}fd\mu \right)\left( \int_{\mathbb{R}}gd\mu \right) \leq \mu(\mathbb{R})\int_{\mathbb{R}}fgd\mu. \]
\end{Proof}


\newpage 
\section*{Problem A}
Consider the following measure spaces
\[ (\Omega_{i} := \mathbb{N}, \mathcal{F}_{i} := \mathcal{P}(\mathbb{N}), \mu_{i} := \text{ counting measure}),\  i = 1, 2. \]
Define $f : (\Omega_{1} \times \Omega_{2}) \rightarrow \mathbb{R}$ by 
\[ f(\omega_{1}, \omega_{2}) := \mathbb{I}_{\{(n,n) : n\in \mathbb{N}\}}(\omega_{1}, \omega_{2}) - \mathbb{I}_{\{(n,n+1) : n \in
\mathbb{N}\}}(\omega_{1}, \omega_{2}). \]
Then,
\begin{enumerate}[label=(\roman*)]
\item Prove $\mathcal{F}_{1} \otimes \mathcal{F}_{2} = \mathcal{P}(\mathbb{N} \times \mathbb{N})$.
\item Is $f$ an $(\mathcal{F}_{1} \otimes \mathcal{F}_{2})$ measurable function?
\item Is $f$ integrable with respect to $\mu_{1} \times \mu_{2}$?
\item Compute the two iterated integrals:
\[ \int_{\Omega_{1}}\int_{\Omega_{2}}f(\omega_{1},\omega_{2})d\mu_{2}(\omega_{2})d\mu(\omega_{1}),\ \ 
\int_{\Omega_{2}}\int_{\Omega_{1}}f(\omega_{1},\omega_{2})d\mu_{1}(\omega_{1})d\mu_{2}(\omega_{2}). \]
\end{enumerate}

\subsection*{Solution}
\begin{enumerate}[label=(\roman*)]
\item Let $C := \left\{ A_{1} \times A_{2} : A_{1} \in \mathcal{F}_{1}, A_{2} \in \mathcal{F}_{2} \right\}$. 
Clearly $\mathcal{P}(\mathbb{N} \times \mathbb{N}) \supseteq \mathcal{F}_{1} \otimes \mathcal{F}_{2}$. Now let $A \in \mathcal{P}(\mathbb{N}
\times \mathbb{N})$. Let $A_{1\omega_{1}}$ be defined as in Definition 5.1.2. Then $A = \cup_{\omega_{1} \in \mathbb{N}}A_{1\omega_{1}}$. Since
$A_{1\omega_{1}} \in \mathcal{C}$ for all $\omega_{1} \in \mathbb{N}$, and $\mathbb{N}$ is countable, $A \in \sigma\langle\mathcal{C}\rangle =
\mathcal{F}_{1} \otimes \mathcal{F}_{2}$. Therefore $\mathcal{P}(\mathbb{N} \times \mathbb{N}) \subseteq \mathcal{F}_{1} \otimes \mathcal{F}_{2}$.

\item $f$ is clearly measurable since for all $E \in \mathcal{B}(\mathbb{R})$, $f^{-1}[E] \in \mathcal{P}(\mathbb{N} \times \mathbb{N}) =
\mathcal{F}_{1} \otimes \mathcal{F}_{2}$.

\item Since $|f(\omega_{1}, \omega_{2})| \geq 0$, we can apply Tonelli's Theorem. First note that for each $\omega_{1}' \in \Omega_{1}$,
\[ \int_{\Omega_{2}}|f(\omega_{1}', \omega_{2})|d\mu_{2}(\omega_{2}) = \sum_{\omega_{2}=1}^{\infty}|\mathbb{I}_{\{(\omega_{1}',
\omega_{1}')\}}(\omega_{1}', \omega_{2}) - \mathbb{I}_{\{(\omega_{1}', \omega_{1}' + 1)\}}(\omega_{1}', \omega_{2})| = 2. \]
Thus, by Tonelli's Theorem,
\[ \int_{\Omega_{1}\times \Omega_{2}}|f(\omega_{1}, \omega_{2})|d(\mu_{1}\times\mu_{2}) = \int_{\Omega_{1}}2 d\mu_{1} = \infty. \]
So $f$ is not integrable.

\item For each $\omega_{1}' \in \Omega_{1}$,
\[ \int_{\Omega_{2}}f(\omega_{1}', \omega_{2})d\mu_{2}(\omega_{2}) = \sum_{\omega_{2}=1}^{\infty}\big[\mathbb{I}_{\{(\omega_{1}',
\omega_{1}')\}}(\omega_{1}', \omega_{2}) - \mathbb{I}_{\{(\omega_{1}', \omega_{1}' + 1)\}}(\omega_{1}', \omega_{2})\big] = 1 - 1 = 0. \]
So 
\[ \int_{\Omega_{1}}\int_{\Omega_{2}}f(\omega_{1},\omega_{2})d\mu_{2}(\omega_{2})d\mu(\omega_{1}) = \int_{\Omega_{1}}0d\mu_{1} = 0. \]
On the other hand, if $\omega_{2}' = 1$,
\[ \int_{\Omega_{1}}f(\omega_{1}, \omega_{2}')d\mu_{1}(\omega_{1}) = \sum_{\omega_{1} = 1}^{\infty}\left[\mathbb{I}_{\{(1,1)\}}(\omega_{1}, 1) -
\mathbb{I}_{\{(0,1)\}}(\omega_{1}, 1)\right] = 1. \]
But if $\omega_{2}' > 1$,
\[ \int_{\Omega_{1}}f(\omega_{1}, \omega_{2}')d\mu_{1}(\omega_{1}) = \sum_{\omega_{1} = 1}^{\infty}\left[\mathbb{I}_{\{(\omega_{2}',\omega_{2}')\}}(\omega_{1},
\omega_{2}') - \mathbb{I}_{\{(\omega_{2}' - 1, \omega_{2}')\}}(\omega_{1}, \omega_{2}')\right] = 1 - 1 = 0. \]
Therefore, 
\[ \int_{\Omega_{2}}\int_{\Omega_{1}}f(\omega_{1},\omega_{2})d\mu_{1}(\omega_{1})d\mu_{2}(\omega_{2}) = \int_{\{1\}}1d\mu_{2} +
\int_{\mathbb{N}\setminus \{1\}}0d\mu_{2} = 1. \]
\end{enumerate}


\end{document}

