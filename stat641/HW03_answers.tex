\documentclass[12pt]{article}
\usepackage{amsmath}
\usepackage{amsfonts}
\usepackage{parskip}
\usepackage{amsthm}
\usepackage{thmtools}
\usepackage[headheight=15pt]{geometry}
\geometry{a4paper, left=20mm, right=20mm, top=30mm, bottom=30mm}
\usepackage{graphicx}
\usepackage{bm} % for bold font in math mode - command is \bm{text}
\usepackage{enumitem}
\usepackage{fancyhdr}
\usepackage{amssymb} % for stacked arrows and other shit
\pagestyle{fancy}

\declaretheoremstyle[headfont=\normalfont]{normal}
\declaretheorem[style=normal]{Theorem}
\declaretheorem[style=normal]{Proposition}
\declaretheorem[style=normal]{Lemma}
\newenvironment{claimproof}[1]{\par\noindent\underline{Claim Proof:}\space#1}{\hfill $\blacksquare$}

\title{STAT 641: HW 3}
\author{Evan ``Pete'' Walsh}
\makeatletter
\let\runauthor\@author
\let\runtitle\@title
\makeatother
\lhead{\runauthor}
\chead{\runtitle}
\rhead{\thepage}
\cfoot{}

\begin{document}
\maketitle

\section*{1.3}
Let $\Omega$ be a nonempty set and let 
\[ \mathcal{F}_{6} \equiv \mathcal{F}  = \left\{ A \subseteq \Omega : A\text{ is countable or }A^{c}\text{ is countable} \right\}. \]
(a) Show that $\mathcal{F}$ is a $\sigma$-algebra.

(b) Define $\mu$ on $\mathcal{F}$ as $\mu(A) = 1$ if $A$ is uncountable and $\mu(A) = 0$ is $A$ is countable. Is $\mu$ a measure of $\mathcal{F}$?

\section*{Solution}

{\bf (a)}
\begin{proof}
The empty set is countable, so $\emptyset^{c} = \Omega \in \mathcal{F}$. Now let $A \in \mathcal{F}$. If $A$ is countable, then $(A^{c})^{c}$ is
countable, so $A^{c} \in \mathcal{F}$. If $A^{c}$ is countable, then $A^{c} \in \mathcal{F}$. Hence $\mathcal{F}$ is closed under complementation. It
remains to verify that $\mathcal{F}$ is closed under countable unions. So, let $\left\{ A_{n} \right\}_{n\geq 1} \subseteq \mathcal{F}$. Then for each $n \in
\mathbb{N}$, either $A$ is countable or $A^{c}$ is countable. Let 
\[ I_{0} = \left\{ n : A_{n} \text{ is countable} \right\} \qquad \text{and} \qquad I_{1} = \left\{ n : A_{n}^{c}\text{ is countable} \right\}. \]
Note that clearly $I_{0} \cup I_{1} = \mathbb{N}$. Now, the intersection of countable sets is countable, so 
\[ \bigcap_{k \in I_{1}}A_{k}^{c} \in \mathcal{F}. \]
Therefore 
\[ \bigcap_{k \in I_{1}}A_{k}^{c} \supseteq \left( \bigcup_{k \in I_{0}}A_{k} \right)^{c} \cap \left( \bigcap_{k\in I_{1}}A_{k}^{c} \right) =
\bigcap_{k \in \mathbb{N}}A_{k}^{c} \in \mathcal{F}, \]
since the subset of a countable set is countable. Thus, by closure under complementation, 
\[ \left( \bigcap_{k\in\mathbb{N}}A_{k}^{c} \right)^{c} = \bigcup_{k\in\mathbb{N}}A_{k} \in \mathcal{F}. \]
\end{proof}

{\bf (b)} $\mu$ is a measure.

By definition, $\mu(\cdot) \in \left\{ 0, 1 \right\} \subset [0, \infty]$. Further, since $\emptyset$ is countable, $\mu(\emptyset) = 0$. Then it just
remains to show that countable additivity holds. Thus, let $\left\{ A_{n} \right\}_{n\in\mathbb{N}}$ be a family of disjoint sets where $A_{n} \in
\mathcal{F}$ for each $n \in \mathbb{N}$.

\underline{Claim:} At most one $A_{n}$ is uncountable.
\begin{claimproof}
By way of contradiction, assume not. That is, assume at least two sets in the family $\left\{ A_{n} \right\}_{n\in\mathbb{N}}$ are uncountable, call
them $A_{1}$ and $A_{2}$. However, since $A_{1} \in \mathcal{F}$, $A_{1}^{c}$ is countable. Thus, since $A_{1} \cap A_{2} = \emptyset$, $A_{1}^{c}
\supseteq A_{2}$. Therefore $A_{2}$ must be countable. This is a contradiction.
\end{claimproof}

By the above claim, there are two cases to consider.

{\bf Case 1:} There is one $A_{k^{*}} \in \left\{ A_{n} \right\}_{n\in\mathbb{N}}$ that is uncountable. Then
\[ \sum_{n=1}^{\infty}\mu(A_{n}) = \mu(A_{k^{*}}) + \sum_{n \neq k^{*}}\mu(A_{n}) = 1 + 0 = 1. \]
Further $\cup_{n\in\mathbb{N}}A_{n}$ is uncountable since $A_{k^{*}}$ is uncountable, so 
\[ \mu^{*}\left( \bigcup_{n\in\mathbb{N}}A_{n} \right) = 1. \]


{\bf Case 2:} Every $A_{n}$ is countable for each $n \in \mathbb{N}$. Thus, since the countable union of countable sets is countable,
$\cup_{n\in\mathbb{N}}A_{n}$ is countable. Hence 
\[ \mu\left( \bigcup_{n\in\mathbb{N}}A_{n} \right) = 0. \]
Further, since each $A_{n}$ is countable,
\[ \sum_{n=1}^{\infty}\mu(A_{n}) = \sum_{n=1}^{\infty}0 = 0. \]

In each case, countable additivity is satisfied. Therefore $\mu$ is a measure on $\mathcal{F}$.




\newpage
\section*{1.8}

Let $\Omega = \left\{ 1, 2, \dots \right\} = \mathbb{N}$ and $A_{i} = \left\{ j : j \in \mathbb{N}, j \geq i \right\}, i \in \mathbb{N}$. Show that
$\sigma \langle \mathcal{A} \rangle = \mathcal{P}(\Omega)$, where $\mathcal{A} = \left\{ A_{i} : i \in \mathbb{N} \right\}$.

\section*{Solution:}

\begin{proof}
Since $\mathcal{P}(\Omega)$ is a $\sigma$-algebra that contains $\mathcal{A}$, 
\begin{equation}
\sigma\langle \mathcal{A}\rangle \subseteq \mathcal{P}(\Omega).
\label{eqn:1.8a}
\end{equation}
Therefore it remains to show that $\mathcal{P}(\Omega) \subseteq \sigma \langle \mathcal{A} \rangle$. Let $i \in \mathcal{N}$. Since
$\sigma\langle\mathcal{A}\rangle$ is closed under finite unions and complements, 
\[ i = \left( A_{i}^{c} \cup A_{i+1} \right)^{c} \in \sigma\langle\mathcal{A}\rangle. \]
So every singleton set is in $\sigma\langle\mathcal{A}\rangle$.
Now, let $\emptyset \neq B \subseteq \Omega$. $B$ must be countable since it is the subset of a countable set. Thus, 
\[ B = \bigcup_{i\in B}\{i\} \in \sigma\langle\mathcal{A}\rangle, \]
since $\sigma\langle\mathcal{A}\rangle$ is closed under countable unions. Hence every subset of $\Omega$ is in $\sigma\langle\mathcal{A}\rangle$, so
\begin{equation}
\mathcal{P}(\Omega) \subseteq \sigma\langle\mathcal{A}\rangle.
\label{eqn:1.8b}
\end{equation}
So by \ref{eqn:1.8a} and \ref{eqn:1.8b}, we have equality.
\end{proof}

\newpage
\section*{1.10}

Show that in Example 1.1.6, $\mathcal{O}_{j} \subseteq \sigma\langle\mathcal{O}_{i}\rangle$ for all $1 \leq i,j \leq 4$.

\section*{Solution}

\begin{proof}
Clearly $\mathcal{O}_{j} \subseteq \sigma \langle \mathcal{O}_{j}\rangle$ for all $1 \leq j \leq 4$. Also, by definition,
\[ \mathcal{O}_{3} \subseteq \mathcal{O}_{1} \subseteq \sigma\langle \mathcal{O}_{1}\rangle \qquad \text{and} \qquad \mathcal{O}_{4} \subseteq
\mathcal{O}_{2} \subseteq \sigma\langle \mathcal{O}_{2}\rangle. \]
This implies that 
\begin{equation}
\sigma\langle\mathcal{O}_{3}\rangle \subseteq \sigma\langle\mathcal{O}_{1}\rangle \qquad \text{and} \qquad \sigma\langle\mathcal{O}_{4}\rangle
\subseteq \sigma\langle \mathcal{O}_{2}\rangle. 
\label{eqn:10.1}
\end{equation}
We will proceed by showing that $\sigma\langle\mathcal{O}_{j}\rangle = \sigma\langle\mathcal{O}_{i}\rangle$ for all $1 \leq i,j\leq 4$.

{\bf Claim 1:} $\mathcal{O}_{2} \subseteq \sigma\langle\mathcal{O}_{3}\rangle$.

\begin{claimproof}
Let $x_{1}, x_{2} \in \mathbb{R}$. We need to show that $(-\infty, x_{1}) \times (-\infty, x_{2}) \in \sigma\langle\mathcal{O}_{3}\rangle$. Without
loss of generality, assume $x_{1}, x_{2} \geq 0$. Now, for each $n \in \mathbb{N}$, define $a_{1n}, a_{2n} \in \mathbb{Q}$ such that $x - 1/n < a_{1n}
\leq x_{1}$, and $x_{2} - 1/n < a_{2n} \leq x_{2}$. It is possible to define $a_{1n}, a_{2n}$ this way since $\mathbb{Q}$ is dense in $\mathbb{R}$.
Then, let
\[ A_{n} = (-n, a_{1n}) \times (-n, a_{2n}) \in \mathcal{O}_{3} \text{ for all } n \in \mathbb{N}. \]
Hence,
\[ \bigcup_{n=1}^{\infty}A_{n} = (-\infty, x_{1}) \times (-\infty, x_{2}) \in \sigma\langle \mathcal{O}_{3}\rangle. \]
So $\mathcal{O}_{2} \subseteq \sigma\langle\mathcal{O}_{3}\rangle$.
\end{claimproof}

It follows directly from claim 1 that 
\begin{equation}
\sigma\langle\mathcal{O}_{2}\rangle \subseteq \sigma\langle\mathcal{O}_{3}\rangle.
\label{eqn:10.2}
\end{equation}

{\bf Claim 2:} $\mathcal{O}_{1} \subseteq \sigma\langle\mathcal{O}_{4}\rangle$.

\begin{claimproof}
Let $a_{1}, b_{1}, a_{2}, b_{2} \in \mathbb{R}$ such that $a_{1}, b_{1}$ and $a_{2} < b_{2}$. We need to show that $(a_{1}, b_{1}) \times (a_{2},
b_{2}) \in \sigma\langle\mathcal{O}_{4}\rangle$. Now, for each $n \in \mathbb{N}$, define $c_{1n}, c_{2n} \in \mathbb{Q}$ such that $b_{1} - 1/n <
c_{1n} \leq b_{1}$ and $b_{2} - 1/n < c_{2n} \leq b_{2}$. Then define $C_{n} = (-\infty, c_{1n})\times (-\infty, c_{2n}) \in \mathcal{O}_{4}$ for each
$n \in \mathbb{N}$. Then,
\[ C \equiv \bigcup_{n=1}^{\infty}C_{n} = (-\infty, b_{1}) \times (-\infty, b_{2}) \in \sigma\langle\mathcal{O}_{4}\rangle. \]
Further, for each $n \in \mathbb{N}$, define $a_{1n}, a_{2n} \in \mathbb{Q}$ such that $a_{1} \leq a_{1n} < a_{1} + 1/n$ and $a_{2} \leq a_{2n} <
a_{2} + 1/n$. Also, let $b_{1}^{*}, b_{2}^{*} \in \mathbb{Q}$ such that $b_{1}^{*} > b_{1}$ and $b_{2}^{*} > b_{2}$. Then, let
\begin{align*}
& A_{n} = (-\infty, b_{1}^{*}) \times (-\infty, a_{2n}) \in \mathcal{O}_{4} \text{ and } \\
& B_{n} = (-\infty, a_{1n}) \times (-\infty, b_{2}^{*}) \in \mathcal{O}_{4}.
\end{align*}
Thus,
\begin{align*}
& A = \bigcap_{n=1}^{\infty}A_{n} = (-\infty, b_{1}^{*}) \times (-\infty, a_{2}] \in \sigma \langle \mathcal{O}_{4}\rangle, \text{ and } \\
& B = \bigcap_{n=1}^{\infty}B_{n} = (-\infty, a_{1}] \times (-\infty, b_{2}^{*}) \in \sigma \langle \mathcal{O}_{4}\rangle.
\end{align*}
Hence 
\[ C \cap (A\cup B)^{c} = C \setminus (A\cup B) = (a_{1}, b_{1}) \times (a_{2}, b_{2}) \in \sigma\langle\mathcal{O}_{4}\rangle. \]
So $\mathcal{O}_{1} \subseteq \sigma\langle\mathcal{O}_{4}\rangle$.
\end{claimproof}

It follows from claim 2 that 
\begin{equation}
\sigma\langle\mathcal{O}_{1}\rangle \subseteq \sigma\langle\mathcal{O}_{4}\rangle.
\label{eqn:10.3}
\end{equation}
Therefore, by equations \ref{eqn:10.1}, \ref{eqn:10.2}, and \ref{eqn:10.3}, 
\[ \sigma\langle\mathcal{O}_{3}\rangle \subseteq \sigma\langle\mathcal{O}_{1}\rangle \subseteq \sigma\langle\mathcal{O}_{4}\rangle \subseteq
\sigma\langle\mathcal{O}_{2}\rangle \subseteq \sigma\langle\mathcal{O}_{3}\rangle. \]
So $\sigma\langle\mathcal{O}_{j}\rangle = \sigma\langle\mathcal{O}_{i}\rangle$ for each $1 \leq i,j\leq 4$. Therefore $\mathcal{O}_{j} \subseteq
\sigma\langle\mathcal{O}_{i}\rangle$ for each $1 \leq i,j \leq 4$.
\end{proof}

\newpage
\section*{1.18}

Let $\Omega \equiv \mathbb{N}$, $\mathcal{F} = \mathcal{P}(\Omega)$, and $A_{n} = \left\{ j : j \in \mathbb{N}, j \geq n \right\}$, for each $n \in
\mathbb{N}$. Let $\mu$ be the counting measure on $(\Omega, \mathcal{F})$. Verify that $\lim_{n\rightarrow\infty}\mu(A_{n}) \neq \mu\left( \cap_{n\geq
1}A_{n} \right)$.

\section*{Solution}

\begin{proof}
Since $\mu(A_{n}) = |A_{n}| = \infty$ for each $n \in \mathbb{N}$, $\lim_{n\rightarrow\infty}\mu(A_{n}) = \infty$. However, we will show that
$\cap_{n\geq 1}A_{n} = \emptyset$. By way of contradiction, assume that $\cap_{n\geq 1}A_{n} \neq \emptyset$. Then there exists $k^{*} \in
\cap_{n\geq 1}A_{n}$. But $k^{*} \notin A_{k^{*}+1}$. This is a contradiction. Thus $\cap_{n\geq 1}A_{n} = \emptyset$. Hence, by definition of measure,
\[ \mu\left( \bigcap_{n\geq 1}A_{n} \right) = \mu(\emptyset) = 0, \]
So,
\[ \lim_{n\rightarrow \infty}\mu(A_{n}) \neq \mu\left( \bigcap_{n\geq 1}A_{n} \right). \]
\end{proof}



\newpage
\section*{1.19}

Let $\Omega$ be nonempty and $\mathcal{C} \subseteq \mathcal{P}(\Omega)$ be a semialgebra. Let 
\[ \mathcal{A}(\mathcal{C}) = \left\{ A : A = \cup_{i=1}^{k}B_{i} : B_{i} \in \mathcal{C}, i = 1, \dots, k \in \mathbb{N} \right\}. \]
(a) Show that $\mathcal{A}(\mathcal{C})$ is the smallest algebra containing $\mathcal{C}$.

(b) Show that $\sigma\langle\mathcal{C}\rangle = \sigma\langle\mathcal{A}(\mathcal{C})\rangle$.

\section*{Solution}

{\bf (a)} 
\begin{proof}
First we will verify that $\mathcal{A}(\mathcal{C})$ is, in fact, an algebra. Well, it follows from the definition of semialgebra that $\emptyset \in
\mathcal{C}$. Therefore $\Omega = (\emptyset)^{c} = \cup_{i=1}^{k}B_{i}$, where $\left\{ B_{i} \right\}_{i=1}^{k}$, is a disjoint collection of sets
in $\mathcal{C}$. Thus $\Omega \in \mathcal{A}(\mathcal{C})$ by definition. Now, let $X, Y \in \mathcal{A}(\mathcal{C})$. Then $X =
\cup_{i=1}^{m}A_{i}$ and $Y = \cup_{i=1}^{n}B_{i}$ where $A_{1}, \dots, A_{m} \in \mathcal{C}$ and $B_{1}, \dots, B_{n} \in \mathcal{C}$. 
We will show that $\mathcal{A}(\mathcal{C})$ is closed under finite unions and use that to show closure under finite intersections. We will then use
closure under finite intersections to show closure under complementation. Closer under finite unions is trivial, for
\[ X\cup Y = \left( \cup_{i=1}^{m}A_{i} \right) \cap \left( \cup_{j=1}^{n}B_{j} \right) = A_{1} \cup A_{2} \cup \dots \cup A_{m} \cup B_{1} \cup B_{2}
\cup \dots \cup B_{n} \in \mathcal{A}(\mathcal{C}), \]
by definition. Now,
\begin{align*}
X \cap Y & = \left( \cup_{i=1}^{m}A_{i} \right) \cap \left( \cup_{j=1}^{n}B_{j} \right) \\
& = \cup_{i=1}^{m}A_{i} \cap \left( \cup_{j=1}^{n}B_{j} \right) \\
& = \cup_{i=1}^{m}\cup_{j=1}^{n}A_{i}\cap B_{j} \in \mathcal{A}(\mathcal{C}),
\end{align*}
since $A_{i}\cap B_{j} \in \mathcal{C}$ for each $1\leq i \leq m, 1\leq j \leq n$ and $\mathcal{A}(\mathcal{C})$ is closed under finite unions. So
$\mathcal{A}(\mathcal{C})$ is closed under finite intersections. It just remains to show that $\mathcal{A}(\mathcal{C})$ is closed under
complementation to verify that it is an algebra. Well, it follows from the definition of $\mathcal{A}(\mathcal{C})$ that $A_{i}^{c} \in \mathcal{A}(\mathcal{C})$ since $A_{i} \in \mathcal{C}$ for each $1 \leq i \leq
m$. Thus,
\[ X^{c} = \left( \cup_{i=1}^{m}A_{i} \right)^{c} = \cap_{i=1}^{m}A_{i}^{c} \in \mathcal{A}(\mathcal{C}), \]
since $\mathcal{A}(\mathcal{C})$ is closed under finite intersections. Hence $\mathcal{A}(\mathcal{C})$ is indeed an algebra.

Now let $\mathcal{F} \supseteq \mathcal{C}$ be an algebra, and let $A \in \mathcal{A}(\mathcal{C})$. Then $A = \cup_{i=1}^{n}A_{i}$, where $A_{1},
\dots, A_{n} \in \mathcal{C}$, for each $n \in \mathbb{N}$. But therefore $A \in \mathcal{F}$ since $\mathcal{F}$ is closed under finite unions. So
$\mathcal{A}(\mathcal{C}) \subseteq \mathcal{F}$. Since $\mathcal{F} \supseteq \mathcal{C}$ was arbitrary, $\mathcal{A}(\mathcal{C})$ is the smallest
algebra containing $\mathcal{C}$.
\end{proof}

{\bf (b)}

\begin{proof}
Since $\mathcal{C} \subseteq \mathcal{A}(\mathcal{C})$, $\sigma\langle\mathcal{C}\rangle \subseteq \sigma\langle\mathcal{A}(\mathcal{C})\rangle$. Now,
let $A \in \mathcal{A}(\mathcal{C})$. Then $A = \cup_{i=1}^{n}A_{i}$, where $A_{i} \in \mathcal{C}$ for each $1\leq i \leq n \in \mathbb{N}$. But then
$A \in \sigma\langle\mathcal{C}\rangle$ since $\sigma\langle\mathcal{C}\rangle$ is closed under finite unions. Therefore $\mathcal{A}(\mathcal{C})
\subseteq \sigma\langle\mathcal{C}\rangle$. Thus $\sigma\langle\mathcal{A}(\mathcal{C})\rangle \subseteq \sigma\langle\mathcal{C}\rangle$. So 
\[ \sigma\langle\mathcal{C}\rangle = \sigma\langle\mathcal{A}(\mathcal{C})\rangle. \]
\end{proof}






\newpage
\section*{1.20}

Let $\mu$ be a measure on a semialgebra $\mathcal{C}$. Let $A \in \mathcal{P}(\Omega)$. Define
\[ \mu^{*}(A) = \inf\left\{ \sum_{n=1}^{\infty}\mu(A_{n}) : \left\{ A_{n} \right\}_{n\geq 1} \subseteq \mathcal{C}, A \subseteq \cup_{n\geq 1}A_{n}
\right\}. \]
Verify (3.4) - (3.6).

\section*{Solution}

(3.4) $\mu^{*}(\emptyset) = 0$.

\begin{proof}
First note that 
\[ \mu^{*}(\emptyset) = \inf\left\{ \sum_{n=1}^{\infty}\mu(A_{n}) : \left\{ A_{n} \right\}_{n\geq 1}\subseteq \mathcal{C}, A\subseteq\cup_{n\geq
1}A_{n} \right\} \geq 0 \]
since $\mu$ only takes on values in $[0,\infty]$.
Now, it follows from the definition of semialgebra that $\emptyset \in \mathcal{C}$. Thus, let $A_{n} = \emptyset$ for each $n \in \mathbb{N}$. Then
$\emptyset \subseteq \cup_{n\geq 1}A_{n} = \emptyset$ and $\sum_{n=1}^{\infty}\mu(A_{n}) = \sum_{n=1}^{\infty}\mu(\emptyset) = 0$.
Therefore 
\[ \mu^{*}(\emptyset) = \inf\left\{ \sum_{n=1}^{\infty}\mu(A_{n}) : \left\{ A_{n} \right\}_{n\geq 1}\subseteq \mathcal{C}, A\subseteq\cup_{n\geq 1}A_{n} \right\} \leq 0,
\]
so $\mu^{*}(\emptyset) = 0$.
\end{proof}

(3.5) $A \subseteq B \Rightarrow \mu^{*}(A) \leq \mu^{*}(B)$.

\begin{proof}
Let $A, B \in \mathcal{P}(\Omega)$. Let 
\begin{align*}
S_{A} & = \left\{ \sum_{n=1}^{\infty}\mu(A_{n}) : \left\{ A_{n} \right\}_{n\geq 1}\subseteq \mathcal{C}, A\subseteq\cup_{n\geq 1}A_{n} \right\}, \text{
and } \\
S_{B} & = \left\{ \sum_{n=1}^{\infty}\mu(A_{n}) : \left\{ A_{n} \right\}_{n\geq 1}\subseteq \mathcal{C}, B\subseteq\cup_{n\geq 1}A_{n} \right\}.
\end{align*}
Now, let $\left\{ B_{n} \right\}_{n\geq 1} \subseteq \mathcal{C}$ such that $A\subseteq B\subseteq \cup_{n\geq 1}B_{n}$. Then
$\sum_{n=1}^{\infty}\mu(B_{n}) \in S_{A}$. Hence $S_{B} \subseteq S_{A}$. Therefore 
\[ \inf S_{A} \leq \inf S_{B}. \]
Thus $\mu^{*}(A) \leq \mu^{*}(B)$.
\end{proof}

(3.6) Let $\left\{ A_{n} \right\}_{n\geq 1} \subseteq \mathcal{P}(\Omega)$. Then 
\[ \mu^{*}\left( \cup_{n\geq 1}A_{n} \right) \leq \sum_{n=1}^{\infty}\mu^{*}(A_{n}). \]

\begin{proof}
Let $\left\{ A_{n} \right\}_{n\geq 1} \subseteq \mathcal{P}(\Omega)$. If $\mu^{*}(A_{n}) = \infty$ for any $n \in \mathbb{N}$, then the inequality
holds trivially. Thus, assume that $\mu^{*}(A_{n}) < \infty$ for each $n \in \mathbb{N}$. Let $\epsilon > 0$. For each $n \in \mathbb{N}$, choose a
cover $\left\{ A_{nj} \right\}_{j \geq 1} \subseteq \mathcal{C}$ such that $A_{n}\subseteq \cup_{j\geq 1}A_{nj}$ and 
\begin{equation}
\sum_{j=1}^{\infty}\mu(A_{nj}) < \mu^{*}(A_{n}) + \epsilon / 2^{n}. 
\label{eqn:1.20a}
\end{equation}
Therefore $\left\{ A_{nj} \right\}_{n,j\geq 1}\subseteq \mathcal{C}$ is a cover of $\cup_{n\geq 1}A_{n}$, so by definition of outer measure,
\begin{equation}
\mu^{*}\left( \bigcup_{n\geq 1} A_{n}\right) \leq \sum_{n,j\geq 1}\mu(A_{nj})
\label{eqn:1.20b}
\end{equation}
Further,
\begin{equation}
\sum_{n,j\geq 1}\mu(A_{nj}) = \sum_{n=1}^{\infty}\sum_{j=1}^{\infty}\mu(A_{nj}) \stackrel{\ref{eqn:1.20a}}{<} \sum_{n=1}^{\infty}\left( \mu^{*}(A_{n}) + \epsilon / 2^{n} \right)
 = \sum_{n=1}^{\infty}\mu^{*}(A_{n}) + \epsilon.
\label{eqn:1.20c}
\end{equation}
(Note that we don't have to worry about the order of terms in the summation since all terms are positive)

So by \ref{eqn:1.20b} and \ref{eqn:1.20c}, 
\[ \mu^{*}\left( \bigcup_{n\geq 1}A_{n} \right) \leq \sum_{n=1}^{\infty}\mu^{*}(A_{n}) + \epsilon. \]
Since the inequality holds for all $\epsilon > 0$, 
\[ \mu^{*}\left( \bigcup_{n\geq 1}A_{n} \right) \leq \sum_{n=1}^{\infty}\mu^{*}(A_{n}). \]
\end{proof}


\newpage
\section*{1.24}

Let $\mathcal{C}_{2} = \left\{ I_{1} \times I_{2} : I_{1}, I_{2} \in \mathcal{C} \right\}$, where 
\[ C \equiv \left\{ (a,b] : -\infty \leq a \leq b < \infty \right\} \cup \left\{ (a,\infty) : -\infty  \leq a < \infty \right\}. \]
Verify that $C_{2}$ is a semialgebra.

\section*{Solution}

\begin{proof}
First we will verify the following claim.

{\bf Claim 1:} If $I_{1}, I_{2} \in \mathcal{C}$, then $I_{1}\cap I_{2} \in \mathcal{C}$.

\begin{claimproof}
There are three cases to consider.

Case 1: $I_{1} = (a_{1}, b_{1}], I_{2} = (a_{2}, b_{2}]$, where $-\infty \leq a_{1}\leq b_{1} < \infty$, and $-\infty \leq a_{2} \leq b_{2}] <
\infty$.

If $a_{1} = b_{1}$ or $a_{2} = b_{2}$, then $I_{1} = \emptyset$ or $I_{2} = \emptyset$, so $I_{1}\cap I_{2} = \emptyset \in \mathcal{C}$. Now assume
that $a_{1} < b_{1}, a_{2} < b_{2}$.
\begin{itemize}[label={},leftmargin=4mm, itemsep=1em, parsep=1em]
\item Case 1.1: If $a_{2} \leq a_{1} < b_{1} \leq b_{2}$, then $I_{1} \subseteq I_{2}$, so $I_{1} \cap I_{2} = I_{1} \in \mathcal{C}$, and vica-versa.

\item Case 1.2: If $a_{2} \leq a_{1} \leq b_{2} \leq b_{1}$, then $I_{1} \cap I_{2} = (a_{1}, b_{2}] \in \mathcal{C}$. Similarly, if $a_{1} \leq a_{2}
\leq b_{1} \leq b_{2}$, then $I_{1}\cap I_{2} = (a_{2},b_{1}] \in \mathcal{C}$.

\item Case 1.3: If $a_{1} < b_{1} < a_{2} < b_{2}$, or if $a_{2} < b_{2} < a_{1} < b_{1}$, then $I_{1} \cap I_{2} = \emptyset \in \mathcal{C}$. 
\end{itemize}

Case 2: Now let $I_{1} = (a_{1}, b_{1}]$, $I_{2} = (a_{2}, \infty)$, where $-\infty \leq a_{1} \leq b_{1} < \infty$, and $-\infty \leq a_{2} <
\infty$.

If $b_{1} < a_{2}$, then $I_{1}\cap I_{2} = \emptyset \in \mathcal{C}$. But if $a_{2} < b_{1}$, then 
\[ I_{1} \cap I_{2} = (\max\left\{ a_{1},a_{2} \right\}, b_{1}] \in \mathcal{C}. \]

Case 3: Let $I_{1} = (a, \infty), I_{2} = (a_{2}, \infty)$, where $-\infty \leq a_{1},a_{2} < \infty$. Then 
\[ I_{1} \cap I_{2} = (\max \left\{ a_{1},a_{2} \right\}, \infty) \in \mathcal{C}. \]
\end{claimproof} 

{\bf Claim 2:} If $A,B \in \mathcal{C}_{2}$, then $A\cap B \in \mathcal{C}_{2}$.

\begin{claimproof}
Let $A = I_{A1} \times I_{A2} \in \mathcal{C}_{2}, B = I_{B1}\times I_{B2} \in \mathcal{C}_{2}$. Then 
\[ A\cap B = (I_{A1} \times I_{A2}) \cap (I_{B1}\times I_{B2}) = (I_{A1}\cap I_{B1}) \times (I_{A2}\cap I_{B2}) \in \mathcal{C} \]
by claim 1.
\end{claimproof}

{\bf Claim 3:} If $A \in \mathcal{C}_{2}$, then $A^{c} = \cup_{i=1}^{k}B_{i}$, wher $B_{i} \in \mathcal{C}_{2}$ and $B_{i} \cap B_{j} = \emptyset$ for
all $i,j \in \left\{ 1, \dots, k \right\}$ with $i \neq j$.

\begin{claimproof}
There are 4 cases to consider.

Case 1: Let $A = (a_{1}, b_{1}] \times (a_{2}, b_{2}]$, where $-\infty \leq a_{1} \leq b_{1} < \infty$ and $-\infty \leq a_{2} \leq b_{2} < \infty$.
Then,
\begin{align*}
A^{c} = & (-\infty, a_{1}] \times (-\infty, a_{2}] \cup (a_{1},b_{1}] \times (-\infty, a_{2}]\ \cup \\
& (-\infty, a_{1}] \times (a_{2}, b_{2}] \cup (-\infty, b_{1}] \times (b_{2}, \infty)\ \cup \\
& (b_{1}, \infty) \times (-\infty, b_{2}] \cup (b_{1}, \infty) \times (b_{2}, \infty),
\end{align*}
which are all disjoint sets in $\mathcal{C}_{2}$.

Case 2: Let $A = (a_{1}, b_{1}] \times (a_{2}, \infty)$, where $-\infty \leq a_{1} \leq b_{1} < \infty$ and $-\infty \leq a_{2} < \infty$. Then 
\begin{align*}
A^{c} = & (-\infty, a_{1}] \times (-\infty, a_{2}] \cup (a_{1}, b_{1}] \times (-\infty, a_{2}]\ \cup \\
& (-\infty, a_{1}] \times (a_{2}, \infty) \cup (b_{1}, \infty) \times (a_{2}, \infty),
\end{align*}
which are all disjoint sets in $\mathcal{C}_{2}$.

Case 3: Let $A = (a_{1}, \infty) \times (a_{2}, b_{2}]$, where $-\infty \leq a_{1} < \infty$ and $-\infty \leq a_{2} \leq b_{2} < \infty$. Then the
result follows from case 2 by switching the coordinates.

Case 4: Let $A = (a_{1}, \infty) \times (a_{2}, \infty)$, where $-\infty \leq a_{1}, a_{2} < \infty$. Then
\begin{align*}
A^{c} = & (-\infty, a_{1}] \times (-\infty, a_{2}] \cup (a_{1},\infty) \times (-\infty, a_{2}]\ \cup \\
& (-\infty, a_{1}] \times (a_{2}, \infty),
\end{align*}
which are all disjoint sets in $\mathcal{C}_{2}$.
\end{claimproof}

Hence, by claims 2 and 3, $\mathcal{C}_{2}$ is a semialgebra.
\end{proof}


\newpage 
\section*{Additional Problem}

Let $F(x) = xI\left\{ 0 < x < 1 \right\} + I\left\{ x \geq 1 \right\}, x \in \mathbb{R}$.

\section*{Solution}

{\bf (i)} There are three cases to consider.

Case 1: $x \leq 0$.

Let $x < 0$. Then $F(x) = x(0) + 0 \equiv 0$. So $F(\cdot)$ s non-decreasing for all $x \leq 0$, and right-continuous for all $x < 0$. Further, 
\[ \lim_{y\downarrow 0}F(y) = \lim_{y\downarrow 0}y(1) = 0 = F(0), \]
so $F(\cdot)$ is right-continouos over all $x \leq 0$.

Case 2: $0 < x < 1$.

Let $0 < x < y < 1$. Then $F(x) = x(1) + 0 = x < y = y(1) + 0 = F(y)$. So $F(\cdot)$ is non-decreasing over $(0,1)$. Further, 
\[ \lim_{y\downarrow x}F(y) = \lim_{y\downarrow x}y(1) = x. \]
So $F(\cdot)$ is right-continuous over $(0,1)$.

Case 3: $x \geq 1$.

Let $x \geq 1$. Then $F(x) = x(0) + 1 \equiv 1$. So $F(\cdot)$ is non-decreasing and right-continuous for all $x \geq 1$.

{\bf (ii)} Using the same definitions as on page 25, let $C = \left\{ (a,b], (a, \infty) : -\infty \leq a \leq b < \infty \right\}$. Then
\begin{align*}
& \mu\left( (a,b] \right) = F(b+) - F(a+) = F(b) - F(a) \\
& \mu\left( (a, \infty) \right) = F(\infty) - F(a+) = 1 - F(a),
\end{align*}
since $F(\cdot)$ is right-continouos and $F(\infty) = \lim_{x\rightarrow\infty}F(x) = \lim_{x\rightarrow \infty} 1 = 1$.

{\bf (iii)} We will prove that $\sigma\langle C\rangle = \mathcal{B}(\mathbb{R})$.

\begin{proof}
First we will show that $\sigma \langle \mathcal{C}\rangle \subseteq \mathcal{B}(\mathbb{R})$. Let $A \in \mathcal{C}$. Then either (1) $A = (a,b]$ for $-\infty \leq a \leq b < \infty$, or (2) $A = (a, \infty)$ for $-\infty \leq a < \infty$.

Case 1: Let $A = (a,b]$ where $-\infty \leq a \leq b < \infty$. Then 
\[ A = \cap_{n=1}^{\infty}(a, ,b + 1/n) \in \mathcal{B}(\mathbb{R}). \]

Case 2: Let $A = (a, \infty)$ where $-\infty \leq a < \infty$. Then 
\[ A = \cup_{n=1}^{\infty}[a + 1/n, \infty) = \left( \bigcap_{n=1}^{\infty}(-\infty, a + 1/n) \right)^{c} \in \mathcal{B}(\mathbb{R}), \]
since $\left( \cap_{n=1}^{\infty}(-\infty, a + 1/n) \right)^{c} \in \sigma\langle \mathcal{O}_{2}\rangle = \mathcal{B}(\mathbb{R})$.

Therefore $C \subseteq \mathcal{B}(\mathbb{R})$, so $\sigma\langle \mathcal{C}\rangle \subseteq \mathcal{B}(\mathbb{R})$. We will now show that
$\mathcal{B}(\mathbb{R}) \subseteq \sigma \langle \mathcal{C}\rangle$ using the fact that $\mathcal{B}(\mathbb{R}) = \sigma \langle
\mathcal{O}_{1}\rangle$. So, let $A = (a,b) \in \mathcal{O}_{1}$, where $-\infty \leq a \leq b \leq \infty$. If $a = b$ then $(a,b) = \emptyset \in
\langle \mathcal{C} \rangle$. Also, if $b = \infty$, then $(a, \infty) \in \mathcal{C} \subseteq \sigma \langle \mathcal{C}\rangle$. If $b < \infty$,
then 
\[ A = \bigcup_{n=1}^{\infty}(a, b - 1/n] \in \sigma \langle \mathcal{C}\rangle. \]
Hence $\mathcal{O}_{1} \subseteq \sigma \langle \mathcal{C}\rangle$, so $\mathcal{B}(\mathbb{R}) = \sigma\langle \mathcal{O}_{1}\rangle \subseteq
\sigma\langle \mathcal{C} \rangle$.

Thus $\sigma\langle \mathcal{C} \rangle = \mathcal{B}(\mathbb{R})$.
\end{proof}

{\bf (iv)} Define $\mu^{*}(A) = \inf\left\{ \sum_{n=1}^{\infty}\mu(A_{n}) : \left\{ A_{n} \right\}_{n\geq 1}\subseteq \mathcal{C}, A \subseteq
\cup_{n\geq 1}A_{n} \right\}$. By Theorem 1.3.3,
\[ C \subseteq M_{\mu^{*}} \equiv \left\{ A : A \text{ is $\mu^{*}$-measurable} \right\}. \]
Further, by Theorem 1.3.2, $M_{\mu^{*}}$ is a $\sigma$-algebra. Thus $\mathcal{B}(\mathbb{R}) = \sigma\langle\mathcal{C}\rangle \subseteq
M_{\mu^{*}}$. Hence every set $A \subseteq \mathcal{B}(\mathbb{R})$ is measurable with respect to $\mu^{*}$.

{\bf (v)} Let $\mu'$ be another extension of $\mu$ that is a measure on $\mathcal{B}(\mathbb{R})$. Since $\mu^{*}$ is finite, 
$\mu' = \mu^{*}$ on all of $\sigma\langle\mathcal{C}\rangle = \mathcal{B}(\mathbb{R})$ by Theorem 1.2.4. Thus the extension is unique.
The common name for this measure is the uniform (0,1) cdf.
\end{document}

